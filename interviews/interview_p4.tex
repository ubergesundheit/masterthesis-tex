\textbf{Teil 1 -- B{\"u}rgerbeteiligung}
\begin{itemize}
    \item[I:] Erz{\"a}hlen Sie mir {\"u}ber ihre Rolle und Aufgaben in der B{\"u}rgerbeteiligung.
    \item[P4:] Durch mein FSJ in der B{\"u}rgerstiftung stehe ich mit vielen Menschen in Kontakt die sich engagieren m{\"o}chten, die sich engagieren in verschiedenen Projekten der B{\"u}rgerstiftung. Und meine Rolle ist da konkret dass ich denen halt weiterhelfe und die in diese Projekte vermittele und daf{\"u}r sorge, dass die da das machen k{\"o}nnen, was sie gerne m{\"o}chten. Und dann einen Platz finden wie sie sich engagieren k{\"o}nnen.
    \item[I:] Und sind Sie da dann eher "`Organisator"' oder "`an der Basis"'?
    \item[P4:] Ich bin schon mehr Organisator, also f{\"u}r den administrativen Teil zust{\"a}ndig. In der Stiftung sind die Projekte halt so aufgebaut, dass jemand aus dem Vorstand ein Projekt betreut. Und da drunter gibts dann jeweils Projektleiter. Wenn man sich jetzt mal die Mentoren oder Lesepatenprojekte anschaut, gibts eine Projektleiterin, dadrunter sind dann jeweils an den Schulen noch Projektleiter, die sich halt selber auch noch engagieren als Mentor oder Lesepate, aber die k{\"u}mmern sich halt darum, dass Leute, die neu einsteigen m{\"o}chten, da einen Platz finden, eine Zeit bekommen, Sch{\"u}ler bekommen die sie betreuen. Genau. Und ich bin im B{\"u}ro f{\"u}r die Verwaltung entsprechend zust{\"a}ndig. Das hei{\ss}t, wenn Mails an die Leute verschickt werden m{\"u}ssen. Wenn Briefe fertig gemacht werden. Pflege von Listen. Und dann halt Telefonkontakte zu denen l{\"a}uft dann halt viel {\"u}ber das B{\"u}ro, {\"u}ber mich, weil wir die ganzen Daten haben. Und Verwaltung von F{\"u}hrungszeugnissen, dass die dann halt auch wirklich anfangen k{\"o}nnen. So der administrative Teil.
    \item[I:] Bitte beschreiben Sie mir die aus ihrer Sicht wichtigsten Aspekte der B{\"u}rgerbeteiligung.
    \item[P4:] (\dots) Der Nutzen ist halt, finde ich, (\dots). B{\"u}rger haben ja verschiedene Stellungen, haben ja verschiedene Erfahrungen, und k{\"o}nnen damit Dinge weitergeben. K{\"o}nnen anderen ziemlich simpel helfen und anderen Leuten einfach etwas gutes tun, ohne dass da viel Aufwand betrieben werden muss. Man geht halt selbst einfach irgendwo nur in die Schule und liest einfach ein bisschen vor. Und das Kind profitiert dadurch einfach dass es in der Schule dadurch bessere Chancen hat, im Deutschunterricht klar zu kommen. Man erm{\"o}glicht den Kindern dadurch einfach, (\dots) ein besseres Leben. Das ist jetzt vielleicht ein bisschen hoch gegriffen, aber man kann dadurch einfach ein bisschen was verbessern. Ja und dieses selbstlose finde ich da sehr wichtig, dass man da einfach sagt "`Ich habe Sachen die ich machen kann"' oder "`Ich habe freie Resourcen und damit setze ich mich f{\"u}r andere einfach ein, die eben nicht den Luxus haben, dass man eben selbst die freie Zeit hat, dass man da Mittel zur Verf{\"u}gung hat."'
    \item[I:] Bitte geben Sie mir eine Einf{\"u}hrung in ein laufendes oder abgeschlossenes Projekt oder Initiative bei der Sie denken, dass dort Dialoge zwischen den Akteuren am wichtigsten war oder ist.
    \item[P4:] (\dots) Da sind nat{\"u}rlich solche Projekte wie der B{\"u}rgerpreis. Das ist ein Preis, der vergeben wird, wo halt wirklich Einsatz  aus der B{\"u}rgerschaft gew{\"u}rdigt wird. Das sorgt nat{\"u}rlich f{\"u}r entsprechende Gespr{\"a}chsstoffe, wenn da jedes Jahr wieder andere Projekte und Themen gew{\"u}rdigt werden. Und dann da auch verschiedene Leute ins Gespr{\"a}ch kommen durch ihre eigenen Bewerbungen. Dass der {\"O}ffentlichkeit da einfach viel vorgestellt wird. Das sind dann ja auch Akteure aus allen m{\"o}glichen Schichten dabei, die komplett verschieden auch sind. Also ansonsten bei den anderen Projekten der B{\"u}rgerstiftung nicht direkt Dialog zwischen den Akteuren und B{\"u}rgern. In den Projekten intern, da finden eigentlich Dialoge statt zwischen Akteuren und Sch{\"u}lern oder Jugendlichen. Aber nicht Dialog nach au{\ss}en hin.
    \item[I:] Bleiben wir einfach bei dem B{\"u}rgerpreis. Wie findet dort dann das Vorschlagen statt?
    \item[P4:] Der Preis wird von der Stiftung ausgeschrieben zu einem bestimmten Thema. Und da es ja die "`B{\"u}rgerstiftung -- B{\"u}rger f{\"u}r M{\"u}nster"' hei{\ss}t, ist ganz klar, dass ein gewisser M{\"u}nsterbezug da sein muss. Das hei{\ss}t, Bewerbungen werden nur der Jury weitergegeben, wenn sie aus M{\"u}nster wirklich kommen, also der M{\"u}nsterbezug da ist. Und es muss komplett rein ehrenamtlich sein. Das hei{\ss}t im Prinzip keine Verg{\"u}tung f{\"u}r die Arbeit. Und es muss entsprechend zum Thema passen. Ja. Man bewirbt sich normal als Verein oder Initiative dort selber. Dass Leute vorgeschlagen werden, ist eher selten, da der Preis eher auf Gruppen abzielt, anstatt auf Einzelpersonen. Da halt breites b{\"u}rgerschaftliches Engagement sichtbar gemacht werden soll. Es gibt zwar viele Einzelpersonen die sich engagieren. Das ist auch gut so, aber es soll mehr das gezeigt werden, wo sich B{\"u}rger zusammentun und zusammen was gutes machen, daran Spa{\ss} haben und dann dabei auch noch was f{\"u}r die Stadt tun.
\end{itemize}

\textbf{Teil 2 -- Einsatz der Anwendung}
\begin{itemize}
    \item[I:] Bitte geben Sie mir eine Einf{\"u}hrung in das Projekt in dem die Anwendung eingesetzt werden soll.
    \item[P4:] Das Gutscheinheft tr{\"a}gt den Beititel "`1000 Stunden f{\"u}r M{\"u}nster"' und soll so sein, dass dort f{\"u}nfundzwanzig Vereine und Initiativen dabei sind, die jeweils einen Gutschein {\"u}ber ein bis zwei Stunden b{\"u}rgerschaftliches Engagement anbieten. Das hei{\ss}t dass man da einfach hingehen kann, und sich dort kurzfristig engagiert. So sollen dann insgesamt eintausend Stunden von dem B{\"u}rgerengagement f{\"u}r die Stadt zusammenkommen. Und dort ist es dann nat{\"u}rlich so, dass die Vereine ja {\"u}ber die ganze Stadt verteilt sind. Die stellen sich in dem Heft kurz vor. Also eine Projektbeschreibung und Allgemein was die Einrichtung oder Vereine machen. Das wird dann so da drin sein. Allerdings ist das ja ein Unterschied, ob man jetzt sich eine Karte anschaut, und da einfach mit der Maus {\"u}ber ein paar Punkte geht und einem dass dann eingeblendet wird, als wenn man ein Heft vor sich liegen hat und da dann jede Adresse anschaut und dann guckt, wie weit das jetzt von sich entfernt ist. "`Wie weit ist das von mir weg, kann ich da nicht eben mal so gleich vorbeigehen?"'. Dementsprechend k{\"o}nnte ich mir das gut vorstellen, dass die Anwendung sich da sehr gut erg{\"a}nzen. Weil das ja im Prinzip ja nur eine digitale Form davon ist. Das dass einfach einem die M{\"o}glichkeit gibt, dass man Fokus auf seinen Standort gerade legt, und dann schaut wo ist in der N{\"a}he denn was, wo ich mit einer kurzen Strecke schnell hinkomme. Und das ist halt bei einem Heft (\dots) ist halt keine interaktive Karte mit drin.
    \item[I:] Welche Anreize k{\"o}nnte man geben, damit sich B{\"u}rger {\"u}ber die eingestellten Projekte austauschen?
    \item[P4:] Anreize zum Austauschen. (\dots) Also Leute berichten ja gerne {\"u}ber Sachen die entweder sehr schlecht waren oder sehr gut waren. (\dots) Aber wie man einen konkreten Anreiz erschafft, da f{\"a}llt mir konkret nichts ein. Wie man da ja Beitr{\"a}ge bekommen w{\"u}rde, wenn die Leute da sind, dass man denen dann direkt da vor Ort einmal die M{\"o}glichkeit gibt da direkt was einzutragen. Das w{\"a}re auf jeden Fall was sinnvolles, wie man da dann dazu kommt, dass da erstmal was steht, und ich denke wenn da was steht, und andere Leute das lesen, dass es da sehr gut war, dann ist das nat{\"u}rlich auch eine Motivation f{\"u}r die "`Ah, der hat da was spannendes erlebt, vom Titel her reizt mich das auch schon, nach den Sachen was der da schreibt, w{\"u}rde ich da glaube ich auch gerne hin"'. Aber direkt den Anreiz zu schaffen, w{\"u}sste ich gerade keinen.
    \item[I:] Gegen{\"u}ber anderen angedachten L{\"o}sungen, welche L{\"o}sungen sprechen f{\"u}r den Einsatz dieser L{\"o}sung?
    \item[P4:] Die andere L{\"o}sung, oder die andere M{\"o}glichkeit wie man das publik machen k{\"o}nnte, w{\"a}re {\"u}ber eine Internet-Seite. Wahrscheinlich auch erstmal nur in Listenform. Und f{\"u}r mich so was interaktives nat{\"u}rlich ein Schritt auf die Leute zu. Und dass man auch zeigt, dass man eben auch mit den neuen Medien gut arbeitet. Und eben nicht halt nur steife Listen dem Benutzer vorsetzt. Da gef{\"a}llt mir das Interaktive sehr gut, weil es halt einfacher f{\"u}r die Leute ist. Und viele M{\"o}glichkeiten eben bietet.
    \item[I:] Welche Eigenschaften w{\"u}rden Sie davon abhalten diese L{\"o}sung einzusetzen?
    \item[P4:] Ein eventueller Grund k{\"o}nnte sein, wenn damit noch h{\"o}here Kosten verbunden sind. Da man als Stiftung eigentlich eher das Geld f{\"u}r andere Vereine und zur Unterst{\"u}tzung verwenden will. Und der Verwaltungsaufwand und Werbeaufwand immer sehr gering gehalten werden soll. Das k{\"o}nnte ich mir vorstellen, dass das ein Grund w{\"a}re sich dagegen zu entscheiden. Und ansonsten wenn damit noch Verwaltungsaufwand verbunden ist. Dass man sein kleines Profil dass da sozusagen drin ist, dass man das sehr sehr viel pflegen muss. Weil das dann immer noch zus{\"a}tzliche Aufgaben sind, die dann anfallen, die halt bei einer einfachen Karte oder bei einer einfachen Liste wo die Sachen drin stehen halt nicht anfallen w{\"u}rden, da dass ja nur reines Informationsmaterial da f{\"u}r die Personen w{\"a}re.
    \item[I:] Was k{\"o}nnte Ihrer Meinung nach B{\"u}rger davon abhalten sich zu beteiligen?
    \item[P4:] Zum einen k{\"o}nnte dass die Anmelung eventuell sein. Dass da vielleicht eine kleine Hemmschwelle ist, oder dass das einen davon abh{\"a}lt, dass man sich da auch noch anmelden muss. Das w{\"u}rde ja aber nicht hei{\ss}en, dass die Leute das nicht benutzen k{\"o}nnten um sich nur zu informieren. Das w{\"u}rde ja so gehen. Ja die Anmeldung f{\"u}r junge Leute, die Facebook oder Twitter haben, ist das glaube ich kein Problem das zu nutzen, Ja aber wenn man halt die Generation f{\"u}nfzig plus oder sechzig plus ist, die ja da doch auf die Daten doch sehr viel mehr achten, dass die da dann eventuell zur{\"u}ckschrecken, und das dann nur zur Information nutzen.
    \item[I:] K{\"o}nnen Sie sich weitere Anwendungsf{\"a}lle f{\"u}r die Verkn{\"u}pfung von Texten mit Karten neben der B{\"u}rgerbeteiligung vorstellen?
    \item[P4:] (\dots) Also das Programm noch f{\"u}r andere Zwecke? (Ja) Ja im Prinzip gibts das ja schon f{\"u}r alle m{\"o}glichen Suchen. Ob es jetzt einfach (\dots) Ja ich denke gerade an Foursquare. Da wird mir ja auch angezeigt, was in der N{\"a}he f{\"u}r L{\"a}den, Gesch{\"a}fte und {\"a}hnliches gibt. Das k{\"o}nnte man nat{\"u}rlich allgemein auf Vereine ausweiten wenn man in dem B{\"u}rgerschaftlichen bleiben m{\"o}chte. Dass man Sportvereine und vielleicht noch sonstige Aktivit{\"a}ten einflechtet, und damit eine Karte h{\"a}tte, wo allgemein was drin ist, wie man sich (\dots) ja was man wo in der Stadt machen kann. Was f{\"u}r Angebote es gibt.
\end{itemize}

\textbf{Teil 3 -- Abschlie{\ss}ende Fragen}
\begin{itemize}
    \item[I:] Kennen Sie Beispiele f{\"u}r die Verkn{\"u}pfung geographischer Daten mit Diskussionsbeitr{\"a}gen?
    \item[P4:] Ja hatte ich ja gerade schon gesagt. Foursquare oder wie hei{\ss}t das. Yelp. Da kann man ja auch so Dinge eintragen kann. Bei Google gibts die Funktion ja glaube ich auch. Dass in den Karten was angezeigt wird. Dass ist dann ja immer noch {\"u}ber verschiedene Seiten verkn{\"u}pft. Was ich selber auch nur zur Information nutze. Mir ist die Anmeldung da halt (\dots) Dass man sich ja daf{\"u}r wieder anmelden muss, das hat mich bisher gehindert da selber mal aktiv zu werden.
    \item[I:] Also auch noch nicht beteiligt?
    \item[P4:] Nur bei Foursquare mal.
    \item[I:] Kennen Sie Werkzeuge um interaktive Karten mit eigenen Inhalten zu erzeugen?
    \item[P4:] Ja ich habe mal f{\"u}r eine Internetseite (\dots) Da kann man halt bei Google Routen erstellen, Routen anzeigen. Oder mit Openstreetmap gibts ja glaube ich auch. Das habe ich mal genutzt. Ansonsten nichts.
    \item[I:] Gut. Gibt es dann noch Anmerkungen oder Fragen von Ihrer Seite aus?
    \item[P4:] Ne h{\"o}chstens zur Umsetzung, ob das dann wirklich dann bald ins Netz gehen soll?
    \item[I:] Ja online ist das schon. Aber {\"u}ber die genauen Rahmenbedingungen muss da nochmal an anderer Stelle gesprochen werden. Die Seite wird wohl n{\"a}chstes Jahr f{\"u}r den Tag der Nachhaltigkeit eingesetzt werden.
    \item[P4:] Das hei{\ss}t, man platziert dann im Prinzip auch so ein Kartentool auf auf so einer Seite? Und ansonsten wie findet man das?
    \item[I:] Ja genau. Die Adresse kann ich Ihnen gleich auch nochmal aufschreiben.
    \item[P4:] Ja das w{\"a}re sch{\"o}n.
    \item[I:] Alles klar Dann auf jeden Fall vielen Dank.
\end{itemize}