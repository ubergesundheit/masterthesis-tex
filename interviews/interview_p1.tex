\textbf{Teil 1 -- B{\"u}rgerbeteiligung}
\begin{itemize}
    \item[I:] Erz{\"a}hlen Sie mir {\"u}ber ihre Rolle und Aufgaben in B{\"u}rgerbeteiligung
    \item[P1:] Ich kann nur sagen was f{\"u}r mich wichtig ist
    \item[I:] Ja das ist auch eine Sache die Sie mir erz{\"a}hlen k{\"o}nnen. Dann beschreiben Sie mir bitte die aus ihrer Sicht wichtigsten Aspekte der B{\"u}rgerbeteiligung
    \item[P1:] Die B{\"u}rgerbeteiligung f{\"u}hrt dazu, das erstmal Leute sich informieren, dass sie mehr wissen als nur {\"u}ber Zeitung. Dann k{\"o}nnen sie sich auch zusammenschlie{\ss}en und diskutieren und Aktionen besprechen. Ja und auch entsprechend Aktionen machen. Das st{\"a}rkt auch im Grunde eine Stadt.
    \item[I:] An welchen B{\"u}rgerbeteiligungsaktionen haben Sie dann schonmal teilgenommen?
	\item[P1:] Ja die Frage ist jetzt was alles unter B{\"u}rgerbeteiligung f{\"a}llt?
	\item[I:] Da kann alles drunter fallen, was f{\"u}r die {\"O}ffentlichkeit geschieht.
	\item[P1:] Also, ich habe zum Beispiel mit Leuten zusammen einen Gemeinschaftsgarten, der ist gegr{\"u}ndet worden und da treffen wir uns, da wird Gem{\"u}se angebaut, da gibts Bienen und da ist auch gedacht, dass man sich noch in Zukunft wenn das mal l{\"a}uft sich vernetzt mit anderen G{\"a}rten. Zum Beispiel zum Thema Bienen haben wir einen Nachmittag gehabt. Aber das kann man nat{\"u}rlich auch in einen gr{\"o}{\ss}eren Ma{\ss}stab machen.
	\item[I:] Und w{\"u}rden Sie denken, dass in diesem Kontext so eine Art von Anwendung dann sinnvoll einzusetzen w{\"a}re um das ganze bekannter zu machen und die Inhalte nach au{\ss}en zu kommunizieren?
	\item[P1:] Ja erstmal stell ich mir das so vor, dass jemand der, meinetwegen, neu ist oder keine Kontakte hat, sich mit Hilfe der Karte {\"u}berhaupt mal ein Bild machen was es f{\"u}r M{\"o}glichkeiten gibt. Und dann geht es ja in die Feindifferenzierung. Da w{\"u}rde er sagen: "`Gut, ich interessiere mich f{\"u}r Umwelt. Wer ist zust{\"a}ndig f{\"u}r Umwelt. Naja Greenpeace kann ich mal anklicken. Wo treffen die sich. Wann treffen die sich. Was haben die f{\"u}r Aktivit{\"a}ten zum Beispiel. Oder Transition Town. Was machen die eigentlich. Muss ich mal lesen was das {\"u}berhaupt ist. Ich wei{\ss} gar nicht genau was das ist. Also kann ich das mal lesen und vielleicht auch Kontakt aufnehmen."' Ich muss mir jetzt nicht m{\"u}hsam diese ganzen Adressen zusammensuchen. Diese WWW-Adressen, sondern die sind ja auf deiner Karte schon angegeben. Das ist nat{\"u}rlich schon auch erleichternd ist. Denn manchmal scheitert es an solchen Sachen. Auch an Bequemlichkeit.
	\item[I:] Wie l{\"a}uft dann im Moment die Kommunikation intern f{\"u}r diesen Garten ab?
	\item[P1:] {\"U}ber E-Mail und {\"u}ber Treffen.
	\item[I:] Wie oft treffen sie sich da?
	\item[P1:] Ja da gibts dann Einladungen. Aber das ist unterschiedlich. Alle zwei Monate wenn was ansteht. Jetzt wo das Wasser da ist, da trifft man sich mal um aufzur{\"a}umen oder um Projekte zu besprechen.	
\end{itemize}

\textbf{Teil 2 -- Einsatz der Anwendung}
\begin{itemize}
	\item[I:] Wie soll dann die Beteiligung von Transition Town oder dem Garten auf dem Nachhaltigkeitstag aussehen?
	\item[P1:] Naja das kann ich jetzt nur erfinden. Letztenendes m{\"u}ssen wir das ja als Gruppe besprechen.
	\item[I:] Also am besten wie Sie sich das vorstellen
	\item[P1:] Themen die Transition Town wichtig sind w{\"u}rden da einen Raum finden und den Rahmen m{\"u}ssen die sich dann geben. Ob das jetzt in Form von (\dots) dass man gesundes Essen anbietet, oder mal so ne Karte entwirft wo Transition Town ist. Es gibt ja auch einen Film {\"u}ber Transition Town. Da gibts ja vielf{\"a}ltige M{\"o}glichkeiten. (\dots) Zum Beispiel in unserem Garten da hat die Bienen-Frau einen Vortrag gehalten {\"u}ber die Bienen und das soziale Miteinander. Das ist ja hoch differenziert. Zum Beispiel die Drohnen, die treffen sich an ganz bestimmten Pl{\"a}tzen vierzig Meter {\"u}ber der Erde. Solche die Detailinformationen die kein Mensch eigentlich wei{\ss}, die k{\"o}nnte man dann geben, in dem diese Bienen-Frau vielleicht was mitbringt und dann dar{\"u}ber redet und das dann auch Kindern zeigt wie so ein Bienenstock aussieht und mal Honig probieren l{\"a}sst. Und dann auch zum engagieren auffordert. Oder hab ich heute in der Zeitung gelesen, dass es einen Jungen gibt, der hat ein Bienenhaus gebaut f{\"u}r den Balkon. Der w{\"u}rde dann eingeladen und w{\"u}rde das vorstellen.
	\item[I:] Wer w{\"a}re dann die Zielgruppe? 
	\item[P1:] Wie meinst du die Zielgruppe?
	\item[I:] Ich meine damit die Personenkreise die man ansprechen m{\"o}chte
	\item[P1:] Naja an dem Tag werden ja viele Menschen da sein. Und das w{\"a}re ja dann ein wichtiger Aspekt zum Nachhaltigkeitsthema. Ich meine, da dass ja wahrscheinlich drau{\ss}en stattfindet, kann man ja von Zielgruppe nicht so unbedingt sprechen, oder? Wer will, der kommt.
	\item[I:] Gibt es andere Ans{\"a}tze die Sie zur Kommunikation bez{\"u}glich des Nachhaltigkeitstages in Betracht gezogen haben?
	\item[P1:] Nein im Moment nicht.
	\item[I:] Was f{\"u}r Gr{\"u}nde w{\"u}rden f{\"u}r den Einsatz der Karte sprechen? 
    \item[P1:] Also du meinst was f{\"u}r Vorteile es f{\"u}r uns h{\"a}tte den Garten in deine Karte einzutragen?
    \item[I:] Richtig.
	\item[P1:] Das h{\"a}tte den Vorteil, dass man auf einen Blick sehen kann, da und da und da gibt es einen freien Garten. Man sieht welche Adresse das sind. Man sieht vielleicht auch wann die da sind. Und dann ist das nat{\"u}rlich sehr {\"u}bersichtlich. Mit einem Klick hat man sozusagen die Information. Es gibt ja noch mehrere G{\"a}rten. Es gibt da unseren Paradies-Garten, dann gibt es am Campus noch einen Garten, dann gibts noch an der Gasselstiege einen Garten. Ja. Die w{\"u}rde man dann da sehen und dann k{\"o}nnte man auch Leute die das wollen, meinetwegen, eine Fahhradtour machen lassen und die G{\"a}rten angucken.
	\item[I:] Die Karte k{\"o}nnte man auch benutzen um die Fahrradtour zu organisieren. Dass man Start, Ziel und Zwischenhalte markiert. 
	\item[P1:] Ja genau.
	\item[I:] Was k{\"o}nnten Sachen sein die B{\"u}rger davon abhalten w{\"u}rden diese Karte zu benutzen?
	\item[P1:] (\dots) Ja also die Karte ist ja elektronisch. Geht ja nur {\"u}ber das Internet. Also sofern man einen Internetanschluss hat und einen Laptop oder einen Computer, gibts da nichts was dagegen spricht.
\end{itemize}
\textbf{Teil 3 -- Abschlie{\ss}ende Fragen}
\begin{itemize}
    \item[I:] Kennen Sie Beispiele f{\"u}r die Verkn{\"u}pfung geographischer Daten mit Diskussionsbeitr{\"a}gen?
	\item[P1:] Sag mir nochmal was man alles unter geographische Daten fasst.
	\item[I:] Orte und Objekte mit einem Ort
	\item[P1:] Naja ich war jetzt auf dem Jakobsweg, da hat man auch Karten. Aber die nutzt man nicht so oft. Da hat man B{\"u}cher in denen die Adressen drin stehen. Und die Zeichen sind an den B{\"a}umen.
	\item[I:] Ja ist auch ne M{\"o}glichkeit. Ich ziele mit der Frage eher ab auf auf elektronische Anwendungen.
	\item[P1:] Nein, ich nicht da auch nicht so firm. Ich mag das auch nicht.
	\item[I:] Also haben Sie sowas auch noch nie benutzt?
	\item[P1:] Nein.
	\item[I:] Kennen Sie Werkzeuge um interaktive Karten mit eigenen Inhalten zu erzeugen?
	\item[P1:] Nein. Es gibt ja viele Leute die nicht so interessiert sind mit den neuen Medien. 
	\item[I:] Ja. Alles klar. Das waren dann die Fragen von meiner Seite. Gibt es noch Fragen von ihrer Seite? 
    \item[P1:] Nein. Eigentlich nicht. Gute Sache.
    \item[I:] Vielen Dank f{\"u}r ihre Zeit.
\end{itemize}