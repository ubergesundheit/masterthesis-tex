\textbf{Teil 1 -- B{\"u}rgerbeteiligung}
\begin{itemize}
    \item[I:] Erz{\"a}hlen Sie mir {\"u}ber ihre Rolle und Aufgaben in der B{\"u}rgerbeteiligung.
    \item[P6:] Also wo ich mich sehe? (Genau) Ich glaube einfach als aktive M{\"u}nsteranerin, die schon an ganz verschiedenen Stellen sich ehrenamtlich engagiert hat, und Spa{\ss} daran hat, bestimmte Projekte auf die Beine zu stellen und es auch ein St{\"u}ck weit als Verantwortung sieht, sich Gesellschaftlich zu engagieren und sich einzubringen.
    \item[I:] Wie lange sind Sie dann jetzt schon aktiv?
    \item[P6:] Also irgendwie schon seit Schulzeiten. Schon bestimmt zwanzig Jahre oder so.	Also immer wieder mal ganz unterschiedliche Stellen. Auch mal Pause wieder zwischendurch.
    \item[I:] Welche Aspekte der B{\"u}rgerbeteiligung sind die aus Ihrer Sicht die Wichtigsten?
    \item[P6:] Ich glaube es geht mir ganz wichtig darum ein gutes Zusammensein zu haben. Und ein lebenswertes Leben wo man als Mensch mit dabei ist. Wo man sich einbringen kann, f{\"u}r Sachen die einem selber wichtig sind. Wo man anderen Menschen begegnen kann. Wo man was gemeinsam miteinander macht, und irgendwie das Gef{\"u}hl hat, es ist etwas sinnvolles. Man lernt neue Leute kennen. Man lernt neue Sichtweisen kennen. Das Miteinander ist mir wichtig vor allen Dingen. Und dann sicherlich auch eine gute Arbeit zu machen.
    \item[I:] Bitte geben Sie mir eine Einf{\"u}hrung in ein Projekt bei dem Sie denken dass es besonders auf eine gute Kommunikation und Diskussion zwischen den Akteuren angekommen ist.
    \item[P6:] Ich habe mich vor Jahren in der Queer-Gemeinde engagiert. War im Leitungsteam und Sprecherin der Gruppe. Ich habe vor Jahren Theologie studiert. Und das war eine Gruppe von Leuten, mit sehr vielen Schwulen und Lesben. Aber auch Leute die sich irgendwie nicht so ganz mit der Kirche identifizieren k{\"o}nnen. Da war es schon sehr wichtig, dass wir uns vernetzt haben, weil es erstens am Anfang eine sehr kleine Gruppe war. Und um so das Projekt am laufen zu halten, und auch in der Au{\ss}einandersetzung mit dem Bistum und der offiziellen Bistumspolitik, war schon auch wichtig, Kontakt zu anderen Gruppen zu haben. Also der Hook, der der anderen Queer-Initiativen in ganz Deutschland oder anderen Gruppierungen, Gespr{\"a}chskreisen, Universit{\"a}tskreisen, Gruppierungen innerhalb des Bistums. Und da war das sehr sehr wichtig, und auch bereichernd und hilfreich.
    \item[I:] Wie ist dann die Kommunikation abgelaufen?
    \item[P6:] Also so ein Internettool hatten wir noch nicht. Wir hatten selber einen Internetauftritt, und haben uns als Gruppe vorgestellt. So offen wie das mit dieser Thematik halt m{\"o}glich ist. Da ist nat{\"u}rlich eine heikle Sache, wenn sich schwule Priester oder schwule Theologen (\dots) schwule Theologen k{\"o}nnen sich nicht unbedingt outen. Lesbische eben auch nicht. Und in sofern eben tats{\"a}chlich eben E-Mail-Verteiler, Linklisten im Internet auf andere Initiativen. Und nat{\"u}rlich haben wir uns als Gruppe regelm{\"a}{\ss}ig mindestens einmal im Monat getroffen und {\"u}berlegt und geplant. Zumindest das Orga-Team. Und dar{\"u}ber hinaus eben noch andere Veranstaltungen. Das waren dann Gottesdienste f{\"u}r diese Randgruppe. Dann haben wir zum Beispiel Gemeindefeste oder Wochenendbegegnungen organisiert. Und da sind dann teilweise auch Leute aus ganz Deutschland dazu gekommen.
    \item[I:] Die Kommunikation ist dann eher in der Gruppe geblieben, oder wurde auch aktiv nach au{\ss}en kommuniziert?
    \item[P6:] Wir haben Anfragen von au{\ss}en bekommen. Es gab immer mal wieder Radiosender oder Journalisten, die was von uns wissen wollten. Selber so ganz aktiv haben wir durch die Internetseite oder durch bekannt machen in bestimmten Publikationen, es gab Na-Dann zum Beispiel, oder andere Printmedien die es glaube ich mittlerweile gar nicht mehr gibt, und eben {\"u}ber den universit{\"a}ren Kontext bekannt gemacht, dass es uns gibt. Aber eben auch nicht zu offensiv, weil es ist halt auch ein heikles Thema f{\"u}r viele.
\end{itemize}

\textbf{Teil 2 -- Einsatz der Anwendung}
\begin{itemize}
    \item[I:] Bitte geben Sie mir eine Einf{\"u}hrung in das Projekt, in dem die Anwendung eingesetzt werden soll.
    \item[P6:] Im n{\"a}chsten Jahr ist so ein Nachhaltigkeitstag geplant in M{\"u}nster, und da soll es darum gehen, diese Thematik "Nachhaltigkeit, nachhaltiges Leben" einer breiteren Stadtgemeinschaft bekannt zu machen, und auch einzuladen, sich damit auseinanderzusetzen. Die Zielgruppe ist unglaublich heterogen. Vom Kindergarten bis Universit{\"a}t. Von Wissenschaftlern bis Kreative. Da gibt es eine Menge Ideen. Ein wichtiger Punkt ist eben sicherlich auch eine M{\"o}glichkeit zu haben, dass man erstmal sagt was {\"u}berhaupt schon da ist. Weil ich glaube, dass es eine Menge Initiativen und M{\"o}glichkeiten hier in M{\"u}nster gibt, die aber noch nicht so optimal vernetzt sind, dass man wirklich von einander wei{\ss}, oder sich auch schnell und komplikationslos {\"u}ber deren Aktivit{\"a}ten informieren kann. Und daf{\"u}r k{\"o}nnte ich mir vorstellen, dass das gut w{\"a}re.
    \item[I:] Was f{\"u}r redaktionelle Inhalte wollen Sie einstellen?
    \item[P6:] Also wenn ich es mir einfach so als Nutzerin vorstelle, dann finde ich das mit der Karte gut. Dass man sich sagen kann, "Ich m{\"o}chte gerne nach bestimmten Kategorien sortieren. Beispielsweise welche Gesch{\"a}fte nachhaltige Produkte verkaufen". Ich glaube es w{\"a}re mir wichtig, dass es aktuell ist. Ich glaube dass ich die klassischen Fragen habe "Wo steht es, wie kann ich Kontakt aufnehmen, was machen die Inhaltlich". Also eine Verlinkung wenn m{\"o}glich zu deren Internetseite, dass man fragen kann "Was machen die so? Passt das zu mir?" Ich glaube das w{\"a}ren erstmal so die Grundinformationen, die erstmal wichtig w{\"a}ren.
    \item[I:] Welche Anreize f{\"u}r B{\"u}rger sich dann in Dialogen {\"u}ber die Anwendung auszutauschen sehen sie hier dann?
    \item[P6:] Ich glaube jetzt eher weniger durch die Internetseite, sondern durch die Aktivit{\"a}ten der Leute selber.
    \item[I:] Welche Gr{\"u}nde sprechen f{\"u}r den Einsatz dieser Anwendung gegen{\"u}ber anderen Anwendungen?
    \item[P6:] Also das einzige, was ich mal mitgekriegt habe, dass wohl der Asta mal vor zwei oder drei Jahren ein PDF-Dokument ver{\"o}ffentlicht hat, wo eben auch eine relativ ausf{\"u}hrliche Liste mit Nachhaltigkeitsinitiativen drin war. Die hat nat{\"u}rlich erstmal das Problem, dass sie erstmal m{\"u}hsam suchen muss im Internet und erst durch Gl{\"u}ck findet. Und dass nat{\"u}lich so ein PDF-Dokument innerhalb von ein bis zwei Jahren sp{\"a}testens veraltet ist. Dass das eine M{\"o}glichkeit ist, Daten und Informationen auch aktuell zu halten. Und wenn es m{\"o}glich w{\"a}re, sie eben sehr prominent zu platzieren. Meintetwegen auf der Seite der Stadt M{\"u}nster. Eben auch sehr gut zug{\"a}nglich und mit der Chance dann auch vielleicht Leute anzusprechen, die, wei{\ss} ich nicht, neu sind, und so einfach mal gucken wollen was in dieser Stadt los ist und dann dazu sto{\ss}en k{\"o}nnen.
    \item[I:] Was w{\"a}ren Dinge, von denen Sie denken, dass sie B{\"u}rger davon abhalten k{\"o}nnten, sich {\"u}ber die Anwendung zu beteiligen?
    \item[P6:] Es muss funktional sein. Also es m{\"u}sste relativ selbsterkl{\"a}rend sein. Ich finde es nicht gut, wenn man unglaublich lang durchklicken muss. Es muss ziemlich schnell sein. Also wie ist es aufgebaut, wie schnell komme ich an meine Informationen? Die wichtigsten Sachen m{\"u}ssten da sein. Also wie, wo, was, wann. Vielleicht noch die M{\"o}glichkeit Kontakt aufzunehmen. Das w{\"a}re glaube ich so das wichtigste. So als Ersteinstieg in diese Thematik.
    \item[I:] K{\"o}nnen Sie sich weiter Anwendungsf{\"a}lle f{\"u}r die Verkn{\"u}pfung von Texten mit Karten au{\ss}erhalb der B{\"u}rgerbeteiligung vorstellen?
    \item[P6:] Also das w{\"a}re jetzt so erst mal meine erste Idee gewesen. Was ich mir nat{\"u}tlich vorstellen k{\"o}nnte, dass solche Karten auch in anderen St{\"a}dten programmiert werden. Und man von daher dann eben auch Verkn{\"u}pfungen zwischen diesen St{\"a}dten herstellt. Dass man also vielleicht guckt, in welchen St{\"a}dten gibt es beispielsweise so eine Givebox, die wir da eben programmiert haben. Oder wo gibt es bestimmte regionale Untertreffen von beispielsweise Greenpeace oder irgendwelchen anderen Initiativen. Vielleicht auch die M{\"o}glichkeit, dass bestimmte Gruppen oder gr{\"o}{\ss}ere, auch {\"u}bergreifende Initiativen, auch Daten	oder so in diese Seite mit einbauen, dass man gucken kann, wo findet vielleicht so etwas wie der Nachhaltigkeitstag statt. Ist das nicht nur bei uns in M{\"u}nster, oder in Bremen gibt es etwas anderes. Wobei die glaube ich auch ein anderes Konzept haben. Gibt es das in K{\"o}ln, gibt es das in Aachen oder anderen gr{\"o}{\ss}eren St{\"a}dten. Das f{\"a}nde ich auch vielleicht ganz gut, dass da sich nicht nur aussenstehende B{\"u}rger und B{\"u}rgerinnen informieren k{\"o}nnen. Sondern auch Initiativen selber.
\end{itemize}

\textbf{Teil 3 -- Abschlie{\ss}ende Fragen}
\begin{itemize}
    \item[I:] Kennen Sie Beispiele f{\"u}r die Verkn{\"u}pfung geographischer Daten mit Diskussionsbeitr{\"a}gen?
    \item[P6:] Nein. Kenne ich selber noch nicht.
    \item[I:] Und Werkzeuge im interaktive Karten mit eigenen Inhalten zu erzeugen?
    \item[P6:] Habe ich bisher auch noch nicht benutzt, finde ich aber ziemlich gut, dass man von anderen Seiten auf diese Karte zugreifen k{\"o}nnte. Das meinen Sie doch, dass ich diese Karte jetzt in meine eigene einbinden k{\"o}nnte?
    \item[I:] Nein, die Frage zielt konkret auf Werkzeuge die es erlauben eigene Karten zu erzeugen. Bestes Beispiel war jetzt vor kurzem die Karte zur {\"U}berschwemmungshilfe.
    \item[P6:] Achso. Ja das habe ich nur am Rande mitbekommen. Aber ich kenne das aus anderen Bereichen. Habe ich auch selbst aber nicht genutzt.
    \item[I:] Okay. Gibt es dann noch abschlie{\ss}ende Fragen oder Anmerkungen von Ihrer Seite?
    \item[P6:] Nein. Ich denke h{\"o}chstens dann wenn es konkret mit der Umsetzung (\dots) Und, wird das angewandt, wer wird dann zust{\"a}ndig sein. Die Idee finde ich schon wirklich gut. Und dann auch immer ausgedr{\"u}ckt, mit der praktischen Erfahrung, dass solche Ideen am Anfang immer sehr gut sind, aber eben auch die langfristige Perspektive brauchen. Dass dann auch da anzubinden, wo das auch m{\"o}glichst gesichert ist.
    \item[I:] Ja das wird dann wohl noch im Detail entschieden werden. Dann vielen Dank!
\end{itemize}