\begin{itemize}
\item[I:] Hallo, erstmal vielen Dank, dass Sie sich hier Zeit f{\"u}r mich und meine Masterarbeit nehmen. Es soll jetzt gleich hier um die von mir entwickelte Anwendung gehen. Danach werde ich ihnen noch ein paar Fragen stellen. Also meine Masterarbeit hat das Thema "`Supporting public deliberation through spatially enhanced dialogues"'. Das bedeutet grob, dass ich herausfinden will wie man Dialoge in der B{\"u}rgerbeteiligung durch kartenbasierte Anwendungen unterst{\"u}tzen kann. Also, ich fange dann einfach mal mit der Demo der Anwendung an. (ruft die Anwendung auf) Wenn man als Benutzer auf die Webseite kommt, dann wird man beim ersten Besuch erstmal hier mit so einem Text begr{\"u}{\ss}t, der erstmal ein bisschen das Thema vom Nachhaltigkeitstag erkl{\"a}rt. Bei der Entwicklung der Karte wurde ich von Mitgliedern des "`Arbeitskreis Gemeinsam Nachhaltig"' unterst{\"u}tzt. Die sind vom Institut f{\"u}r Soziologie hier in M{\"u}nster. Das sah dann so aus, dass wir uns an mehreren Treffen {\"u}ber die Entwicklung der Anwendung unterhalten haben. Die Soziologen haben mir dann ihre Meinung und Ideen gesagt, die ich dann versucht habe umzusetzen. Also letztendlich soll wohl wahrscheinlich die Anwendung dann n{\"a}chstes Jahr in dem geplanten Nachhaltigkeitstag eingesetzt werden. (klickt auf weiter) Hier kriegt man so kleine Einf{\"u}hrungsvideos, aber das mache ich ja jetzt m{\"u}ndlich, da brauchen wir uns die nicht anschauen. (klickt weiter) Dann hier die Zeichenerkl{\"a}rung f{\"u}r die Marker in der Karte. Man sieht ja hier, dass es zwei Dimensionen gibt. Das sind hier die Akteure, gekennzeichnet durch die Farbe und dann die Aktivit{\"a}t durch den Buchstaben im Marker. (f{\"a}hrt mit der Maus {\"u}ber die Tabellenzellen) Hier kann man sich dann noch kleine Erkl{\"a}rungen zu den Akteuren und Aktivit{\"a}ten durchlesen. (klickt auf weiter) Dann gibts hier noch einen Text zum Kontext der Anwendung. (klickt auf "`Alles klar, ich will loslegen!"') So dann sind wir hier jetzt in der {\"U}bersicht der Anwendung. Man sieht direkt die Karte und am rechten Rand gibts noch so eine Seitenleiste. In der Seitenleiste finden sich die Beitr{\"a}ge, ein Filter und die Eingabemaske f{\"u}r das Einstellen von neuen Themen. (f{\"a}hrt mit der Maus {\"u}ber einen Marker) Hier wenn man jetzt mit der Maus {\"u}ber so einen Marker f{\"a}hrt, dann sieht man direkt, dass der Marker visuell hervorgehoben wird. Also der orangene Ring und das Popup. Gleichzeitig wird der zugeh{\"o}rige Beitrag in der Seitenleiste hervorgehoben, damit man direkt sehen kann, zu welchem Beitrag der Marker geh{\"o}rt. (f{\"a}hrt mit der Maus {\"u}ber einen Beitrag in der Seitenleiste) Das ganze funktioniert dann auch in der anderen Richtung. Man kann direkt dann sehen welche Marker zu dem Beitrag geh{\"o}ren. (klickt auf einen Beitrag in der Seitenleiste) So dann kann man die Beitr{\"a}ge hier noch so ausklappen, um dann auch den Beschreibungstext lesen zu k{\"o}nnen. Ja also neben dem Beschreibungstext kann man dann hier auch lesen von wann der Beitrag ist und wer ihn geschrieben hat. Sonst sind hier noch der Akteur, Aktivit{\"a}t und Inhalte zu lesen. Dann gibts hier noch dieses Herzchen mit der Zahl davor. Das zeigt an wie oft der Beitrag von den Benutzern favorisiert worden ist. Das ist so {\"a}hnlich wie ein Facebook "`Gef{\"a}llt mir"'. (tippt ein paar Buchstaben in den Filter) So dann gibts hier noch den Filter. Hier kann man entweder direkt nach einem Wort suchen, oder dann mit den Filteroptionen (klickt auf "`Filter einblenden"') entweder nach Akteur, Aktivit{\"a}t oder Inhalt filtern. Hier ganz unten gibts noch die Filteroption "`Zeitraum unbegrenzt"'. Das ist noch ne Besonderheit. Die Beitr{\"a}ge k{\"o}nnen ein Ablaufdatum bekommen. Dann werden die nach Ablauf des Ablaufdatums dann auch nicht mehr auf der Karte und in der Seitenleiste angezeigt. Hier mit der Option kann man sie dann wieder einblenden. (klickt auf "`alle Filter zur{\"u}cksetzen"' und dann auf "`Filter ausblenden"') So und dann kann man noch hier noch die Sortierreihenfolge in der Seitenleiste ver{\"a}ndern. Da kann man dann sortieren wie man die Beitr{\"a}ge hier gerne h{\"a}tte. Achso, hier in den Beitr{\"a}gen kann man dann auch schon sehen wie viele Antworten das Thema schon hat. (klickt auf "`Antworten anzeigen"') Wie Ihnen wahrscheinlich schon aufgefallen ist, hat sich nicht nur der Kartenausschnitt ver{\"a}ndert, sondern es werden jetzt auch andere Marker als gerade angezeigt. Das ist auch so ne Besonderheit. In der {\"U}bersicht werden nur die Marker angezeigt, die zu den initialen Beitr{\"a}gen der Themen erstellt worden sind. Das ist damit die Karte nicht zu schnell zu voll wird, und man noch den {\"U}berblick hat. Hier in den Antworten k{\"o}nnen Sie ja sehen, da sind so unterschiedlich farbige Boxen. Die gr{\"u}nen zeigen an, dass da ein Marker verlinkt worden ist, die braunen zeigen, dass ein bestehender Marker referenziert worden ist, und blau bedeutet, dass dort eine Webseite verlinkt worden ist. Also schreiben wir jetzt einfach mal eine Antwort. (klickt auf "`Antwort verfassen"') Hier hat sich jetzt die Eingabemaske f{\"u}r Antworten ausgeklappt. Da kann man nur eine Beschreibung mit Geoobjekten, referenzierten Geoobjekten und Link angeben und ein Bild anh{\"a}ngen. Der Rest Attribute wie Akteur und so werden vom Thema geerbt. (tippt eine Beschreibung) So nachdem man nun seinen Text verfasst hat, kann man noch W{\"o}rter verlinken. Das geht wie gesagt mit einem neuen Marker, einem bestehenden Marker, oder einem Link zu einer Webseite. Dazu muss man hier ein Wort oder halt mehrere W{\"o}rter markieren. (markiert ein Wort mit der Maus) Dann sieht man hier so ein Kontextmen{\"u} mit drei Buttons. Der erste ist f{\"u}r einen neuen Marker, der zweite um einen bestehenden Marker zu verkn{\"u}pfen und der dritte um eine Webseite zu verlinken. (klickt auf den "`Marker"'-Button) Ich mach jetzt hier mal einen neuen Marker. (klickt in die Karte) So jetzt wurde das Wort und der Ort, an dem ich den Marker gesetzt habe, verkn{\"u}pft. (markiert ein anderes Wort und klickt auf den "`Verkn{\"u}pfen"'-Button) Genauso funktioniert das dann auch mit dem Verkn{\"u}pfen. (klickt auf einen bestehenden Marker in der Karte). Und dann nochmal mit der Webseite (markiert ein drittes Wort und klickt auf den "`Link"'-Button) Hier beim Webseiten verlinken {\"o}ffnet sich dann so ein kleines Eingabefeld wo man dann den Link reinschreiben kann. (schreibt einen Link in das Feld und dr{\"u}ckt die Enter-Taste auf der Tastatur) So lange man die Antwort noch nicht abgeschickt hat, kann man auch noch alles l{\"o}schen und dann ist es auch weg. Sp{\"a}ter beim Editieren geht das nicht mehr. So wenn man jetzt noch ein Bild hat, kann man das hier unten anh{\"a}ngen noch. Das funktioniert aber so wie man es erwartet, ich hab jetzt auch keines gerade. (klickt auf "`Abschicken"') So, da man noch nicht eingeloggt, ist kann man nat{\"u}rlich den Beitrag jetzt noch nicht abschicken. Also verfassen geht, abschicken aber nicht. Da kommt man dann hier zu dem Login-Dialog. Hier kann man sich entweder mit Facebook, Twitter oder Google einloggen, oder halt auch ganz traditionell mit E-Mail und Passwort wenn man sich vorher hier auch registriert hat. (loggt sich ein). Dann kann man jetzt auch die Antwort abschicken. So wenn man jetzt merkt, dass man sich verschrieben hat oder dass der Marker falsch ist, kann man jetzt den Fehler korrigieren. Dazu muss man hier auf den kleinen Stift dr{\"u}cken, (klickt auf den Stift) und kann dann den Beitrag bearbeiten. ({\"a}ndert einen Buchstaben, verschiebt den Marker und {\"a}ndert den Link) Man kann das hier durch draufklicken auf den kleinen Kasten ausl{\"o}sen, das {\"a}ndern des Links. (klickt auf "`Abschicken"') So jetzt sind die {\"A}nderungen gespeichert, und man kann direkt sehen, dass jetzt hier auch "`ge{\"a}ndert am"' steht. So dann gibts hier noch den Button mit der kleinen Tonne. Damit kann man den Beitrag l{\"o}schen. Editieren und l{\"o}schen geht nat{\"u}rlich nur bei Beitr{\"a}gen, von denen man selber der Autor ist. Das l{\"o}schen ist dann auch kein richtiges L{\"o}schen, sondern da kann man dann einen Grund angeben, warum der Beitrag gel{\"o}scht werden soll. (klickt auf die kleine Tonne) Also hier kann man den Grund angeben (tippt Grund ein und klickt auf "`Ja, Beitrag l{\"o}schen"') So dann sieht man direkt dass der Beitrag ausgegraut wird, durchgestrichen und dann auch noch die Marker in hellerer Farbe dargestellt werden. Man kann nicht komplett l{\"o}schen, weil sonst der Sinn von den Diskussionen verloren gehen k{\"o}nnte. Die Themenstarter kann dann auch nichtmal der Autor l{\"o}schen. So wenn man denn nun jetzt nicht der Autor eines Beitrages ist, dann kann man hier mit dem kleinen Herzchen den Beitrag favorisieren. (klickt auf das Herzchen) Dann sieht man auch direkt, die Zahl hier unten neben dem anderen Herzchen erh{\"o}ht sich und das Herzchen wird ausgef{\"u}llt. Das bedeutet, dass man selbst den Beitrag favorisiert hat. Das ganze kann man dann nat{\"u}rlich auch wieder entfavorisieren wenn man wieder auf das kleine Herzchen klickt. (klickt auf den "`zur{\"u}ck"'-Button) So okay, dann wollen wir auch nochmal ein neues Thema erstellen. (klickt in das "`Titel"'-Feld) So hier hat sich jetzt die Eingabemaske f{\"u}r neue Themen ausgeklappt. Hier kann man dann den Titel, einen Akteur, eine Aktivit{\"a}t und mehrere Inhalte ausw{\"a}hlen. Dann kann man hier einen Start- und Endzeitpunkt ausw{\"a}hlen.  Darunter, das kennen Sie ja schon von eben, kann man eine Beschreibung eingeben und darunter noch ein Bild anh{\"a}ngen. (f{\"u}llt die Felder aus und klickt "`Abschicken"') So und dann hat man hier ein neues Thema. Hier unten auf der Seite kommt man dann auch nochmal zu der Zeichenerkl{\"a}rung, und hier oben nochmal zu den Erkl{\"a}rungsvideos. Ja also das war es jetzt erstmal zur Anwendung.
\end{itemize}