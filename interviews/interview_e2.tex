\begin{itemize}
    \item[I:] Kennen Sie Anwendungen die Diskussionen durch Geoobjekte unterst{\"u}tzen?
    \item[E2:] Ja ArguMap nat{\"u}rlich. (lacht) Sonst, kenn ich nat{\"u}rlich einige. Oder war das jetzt eine Ja-Nein-Frage? (Nein, das sollen Erz{\"a}hlaufforderungen sein.) Was denn noch? (\dots) H{\"a}tte ich jetzt vorher noch mal ein meine Diplomarbeit gucken sollen. Also jetzt spontan habe ich jetzt keine mehr so auf dem Schirm. Es gibt auf jeden Fall mehrere.
    \item[I:] Welche Anwendungsf{\"a}lle gibt es f{\"u}r die Idee der "`Argumentation Map"'?
    \item[E2:] Ja einmal nat{\"u}rlich so B{\"u}rgerbeteiligungsgeschichten. Also {\"o}ffentliche Planung, wo es darum geht irgendwie Bauvorhaben {\"o}ffentlich zur Diskussion zu stellen, um halt einfach meistens um eine breitere Akzeptanz zu sicherzustellen. Dass man so nicht irgendwas plant und dass dann den B{\"u}rgern so vorsetzt, und sagt "`Okay, das ist es, und so wird es gemacht."' Sondern, damit schon m{\"o}glichst fr{\"u}h die B{\"u}rger einzubinden. Das ist einmal so eine offizielle Nutzung und dann gibts nat{\"u}rlich so privat. So "`Freizeit"'-Nutzungen sage ich mal. Wo man irgendwie einen gemeinsamen Urlaub oder eine Radtour plant. Oder solche Sachen.
    \item[I:] Welche L{\"o}sungen um B{\"u}rger mit Initiativen oder Politik zusammenzubringen kennen Sie?
    \item[E2:] Naja, klassisch l{\"a}uft das nat{\"u}rlich so (\dots) Also es kommt nat{\"u}rlich darauf an, wo man hinguckt. Hier in Amerika gibt es ja klassisch diese Town Hall Meetings wo man dann einfach hingehen kann, und seinen Senf dazugeben kann. Und in Deutschland l{\"a}uft das ja glaube ich offiziell eher {\"u}ber Aush{\"a}nge. Das halt so Entw{\"u}rfe gemacht werden. Die werden dann halt ausgeh{\"a}ngt, und dann kann man da das irgendwie kommentieren auf diesem Weg. Das ist so ein bisschen asynchroner. (\dots) Was gibts noch? (\dots) Ich glaube das jetzt wirklich so mit elektronischen Hilfsmitteln zu gestalten, das waren bislang eher alles so Experimente. Mir ist jetzt kein Beispiel bekannt wo das mal wirklich breit ausgerollt wurde sozusagen, dass man sowas wirklich online machen konnte. Habe ich noch nirgends gesehen.
    \item[I:] Denken Sie die explizite Verkn{\"u}pfung von Geoobjekten mit Diskussionsgegenst{\"a}nden ist generell hilfreich im B{\"u}rgerbeteiligungskontext?
    \item[E2:] Finde ich einerseits nat{\"u}rlich super (lacht), sonst h{\"a}tte ich mich auch nicht damit besch{\"a}ftigt. Aber ist nat{\"u}rlich auch immer so eine zweischneidige Sache, weil man nat{\"u}rlich so eine gewisse Lernkurve hinzuf{\"u}gt. Also das muss schon extrem einfach zu bedienen sein, damit auch wirklich jeder Zugriff hat. Und man m{\"u}sste dann wahrscheinlich auch noch so einen Extra-Schritt machen, und irgendwie noch so eine Station oder Kiosk im Rathaus irgendwie zu Verf{\"u}gung stellen. F{\"u}r Leute eben die das nicht von zuhause aus machen k{\"o}nnen. Vielleicht dann auch mit jemandem, der das irgendwie bedienen kann und da helfen kann. Sonst hat man da nat{\"u}rlich die Gefahr, dass man da Nutzergruppen, wahrscheinlich vor allem {\"a}ltere Leute, einfach ausschlie{\ss}t.
    \item[I:] Dann konkret zur Anwendung im Vergleich zu bestehenden Anwendungen die Sie schon kennen. Was denken Sie speziell zu der Gegebenheit dass in der {\"U}bersicht nur die Geoobjekte des ersten Beitrages zum Thema angezeigt werden und dass in der Themendetailansicht nur die Geoobjekte zu dem Thema angezeigt werden?
    \item[E2:] Ich denke das ist ganz clever gel{\"o}st, weil sonst wird es wahrscheinlich sehr schnell un{\"u}bersichtlich. Wenn du jetzt noch eine r{\"a}umliche Suche drin h{\"a}ttest, dann w{\"u}rde man das wahrscheinlich haben wollen, dass auch in in den Antworten die Georeferenzen durchsucht werden. Das hast du glaube ich noch nicht, oder? (Nein, da ist keine r{\"a}umliche Suche drin) Das w{\"u}rde wahrscheinlich Sinn machen. Aber so jetzt zum browsen, sage ich mal, wenn man einsteigt, macht das schon Sinn finde ich, dass man nur das sieht, was auf der obersten Ebene ist.
    \item[I:] Dann die Zwei-Wege Highlights bei Mausinteraktion?
    \item[E2:] Finde ich gut, das hatten wir ja auch in dieser ArguMap-Anwendung schon so. Gibt es denn da auch die M{\"o}glichkeit, dass mehrere Beitr{\"a}ge auf das gleiche Objekt referenzieren? (Ja, das geht) Ja das ist auf jeden Fall super dann. Weil dann kann man direkt auf einen Blick sehen, welche Beitr{\"a}ge gibts jetzt zum ifgi, oder zum Schloss.
    \item[I:] Die Filter- und Sortierfunktion?
    \item[E2:] Ja das macht auf jeden Fall beides Sinn. Und wenn man jetzt gerade neu rein kommt, und da ist schon einiges auf der Karte (\dots) Da will man ja wahrscheinlich auch erstmal gucken was schon da ist, bevor man jetzt irgendwas rein schreibt, was schon drei andere geschrieben haben. Macht auf jeden Fall Sinn. Also bis jetzt wie ich es gesehen habe aus so wie ich es erwarten w{\"u}rde.
    \item[I:] Dann diese Verfassen- und Antworten- Funktion?
    \item[E2:] Finde ich gut, aber was auf die Dauer vielleicht ein bisschen verwirrend werden k{\"o}nnte, ist die M{\"o}glichkeit da neue Akteure, Aktivit{\"a}ten und Inhalte hinzuzuf{\"u}gen. Weil, wenn da ordentlich gebrauch von gemacht wird, dann k{\"o}nnte es glaube ich un{\"u}bersichtlich werden. Also in einer echten Anwendung w{\"u}rde man das vielleicht abschalten wollen k{\"o}nnte ich mir vorstellen.
    \item[I:] Normalerweise ist das auch abgeschalten, aber hier auf meiner lokalen Version ist das noch m{\"o}glich.
    \item[E2:] Ah okay.
    \item[I:] Dann das Verkn{\"u}pfen von den W{\"o}rtern mit den Geoobjekten, bestehenden Geoobjekten und Hyperlinks?
    \item[E2:] Macht Sinn. Ich wei{\ss} jetzt nicht wie intuitiv das ist. Was da vielleicht noch hilfreich w{\"a}re, w{\"a}re eine Anbindung an einen Geocoder, der dann aus dem aktuellen Kartenausschnitt nach Objekten durchsucht, die dem eingegebenen Wort schon entsprechen. Das w{\"u}rde das ganze wahrscheinlich nochmal ein gutes St{\"u}ck einfacher machen, als das einzugeben oder manuell den Marker zu setzen. (Das ist eigentlich eine gute Idee)
    \item[I:] Dann die Favorisierung der Beitr{\"a}ge?
    \item[E2:] Finde ich auch gut. M{\"u}sste man sich dann konkret mal angucken, wie die Leute das benutzen. Also ist das wahrscheinlich eher so gedacht wie so eine Art "`Finde ich gut. Das unterst{\"u}tze ich, diese Idee"'-Funktion. Aber ich kann mir auch vorstellen, dass einige Leute das so als eine Art Bookmark benutzen f{\"u}r Diskussionen, die sie irgendwie weiterverfolgen wollen oder so. M{\"u}sste man sich dann konkret einfach angucken, wie die Leute das benutzen. Kann ich nicht sagen.
    \item[I:] Die Benutzerregistrierung und Anmeldung?
    \item[E2:] Ja ist ja ziemlich Standard. Was man erwarten w{\"u}rde, w{\"u}rde ich sagen. Schien mir intuitiv zu sein.
    \item[I:] Dann konkret der Social Login?
    \item[E2:] Finde ich gut. Ich glaube das macht es einfacher f{\"u}r viele Leute.
    \item[I:] Dann ganz speziell jetzt auf die Anwendung bezogen. Wie werden Dialoge damit vereinfacht?
    \item[E2:] Kommt auf den Dialog an wahrscheinlich. Wollen die die Anwendung auch f{\"u}r Planung benutzen? (Ja w{\"a}re ja prinzipiell m{\"o}glich) Ist wohl sowohl f{\"u}r den angedachten Einsatz als auch f{\"u}r die Planung n{\"u}tzlich. Wobei ich mir vorstellen k{\"o}nnte, dass es noch f{\"u}r die Planung vielleicht noch hilfreicher w{\"a}re. F{\"u}r mal so eine Gruppe von Leuten, die sich jetzt {\"u}ber Wochen und Monate mit diesem Tag besch{\"a}ftigen und da irgendwie so festhalten wollen, was es da f{\"u}r Ideen gibt. Und so Pro- und Contra-Argumente zu sammeln. Dann als R{\"u}ckkanal f{\"u}r die Leute, um jetzt hinterher zu sagen "`Fand ich gut, da w{\"u}rde ich wieder hingehen"'. Das ist vielleicht sogar so wie "`Kanonen auf Spatzen geschossen."' Aber das ist auch nur eine Vermutung. Muss man ausprobieren. 
    \item[I:] Haben Sie Funktionen vermisst?
    \item[E2:] In diesem Kontext jetzt nicht. Also diese ganze ArguMap-Sache, die wir da damals gemacht haben, die war ja eher so in einem B{\"u}rgerbeteiligungskontext. Und dann war so das Grundszenario dass es da so verschiedene Planungsvarianten auf die man dann antworten konnte. Da hatten wir dann auch diese Funktion, dass man angeben konnte, ob das jetzt ein Pro oder Kontra Argument ist. Dass man halt schnell einen {\"U}berblick kriegt, das eine hat irgendwie eine breite Zustimmung oder breite Ablehnung. Da hatten wir irgendwie glaube ich so Plus und Minus Symbole benutzt. Aber f{\"u}r den Anwendungskontext hier, brauch man das glaube ich nicht. Ja sonst das einzige was ich ja schon gesagt hab, war halt die Einbindung eines Geocoders. Das w{\"u}rde es wahrscheinlich noch ein ganze Ecke benutzbarer machen.
    \item[I:] Was f{\"u}r Gr{\"u}nde k{\"o}nnen Sie sich vorstellen die Leute davon abhalten k{\"o}nnten, die Karte zu benutzen?
    \item[E2:] Also man hat ja immer noch eine gewisse Lernkurve. Ich glaube das ist auch nicht weg zu kriegen bei solchen Tools, weil (\dots) ich w{\"u}sste nicht wie man das noch einfacher machen soll. Man muss ja eh immer durch die einzelnen Screens klicken. Und wenn man das ignoriert, dann muss man halt einfach ein bisschen rumgucken, wie das funktioniert. Das k{\"o}nnte ich mir vorstellen, dass Leute, die gerade nicht so viel online machen, (\dots) ein bisschen abgeschreckt sind, und die Finger davon lassen.
    \item[I:] Gut. Das war es von meiner Seite so. Gibt es von Ihrer Seite noch Anmerkungen oder Fragen?
    \item[E2:] Machst du das als Open Source verf{\"u}gbar? (lacht)
    \item[I:] Ja. Das ist auf jeden Fall geplant.
    \item[E2:] Cool. K{\"o}nnte ich mir auf jeden Fall auch noch vorstellen, dass man das hier mal irgendwie benutzen k{\"o}nnte. Was man nat{\"u}rlich auch mal {\"u}berlegen k{\"o}nnte, w{\"a}re so langfristig so eine Art Vergleich zu machen zwischen dem ArguMap und deiner Anwendung. Gut, das ArguMap-Tool was ich damals gemacht habe, ist auch schon ein bisschen in die Jahre gekommen. Dass man das vielleicht nochmal reaktiviert. Und dann so eine Art Vergleich macht zwischen den beiden.
    \item[I:] Ja. Ich muss zugeben, ich habe es nicht zum laufen bekommen.
    \item[E2:] Das kann gut sein, das ist ja auch eine Version von der Google Maps API, die schon {\"a}lter ist auf jeden Fall. Ich wei{\ss} auch gar nicht was ich da Serverseitig benutzt habe. Das war glaube ich in PHP.
    \item[I:] Ja, dann war es das schon jetzt. Vielen Dank!        
\end{itemize}