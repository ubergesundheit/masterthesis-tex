\begin{itemize}
    \item[I:] Kennen Sie Anwendungen die Diskussionen durch Geoobjekte unterstützen?
    \item[P1:] Ja ArguMap natürlich. (lacht) Sonst, kenn ich natürlich einige. Oder war das jetzt eine Ja-Nein-Frage? (Nein, das sollen Erzählaufforderungen sein.) Was denn noch? (\dots) Hätte ich jetzt vorher noch mal ein meine Diplomarbeit gucken sollen. Also jetzt spontan habe ich jetzt keine mehr so auf dem Schirm. Es gibt auf jeden Fall mehrere.
    \item[I:] Welche Anwendungsfälle gibt es für die Idee der "`Argumentation Map"'?
    \item[P1:] Ja einmal natürlich so Bürgerbeteiligungsgeschichten. Also öffentliche Planung, wo es darum geht irgendwie Bauvorhaben öffentlich zur Diskussion zu stellen, um halt einfach meistens um eine breitere Akzeptanz zu sicherzustellen. Dass man so nicht irgendwas plant und dass dann den Bürgern so vorsetzt, und sagt "`Okay, das ist es, und so wird es gemacht."' Sondern, damit schon möglichst früh die Bürger einzubinden. Das ist einmal so eine offizielle Nutzung und dann gibts natürlich so privat. So "`Freizeit"'-Nutzungen sage ich mal. Wo man irgendwie einen gemeinsamen Urlaub oder eine Radtour plant. Oder solche Sachen.
    \item[I:] Welche Lösungen um Bürger mit Initiativen oder Politik zusammenzubringen kennen Sie?
    \item[P1:] Naja, klassisch läuft das natürlich so (\dots) Also es kommt natürlich darauf an, wo man hinguckt. Hier in Amerika gibt es ja klassisch diese Town Hall Meetings wo man dann einfach hingehen kann, und seinen Senf dazugeben kann. Und in Deutschland läuft das ja glaube ich offiziell eher über Aushänge. Das halt so Entwürfe gemacht werden. Die werden dann halt ausgehängt, und dann kann man da das irgendwie kommentieren auf diesem Weg. Das ist so ein bisschen asynchroner. (\dots) Was gibts noch? (\dots) Ich glaube das jetzt wirklich so mit elektronischen Hilfsmitteln zu gestalten, das waren bislang eher alles so Experimente. Mir ist jetzt kein Beispiel bekannt wo das mal wirklich breit ausgerollt wurde sozusagen, dass man sowas wirklich online machen konnte. Habe ich noch nirgends gesehen.
    \item[I:] Denken Sie die explizite Verknüpfung von Geoobjekten mit Diskussionsgegenständen ist generell hilfreich im Bürgerbeteiligungskontext?
    \item[P1:] Finde ich einerseitz natürlich super (lacht), sonst hätte ich mich auch nicht damit beschäftigt. Aber ist natürlich auch immmer so eine zweischneidige Sache, weil man natürlich so eine gewisse Lernkurve hinzufügt. Also das muss schon extrem einfach zu bedienen sein, damit auch wirklich jeder Zugriff hat. Und man müsste dann wahrscheinlich auch noch so einen Extra-Schritt machen, und irgendwie noch so eine Station oder Kiosk im Rathaus irgendwie zu Verfügung stellen. Für Leute eben die das nicht von zuhause aus machen können. Vielleicht dann auch mit jemandem, der das irgendwie bedienen kann und da helfen kann. Sonst hat man da natürlich die Gefahr, dass man da Nutzergruppen, wahrscheinlich vor allem ältere Leute, einfach ausschließt.
    \item[I:] Dann konkret zur Anwendung im Vergleich zu bestehenden Anwendungen die Sie schon kennen. Was denken Sie speziell zu der Gegebenheit dass in der {\"U}bersicht nur die Geoobjekte des ersten Beitrages zum Thema angezeigt werden und dass in der Themendetailansicht nur die Geoobjekte zu dem Thema angezeigt werden?
    \item[P1:] Ich denke das ist ganz clever gelöst, weil sonst wird es wahrscheinlich sehr schnell unübersichtlich. Wenn du jetzt noch eine räumliche Suche drin hättest, dann würde man das wahrscheinlich haben wollen, dass auch in in den Antworten die Georeferenzen durchsucht werden. Das hast du glaube ich noch nicht, oder? (Nein, da ist keine räumliche Suche drin) Das würde wahrscheinlich Sinn machen. Aber so jetzt zum browsen, sage ich mal, wenn man einsteigt, macht das schon Sinn finde ich, dass man nur das sieht, was auf der obersten Ebene ist.
    \item[I:] Dann die Zwei-Wege Highlights bei Mausinteraktion?
    \item[P1:] Finde ich gut, das hatten wir ja auch in dieser ArguMap-Anwendung schon so. Gibt es denn da auch die Möglichkeit, dass mehrere Beiträge auf das gleiche Objekt referenzieren? (Ja, das geht) Ja das ist auf jeden Fall super dann. Weil dann kann man direkt auf einen Blick sehen, welche Beiträge gibts jetzt zum ifgi, oder zum Schloss.
    \item[I:] Die Filter- und Sortierfunktion?
    \item[P1:] Ja das macht auf jeden Fall beides Sinn. Und wenn man jetzt gerade neu rein kommt, und da ist schon einiges auf der Karte (\dots) Da will man ja wahrscheinlich auch erstmal gucken was schon da ist, bevor man jetzt irgendwas rein schreibt, was schon drei andere geschrieben haben. Macht auf jeden Fall Sinn. Also bis jetzt wie ich es gesehen habe aus so wie ich es erwarten würde.
    \item[I:] Dann diese Verfassen- und Antworten- Funktion?
    \item[P1:] Finde ich gut, aber was auf die Dauer vielleicht ein bisschen verwirrend werden könnte, ist die Möglichkeit da neue Akteure, Aktivitäten und Inhalte hinzuzufügen. Weil, wenn da ordentlich gebrauch von gemacht wird, dann könnte es glaube ich unübersichtlich werden. Also in einer echten Anwendung würde man das vielleicht abschalten wollen könnte ich mir vorstellen.
    \item[I:] Normalerweise ist das auch abgeschalten, aber hier auf meiner lokalen Version ist das noch möglich.
    \item[P1:] Ah okay.
    \item[I:] Dann das Verkn{\"u}pfen von den W{\"o}rtern mit den Geoobjekten, bestehenden Geoobjekten und Hyperlinks?
    \item[P1:] Macht Sinn. Ich weiß jetzt nicht wie intuitiv das ist. Was da vielleicht noch hilfreich wäre, wäre eine Anbindung an einen Geocoder, der dann aus dem aktuellen Kartenausschnitt nach Objekten durchsucht, die dem eingegebenen Wort schon entsprechen. Das würde das ganze wahrscheinlich nochmal ein gutes Stück einfacher machen, als das einzugeben oder manuell den Marker zu setzen. (Das ist eigentlich eine gute Idee)
    \item[I:] Dann die Favorisierung der Beiträge?
    \item[P1:] Finde ich auch gut. Müsste man sich dann konkret mal angucken, wie die Leute das benutzen. Also ist das wahrscheinlich eher so gedacht wie so eine Art "`Finde ich gut. Das unterstütze ich, diese Idee"'-Funktion. Aber ich kann mir auch vorstellen, dass einige Leute das so als eine Art Bookmark benutzen für Diskussionen, die sie irgendwie weiterverfolgen wollen oder so. Müsste man sich dann konkret einfach angucken, wie die Leute das benutzen. Kann ich nicht sagen.
    \item[I:] Die Benutzerregistrierung und Anmeldung?
    \item[P1:] Ja ist ja ziemlich Standard. Was man erwarten würde, würde ich sagen. Schien mir intuitiv zu sein.
    \item[I:] Dann konkret der Social Login?
    \item[P1:] Finde ich gut. Ich glaube das macht es einfacher für viele Leute.
    \item[I:] Dann ganz speziell jetzt auf die Anwendung bezogen. Wie werden Dialoge damit vereinfacht?
    \item[P1:] Kommt auf den Dialog an wahrscheinlich. Wollen die die Anwendung auch für Planung benutzen? (Ja wäre ja prinzipiell möglich) Ist wohl sowohl für den angedachten Einsatz als auch für die Planung nützlich. Wobei ich mir vorstellen könnte, dass es noch für die Planung vielleicht noch hilfreicher wäre. Für mal so eine Gruppe von Leuten, die sich jetzt über Wochen und Monate mit diesem Tag beschäftigen und da irgendwie so festhalten wollen, was es da für Ideen gibt. Und so Pro- und Contra-Argumente zu sammeln. Dann als Rückkanal für die Leute, um jetzt hinterher zu sagen "`Fand ich gut, da würde ich wieder hingehen"'. Das ist vielleicht sogar so wie "`Kanonen auf Spatzen geschossen."' Aber das ist auch nur eine Vermutung. Muss man ausprobieren. 
    \item[I:] Haben Sie Funktionen vermisst?
    \item[P1:] In diesem Kontext jetzt nicht. Also diese ganze ArguMap-Sache, die wir da damals gemacht haben, die war ja eher so in einem Bürgerbeteiligungskontext. Und dann war so das Grundszenario dass es da so verschiedene Planungsvarianten auf die man dann antworten konnte. Da hatten wir dann auch diese Funktion, dass man angeben konnte, ob das jetzt ein Pro oder Kontra Argument ist. Dass man halt schnell einen Überblick kriegt, das eine hat irgendwie eine breite Zustimmung oder breite Ablehnung. Da hatten wir irgendwie glaube ich so Plus und Minus Symbole benutzt. Aber für den Anwendungskontext hier, brauch man das glaube ich nicht. Ja sonst das einzige was ich ja schon gesagt hab, war halt die Einbindung eines Geocoders. Das würde es wahrscheinlich noch ein ganze Ecke benutzbarer machen.
    \item[I:] Was f{\"u}r Gr{\"u}nde k{\"o}nnen Sie sich vorstellen die Leute davon abhalten k{\"o}nnten, die Karte zu benutzen?
    \item[P1:] Also man hat ja immer noch eine gewisse Lernkurve. Ich glaube das ist auch nicht weg zu kriegen bei solchen Tools, weil (\dots) ich wüsste nicht wie man das noch einfacher machen soll. Man muss ja eh immer durch die einzelnen Screens klicken. Und wenn man das ignoriert, dann muss man halt einfach ein bisschen rumgucken, wie das funktioniert. Das könnte ich mir vorstellen, dass Leute, die gerade nicht so viel online machen, (\dots) ein bisschen abgeschreckt sind, und die Finger davon lassen.
    \item[I:] Gut. Das war es von meiner Seite so. Gibt es von Ihrer Seite noch Anmerkungen oder Fragen?
    \item[P1:] Machst du das als Open Source verfügbar? (lacht)
    \item[I:] Ja. Das ist auf jeden Fall geplant.
    \item[P1:] Cool. Könnte ich mir auf jeden Fall auch noch vorstellen, dass man das hier mal irgendwie benutzen könnte. Was man natürlich auch mal überlegen könnte, wäre so langfristig so eine Art Vergleich zu machen zwischen dem ArguMap und deiner Anwendung. Gut, das ArguMap-Tool was ich damals gemacht habe, ist auch schon ein bisschen in die Jahre gekommen. Dass man das vielleicht nochmal reaktiviert. Und dann so eine Art Vergleich macht zwischen den beiden.
    \item[I:] Ja. Ich muss zugeben, ich habe es nicht zum laufen bekommen.
    \item[P1:] Das kann gut sein, das ist ja auch eine Version von der Google Maps API, die schon älter ist auf jeden Fall. Ich weiß auch gar nicht was ich da Serverseitig benutzt habe. Das war glaube ich in PHP.
    \item[I:] Ja, dann war es das schon jetzt. Vielen Dank!        
\end{itemize}