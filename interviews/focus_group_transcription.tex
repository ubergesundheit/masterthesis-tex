\begin{itemize}
    \item[I:] Habt ihr irgendwelche Anmerkungen zu ersten Eindrücken oder etwas was ihr sofort loswerden wollt?
    \item[F1:] Ja wir haben ja jetzt quasi alle drei einen eigenen Ort eingegeben. Also anstatt jetzt irgendwie (\dots)
    \item[F2:] Ich war gerade dabei zu überlegen, wir sollten uns jetzt aber mal auf was einigen.
    \item[F1:] Also so nach dem Motto, irgendwie hätte jetzt einer sagen müssen "`Ich ziehe meinen Ort zurück"' oder so. Wenn jetzt einer nur einen Ort eingegeben hätte und einen Vorschlag gemacht hätte, dann hätte ich ja um diesen Ort die Diskussion (\dots)
    \item[I:] Es hätte ja auch die Möglichkeit gegeben innerhalb der Themen auch neue Orte vorzuschlagen.
    \item[F3:] Achso, ja. Ansonsten hat das ganz gut geklappt mit dem Schreiben. Was mich halt irritiert hat, mit der Maus am Bildschirm ziehen. Und wenn man schreibt, kann man mit dem Cursor nicht zurück gehen.
    \item[F1:] Achja das hatte ich auch.
    \item[F3:] Ich weiß nicht, ist das so gewollt?
    \item[I:] Nein, dass muss noch ein Fehler sein. Gut dass du das gesagt hast.
    \item[F1:] Also das ist ja jetzt auch nicht so ein Medium, dass man (\dots) Für mich ist das eher so eine Landkarte, wo man auch halt Fragen loswerden kann. Also wie wenn man eine Internetseite hat, und Kommentare, Anmerkungen oder irgendwas machen kann, sagen kann "`Wo ist das genau"' oder "`Kann ich auch später kommen?"'. Solche Sachen. Aber nicht jetzt so dass wenn man da jetzt intensive Diskussionen führen möchte. Also so ne Aktion, dass man die Veranstalter was fragen kann. Also "`Wo kann ich parken"' oder wenn das jetzt ein Veranstaltungshinweis ist. Wo kann ich dieses oder jenes machen. Dass man die halt eben Fragen kann, ohne direkt da in die speziellen Auftritte geht. Dass man direkt die Frage adressieren kann. Also aber für ein Diskussionsmedium würde ich mir was anderes wünschen. Wir in der Gruppe reden untereinander anders. {\"U}ber unsere E-Mail Liste zum Beispiel jetzt.
    \item[F2:] Ja und ich finde überhaupt diese Chats für Diskussionen nicht sehr geeignet. Aber das ist jetzt eine grundsätzliche Einstellung von mir. Und wenn, dann müsste mindestens erlaubt sein, dass man hierarchisch auf jeden Beitrag einzeln antworten kann. Dass man speziell sagen kann, diese Antwort geht jetzt speziell nur an diese andere Antwort. Weil es sonst sehr schnell unübersichtlich wird. Und selbst dann, wenn es hierarchisch ist, dann ist es immer noch nicht so gut, weil es so ellenlang ist. Da finde ich dann, dass so ein Chat einfach kein gutes Medium für wirkliche Diskussionen ist. Konkrete Anfragen, da ist das sicherlich eine gute Sache.
    \item[F1:] Es wird eine Tapete dann. Und wenn man jetzt den Nachhaltigkeitstag da eingegeben hätte, oder so, "`Wo kann ich mich da melden"' oder wenn da einer eine Frage hat, oder mit machen möchte, dass man halt die Antwort da bekommt. Dass das eine etwas interaktivere FAQ-Liste hat, die sich selbst so ein bisschen selber erstellt zu dieser Veranstaltung. Das finde ich klasse. Aber so eine Diskussion, das war ja auch schon sehr speziell, was wir da jetzt diskutiert haben. So eine Diskussion, das ist eine andere Sache.
    \item[F3:] Also nochmal für mich zur Information: Also ich dachte, das ganze soll nur ins Netz gestellt werden zur Information.
    \item[I:] Ja wie ich eben am Anfang schon mal gesagt hatte, die Möglichkeit zu Diskutieren ist einfach noch mal eine Ebene über der des reinen Informieren. Damit könnt ihr Außenstehenen die Möglichkeit geben sich zu beteiligen. Und das ist ja im Endeffekt genau das was ihr wollt.
    \item[F1:] Ich könnte mir auch vorstellen, dass das auch einfach eine Hemmschwelle ist, da steht man dann ja auch als derjenige der etwas eingestellt hat. Also mit vollem Namen im Internet für alle einsehbar.
    \item[I:] Man kann ja auch einfach seinen Namen ändern.
    \item[F2:] Was sinnvoll wäre, wenn man, genau im Sinne, wenn es Nachfragen gibt, die öfter auftauchen, dass man die zusätzlich noch mit da hinschreiben könnte. Also editieren.
    \item[I:] Ja also das editieren habe ich euch doch auch erklärt eben.
    \item[F2:] Ja das wäre trotzdem gut, dass halt für jeden direkt klar ist, das wurde schonmal gefragt, das brauch ich nicht nochmal zu fragen und wurde hier nachträglich noch eingestellt. Also irgendwie sowas wie "`Weitere Informationen"'. Also dass der jenige, der eine Anfrage kriegt, dass der dann direkt auch beim beantworten direkt oben mit reinschreiben kann.
    \item[F3:] So allgemeingültige Fragen kann man sicherlich dann auch mit in diesen allgemeinen Beschreibungsteil mit einfügen.
    \item[F1:] Also dass man direkt mit den Leuten kommunizieren kann. Das finde ich klasse. Ja und einfach so als Meta-Funktion, einfach so eine Landkarte zu haben. Das ist ja schonmal richtig viel Wert. Dass man einfach mal gucken kann, was es alles wo gibt.
    \item[F2:] Ja das finde ich auch sehr gut.
    \item[F1:] Ich weiß jetzt nicht ob das technisch vorgesehen ist, dass man vielleicht noch so einen Feed oder Newsletter anbieten kann. So dass man irgendwie eine Nachricht bekommt, wenn irgendwo was neues geschrieben wurde. Das fände ich superklasse!
    \item[I:] Das könnte man noch einbauen.
    \item[F3:] Oder so eine Unter-Seite wo das steht. Ich mein ich bekomm schon genug E-Mails.
    \item[F1:] Wir sind aber auch alle schon ein bisschen älter. Auch mit dieser Diskussionsfuntion. So digital, das ist jetzt eher nichts für uns gebräuchlich.
    \item[I:] Ihr habt ja schon ein bisschen angefangen Sachen zu nennen, die euch gut oder nicht so gut gefallen haben. Vielleicht können wir in diese Richtung ein bisschen weiter gehen.
    \item[F3:] Das ganze Projekt hier gefällt mir sehr gut. Da ist echt Bedarf dran. Echt klasse.
    \item[F1:] Was ich nicht schlecht fände, das wäre so, ob man da nicht irgendwo noch eine Adresse hinterlegen könnte. Jetzt ist man quasi darauf angewiesen, dass der jenige der etwas einstellt, auch von sich aus eine Kontaktadresse oder Telefonnumer dazu schreibt. Das wäre auch noch so ne Frage, ob man sowas noch mit einbauen könnte.
    \item[I:] Das könnte man noch mit in die Benutzerprofile einbauen, wäre das noch eine Möglichkeit?
    \item[F1:] Ja das ist eine gute Idee. Aber nicht Pflicht, zumindest dass man die Möglichkeit hat das anzugeben. Auf der anderen Seite gibt es ja die Möglichkeit da so einen Link zu erstellen, also (\dots) ja.
    \item[F2:] Und die Frage ist natürlich, wie offen soll das sein. Kann da jeder (\dots). Allein jetzt haben wir ja schon wieder so-und-so viel zusätzlich Dinge reingestellt. Die wären ja auf lange Sicht gesehen ziemlich nutzlos.
    \item[I:] Da müsste man dann sehen wie man das Tool jetzt einsetzt. Ob nun zur internen Planung oder dass man den Leuten die Möglichkeit gibt sich zu beteiligen.
    \item[F3:] Ah! Also dass eines so ein öffentlicher Bereich ist und das andere zur internen Planung genutzt wird. Was mir gerade noch so durch den Kopf ging: Wenn man da irgendwo noch einen Kalender einbauen könnte, das fände ich klasse. So dass man sich den angucken kann und alle Termine auf einen Blick hat. So zum Beispiel die Sperrmülltage. Das gehört ja auch zu Nachhaltigkeit dazu.
    \item[F1:] Oder dass man gezielt gucken kann. Ich bin von dann bis dann im Urlaub, was verpasse ich denn da.
    \item[F3:] Also nachhaltige Projekte. Irgendwie so einen Bauernmarkt, Apfeltag, was da so alles kommt.
    \item[I:] Okay, dann frage ich euch mal direkt ein bisschen zu einzelnen Funktionen der Karte. Wie findet ihr das dass die Marker von den Antworten nicht direkt mit angezeigt werden? Also dass man erst zu den Antworten gehen muss um die Geoobjekte für die Antworten sehen zu können?
    \item[F1:] Da habe ich mir ehrlich gesagt gar keine Gedanken zu gemacht.
    \item[F2:] Ich finde das sinnvoll. Aber wenn jemand antwortet und dann in seiner Antwort auch nochmal einen Ort markiert oder einen existieren Marker verlinkt, ist das auch kein Problem. Man sieht ja auf der Hauptseite immer dass da Antworten existieren.
    \item[I:] Was sind euere Gedanken zu der Zwei-Wege Highlight Funktion? Was würdet ihr daran noch ändern?
    \item[F3:] Wenn der Standort keine Relevanz hat, muss ich auch keinen Marker setzen oder?
    \item[I:] Nein, das stimmt.
    \item[F1:] Man muss das nur wissen, dass es dann diesen Heiligenschein gibt. Das hilft dann aber ungemein, dass einem dann direkt diese visuelle {\"A}nderung ins Auge fällt.
    \item[F2:] Ja
    \item[F3:] Vielleicht noch eine bisschen schrillere Farbe.
    \item[I:] Wie fandet ihr die Form zum Erstellen von Themen und Antworten?
    \item[F3:] Da habe ich die ersten zwei Male gesucht wo das denn überhaupt ist, aber dann war dann war es gut.
    \item[F1:] Ja das sind sicherlich die Anfängerfehler, aber wenn man dann einmal weiß, was man machen muss (\dots).
    \item[F2:] Also ich fand das völlig okay, hatte keine Probleme. Dass man allerdings auf Antworten klicken muss, um seinen eigenen Beitrag zu bearbeiten, das fand ich nicht ganz so eindeutig und umständlich. Wenn man es weiß, dann ist es gut, aber ich hab doch sehr lange gesucht. Das kommt dann bestimmt öfter vor, dass Leute was reinschreiben, und dann merkt "`Moment jetzt möchte ich nochmal was ändern"'.
    \item[F1:] Aber allein dass das möglich ist, finde ich schon sehr Anwenderfreundlich. Du hast uns ja jetzt nicht so ne riesen Einführung gegeben. Ich fand das schon sehr selbsterklärend muss ich sagen. Da finde ich manch anderes im Netz komplexer.
    \item[F3:] Ja also sobald du dass dann nochmal gesagt hattest, dann gings auch wirklich gut. Ich bin ja nun wirklich nicht so ein technischer Mensch. Wenn ich es kapiere, dann ist es schon gut.
    \item[I:] Kennt ihr denn noch Anwendungen die so ähnlich funktionieren?
    \item[F2:] Nein. Trello vielleicht.
    \item[F3:] Nein ist mir noch nicht untergekommen.
    \item[F1:] Nein. Trello ist ja nun nichts was mit {\"O}rtlichkeiten zu tun hat.
    \item[I:] Ihr habt ja eben schonmal durchscheinen lassen, dass ihr denkt, dass euch Diskussionen jetzt nicht so gut gefallen haben. Vielleicht könnt ihr mir nochmal ein paar Gründe nennen.
    \item[F3:] Man muss unterscheiden zwischen öffentlichkeitsrelevanten Fragen und internen Fragen. Was so intern diskutiert wird, dass muss nicht die ganze Stadt mitlesen, das sollte irgendwie separat sein. Wir verschicken ja unsere Protokolle auch nicht in alle Welt, sondern auch nur an uns. Das sind ja Sitzungen, bei denen man sich ja zusammenrauft, Vor- und Nachteile bespricht. Da wird nicht so sauber drin gesprochen. So allgemeingültiger Informationsgehalt sollte sauber und ohne Schnörkel, Diskussionen und Kommentare sein.
    \item[F2:] Es gibt auch einfach Sachen, die diskutiert werden müssen, die aber nicht unbedingt örtlich referenziert werden müssen. Wichtig sind dann eher so Sachen, die im Trello drin sind. Also wer ist für was verantwortlich. Dafür wäre die Seite hier zu unübersichtlich finde ich. Also es ist gut zu wissen, wenn etwas stattfindet, wo dass das dann stattfindet. Dafür ist es gut. Aber so als allgemeines Diskussionstool finde ich das nicht so sehr geeignet. Tut mir leid wenn ich das jetzt so sage.
    \item[I:] Nein, das ist völlig in Ordnung. Ist ja auch ein Ergebnis.
    \item[F1:] Ja aber interessant ist es dass das ganze dezentral gepflegt wird. Dass es da nicht einen gibt, der bearbeiten kann, sondern dass da jeder was reinstellen kann und sich beteiligen kann. Bei anderen Seiten, da muss man ja seinen Kram an einen schicken, und der stellt das dann ein. Also weiß ich jetzt nicht, ob man da nach nem Jahr wieder nochmal eine Erinnerung bekommt, dass man dann seine Informationen nochmal überprüfen soll, aber sonst (\dots). Also dass da nicht altes Zeug drinsteht.
    \item[F3:] Also das läuft ganz ohne Administrator?
    \item[I:] Nein. Es gibt eine Administrationsübersicht, aber man braucht das nur, wenn man eingreifen möchte. Sonst kann ja jeder seine Beiträge selbst ändern.
    \item[F1:] Ja das wäre doch auch noch eine Idee. Dass man da in regelmäßigen Abständen eine Benachrichtigung bekommt, dass man seine Informationen überprüfen soll.
    \item[F3:] Was ist mit den abgelaufenen Terminen? Scrollen die dann ganz nach unten?
    \item[I:] Die werden ausgeblendet. Dafür gibt es ja dann in den Filtern diese "`Zeitraum unbegrenzt"' Option.
    \item[F2:] Ach, die werden nicht aus der Datenbank gelöscht, sind also noch da?
    \item[I:] Ja, die sind nicht verloren, können also mit der Filteroption wieder angezeigt werden. Der Benutzer muss dann explizit sagen: "`Ich will das sehen"'.
    \item[F2:] Also da fällt mir gerade auch ein, da war noch so eine Sache: Wenn man Filter an und ausschaltet, dann bleibt die Karte nicht mehr da wo ich sie hatte.
    \item[I:] Also ich weiß was du meinst, das lässt sich glaube ich nicht besser lösen, Die Karte versucht immer die gerade eingeblendeten Marker dem Benutzer komplett darzustellen. Hättest du eine Idee, wie man das anders machen könnte?
    \item[F2:] Ja also vielleicht das man diese Anpassung nur macht, wenn man den Filter anhakt, und nicht wenn man die Optionen abhakt.
    \item[F1:] Wäre das viel zu programmieren?
    \item[I:] Müsste ich mir nochmal anschauen. Aber ich denke nicht so ganz so viel. Wenn ihr dann sonst nichts mehr an Anmerkungen habt, dann bedanke ich mich bei euch. Vielen Dank dass ihr für mich euren Abend geopfert habt.
    \item[F1:] Ja also nochmal: Was mir so richtig gut gefällt, ist der denzentrale Charakter. Also nicht dass einer die Zügel in der Hand hat, sondern dass da jeder seinen "`Senf"', sag ich mal, dazugeben kann. Und auch dass man sortieren kann, das finde ich auch wichtig. Ich kenn da so einen Veranstaltungskalender, da kann man auch seine Veranstaltungen eintragen, aber da weiß ich dann nicht wo die Sachen stattfinden.
    \item[F2:] Also wenn man dann so einen Kalender kombinieren könnte mit dem räumlichen Charakter deiner Anwendung hier jetzt. Das fände ich auch super. Dann aber weniger als Diskussionstool sondern nur zur Information. Quasi "`Wann/Wo"'. Wenn es dafür dann auch brauchbar ist, dann müsste man sich da auch bekannter machen irgendwie.
    \item[I:] Würdet ihr denn generell eine Chance für geographisch referenzierte Dialoge sehen?
    \item[F1:] Ich überlege jetzt mal zu anderen Themen. Wo ein Endlager hin soll, das ist ja weiß Gott auch ein Thema. Aber ich könnte mir nicht vorstellen, dass man so eine Diskussion über so ein Tool führt.
    \item[F2:] Also der einzige Nutzen, der mir jetzt so spontan einfallen würde, wären taktische {\"U}berlegungen fürs Militär, also für die Kriegsführung. Da muss man wissen, was ist wo. Aber dann braucht man auch entsprechendes Kartenmaterial.
    \item[I:] Das ist natürlich eine ganz andere Richtung als der Kontext in dem wir uns jetzt hier bewegen.
    \item[F1:] Mir fällt noch etwas ein. Im Sommer war ich im Urlaub. Da habe ich auch allerlei so Faltkarten bekommen. Die hätte ich gerne alle übereinander gelegt. Ich bin teilweise mit drei Plänen durch die Gegend gelaufen und hab dann immer vergleicht. Und so eine Funktion fände ich im diesem Kontext auch sehr gut. So dass man irgendwie noch Zusatzinformationen in die Karte einblenden könnte. Die Hintergrundkarte ist ja auch Openstreetmap wenn ich das richtig gesehen habe. Da könnte man doch auch versuchen irgendwie nochmal Filter mit einzubauen. Also irgendwie zum Beispiel Bushaltestellen mehr hervorheben oder so.
    \item[I:] Man könnte sicherlich noch auswechselbare Hintergrundkarten einbauen oder irgenwie schauen, dass man sich eine Hintergrundkarte bastelt, die passend zu dem Kontext steht. (\dots) Also gut. Dann nochmal vielen Dank und einen schönen Abend noch.
\end{itemize}