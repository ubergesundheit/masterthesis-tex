\textbf{Teil 1 -- B{\"u}rgerbeteiligung}
\begin{itemize}
    \item[I:] Bitte erz{\"a}hlen Sie mir {\"u}ber ihre Rolle und Aufgaben in der B{\"u}rgerbeteiligung.
    \item[P8:] Ich bin der Vorsitzende des Vorstandes der Stiftung "`B{\"u}rger f{\"u}r M{\"u}nster"', die Ende 2004 zur F{\"o}rderung des B{\"u}rgerengagements in M{\"u}nster gegr{\"u}ndet wurde. Es gibt zwei Gremien: das Kuratorium legt den generellen Kurs fest; der Vorstand, der aus drei Personen besteht, f{\"u}hrt die operativen Gesch{\"a}fte.
    \item[I:] Und waren Sie davor schon in der B{\"u}rgerbeteiligung aktiv?
    \item[P8:] Ja. Meine Liste von ehrenamtlichen T{\"a}tigkeiten ist relativ lang. Ich habe das sozialpolitische Projekt "`Ma{\ss}arbeit f{\"u}r M{\"u}nster"' ins Leben gerufen, das ist aber schon 15 Jahre her. Ich war der Initiator des Projektes "`PCs f{\"u}r Schulen"'. In einer der gro{\ss}en Parteien hier in M{\"u}nster habe ich den Arbeitskreis "`Wirtschaftspolitik"' geleitet. Ich bin in einem der gro{\ss}en Service-Clubs hier in M{\"u}nster, bei Rotary, aktiv. In dieser Rolle habe ich vor zehn Jahren mitgewirkt, die B{\"u}rgerstiftung zu gr{\"u}nden, und die Rotary-Clubs als Gr{\"u}ndungsstifter gewinnen k{\"o}nnen.
    \item[I:] Bitte beschreiben Sie mir die aus ihrer Sicht wichtigsten Aspekte der B{\"u}rgerbeteiligung.
    \item[P8:] In M{\"u}nster gibt es sehr viel B{\"u}rgerengagement. Es gibt 1500 eingetragene Vereine, aber auch viele Initiativen und informelle Zusammenschl{\"u}sse von B{\"u}rgern. Eine der j{\"u}ngsten hei{\ss}t "`Regen in M{\"u}nster"'. Da haben sich {\"u}ber Facebook viele Leute getroffen. Nicht nur Junge oder Studenten, sondern B{\"u}rger aller Herkunft und Altersklassen haben sich zusammengetan: "`Wir m{\"u}ssen Mitb{\"u}rgern helfen, die vom Regen und vollaufenden Kellern und Wohnungen betroffen sind"'. Aber zur{\"u}ck zum Allgemeinen. Es gibt reichhaltiges B{\"u}rgerengagement in M{\"u}nster, und die B{\"u}rgerstiftung ist eine Institution, die in dieser Landschaft agiert. Wir wollen nicht als tausendf{\"u}nfhundertster Verein aktiv werden, sondern diese Vielfalt darstellen, sichtbar machen und den B{\"u}rgern verdeutlichen, was es alles an guten Engagement-Beispielen gibt. Wir wollen die B{\"u}rger ermuntern "`Guckt mal, das gibt es! Da sind die und die F{\"a}higkeiten gefragt. Habt ihr Lust, mit zumachen?"'. Das schlie{\ss}t auch ein, B{\"u}rger zu ermuntern, zu spenden, zum Beispiel beim B{\"u}rgerbrunch. Dort kommen in diesem Jahr 1600 B{\"u}rger zusammen, und jeder, der einen Tisch mietet, ist aufgerufen, eine Spende f{\"u}r Kinder- und Jugendprojekte zu leisten. In diesem Jahr haben wir sicherlich das dreizehnte bis f{\"u}nfzehnte Kinder- und Jugendprojekt, f{\"u}r die nicht nur gespendet wird, sondern die vielen Teilnehmern {\"u}berhaupt erst bekannt gemacht werden. An der Spitze unserer Aktivit{\"a}ten in diesem Netzwerkgedanken steht der B{\"u}rgerpreis, den wir jedes Jahr zu einem Thema ausloben, in diesem Jahr zum Thema "`Internationales Engagement"'. Wir rufen alle international engagierten Vereine, Initiativen und Projekte, zum Beispiel St{\"a}dtepartnerschaften, Engagements in der dritten Welt oder Integrationsprojekte hier in M{\"u}nster, zu Bewerbungen auf. Eine Jury w{\"a}hlt die besten Bewerbungen aus. In einer Festveranstaltung Anfang Dezember werden die Finalisten und die Preistr{\"a}ger vorgestellt; die Preistr{\"a}ger werden ausgezeichnet und erhalten ein Preisgeld. Die Pr{\"a}sentation der Finalisten und Preistr{\"a}ger bedeutet auch, gute Beispiele von B{\"u}rgerengagement in der {\"O}ffentlichkeit vorzustellen und die B{\"u}rgerschaft aufzurufen: "`Da k{\"o}nnte man mitmachen"'. Die B{\"u}rgerstiftung ist also in erster Linie ein Akteur im Netzwerk des B{\"u}rgerengagements. Wir machen aber auch operative Projekte. Es gibt n{\"a}mlich in dieser gro{\ss}en Engagements-Landschaft auch wei{\ss}e Flecken. Ich nenne ein Beispiel. Vor vier Jahren, als das Thema des B{\"u}rgerpreises "`Jung und Alt"' war, wurde ein Lesepaten-Projekt in Kinderhaus mit dem Silberpreis ausgezeichnet. Wir fanden die Idee "`Lesepaten"', also Menschen, die Grundschulkinder im zweiten, dritten Schuljahr animieren zu lesen und dabei zu lernen, dass Lesen mehr als ein Schulfach ist, so gut, dass wir uns gefragt haben "`Warum gibt es das nicht in anderen Stadtteilen?"'. Inzwischen ist es uns gelungen, {\"a}hnliche Projekte in anderen Stadtteilen in Gang zu bringen und zu einem Lesepaten-Netzwerk zu verbinden. Wir sind also auch selbst ein Anbieter von B{\"u}rgerengagement; aber wir legen gro{\ss}en Wert darauf, dass wir wei{\ss}e Flecken bedienen und nicht in Konkurrenz zu bestehendem Engagement treten.
    \item[I:] Bitte geben Sie mir eine Einf{\"u}hrung in ein laufendes oder abgeschlossenes Projekt, bei dem es besonders auf gute Kommunikation oder Dialoge zwischen den Akteuren angekommen ist.
    \item[P8:] Das eben beschriebene Programm "`Lesepaten"' geh{\"o}rt sicher dazu. Erstens m{\"u}ssen sich die Lesepaten pro Stadtteil untereinander austauschen, es muss organisiert werden, dass sie mit der Schule und den Lehrern kommunizieren, es muss gekl{\"a}rt werden, f{\"u}r welche Sch{\"u}ler vorgelesen werden soll. Zweitens wollen wir auch erreichen, dass die Idee "`Lesepaten"' in der Stadt verbreitet wird. Wir wollen neue Menschen ermuntern "`Werdet auch Lesepaten. Macht mit! Das Programm kann noch wachsen." Und der ein oder andere, der ein paar Jahre vorgelesen hat, sagt "`nun ist es gut"' und muss "`ersetzt"' werden. In  anderen Projekten ist es {\"a}hnlich, z.B. im "`Mentoren-Programm"'. Da geht es um die Begleitung und das Coaching von 14 bis 16 Jahre alten Realsch{\"u}lern, die an der Schwelle zur Berufst{\"a}tigkeit stehen und denen wir helfen, diese Schritte noch etwas systematischer, {\"u}berlegter, vielleicht auch mit mehr Mut und Kreativit{\"a}t zu gehen. Insbesondere machen wir das Mentoring f{\"u}r Sch{\"u}ler, die von ihren Elternh{\"a}usern nicht so gef{\"o}rdert werden. Auch da kommt es darauf an, die Idee zu erz{\"a}hlen, in der B{\"u}rgerschaft zu verbreiten und immer wieder neue Mentoren zu gewinnen. Andere Projekte sind zeitlich befristet, z.B. auf ein Jahr. Beispielweise findet der B{\"u}rgerpreis jedes Jahr statt, hat aber jedes Jahr ein neues Thema. Also bildet man auch jedes Jahr ein neues Projektteam, dessen Arbeit zeitlich begrenzt ist. Alle unsere Aktivit{\"a}ten zielen darauf, beispielhaft Engagement zu praktizieren, dar{\"u}ber zu reden, zu informieren. Wir wollen so immer wieder B{\"u}rger motivieren, mitzumachen, sei es direkt bei der B{\"u}rgerstiftung, sei es bei einem der Vereine und Initiativen, die wir bei unseren Netzwerkaktivit{\"a}ten vorstellen und f{\"o}rdern.
\end{itemize}

\textbf{Teil 2 -- Einsatz der Anwendung}
\begin{itemize}
    \item[I:] Bitte geben Sie mir eine Einf{\"u}hrung in das Projekt in dem Sie die Anwendung einsetzen wollen.
    \item[P8:] Wenn wir Menschen gewinnen wollen, sich an b{\"u}rgerschaftlichen Projekten zu beteiligen, brauchen wir eine Informationsplattform. Nat{\"u}rlich bieten Zeitungen und Rundfunk M{\"o}glichkeiten, sich zu informieren, aber im Internetzeitalter ist eigentlich jeder gewohnt, sich auch im Internet informieren zu k{\"o}nnen. Google und andere Informationsquellen bieten viele M{\"o}glichkeiten; f{\"u}r viele B{\"u}rger stellt sich aber die Frage "`Wo finde ich das was ich suche?" oder sogar: "`Was suche ich eigentlich?" Ich m{\"o}chte vielleicht keine langen Anfahrtzeiten haben, ich m{\"o}chte erstmal gucken, ob in meiner Nachbarschaft Menschen an einem Thema beteiligt sind, das mich interessiert. Vielleicht habe ich von Nachbarn geh{\"o}rt, dass es in meinem Stadtteil ein interessantes Projekt gibt. Daher hat ein Informationsmedium, dass den Ort des Engagements in den Vordergrund stellt, eine Menge Vorteile. Ich glaube, dass Ihre Karte einen guten Beitrag zu leisten kann. Nach meinem Eindruck verbindet sie die rein geographischen Informationen mit der M{\"o}glichkeit, sich {\"u}ber die einzelnen Projekte zu informieren und weitergeleitet zu werden zu zus{\"a}tzlichen Begleitinformationen. Das ist, so glaube ich, eine gute Kombination der verschiedenen Herangehensweisen.
    \item[I:] Was f{\"u}r Inhalte erwarten Sie von B{\"u}rgern, die die Diskussionsfunktion der Anwendung benutzen?
    \item[P8:] Es ist sicherlich neu, eine geographische Informationsplattform zu einer Dialogplattform auszubauen. Wir haben ja allerlei soziale Netzwerke und Blogs, auf denen man sich austauschen kann. In klassischen Informationsbasen, nennen wir das jetzt mal untechnisch Datenbanken, sind Dialoge aber bisher un{\"u}blich. Die meisten Datenbasen sind auch relativ sperrig. Ich will keine Namen nennen, aber es gibt eine ganze Reihe von Informationsbasen, die unterstellen, dass man relativ genau wei{\ss}, wonach man sucht. Ich glaube, dass es ein Fortschritt ist, wenn es Informations{\"u}bersichten gibt, Landkarten sowohl im w{\"o}rtlichen als auch im {\"u}bertragenen Sinne, die einem erst einmal das Spektrum von m{\"o}glichen Engagementfeldern aufzeigen. Ihre L{\"o}sung beschreibt ja neben den Orten auch verschiedene Themenfelder, sie ist nicht nur eine geographische, sondern auch eine inhaltliche Landkarte. Inwieweit B{\"u}rger dieses Angebot annehmen, also auf diesem Medium Kommentare abgeben, Fragen stellen oder versuchen, mit anderen in Kontakt zu kommen, muss sich zeigen. Wenn der Anlass von B{\"u}rgerengagement ein spontaner ist, wie das eben genannte "`Regen in M{\"u}nster"', hat man nat{\"u}rlich einen sehr gro{\ss}en Bedarf, sich unmittelbar, schnell und formlos auszutauschen. Bei anderen Projekten, seien es die, die ich eben f{\"u}r die B{\"u}rgerstiftung beschrieben habe, seien es "`klassische"' Sport-, Kultur- oder Umweltvereine, wird man abzuwarten haben, inwieweit ein {\"u}bergreifendes Medium genutzt wird, oder ob der jeweilige Verein bzw. das Projekt selbst mit diesen Medien arbeitet und Dialogforen oder {\"A}hnliches anbietet. Aber das gilt es zu entwickeln. Sie wollen ja etwas Innovatives machen mit der Karte, und die Nutzung als Dialog-Plattform gilt es auszuprobieren.
    \item[I:] Wollen Sie den Benutzern Anreize geben, sich {\"u}ebr diese Anwendung auszutauschen?
    \item[P8:] Es ist sicher gut, wenn ich auf einer Landkarte Informationen finde und dann direkt Fragen platzieren kann. Die Frage ist, wo geht diese Frage dann hin. Geht sie zum Gesamtorganisator oder Administrator, der die Fragen weiterleiten m{\"u}sste, oder bekomme ich nur einen "`Link"' zu dem jeweiligen Anbieter, Verein oder Projekt. Das wird man ausprobieren m{\"u}ssen. Den Bedarf, dass ich ein Angebot sehe und spontan Fragen dazu habe, gibt es ja h{\"a}ufig. Und wenn es mir einfach gemacht wird, meine Fragen loszuwerden, und sichergestellt ist, dass die Frage bei dem landet, der die Frage auch beantworten kann und das hoffentlich auch tut, dann hat die Anwendung einen Mehrwert.
    \item[I:] Welche Gr{\"u}nde sprechen f{\"u}r den Einsatz dieser Anwendung gegen{\"u}ber anderen Anwendungen?
    \item[P8:] Ein bisschen muss ich mich wiederholen, denn ich glaube, das Innovative ist die Kombination von geographischer mit themenbezogener Landkarte. Heute bieten themenbezogene Internetseiten h{\"a}ufig auch Landkarten dazu. Wenn Sie zum Beispiel auf die B{\"u}rgerstiftungsseite mentoren-muenster.de gehen, sehen Sie eine Karte mit den Schulen, bei denen wir das Projekt machen. Man kann also von beiden Seiten kommen. Aus B{\"u}rgerbefragungen wissen wir, dass viele B{\"u}rger trotz Informationsflut oft Informationsdefizite haben, wo man sich {\"u}berhaupt engagieren kann. Die Herausforderung ist daher, eine {\"u}bergreifende Informationsseite so zu positionieren, dass sie auch tats{\"a}chlich als eine {\"u}bergreifende und {\"u}bergeordnete Informationsquelle wahrgenommen wird. Viele heutige Informationsangebote leiden darunter, dass man relativ gut wissen muss, was man sucht. Und wenn man schon relativ genau wei{\ss}, was man sucht, geht man tendenziell gleich zu den Seiten, wo man die Informationen direkt bekommt. Die Herausforderung f{\"u}r Ihre Seite wird sicher darin bestehen, sie als ein Dach-Informationsangebot zu positionieren, etwa so: "`Da stehen viele Angebote, man sieht die Orte und die Themen. Da kann man sich informieren. Da kann man aber auch in Dialog treten, da kann man Fragen stellen"'. Das Angebot hat Chancen, aber es ist auch eine Herausforderung, sich in der Informationsflut zu positionieren.
    \item[I:] Welche Eigenschaften w{\"u}rden Sie davon abhalten diese Anwendung einzusetzen?
    \item[P8:] Als B{\"u}rgerstiftung beteiligen uns gerne an der Seite, weil die Fragestellungen {\"a}hnlich sind. Auch wir wollen ja die B{\"u}rger in der Breite {\"u}ber M{\"o}glichkeiten von B{\"u}rgerengagement informieren. Insofern passt Ihre Seite grunds{\"a}tzlich zu unseren Zielsetzungen. Deswegen w{\"u}rden wir auch gerne Angebote, die nicht eigenst{\"a}ndig bei Ihnen in der Seite sind, dort einstellen. Das gilt jedenfalls f{\"u}r unsere eigenen Projekte, weil auch die eine r{\"a}umliche Dimension haben; ich hatte zum Beispiel Mentoren- und Lesepaten-Programme an unterschiedlichen Schulen erw{\"a}hnt. Vielleicht k{\"o}nnen wir auch andere Vereine, Initiativen und Projekte einladen, sich an Ihrer Seite zu beteiligen; inwieweit das gelingt, wird sich zeigen. Wenn sich die Seite gut entwickelt, wird man die Seite wechselseitig promoten. Was sicherlich {\"u}ber unsere M{\"o}glichkeiten hinausgehen w{\"u}rde, w{\"a}re als de-facto-Administrator andere Projekte einzustellen; das w{\"u}rde unsere Netzwerkpartner-Rolle {\"u}berdehnen und uns auch personell {\"u}berfordern. Aber wir k{\"o}nnen in unserem Netzwerk werben, auf Ihre Seite hinweisen und ermuntern, sie auszuprobieren und mit diesem Werkzeug zu arbeiten.
    \item[I:] K{\"o}nnen Sie sich weitere Anwendungsf{\"a}lle f{\"u}r die Verkn{\"u}pfung von Texten mit Karten neben der B{\"u}rgerbeteiligung vorstellen?
    \item[P8:] Das ist ja ein weites Feld. Beispielsweise gibt es Landkarten mit volkswirtschaftlichen Informationen. Wir kennen einen gemeinsamen Gespr{\"a}chspartner, der an einer Wissenschaftslandkarte arbeitet. Mir fehlt aber der solide {\"U}berblick, was es wo gibt, und wo es L{\"u}cken gibt. Meine eigene Anregung schon vor zwei Jahren war, den Begriff "`Landkarte"' st{\"a}rker thematisch zu interpretieren. Die Idee war eine "`Weltkarte des Engagements"'. Die Kontinente w{\"a}ren gro{\ss}e Themenfelder wie Sport, Umwelt, Kultur und Bildung. Wenn man in die Kontinente hineinzoomt, k{\"o}nnten sich die Themenfelder weiter aufgliedern. Die Analogie zu weiterem Detail, beispielweise St{\"a}dten, w{\"a}ren konkrete Vereine und Projekte mit den dazugeh{\"o}renden Informationen. Das w{\"a}re eine total innovative Herangehensweise, weil damit noch mehr die Breite und Vielfalt der Engagementfelder visuell unterst{\"u}tzt w{\"u}rde. Bei Ihrer L{\"o}sung muss man ja, wenn ich das richtig verstanden habe, auch entweder eine Textliste von m{\"o}glichen Feldern angucken, oder schon das Thema wissen. Das werden immer die Antipoden sein. Wenn ich das Thema schon wei{\ss}, ist der inhaltliche Informationsgehalt einer prim{\"a}r geographischen Seite relativ beschr{\"a}nkt. F{\"u}r Ihre Seite geht es also darum, wie Menschen angesprochen werden k{\"o}nnen, die noch nicht so genau wissen, was sie machen wollen, aber recht klare Vorstellungen haben, wo sie aktiv werden wollen.
\end{itemize}

\textbf{Teil 3 -- Abschlie{\ss}ende Fragen}
\begin{itemize}
    \item[I:] Kennen Sie Beispiele f{\"u}r die Verkn{\"u}pfung geographischer Daten mit Diskussionsbeitr{\"a}gen?
    \item[P8:] Ich kenne nur themenbezogene Seiten, die mit eingeblendeten Google-Karten arbeiten und dar{\"u}ber informieren "`Wir sind hier oder da"'. Kommentarfunktionen oder Blogs einerseits und Ortsinformationen andererseits stehen nach meiner Wahrnehmung eher nebeneinander; eine direkte Verkn{\"u}pfung von geographischen Daten und Interaktionen im Sinne von blogging oder chatting ist mir nicht bekannt. Aber in der Hinsicht bin ich nicht der wirkliche Fachmann.
    \item[I:] Kennen Sie Werkzeuge um interaktive Karten mit eigenen Inhalten zu erstellen?
    \item[P8:] Nein.
    \item[I:] Dann war es das von meiner Seite. Gibt es von Ihrer Seite noch Fragen oder Anmerkungen?
    \item[P8:] Aus meiner Sicht haben wir die wesentlichen Aspekte Ihres Projektes besprochen. Wir interessieren uns sehr daf{\"u}r, wie Ihre L{\"o}sung in ein Projekt eingebracht werden kann, in dem wir aktiv die B{\"u}rger M{\"u}nsters auf eine Kollektion von alternativen Angeboten ansprechen. Der Projekttitel ist ja "`1000 Stunden f{\"u}r M{\"u}nster"' oder auch "`25 konkrete Angebote zum Schnuppern und Mitmachen"'. Damit haben wir {\"a}hnliche Vorstellungen wie Sie, dass konkrete Angebote zusammengestellt werden und beworben werden. Schnell wird sich den Interessierten die Frage stellen, wo sich die Angebote befinden; wenn's geografisch passt, wie bekommt man n{\"a}here Informationen, Kontaktdaten usw. Was ich sehr spannend bei Ihrer Konzeption finde, dass man sehr einfach die Seiten der Anbieter finden kann. Insofern kann unser Projekt auch eine Verst{\"a}rkung f{\"u}r Ihr Anliegen sein. Und in welchem Ma{\ss}e wir die von Ihnen entwickelte Anwendung einsetzen und in eine Projektseite einbinden, dar{\"u}ber wollen wir ja noch sprechen.
    \item[I:] Ja. Dann vielen Dank f{\"u}r Ihre Antworten und Einblicke!
\end{itemize}