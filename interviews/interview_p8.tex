\textbf{Teil 1 -- B{\"u}rgerbeteiligung}
\begin{itemize}
    \item[I:] Bitte erz{\"a}hlen Sie mir {\"u}ber ihre Rolle und Aufgaben in der B{\"u}rgerbeteiligung.
    \item[P8:] Ich bin ja der Vorsitzende des Vorstandes der B{\"u}rgerstiftung. Der Stiftung "`B{\"u}rger f{\"u}r M{\"u}nster"', die Ende 2004 gegr{\"u}ndet wurde zur F{\"o}rderung des B{\"u}rgerengagements in M{\"u}nster. Da gibt es zwei Gremien. Da gibts ein Kuratorium, dass so die Gesamtsteuerung macht, und einen Vorstand, der die operativen Gesch{\"a}fte f{\"u}hrt. Der Vorstand besteht aus drei Personen und ich leite dieses Gremium.
    \item[I:] Und waren Sie davor schon in der B{\"u}rgerbeteiligung aktiv?
    \item[P8:] Ja. Meine Liste von ehrenamtlichen, oder sagen wir mal nicht bezahlten T{\"a}tigkeiten, ist relativ lang. Ich habe mich in einem Projekt in M{\"u}nster sozialpolitisch engagiert. Ich habe das Projekt "`Ma{\ss}arbeit f{\"u}r M{\"u}nster"' ins Leben gerufen. Das ist aber jetzt schon 15 Jahre her. Habe das Projekt "`PCs f{\"u}r Schulen"' ins Leben gerufen. Ich habe bei einer der gro{\ss}en Parteien hier in M{\"u}nster einen Arbeitskreis "`Wirtschaftspolitik"' geleitet. Also, ich bin in einem der gro{\ss}en Service-Clubs hier in M{\"u}nster aktiv. Das ist einer der Rotary-Clubs. Und ich habe vor zehn Jahren als die B{\"u}rgerstiftung in der Gr{\"u}ndung war, mitgewirkt die Stiftung zu gr{\"u}nden und die Rotary-Clubs zu Gr{\"u}ndungsstiftungen zu machen.
    \item[I:] Bitte beschreiben Sie mir die aus ihrer Sicht wichtigsten Aspekte der B{\"u}rgerbeteiligung.
    \item[P8:] Also in M{\"u}nster gibt es sehr viel B{\"u}rgerengagement. Es gibt 1500 eingetragene Vereine. Es gibt ganz viele Initiativen und informelle Zusammenschl{\"u}sse von B{\"u}rgern. Eine der j{\"u}ngsten hei{\ss}t ja "`Regen in M{\"u}nster"'. Da haben sich {\"u}ber Facebook viele Leute getroffen. Nicht nur Junge oder Studenten, sondern B{\"u}rger aller Art und Altersklassen haben sich zusammengetan und haben gesagt "`Wir m{\"u}ssen Mitb{\"u}rgern helfen die von dem Regen und den vollaufenden Wohnungen betroffen sind"'. Aber zur{\"u}ck zum Allgemeinen. Es gibt eben eine reichhaltige Gemeinschaft von B{\"u}rgerengagement, und insofern ist die B{\"u}rgerstiftung auch eine Institution, die in dieser Landschaft agiert. Es geht also nicht in erster Linie darum, den tausendf{\"u}nfhundertsten Verein zu abzugeben, sondern diese Landschaft darzustellen, sichtbar zu machen und den B{\"u}rgern zu verdeutlichen was es da alles an guten Beispielen gibt. Nat{\"u}rlich auch einmal um die B{\"u}rger zu ermuntern "`Guckt mal, das gibt es! Da sind die und die F{\"a}higkeiten gefragt. Habt ihr Lust mit zumachen?"'. Das schlie{\ss}t auch ein, B{\"u}rger zu ermuntern zu spenden. Zum Beispiel bei dem B{\"u}rgerbrunch. Wo ja 1600 B{\"u}rger in diesen Jahr zusammen kommen, und jeder der einen Tisch mietet auch aufgerufen ist, eine Spende f{\"u}r Kinder- und Jugendprojekte zu leisten. In diesem Jahr haben wir sicherlich das dreizehnte bis f{\"u}nfzehnte Kinder- und Jugendprojekt. Das hei{\ss}t dass nicht nur gespendet wird, sondern dass diese Projekte auch bekannt gemacht werden. Und an der Spitze unserer Aktivit{\"a}ten in diesem Netzwerkgedanken steht der B{\"u}rgerpreis, den wir jedes Jahr zu einem bestimmten Thema ausloben. In diesem Jahr zum Thema "`Internationales Engagement"'. Das hei{\ss}t, wir fordern alle auf, die sich dort engagieren. Von St{\"a}dtepartnerschaften, von Engagements in der dritten Welt, aber auch Integration hier in M{\"u}nster "`Wilkommen f{\"u}r Migranten und Fl{\"u}chtlinge"'. Also ein breites Spektrum von ehrenamtlichen Engagements zum Thema "`Internationales Engagement"' wird angesprochen. Wir rufen zu Bewerbungen auf. Es gibt eine Jury, die die besten Bewerbungen ausw{\"a}hlt. Und in einer Festveranstaltung am 3. Dezember werden dann die Finalisten und die Preistr{\"a}ger vorgestellt. Nat{\"u}rlich wird der Preistr{\"a}ger ausgezeichnet und er bekommt auch ein Preisgeld. Das ganze ordnet sich ja ein in eine Darstellung dieser Preistr{\"a}ger in der {\"O}ffentlichkeit und einer Belobigung wieder mit dem Zweck an die B{\"u}rgerschaft "`Da k{\"o}nnte man ja mal mitmachen"'. Also insofern ist die B{\"u}rgerstiftung in erster Linie ein Akteur im Netzwerk des B{\"u}rgerengagements. Es gibt bei der B{\"u}rgerstiftung aber auch operative Projekte. Es gibt n{\"a}mlich in dieser gro{\ss}en Engagements-Landschaft auch wei{\ss}e Flecken. Ich nenne mal ein Beispiel. Vor vier Jahren, als wir das Thema "`Jung und Alt"' beim B{\"u}rgerpreis hatten, ist der Silberpreis an ein Lesepaten-Projekt in Kinderhaus gegangen. Wir fanden die Idee "`Lesepaten"', also Menschen, die Grundschulkinder im zweiten, dritten Schuljahr animieren zu lesen, und dabei lernen, dass Lesen mehr ist als ein Schulfach, so gut, dass wir uns gefragt haben "`Warum gibt es das nicht in anderen Stadtteilen?"'. Und inzwischen gibt es ein Lesepaten-Netzwerk der B{\"u}rgerstiftung weil es uns gelungen ist, vier {\"a}hnliche Projekte in anderen Stadtteilen zu entwickeln. Also insofern sind wir selbst auch ein Anbieter von B{\"u}rgerengagement. Aber wir legen gro{\ss}en Wert darauf, dass wir wei{\ss}e Flecken bedienen, und nicht in Konkurrenz zu bestehendem Engagement treten.
    \item[I:] Bitte geben Sie mir eine Einf{\"u}hrung in ein laufendes oder abgeschlossenes Projekt bei dem es besonders auf gute Kommunikation oder Dialoge zwischen den Akteuren angekommen ist.
    \item[P8:] Also das eben beschriebene Programm "`Lesepaten"' geh{\"o}rt sicher dazu. Erstens m{\"u}ssen sich die Lesepaten pro Stadtteil untereinander austauschen, es muss organisiert werden, dass sie mit der Schule, den Lehrern kommunizieren, dass dann auch gefragt wird f{\"u}r welche Sch{\"u}ler das denn dann gemacht werden soll. Also insofern ist viel Organisation im eigentlichen Stadtteilprojekt erforderlich. Aber wir wollen ja nat{\"u}rlich auch, dass die Idee "`Lesepaten"' in der Stadt verbreitet wird. Wir wollen aufzeigen, dass das eine gute Idee ist und wir wollen ja auch immer neue Menschen ermuntern "`Werdet auch Lesepaten. Macht mit! Das Programm kann noch wachsen."' Und der ein oder andere, der das vielleicht ein paar Jahre gemacht hat, sagt dann auch "`Nun ist es gut"'. Also zu werben und auf gute Beispiele aufmerksam zu machen. Ich k{\"o}nnte andere Projekte erw{\"a}hnen, die einen {\"a}hnlichen Charakter haben. Ich erw{\"a}hne mal das Stichwort "`Mentorenprogramm"'. Da geht es um die Begleitung und das Coaching von 14 bis 16 Jahre alten Realsch{\"u}lern, die an der Schwelle zur Berufst{\"a}tigkeit stehen, und denen wir helfen, diese Schritte noch etwas systematischer, {\"u}berlegter, vielleicht auch mit mehr Mut und Kreativit{\"a}t zu gehen. Insbesondere weil das Sch{\"u}ler sind, die von ihren Elternh{\"a}usern nicht so gef{\"o}rdert werden. Auch da kommt es darauf an, die Idee zu erz{\"a}hlen, in der B{\"u}rgerschaft zu verbreiten und auch immer wieder neue Mentoren zu gewinnen. Also das sind so laufende Programme, die revolvierend sind und wo es darum geht, auch immer wieder neue Beteiligte und Mitmacher zu finden. Wir haben nat{\"u}rlich auch Projekte wie der schon genannte B{\"u}rgerpreis, bei dem es auf das eine Jahr begrenzte Aktivit{\"a}t ist. Der B{\"u}rgerpreis findet zwar jedes Jahr statt, aber hat jedes Jahr ein neues Thema. Also bildet man auch jedes Jahr ein neues Projektteam. Das ist ein Beispiel f{\"u}r Mitarbeit in einem gemeinn{\"u}tzigen Projekt, dass dann zeitlich begrenzt ist. Also alle Aktivit{\"a}ten die wir machen, zielen immer darauf Beispielhaft Engagement zu praktizieren, dar{\"u}ber zu reden, zu informieren und immer wieder B{\"u}rger zu motivieren mitzumachen. Entweder direkt bei der B{\"u}rgerstiftung oder bei einem der Partner die wir bei unseren Netzwerkaktivit{\"a}ten vorstellen und f{\"o}rdern.
\end{itemize}

\textbf{Teil 2 -- Einsatz der Anwendung}
\begin{itemize}
    \item[I:] Bitte geben Sie mir eine Einf{\"u}hrung in das Projekt in dem Sie die Anwendung einsetzen wollen.
    \item[P8:] Wenn wir Menschen gewinnen wollen, sich an b{\"u}rgerschaftlichen Projekten zu beteiligen, brauchen wir nat{\"u}rlich eine Informationsplattform. Nat{\"u}rlich sind die Zeitungen und der Rundfunk eine M{\"o}glichkeit sich zu informieren, aber im Internetzeitalter ist eigentlich jeder gewohnt, dass es da auch im Internet eine Seite zu allem gibt. Nat{\"u}rlich gibt es da mit Google und anderen Informationsquellen viele M{\"o}glichkeiten, aber f{\"u}r viele B{\"u}rger stellt sich auch die Frage "`Wo ist das denn?"'. Ich m{\"o}chte vielleicht keine langen Fahrtzeiten haben. Ich m{\"o}chte mal gucken ob vielleicht in meiner Nachbarschaft Menschen an diesem Thema beteiligt sind. Oder ich habe auch vielleicht von Nachbarn geh{\"o}rt, dass das bei mir im Stadtteil ist. Und insofern hat ein Informationsmedium, dass den Ort des Engagements in den Vordergrund stellt, eine menge Vorteile und ich glaube dass ihre Karte da einen guten Beitrag zu leisten kann. Nach meinem Eindruck verbindet sie eben die rein geographische Information mit der M{\"o}glichkeit sich auch {\"u}ber die einzelnen Projekte zu informieren und weitergeleitet zu werden zu zus{\"a}tzlichen Begleitinformationen. Das ist glaube ich, eine gute Kombination der verschiedenen Herangehensweisen.
    \item[I:] Was f{\"u}r Inhalte erwarten Sie von B{\"u}rgern, die die Diskussionsfunktion der Anwendung benutzen?
    \item[P8:] Also es ist sicherlich neu, sowas zu einer Dialogplattform zu machen. Wir haben ja allerlei soziale Netzwerke und Blogs auf denen man sich austauschen kann. In klassischen Informationsbasen, nennen wir das jetzt mal untechnisch Datenbanken, ist das bisher un{\"u}blich. Die meisten Datenbasen sind auch relativ sperrig. Ich will jetzt mal keine Namen nennen, aber es gibt eine ganze Reihe von Informationsbasen, die aber unterstellen und erfordern, dass man relativ genau wei{\ss}, wonach man sucht. Ich glaube, dass es ein Fortschritt ist, wenn es Informations{\"u}bersichten gibt, Landkarten jetzt sowohl im w{\"o}rtlichen als auch im {\"u}bertragenen Sinne, die einem erst einmal das Spektrum von m{\"o}glichen Engagementsfeldern aufzeigen. Das finde ich, ist da der wichtigste Aspekt, dass sie eben auch verschiedene Themenbereiche beschreiben. Insofern nicht nur eine geographische, sondern auch inhaltliche Landkarte sind. In wie weit jetzt B{\"u}rger auf diesem Medium Kommentare abgeben, Fragen stellen oder versuchen mit anderen in Kontakt zu kommen, muss sich zeigen. Wenn der Anlass von B{\"u}rgerengagement ein spontaner ist, wie das eben genannte "`Regen in M{\"u}nster"', hat man nat{\"u}rlich einen sehr gro{\ss}en Bedarf, sich unmittelbar, schnell und formlos auszutauschen. Bei anderen Projekten, seien es die, die ich eben f{\"u}r die B{\"u}rgerstiftung beschrieben habe, aber auch wenn es um Sport-, Kultur- oder Umweltvereine geht, wird man abzuwarten haben ob man da so ein ganz-{\"u}bergreifendes Medium braucht, wie Sie es jetzt anbieten, oder ob der jeweilige Verein oder das Projekt selbst mit diesen Medien arbeitet und Dialogforen oder so etwas anbietet. Aber das gilt es zu entwickeln. Sie wollen ja auch etwas innovatives machen mit dieser Karte, und das gilt es dann auszuprobieren.
    \item[I:] Wollen Sie den Benutzern Anreize geben, sich {\"u}ber diese Anwendung auszutauschen?
    \item[P8:] Also es ist sicher gut, wenn ich auf einer Landkarte Informationen finde und dann direkt Fragen platzieren kann. Die Frage ist wo geht diese Frage dann hin. Ob die eben zu dem Gesamtorganisator oder Administrator geht, der das dann wieder weiterleiten m{\"u}sste, oder ob das nur in Anf{\"u}hrungszeichen ein Link zu dem jeweiligen Anbieter, Verein oder Projekt ist. Das wird man ausprobieren m{\"u}ssen. Der Bedarf, dass ich ein Angebot habe und spontan Fragen dazu, das passiert ja h{\"a}ufig. Und wenn es mir einfach gemacht wird, die Frage loszuwerden, und sichergestellt ist dass die Frage bei dem landet, der die Frage auch beantworten kann, dann ist das bestimmt gut.
    \item[I:] Welche Gr{\"u}nde sprechen f{\"u}r den Einsatz dieser Anwendung gegen{\"u}ber anderen Anwendungen?
    \item[P8:] Also ein bisschen muss ich mich wiederholen, denn ich glaube die Kombination von geographischer Landkarte mit themenbezogener Landkarte, dass ist das innovative. Es ist ja h{\"a}ufig heute schon so, dass themenbezogene Internetseiten auch Landkarten dazu anbieten. Wenn Sie auf die B{\"u}rgerstiftungsseite Mentorenprogramm gehen zum Beispiel. Dann sehen Sie die Schulen bei denen wir das machen. Also insofern kann man von beiden Seiten kommen. Wir wissen aber aus B{\"u}rgerbefragungen, dass viele B{\"u}rger noch Informationsdefizite haben, wo man sich {\"u}beraupt engagieren kann. Die Herausforderung ist jetzt sicherlich eine {\"u}bergreifende Informationsseite so zu positionieren, dass die auch tats{\"a}chlich als eine {\"u}bergreifende und {\"u}bergeordnete Informationsquelle wahrgenommen wird. Viele heutige Informationsangebote leiden, in Anf{\"u}hrungszeichen, darunter, dass man relativ gut wissen muss, was man genau sucht. Und wenn man schon relativ genau wei{\ss} was man sucht, dann geht man nat{\"u}rlich tendenziell gleich dahin wo man die Informationen direkt bekommt. Die Herausforderung f{\"u}r Ihre Seite wird sicher darin bestehen, die als ein Dach-Informationsangebot zu positionieren. Vielleicht "`Da stehen die, da kann man gucken. Da kann man dann auch in Dialog treten, da kann man Fragen stellen"'. Also es hat Chancen, aber es ist auch eine Herausforderung sich in der Informationsflut zu positionieren.
    \item[I:] Welche Eigenschaften w{\"u}rden Sie davon abhalten diese Anwendung einzusetzen?
    \item[P8:] Also wir als B{\"u}rgerstiftung beteiligen uns gerne an der Seite weil die Fragestellungen {\"a}hnlich sind. Auch wir wollen ja die B{\"u}rger in der Breite {\"u}ber M{\"o}glichkeiten von B{\"u}rgerengagement informieren. Also in sofern passt das vom Grunds{\"a}tzlichen in unsere Zielsetzung. Und deswegen w{\"u}rden wir auch gerne die Angebote die jetzt nicht eigenst{\"a}ndig bei Ihnen in der Seite sind, gerne auch als B{\"u}rgerstiftung, dort einstellen. Mindestens mal unsere eigenen Projekte. Weil auch die eine r{\"a}umliche Dimension haben, die Mentorenprojekte an unterschiedlichen Schulen, die Lesepaten an unterschiedlichen Schulen. Insofern ist die geographische Dimension auch f{\"u}r uns wichtig. Ob es soweit kommt, dass wir auch andere Vereine, Projekte einladen, sich an der Seite zu beteiligen, das wird sich zeigen. Wenn sich die Seite gut entwickelt, wird man die Seite auch wechselseitig promoten. Was wir sicherlich nicht machen k{\"o}nnen, ist so als Defacto-Administrator andere Projekte einstellen. Das w{\"a}re glaube ich nicht Rollengerecht. Aber man kann werben Und wir k{\"o}nnen auch unser Netzwerk von Vereinen und Partnern darauf hinweisen und sie ermuntern auszuprobieren mit diesem Werkzeug zu arbeiten.
    \item[I:] K{\"o}nnen Sie sich weitere Anwendungsf{\"a}lle f{\"u}r die Verkn{\"u}pfung von Texten mit Karten neben der B{\"u}rgerbeteiligung vorstellen?
    \item[P8:] Also das ist ja ein ganz weites Feld. Es gibt ja dann auch Landkarten die mehr {\"o}konomische Bez{\"u}ge haben. Oder Wissenschaftslandkarten. Wir hatten ja auch mal einen gemeinsamen Gespr{\"a}chspartner, der an so einer Wissenschaftslandkarte arbeitet. Also da fehlt mir aber jetzt der {\"U}berblick, was es wo gibt, und wo es eben L{\"u}cken gibt. Meine eigene Anregung, schon vor zwei Jahren, war ob man den Begriff "`Landkarte"' nicht eben auch noch st{\"a}rker thematisch interpretieren k{\"o}nnte. Also wir hatten mal so eine Idee eine "`Weltkarte des Engagements"' zu machen. Da w{\"a}ren die Kontinente gro{\ss}e Themenfelder wie Sport, Umwelt, Kultur und Bildung. Wenn man dann in die Kontinente hineinzoomen w{\"u}rde, dann k{\"o}nnte sich das weiter aufgliedern. Dann k{\"a}me eben bei Sport eine entsprechende Untergliederung, bei Kultur entsprechende speziellere Themen und am Ende w{\"u}rde man sich dann auf einzelnen St{\"a}dten bewegen, was dann konkrete Vereine und Projekte w{\"a}ren. Das f{\"a}nde ich also eine total innovative Herangehensweise, weil dass dann auch noch mehr die Breite der Engagementsfelder visuelle unterst{\"u}tzen w{\"u}rde. Bei Ihrer L{\"o}sung muss man ja, wenn ich das richtig verstanden habe, auch entweder eine Textliste von m{\"o}glichen Feldern angucken, oder auch schon das Thema wissen. Das werden immer so die Antipoden sein. Wenn ich das Thema schon wei{\ss}, ist der Informationsgehalt relativ beschr{\"a}nkt. Von Ihrer Seite geht es dann darum, wie schaffe ich es Menschen anzusprechen, die noch nicht so genau wissen, was sie machen wollen, aber schon wissen wo. 
\end{itemize}

\textbf{Teil 3 -- Abschlie{\ss}ende Fragen}
\begin{itemize}
    \item[I:] Kennen Sie Beispiele f{\"u}r die Verkn{\"u}pfung geographischer Daten mit Diskussionsbeitr{\"a}gen?
    \item[P8:] Also ich kenne nur themenbezogene Seiten, die dann eben mit eingeblendeten Google-Karten arbeiten und sagen "`Wir sind hier oder da und da und da"'. Wenn die dann noch Kommentarfunktionen oder Blogs haben, steht das nach meiner Wahrnehmung eher nebeneinander und nicht miteinander. Also eine direkte Verkn{\"u}pfung von geographischen Daten und Interaktionen im Sinne von blogging oder chatting ist mir nicht bekannt. Aber in der Hinsicht bin ich auch nicht der wirkliche Fachmann.
    \item[I:] Kennen Sie Werkzeuge um interaktive Karten mit eigenen Inhalten zu erstellen?
    \item[P8:] Nein.
    \item[I:] Dann war es das von meiner Seite. Gibt es von Ihrer Seite noch Fragen oder Anmerkungen?
    \item[P8:] Ich denke wir haben die wesentlichen Aspekte ihres Projektes ja jetzt besprochen. Ich interessiere mich sehr daf{\"u}r, wie das jetzt in ein Projekt eingebracht werden kann, wo wir aktiv die B{\"u}rger auf eine Kollektion von alternativen Angeboten ansprechen. Der Projekttitel ist ja "`1000 Stunden f{\"u}r M{\"u}nster"'. Das ist ja ein Titel, aber dahinter steht ja die {\"a}hnliche Vorstellung wie bei Ihnen, dass konkrete Angebote zusammengestellt werden und beworben werden. Im Rahmen dieser Bewerbung wird sich dann auch die Frage stellen wo die sich befinden und wie k{\"o}nnen Interessierte dazu Informationen finden. Was ich sehr spannend bei Ihrer Konzeption finde, dass man eben sehr einfach die Seiten der Anbieter finden kann. In sofern kann unser Projekt auch eine Verst{\"a}rkung f{\"u}r ihr Anliegen sein. Und in welchem Ma{\ss}e wir die von Ihnen entwickelte Anwendung einsetzen und einbinden, dar{\"u}ber wollen wir ja noch sprechen.
    \item[I:] Ja. Dann vielen Dank f{\"u}r ihre Antworten und Einblicke!
\end{itemize}