\textbf{Teil 1 -- B{\"u}rgerbeteiligung}
\begin{itemize}
    \item[I:] Erz{\"a}hlen Sie mir {\"u}ber ihre Rolle und Aufgaben in der B{\"u}rgerbeteiliung.
    \item[P5:] Also generell halte ich das Thema B{\"u}rgerbeteiligung, das ist ja ein sehr weiter Begriff, f{\"u}r erstmal eine wichtige Sache. Auch wenn man sich das jetzt so unter Demokratieaspekten anguckt. Wenn man sich da anschaut, was es da ja prinzipiell darum gehen sollte, Macht vom Volke ausgehen zu lassen. Dann halte ich so ein Konzept "Nur alle vier Jahre w{\"a}hlen gehen, oder halt wenn Wahlen sind" eigentlich f{\"u}r zu wenig. Und deshalb freue ich mich auch, dass solche Themen, oder das Thema B{\"u}rgerbeteiligung in den letzten Jahren so meiner Wahrnehmenung nach auch gr{\"o}{\ss}er geworden ist. Dass das auch in der etablierten Politik zunehmend Anh{\"a}ngerinnen findet. Und dass das halt, wenn sich das durchsetzt, da dann das auch schon passiert, zumindes gef{\"u}hlt ein besseres Ergebnis entwickelt. Das muss ja nichtmal hinterher sein, dass das da automatisch ein besseres Ergebnis steht, aber dadurch, dass die Leute mitmachen k{\"o}nnen, und sich einbringen k{\"o}nnen, bleibt zumindest nicht so das Gef{\"u}hl "{\"U}ber mich wird potentiell hinwegentschieden.". Ist ja nicht so dass da alle mitmachen und es beschweren sich immer noch genug Leute. Und unter solchen Aspekten, denke ich, dass das auch auszuweiten ist, und solche M{\"o}glichkeiten, dass das die Leute m{\"o}glichst viel was sie Betrifft auch mitbestimmen k{\"o}nnen. Ja das halte ich f{\"u}r erstrebenswert und deshalb hat es mich dann auch	gefreut, dass ich jetzt hier im Rahmen meines Hiwi-Jobs auch die M{\"o}glichkeit hatte, da so ein St{\"u}ck weit das mit zumachen.
    \item[I:] Au{\ss}erhalb dieses Rahmen haben Sie dann noch an keiner Initiative oder Projekt teilgenommen?
    \item[P5:] Vielleicht nicht unbedingt unter dem Aspekt B{\"u}rgerbeteiligung, aber ich bin selber sehr stark engagiert in so politischen Projekten. So "von unten". Also, wo habe ich in den letzten Jahren mitgemacht? In Stadtteilinitiativen oder generell in Osnabr{\"u}ck hat sich jetzt da auch so ein "Recht auf Stadt"-B{\"u}ndnis entwickelt, wo ich mitgemacht habe. Ich bin seit es das gibt in Osnabr{\"u}ck, seit 2008 glaube ich jetzt, mache ich mit bei so einem selbstverwalteten Zentrum f{\"u}r politische und kulturelle Veranstaltungen. In der Klima-Bewegung oder bei so Klima-Aktionen. Zum Beispiel halt im Klima-Camp im Rheinland war ich die letzten Jahre immer so mit dabei. Also halt solche Sachen. Das ist jetzt nicht direkt so klassisch B{\"u}rgerbeteiligung, aber letztlich geht es ja auch darum, dass das Leute, die betroffen sind, mitmachen k{\"o}nnen und sich f{\"u}r ihre Belange auch selber einsetzen.
    \item[I:] Wie lief dann in in diesen Projekten die Kommunikation zwischen den Beteiligten ab?
    \item[P5:] Also wenn man das einmal so pauschal sagen wollte, ist da halt schon immer so das Ziel m{\"o}glichste alle die da hinkommen, alle die irgenwie mitmachen (\dots) also es gibt dann ja verschiedene Sachen, direkte Treffen, Mailinglisten, Telefonkonferenzen und was nicht alles. Das ist schon immer das Ziel, dass sich die Leute nach M{\"o}glichkeit einigen, dass sie sich austauschen. Ich kann mich da gut dran erinnern, dass da bei Entscheidungen dann so ein Satz war "Okay, wir haben das jetzt durchdiskutiert, wenn wir das jetzt entscheiden mit dieser Entscheidung, wer hat dann noch Bauchschmerzen?" Das ist dann halt immer so, dass es auf Konsens zielt. Wo bei das bei manchen nat{\"u}rlich auch nicht m{\"o}glich ist, und trotzdem was entscheiden will, dann versucht man da halt dann doch noch vielleicht andere, meinetwegen andere Abstimmungen hinzukriegen. Aber dann auch immer so dass da niemand quasi mitentscheidet, wo er gar nicht daf{\"u}r ist. So ich glaube das zieht sich relativ durch. Das ist nat{\"u}rlich nicht immer ganz idealtypisch, dass das klappt, oder dass es auch mal ganz anders l{\"a}uft. Dass Leute mal nicht da waren und dann hinterher sagen, eigentlich wollte ich das doch nicht. Dann waren Sie halt nicht da. Kommt halt vor, aber im gro{\ss}en und ganzen zielt das immer darauf, dass das Leute, die irgendwie davon betroffen sind, und die mitentscheiden wollen, auch mitentscheiden k{\"o}nnen. 
\end{itemize}

\textbf{Teil 2 -- Einsatz der Anwendung}
\begin{itemize}
    \item[I:] Bitte geben Sie mir eine kurze Einf{\"u}hrung in das Projekt in dem Sie die Anwenung einsetzen wollen.
    \item[P5:] Genau. Diese Karte, die Idee kam uns ja schon ein bisschen fr{\"u}her. Seit dem ich jetzt hier mitmache bei diesem Hiwi-Job, seit Ende 2012, glaube ich, haben wir ja angefangen so ein Projekt zu starten, dass bisschen die Themen, die halt die beiden Profs hier an diesem Lehrstuhl mitbesch{\"a}ftigt sind, schon vorher l{\"a}ngere Zeit besch{\"a}ftigt haben, also so ein bisschen Stadtforschung und auf der anderen Seite Sozialisation und Gemeinschaftsforschung, Bildung ist da auch noch ein Aspekt der da drin noch mit herumwabert, das wollten die zusammenbringen. Und dann ging es eben darum, eine Plattform zu schaffen, um das "Soziologische Wissen", nenn ich es jetzt mal, einmal verf{\"u}gbar zu machen, auch nach au{\ss}en, dass das nicht immer nur etwas ist, was im Institut bleibt, halt eine Homepage zu gestalten. Und halt solches Soziologisches Wissen, dass halt entstanden ist, dort verf{\"u}gbar zu machen und auch gleichzeitig, das war dann eigentlich immer das Ziel, Leute die in der Stadt sind, und sich mit {\"a}hnlichen Themen besch{\"a}ftigen, und da gibt es hier in M{\"u}nster ja wohl eine ganze Menge, die aber oft so ein bisschen f{\"u}r sich und und unter sich bleiben, die haben halt ihr Spezialthema. Und da ging es ein bisschen darum zu gucken "Okay, Spezialthemen sind auch wichtig, das ist auch gut so dass die das machen, aber vielleicht w{\"a}re es auch eine M{\"o}glichkeit eine st{\"a}rkere Wirkung zu erzielen, einmal diesem ganzen Initiativen, sofern das nicht sowieso schon der Fall war, bewusst zu machen, dass es noch andere Stellen gibt, die zu {\"a}hnlichen Themen arbeiten, die auch {\"a}hnliche Methoden haben, und {\"a}hnliche Ziele" Und das ganze also dann auch untereinander zu vernetzen. Und halt auch den Austausch untereinander.\\
    Damit Diskussionen vielleicht mal woanders ankommen und dort aufgegriffen und mit verarbeitet werden k{\"o}nnen. Das war eigentlich so das Ziel. Und daraufhin haben wir dann auch hier eine Tagung veranstaltet. Beziehungsweise im Schloss veranstaltet. Zusammen auch mit der Stadt M{\"u}nster, und da dann auch eben viele der Initiativen und Parteien hier im Rat eingeladen. Auch nat{\"u}rlich die "normalen" Leute aus der B{\"u}rgerschaft. Das sind dann halt so Versuche solche Themen einzubringen, auch genau solche Vernetzungsprozesse zu starten und die Leute tats{\"a}chlich auch zusammen zu bringen, dass die sich auch kennenlernen. Ja dann im Zuge dessen kam dann eben auch die Idee auf, so eine Karte zu machen. Eine digitale Karte, die eben f{\"u}r alle Leute, zumindest wenn sie Internet haben, aber das ist ja mittlerweile eigentlich doch sehr sehr weit verbreitet, (\dots) Wo alle drauf zugreifen k{\"o}nnen, wo sie selber Vorschl{\"a}ge machen k{\"o}nnen. Vorschl{\"a}ge die schon dastehen kommentieren k{\"o}nnen und darauf antworten k{\"o}nnen. Und einfach sehen k{\"o}nnen, ja was passiert denn in meiner Stadt, kann ich irgendwo was mitmachen. Also da ist dann ein bisschen so der Gedanke, dass viele Leute vielleicht gute Ideen haben, viele Leute auch was machen und auch andere Leute das interessieren k{\"o}nnte oder auch interessiert, das nicht mitkriegen. Das ist ja nicht unbedingt einfach auch in einer gr{\"o}{\ss}eren Stadt, wo es oft so ein bisschen anonym zugeht, und da also die Initiativen von einzelnen Leuten sichtbar zu machen. Das war glaube ich so ein bisschen die Idee die hinter so einer Karte stand. Und genau. Daraufhin gab es dann diese Kooperation mit dem Institut f{\"u}r Geoinformatik. Und da haben wir dann versucht, diese Ideen dann quasi praktisch umzusetzen, mit der programmiertechnischen Hilfe. Und genau. Dann wollen wir halt jetzt sehen wie dass das jetzt anl{\"a}uft. Und gucken, wir sind ja auch gespannt wie das dann funktioniert.
    \item[I:] Dann k{\"o}nnen wir gleich weitermachen. Welche Gr{\"u}nde sprechen f{\"u}r den Einsatz dieser L{\"o}sung gegen{\"u}ber anderen angedachten L{\"o}sungen?
    \item[P5:] Uh das ist ja eine schwere Frage. Ich wei{\ss} gar nicht, ob wir uns andere L{\"o}sungen {\"u}berlegt hatten. Ich kann mich jetzt gar nicht dran erinnern, dass wir noch andere Sachen hatten. Zumindest haben wir die wenn, nicht l{\"a}nger verfolgt. Das w{\"a}re dann halt so ein Brainstoring gewesen, und dann ist es glaube ich bei der Karte h{\"a}ngen geblieben, soweit ich mich jetzt erinnern kann. Ja also was halt daf{\"u}r spricht, ist halt, dass einmal die wie gesagt relativ allgemeine Verf{\"u}gbarkeit, so auch jederzeit. Das ist ja nicht darauf angewiesen, dass Leute zeitgleich auch irgendwo zusammenkommen oder zeitgleich mitmachen. Weil ja auch Tagesabl{\"a}ufe und Rhythmen, doch sehr unterschiedlich sein k{\"o}nnen. Und da ist also die M{\"o}glichkeit eine Plattform zu haben, wo Leute, wenn sie Zeit haben, dann auch zugreifen k{\"o}nnen, sehen k{\"o}nnen, was ist passiert. Selber was machen k{\"o}nnen, und sich dar{\"u}ber dann vielleicht auch absprechen k{\"o}nnen und sich tats{\"a}chlich dann auch zu treffen. Das ist f{\"u}r mich halt immer noch so der gro{\ss}e Vorteil, das gilt ja generell f{\"u}r digitale Medien, (\dots) Es steht da, und es ist verf{\"u}gbar und verschwindet dann nicht sofort wieder. Und gleichzeitig macht es eben in der Form dieser Karte ja direkt sichtbar. Das ist ja quasi die Verkn{\"u}pfung mit dem echten Raum, mit der sozialen Welt, wo die Leute auch sehen k{\"o}nnen, was passiert denn hier bei uns in der Nachbarschaft, in meinem Viertel oder bei mir einfach in der N{\"a}he oder generell in der Stadt, wo ich vielleicht Lust h{\"a}tte vielleicht mitzumachen. Und genau. Da ist es dann, wenn man so will, vielleicht ein Vehikel, das ist ja so ein bisschen unsere Hoffnung, dass {\"u}ber die Vorschl{\"a}ge, die dann in dieser digitalen Karte stehen, dann daraus auch Bekanntschaften, Ideenaustausch und letztlich auch ganz handfeste Sachen entstehen, wo die Leute und B{\"u}rgerinnen sich selber Ideen geben k{\"o}nnen und Ideen umsetzen k{\"o}nnen. Manche Sachen vielleicht direkt vor Ort, manche Sachen dann geb{\"u}ndelt, dann geht man vielleicht, ich wei{\ss} nicht, zu Stadtteilb{\"u}ros oder wendet sich an die Stadt, je nachdem was es halt ist. Aber halt diesen Austausch der Leute und Ideen der Leute und Vernetzung von der digitalen Welt in die echte Welt zu {\"u}bertragen. Und da halt so ein Werkzeug zu haben. Deswegen ist halt eine Karte tats{\"a}chlich eine gute M{\"o}glichkeit um da so eine {\"U}bertragung zwischen echt und virtuell hinzubekommen.
    \item[I:] Welche Eigenschaften w{\"u}rden Sie davon abhalten diese Anwendung einzusetzen?
    \item[P5:] Ja so nach meiner Erfahrung (\dots) Ja einmal wenn es kompliziert ist, also wenn die Sachen, die da passieren, die Funktionen die da vielleicht drin stecken, wenn die nicht gefunden werden, wenn die zu kompliziert wirken, wenn also Leute die da vielleicht nicht sowieso schon Interesse an so einem Thema haben, und auch bereit sind, sich ein bisschen Zeit daf{\"u}r zu investieren sich einzuarbeiten. Also wenn Leute, die mal da so drauf gesto{\ss}en sind, oder vielleicht zuf{\"a}llig drauf kommen oder das irgendwie mitbekommen und sich das angucken. Wenn es dann also zu frustrierend oder kompliziert ist, die Sachen die da m{\"o}glich sind, auch zu machen. Das k{\"o}nnte ich mir vorstellen, k{\"o}nnte einige Leute dann davon abhalten, dass weiter zu nutzen. Und was mir sonst jetzt noch spontan einf{\"a}llt, ist vielleicht auch eine Sache, wenn da wenig los ist. Das kenne ich ja auch von mir selber, wenn ich in Foren bin, in denen nicht viel passiert, da klickt man ein paar mal rein. Aber wenn man merkt, da schreibt sowieso kaum wer, und das was ich geschrieben hat, hat jetzt auch noch keiner drauf geantwortet, dann lasse ich es vielleicht nach einer kurzen Zeit wieder da mitzumachen. Das hei{\ss}t also wenn bei dieser Karte da keine Sachen stehen, und die Leute selber vielleicht auch noch nicht selber Vorschl{\"a}ge machen, sondern erst einmal gucken was da passiert. Wenn da dann also nicht viel passiert, k{\"o}nnte das nat{\"u}rlich auch dazu f{\"u}hren, dass die dann erst mal wieder von ablassen. Also d{\"u}rfte es da dann auch darum gehen, gerade dann am Start zumindest ein paar Sachen zu haben, und das dann auch selber, auch von organisatorischer Seite auch selber Sachen einzutragen, auf der einen Seite dann auch vielleicht Werbung zu machen. Wir sind ja jetzt schon ein St{\"u}ck weit vernetzt mit vielen Initiativen hier aus der Stadt. Da dann auch nochmal sagen, dass die auch selber Sachen rein setzen, so dass da dann auch was passiert, dass die Leute auch vielleicht im besten Fall sehen, "Ja, tats{\"a}chlich auch bei mir um die Ecke sind Sachen". Das Internet kann ja manchmal das Gef{\"u}hl geben, dass das irgendwo stattfindet. Und es ist nicht direkt sp{\"u}rbar. Wenn aber so dieser Impuls vielleicht dann kommt. "Das ist bei mir um die Ecke, ich kann da hingehen, und tats{\"a}chlich auch etwas machen, das was mich wirklich auch ber{\"u}hrt". Dass das einfach daf{\"u}r sorgt, dass Leute selber auch Motivationen haben, mitzumachen und selber dann damit auch bereit sind Erfahrungen zu machen. Und dann auch selber bereit sind Sachen da rein zu stellen. Dass das dann irgendwie so ein bisschen ein Selbstl{\"a}ufer wird. Aber es k{\"o}nnte auch da sein, dass dort erst so eine kritische Masse erreicht werden muss, das k{\"o}nnte ich mir am Anfang als Schwierigkeit vorstellen.
    \item[I:] Ja muss man dann schauen dass man die Seite dann in Gang h{\"a}lt. (Ja dass das auch gepflegt wird)
\end{itemize}

\textbf{Teil 3 -- Abschlie{\ss}ende Fragen}
\begin{itemize}
    \item[I:] Kennen Sie Beispiele f{\"u}r die Verkn{\"u}pfung geographischer Daten mit Diskussionsbeitr{\"a}gen?
    \item[P5:] Ja so ein bisschen. Ich kenne das so aus politischen Zusammenh{\"a}ngen. Zum Beispiel im Wendland. Da sind ja manchmal Castortransporte und das ist ja ein relativ gro{\ss}es Gebiet, sehr weitl{\"a}ufig. Wald, Felder. Und halt immer rund um diese Schiene, wo der Castor dann langl{\"a}uft, da kenne ich das durchaus, dass dann eben Karten, auch noch aus Papier aber bei dem letzten dann auch schon mit digitalen Karten, dass da dann auch Diskussionen gef{\"u}hrt wurde, {\"u}ber Punkte, die vielleicht gut zu erreichen sind, wo es gut w{\"a}re, sich da auf die Schienen zu setzen, oder so {\"a}hnliche Sachen. Dass da dann die Karten genutzt worden sind, um dar{\"u}ber zu diskutieren was man wo, wie machen k{\"o}nnte. Wo Infopunkte hin sollten.
    \item[I:] Und wenn das dann digitale Diskussionen waren, mit welchen Tools wurden die dort dann durchgef{\"u}hrt?
    \item[P5:] Ich glaube wir haben damals noch eine (\dots) So einen IRC\footnote{Internet relay chat}-Channel benutzt. Genau dar{\"u}ber.
    \item[I:] Da konnte man dann aber nicht direkt auf einer Karte diskutieren?
    \item[P5:] Nein die Karte war dann woanders. Die war halt irgendwo hochgeladen. Die hatte man sich dann irgendwie angeguckt. Und dann haben wir {\"u}ber IRC dar{\"u}ber dann uns ausgetauscht. Das war dann aber halt zwei Sachen die man getrennt hatte. Das war nicht verkn{\"u}pft.
    \item[I:] Kennen Sie Werkzeuge um interaktive Karten mit eigenen Inhalten zu erzeugen?
    \item[P5:] Ich wei{\ss} nicht, ob es das genau trifft, ich wei{\ss} dass es bei Google Maps so M{\"o}glichkeiten irgendwie Marker zu setzen, das habe ich aber selber noch nie gemacht. Und ich glaube bei Openstreetmap ist es auch m{\"o}glich. Da irgendwie Punkte in die Karte zu machen, oder sich selber Routen oder sowas anzulegen f{\"u}r eigene Zwecke. Habe ich selber aber auch noch nicht gemacht.
    \item[I:] Gut. Dann war es das jetzt auch von meiner Seite. Gibt es noch Fragen oder Anmerkungen von Ihrer Seite?
    \item[P5:] Nein, mir f{\"a}llt jetzt auch gerade nichts mehr ein. Au{\ss}er, dass ich mich halt freue, dass sich das so gefunden hat. Ich fande das auch insgesamt so ziemlich so nett. Also angenehm, so mit dir das zusammen zu machen. Diesen Austausch. Wobei du nat{\"u}rlich auch die meiste Arbeit hattest, und wir immer nur kluge Vorschl{\"a}ge hatten, oder so. Aber ich fand das war generell eine sch{\"o}ne Zusammenarbeit. Und das hat mir auch so Spa{\ss} gemacht.
    \item[I:] Ja vielen Dank!
\end{itemize}