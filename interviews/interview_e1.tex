\begin{itemize}
    \item[I:] Kennen Sie Anwendungen die Diskussionen durch Geoobjekte unterst{\"u}tzen?
    \item[E1:] (\dots) Ich {\"u}berlege gerade. Es gibt so Emergency-Response-Maps so in Krisenregionen um Hilfseins{\"a}tze zu planen. So was habe ich mal gesehen. Das k{\"o}nnte so in die Richtung gehen weil da eben auch so bestehende Diskussionen mit irgendwie Geoobjekten angereichert werden. Ansonsten Diskussionen nicht wirklich.
    \item[I:] Also dann auch nicht benutzt?
    \item[E1:] Nicht dass ich w{\"u}sste. Also ich meine nat{\"u}rlich habe ich total viele Geo-Anwendungen irgendwie. Also nehmen wir an wie (\dots) Google Maps oder irgendwelche Tank Apps oder Navigations Apps und Apps um Einkaufszentren zu finden oder {\"a}hnliches. Aber das hat ja alles nichts mit einer Diskussion zu tun. Also ich denke nein.
    \item[I:] Also so richtig Problem oder Vorteile davon kennen Sie dann auch nicht?
    \item[E1:] Nein nicht wirklich. Also Vorteile k{\"o}nnte ich mir halt vorstellen, dass es eben Eineindeutig ist dass ich {\"u}ber einen bestimmten Ort schreibe den ich halt gleichzeitg dann noch mit einem Ort auf der Karte verkn{\"u}pfen kann. Dann ist es eben klar, {\"u}ber was ich spreche. Und das ganze ist eben eineindeutig. Und, also nehmen wir an, ich spreche jetzt {\"u}ber M{\"u}nster und es gibt zwei M{\"u}nster in Deutschland und dann ist klar welches M{\"u}nster ich meine.
    \item[I:] Das haben Sie ja gerade schon ein bisschen gesagt, aber welche Anwendungsf{\"a}lle zur Verkn{\"u}pfung von Diskussionen und Geoobjekten k{\"o}nnen Sie sich au{\ss}erhalb des B{\"u}rgerbeteiligungskontextes vorstellen?
    \item[E1:] Ich k{\"o}nnte mir das relativ grob strukturiert eigentlich {\"u}berall vorstellen. Sei es dass ich einen Foreneintrag verfasse oder jedem Diskussionsobjekt eben die M{\"o}glichkeit habe, Orte direkt zu verkn{\"u}pfen und mit dem Vorteil eben dann direkt verweisen kann auf irgendwelche Geoinformationsanwendungen. Und als konkreten Einsatzzweck f{\"a}llt mir eben nur dieses Krisenmanagementsystem ein von dem ich eben schon erz{\"a}hlt habe.
    \item[I:] Welche L{\"o}sungen um B{\"u}rger, Initiativen und die Politik zusammenzubringen kennen Sie?
    \item[E1:] B{\"u}rgerinitiativen k{\"o}nnte man sagen. B{\"u}rgerstammtische. (\dots) So offiziell organisierte Treffen wo dann irgenwie Informationsaustausch stattfindet. Also zum Beispiel zu irgendwelchen lokalen Initiativen. So als Beispiel "`Tagebau in M{\"u}nster"' und dann w{\"u}rde informiert werden. Oder Stuttgart 21 und dann findet irgendwie eine Informationsveranstaltung dazu statt. 
    \item[I:] Und so in Richtung Software-L{\"o}sungen?
    \item[E1:] Also es gibt ja auf jeden Fall so Petitionsportale wo ich jetzt sagen w{\"u}rde das ist ja eher vom B{\"u}rger initiiert. Also vom B{\"u}rger in Richtung Politik. (\dots) Ich glaube die Bundesregierung hat auch irgendwie so ein Ding wo man mitdiskutieren kann. Ich muss gestehen, ich wei{\ss} gerade nicht genau wie das hei{\ss}t. Komme ich irgendwann mal wieder drauf. Also ich glaube aber auch dass es von der Politik Portale gibt, die sich an die B{\"u}rger wendet und dann auch zu aktiver Mitarbeit aufruft. Hab ich aber aktiv noch nicht benutzt. Und wei{\ss} gerade nicht genau wie das hei{\ss}t. 
    \item[I:] Denken Sie die explizite Verkn{\"u}pfung von Geoobjekten mit Diskussionsgegenst{\"a}nden ist generell hilfreich im B{\"u}rgerbeteiligungskontext?
    \item[E1:] Ja absolut! Ich habe mal von sowas geh{\"o}rt, da konnte man, glaube ich, Stra{\ss}ensch{\"a}den melden. Das f{\"a}llt mir gerade auch noch so ein. Quasi zur Frage vorher. Und das ist ja schon ganz interessant, wenn ich mir sage "`Okay, hier ist irgendwie an folgender Stelle die Stra{\ss}e kaputt"' Und dann kann ich das direkt auf ner Karte markieren. Oder ich hab das glaube ich auch mal geh{\"o}rt, dass man zu so einem Blitzmarathon Vorschl{\"a}ge machen an welchen Stellen Gefahrenstellen sind und an welchen Stellen geblitzt werden soll. Da konnte man direkt auf so ner Karte markieren was die Stelle ist die ich konkret vorschlage. Ich denke schon dass das sinnvoll ist. 
    \item[I:] Dann konkret zur Anwendung im Vergleich zu bestehenden Anwendungen die Sie schon kennen. Was denken Sie speziell zu der Gegebenheit dass in der {\"U}bersicht nur die Geoobjekte des ersten Beitrages zum Thema angezeigt werden und dass in der Themendetailansicht nur die Geoobjekte zu dem Thema angezeigt werden?
    \item[E1:] Ich glaube das ist gut f{\"u}r die {\"U}bersichtlichkeit. Also nat{\"u}rlich hat das jetzt einen starken Fokus auf den Ersteller. Es scheint mir weniger Community-fokussiert, sondern hat eher einen Autorenfokus k{\"o}nnte man sagen. Aber auf der anderen Seite ist eben so dass ich nicht wei{\ss} wie gro{\ss} so diese Diskussionen werden k{\"o}nnen. Also wenn man sich jetzt wirklich vorstellt, zum Beispiel hat irgendjemand eine Frage gestellt und dann kommen da sagen wir mal zwanzig Antworten die jeweils auch Geoobjekte referenzieren. Dann w{\"u}rde dieses eine Projekt oder Beitrag sehr dominierend auf der Karte sein. Und deshalb denke ich schon, dass es eine sinnvolle Entscheidung ist nur ein Objekt pro Thema anzuzeigen. Zumindest jetzt gerade in diesem Kontext wie ich es einsch{\"a}tzen kann.
    \item[I:] Dann die Zwei-Wege Highlights bei Mausinteraktion?
    \item[E1:] Absolut sinnvoll! Also sonst w{\"u}sste ich ja gar nicht was wo zu geh{\"o}rt. Nat{\"u}rlich k{\"o}nnte ich das vermutlich auch hier anklicken und komm dann direkt in die Detailansicht. Theoretisch ist die Verlinkung von der Karte zu dem Objekt nicht so wichtig. Weil ich kann ja sowieso in die Detailansicht rein gehen. Aber dass ich, wenn ich nur das Objekt auf der rechten Seite habe, direkt sehe wo es auf der Karte verortet ist, das halte ich f{\"u}r sehr wichtig.
    \item[I:] Die Filter- und Sortierfunktion?
    \item[E1:] Gut da h{\"a}ngts immer davon ab, wie viele Eintr{\"a}ge ich hab. Also momentan mit zehn Eintr{\"a}gen ist nat{\"u}rlich so ein Filter noch nicht so wichtig, wenn aber das ganze mal deutlich mehr werden, ist so ein Filter auf jeden Fall wichtig. Da h{\"a}ngt es glaube ich dann davon ab dass die Funktionen die der Filter bietet, sinnvoll gew{\"a}hlt sind. Und dass sie verst{\"a}ndlich sind finde ich. Also dass ich jetzt hier zum Beispiel bei "`diskutieren"', "`mitmachen"', "`vorschlagen"' wirklich genau wei{\ss}, was ich anklicke. Das hier zum Beispiel bei "`vorschlagen"' w{\"u}rde ich mir noch ein bisschen zus{\"a}tzliche Erkl{\"a}rung was ich genau ich hier jetzt filtern kann w{\"u}nschen oder {\"a}hnliches. Ich denke bei den Akteuren ist das relativ eindeutig. Also "`Bildung"', "`B{\"u}rger"', "`Stadt"', "`Wirtschaft"', das versteht jeder. Bei den Inhalten ist das nat{\"u}rlich so ein bisschen (\dots) Ja, ob das jetzt wirklich trennscharf ist, und ob das alles abdeckt; insbesondere so Dinge wie "`Sonstiges"', da landet h{\"a}ufig viel zu viel in solchen Kategorien. Aber ansonsten, ja Filter eindeutig gut.
    \item[I:] Dann diese Verfassen- und Antworten- Funktion und speziell das Verkn{\"u}pfen von den W{\"o}rtern mit den Geoobjekten, bestehenden Geoobjekten und Hyperlinks?
    \item[E1:] Finde ich gut. Ich wei{\ss} allerdings nicht ob es einhundert Prozent intuitiv ist. Also es ist ja schon so wenn ich hier jetzt irgendwie drauf antworte, dann wird mir erstmal (\dots) Also erstmal h{\"a}tte ich mich hier intiuitv wahrscheinlich, wenn das jetzt hier gerade im Vorfeld nicht so eindeutig erkl{\"a}rt worden w{\"a}re, nicht mit diesen Icons besch{\"a}ftigt. Die h{\"a}tten mir erstmal nichts gesagt. Also ich glaube dieses "`Link"'-Icon, das erkenne ich und eigentlich diesen Geo-Marker erkenne ich auch. Das kenne ich schon aus einem anderen Kontext. Der ist ja schon sehr stark an Google Maps Objekt orientiert. Und so einen Link, kennt man aus jeden Online-Editor oder Text-Editor irgendwie. Aber jetzt zum Beispiel dass ich irgendwas eingetippt h{\"a}tte, dann das Wort markiere, dann dieser Kontextdialog. Das ist eben etwas da w{\"a}re ich glaube ich selber als Bediener nicht drauf gekommen. Und unter dem Standard "`Antwort verfassen"' habe ich auch nicht im Hinterkopf dass meine Antwort eben Geodaten enthalten kann. Nichtsdestotrotz glaube ich dass jemand der so ein bisschen hier mit herumspielt ist dass dann schon klar. Also hier auch vielleicht "`Punkt markieren"' w{\"u}rde ich auch in "`Ort markieren"' umbennen. "`Punkt"', ich wei{\ss} nicht ob ich da automatisch was Geographisches mit verbinden w{\"u}rde. Aber ansonsten ist das sehr gut mit der Funktionalit{\"a}t.
    \item[I:] Dann ganz speziell jetzt auf die Anwendung bezogen. Wie werden Dialoge damit vereinfacht?
    \item[E1:] Das kann ich total schwierig nur einsch{\"a}tzen. Also es ist einfach so, da sprech ich jetzt aus eigener Erfahrung: Diskutieren {\"u}ber Tools ist wirklich schwierig. Insbesondere wenn (\dots) Das h{\"a}ngt hier jetzt insbesondere davon ab, ob es Moderationsfunktionalit{\"a}ten noch gibt. Grunds{\"a}tzlich sieht es hier aus dass f{\"u}r Diskussionen eigentlich relativ wenig Platz ist. Also in der Hinsicht, als dass sowohl das Eingabeformular relativ beschr{\"a}nkt ist vom Platz her, als auch der Platz auf dem die Diskussionen angezeigt werden. Kleinere Diskussionen kann ich mir durchaus vorstellen, aber wenn ich mir jetzt wirklich vorstelle zum Beispiel f{\"u}nfzig Akteure versuchen eine Diskussion zu f{\"u}hren, die dann der menschlichen Natur folgend dann auch nicht perfekt strukturiert abl{\"a}uft, also jeder versucht seinen eigenen Standpunkt durchzubringen (\dots) Dann glaube ich, dass durchaus externe Tools sinnvoller sein k{\"o}nnten. Ist aber pure Vermutung.
    \item[I:] Dann die Favorisierung?
    \item[E1:] Also f{\"u}r mich als Orientierungskriterium ist das nat{\"u}rlich super. Das hei{\ss}t also wenn ich die Themen sortieren kann, was hat die meisten Favorisierungspunkte und mir somit irgendwie so auf die Meinung der anderen Beziehen kann, das finde ich gut. Ob ich selber als Nutzer davon so viel Gebrauch machen w{\"u}rde, wei{\ss} ich nicht so genau. Weil zum einen, gibt es ja wie ich das sehe keine eigene Favoritenliste. Also ich verwalte damit nicht meine Favoriten, wie Bookmarksfunktion sozusagen. Das w{\"u}rde Reize f{\"u}r mich erzeugen dann auch Beitr{\"a}ge zu favorisieren. Und zum anderen ist die Funktionalit{\"a}t relativ versteckt finde ich. Ich wei{\ss} gerade schon wieder nicht wie es geht. Das scheint mir auch relativ versteckt. Ich hab ja eben instinktiv versucht in der {\"u}bersicht schon auf das Herzchen zu klicken, aber es funktioniert ja nur in der Detailansicht. Und dann ist es ja sogar komplett ausgeblendet wenn ich da nicht mit der Maus dr{\"u}ber fahre. Auch ist es sehr klein und ausgegraut und so weiter. Also da wei{\ss} ich nicht, ob die Funktion so intuitiv zu entdecken ist. Also wie gesagt ich finde es sehr gut dass die Funktionalit{\"a}t da ist, ich wei{\ss} nur nicht, ob den Nutzern genug Anreize gegeben werden diese Funktionali{\"a}t wirklich aktiv zu verwenden.
    \item[I:] Die Benutzerregistrierung und Anmeldung und auch ganz speziell der Social Login?
    \item[E1:] Alles absolut {\"u}bersichtlich. Genauso wie man das schon von anderen Webseiten her kennt. H{\"a}lt sich an die Standards. Das ist das wichtigste. Das hei{\ss}t also es gibt genau die Felder die ich mir vorstellen w{\"u}rde. Das einzige was mir eben aufgefallen ist, wenn ich mich einlogge, h{\"a}tte ich das registrieren weiter unten im Dialog gesucht. Also nicht oben im Titel, sondern weiter unten. Ich hab das eben {\"u}berlesen und erst nachdem ich aktiv danach gesucht habe, ist mein Blick nochmal auf den Titel gefallen, und dann war da der Link. Also ist aber ja auch nicht total versteckt. Ansonsten, ja, diese Social Login Funktionalit{\"a}ten finde ich gut. Sind ja auch in vielen Seiten verf{\"u}gbar. Ich selber nutze die fast nie, aber finde ich gut. Ist halt so State-of-the-art das zu bieten.
    \item[I:] Haben Sie insgesamt irgendwelche Funktionen vermisst?
    \item[E1:] Hm. (\dots) Ja ich habe ja eben schonmal gesagt, vielleicht dem Benutzer noch so eine "`Pers{\"o}nliche Favoriten"'-Liste anbieten. Oder eben dass mir Dinge zu denen ich selber schon mitgewirkt habe, also Beitr{\"a}ge geschrieben oder favorisiert habe, hervorgehoben werden. Das w{\"a}ren jetzt Dinge die mir direkt einfallen w{\"u}rden. Also nehmen wir jetzt erstmal den Fall an dass hier hundert Projekte eingetragen sind und da kommen jeden Tag neue Kommentare hinzu. Wie finde ich jetzt wieder, f{\"u}r was ich schonmal irgendwie kommentiert habe. Das w{\"u}rde mir fehlen. Ansonsten konkret nichts mehr. Aber es ist nat{\"u}rlich auch so dass ich nichts mit diesem Nachhaltigkeitsprojekt zu tun habe.
    \item[I:] Was f{\"u}r Gr{\"u}nde k{\"o}nnen Sie sich vorstellen die Leute davon abhalten k{\"o}nnten, die Karte zu benutzen?
    \item[E1:] Also es ist nat{\"u}rlich eine Registrierungsschwelle. (\dots) Ach nein, geht es sogar ohne?
    \item[I:] Nein, man kann verfassen, aber nicht absenden.
    \item[E1:] Ah okay. Ja aber das ist sehr gut gemacht. Finde ich richtig gut. Ich hatte jetzt eigentlich erwartet, dass mir direkt verboten werden w{\"u}rde einen neuen Beitrag zu verfassen. Und ich finde das sehr gut diese diesen Login nachdem der Kommentar verfasst wurde. Also ich hab mir die Arbeit schon gemacht und mehr als mein "`asdf"' da rein geschrieben, und dann versuch ich das abzuschicken und dann kommt die Aufforderung dass ich mich einloggen soll, dann tue ich das auch. Also finde ich sehr gut diese Reihenfolge gerade. Ansonten f{\"a}llt mir gerade kein Grund mehr ein. 	 
    \item[I:] Ja gut, gibt dann noch abschlie{\ss}ende Kommentare ihrerseits? Oder Fragen?
    \item[E1:] Eigentlich nicht wirklich. Doch. In Richtung was f{\"u}r Administrative Funktionen gibts da? Zum einen der Administrator der Seite und auch vielleicht f{\"u}r die Leute die da so ein Projekt "`besitzen"'. Hat der auch irgendwelche Moderationsfunktionen? Oder gibt es Exportfunktionen?
    \item[I:] Es gibt ein Admin-Interface mit dem kann man dann alles machen. Moderationsfunktionen f{\"u}r die Projektbesitzer gibt es allerdings nicht. Und Lesezugriff hat man {\"u}ber eine JSON-API. Deswegen k{\"o}nnte man das Interface auch komplett austauschen.
    \item[E1:] Ah okay.
    \item[I:] Gut, dann war es das. Vielen Dank!
\end{itemize}
