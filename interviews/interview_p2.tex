\textbf{Teil 1 -- B{\"u}rgerbeteiligung}
\begin{itemize}
    \item[I:] Erz{\"a}hlen Sie mir {\"u}ber ihre Aufgaben und Rollen in der B{\"u}rgerbeteiligung
    \item[P2:] Also das ist ne schwierige Frage weil ich da gar nicht so Aufgaben oder Rollen habe sondern mir sie eher selbst suche. Das hei{\ss}t, ich mach meistens dass was ich interessant finde. Und jetzt in dem Kontext halt zum Beispiel diese Organisation des Nachhaltigkeitstages und in dem Kontext dann auch mit der Arbeit mit der Karte.
    \item[I:] Und wie lange sind Sie da jetzt schon aktiv? 
    \item[P2:] Es hat denke ich so angefangen mit der Organisation der Tagung. Vor allem in letzter Zeit wieder mehr. (War das letztes Jahr?) Ja letztes Jahr. Oder es hat angefangen 2012. Wahrscheinlich sogar eher Ende 2011 oder so. Aber davor war ich aber auch schonmal so in dem Bereich w{\"a}hrend des Studiums unterwegs. So NGOs und Entwicklungszusammenarbeit und sowas.
    \item[I:] Bitte beschreiben Sie mir aus ihrer Sicht wichtige Aspekte der B{\"u}rgerbeteiliung.
    \item[P2:] Beteiligung. (lacht) Also das ist nat{\"u}rlich erstmal ein theoretisches Konzept, aber in der Praxis w{\"u}rde ich sagen, ist das wichtige dass die Leute wirklich mitmachen und mitgestalten. Das hei{\ss}t nicht nur passiv so da sitzen, sondern halt eine aktive Rolle haben.
    \item[I:] Und die Ziele und Nutzen davon?
    \item[P2:] Ja die Ziele und Nutzen sind erstmal so ne Art von Legitimation von Ma{\ss}nahmen w{\"u}rde ich sagen. Dabei w{\"u}rde ich nichtmal sagen dass das der Hauptpunkt ist, sondern eigentlich dass den Menschen die M{\"o}glichkeit gegeben wird ihr eigenes Umfeld so zu gestalten, wie sie es gerne wollen. Und dass sie in dem Umfeld wo sie leben nicht so beschr{\"a}nkt sind von {\"a}u{\ss}eren Sachen. Sondern halt eher selbstbestimmt alles zu organisieren.
    \item[I:] Bitte geben Sie mir eine Einf{\"u}hrung in das Nachhaltigkeitsprojekt.
    \item[P2:] Also im Prinzip hat es wie gesagt schon begonnen mit der Tagung letztes Jahr. Das war der Startschuss, dass wir uns gedacht haben, nach dieser Tagung muss es eigentlich irgendwie weitergehen. Das haben damals auch alle gesagt dass man danach einen Nachfolgeprozess organisieren wollte. Dazu haben wir dann ein paar Treffen nach der Tagung gemacht. Auf diesen Treffen haben wir dann entschieden, dass wir so einen Tag der Nachhaltigkeit organisieren, was dahin f{\"u}hren soll, dass n{\"a}chstes Jahr, am 15. Juni 2015 ein Tag in M{\"u}nster stattfindet, an dem an verschiedenen Orten Nachhaltigkeitsprojekte vorgestellt werden und {\"o}ffentlich {\"u}ber das Thema diskutiert werden kann. Dazu soll dann auch vorher ein bisschen {\"O}ffentlichkeitsarbeit in den Medien gemacht werden. Das hei{\ss}t dass man versucht den Diskurs somit weiter zu befeuern um das Projekt im Bewusstsein zu halten. Das ist auch dann das Hauptanliegen im Moment.
    \item[I:] Wieviel Wert wurde da im Vorfeld auf Dialoge gelegt?
    \item[P2:] Es kommt drauf an. Wir haben auf dieser Tagung ne Mailingliste angelegt, so dass wir jetzt verschiedene Verteiler haben. Das ist jetzt dann aber nicht f{\"u}r die gesamte {\"O}ffentlichkeit, sondern eher f{\"u}r diesen Kreis der auch da auf der Tagung war. Gleichzeitig haben wir auch {\"u}ber Zeitungsartikel und Einladungen in den Medien versucht Leute zu mobilisieren au{\ss}erhalb des Kreises. Das war allerdings nicht sehr erfolgreich. Insofern versuchen wir das demn{\"a}chst auch nochmal im Oktober oder so aber haben jetzt noch nicht gezielt auf Breitenbeteiligung geschaut.
    \item[I:] Habt ihr euch im Vorfeld schon nur auf Zeitung festgelegt, oder gab es auch in Richtung Social Media vorschl{\"a}ge?
    \item[P2:] Nein, das wurde eher spontan entschieden. Da war nur die Entscheidung dass wir unsere eigenen Netzwerke aktivieren. Das war der eine Pfad und der andere w{\"a}re halt {\"u}ber Medien die breitere {\"O}ffentlichkeit zu erreichen. Also {\"u}ber Social Media haben wir glaube ich gar nicht geworben. Nur halt E-Mail-m{\"a}{\ss}ig {\"u}ber die Listen.
\end{itemize}

\textbf{Teil 2 -- Einsatz der Anwendung}
\begin{itemize}
    \item[I:] Dann geht es jetzt weiter konkret zur Anwendung. Wer ist die Zielgruppe f{\"u}r die Anwendung?
    \item[P2:] Also die Zielgruppe sind potentiell eigentlich erstmal alle Interessierten. Ich w{\"u}rde das gar nicht so eingrenzen wollen. Nat{\"u}rlich in der Praxis sind das dann meistens die jenigen, die sowie so engagiert sind und in dem Bereich arbeiten.
    \item[I:] Was f{\"u}r Inhalte erwarten Sie?
    \item[P2:] Ich geh mal erstmal vom Idealfall aus. Ideal w{\"a}re es, wenn sich die Leute die sowieso schon aktiv in ihren Bereichen sind, der Karte annehmen w{\"u}rden und dar{\"u}ber die Strukturen der Beteiligung in M{\"u}nster im Bereich Nachhaltigkeit einfach digital sichtbar machen w{\"u}rden von allein. Das w{\"a}re dann so eine Form der Datenerhebung in einer gewissen Art und Weise. Das w{\"a}re dann so der Idealfall, wenn dann {\"u}ber diese Prozesse dann Kommunikation in Gang kommt. Das sollte dann auch m{\"o}glichst weit streuen {\"u}ber m{\"o}glichst viele Schichten und Stadtbezirke und so weiter. Das hei{\ss}t dass sich dann so einen Synergieeffekt ergibt. Realistisch gesehen wird es wahrscheinlich nicht ganz so weit gehen, denke ich. Deswegen w{\"a}re vielleicht so der erste Punkt dass das anl{\"a}uft, dass sich andere beteiligen. Das w{\"a}re schon ein erster kleiner Schritt dass es {\"u}ber den Kreis, die da sowieso schon mitmachen, hinaus bekannter wird.
    \item[I:] K{\"o}nnen Sie sich weitere Anwendungen f{\"u}r die Verkn{\"u}pfung von Geoobjekten mit Karten neben der B{\"u}rgerbeteiligung vorstellen?
    \item[P2:] Sicherlich. Da k{\"o}nnte man einfach erstmal Informationen vermitteln (\dots) Hm, es gibt sicherlich noch (\dots) Ja es ist immer die Frage wie man solche Begriffe definiert. Wie eng oder wie breit. Also ob man informieren schon dazu z{\"a}hlt oder ob man sagt, das ist was ganz anderes und so weiter. Oder auch in dem Sinne von so Web 2.0 Anwendungen. Das halt jemand was dazu beitragen kann, ob das schon ne Form von B{\"u}rgerbeteiligung ist, oder ob dazu wirklich eben nur was konkretes in der Stadt, ein Projekt oder so, z{\"a}hlt. Ich k{\"o}nnte mir vorstellen, dass das durchaus auch ne wissenschaftliche Frage ist. Also zum Beispiel was f{\"u}r Projekte sich da eintragen. Ich k{\"o}nnte mir durchaus vorstellen dass man das ganze als Datenbank verwenden k{\"o}nnte in einem wissenschaftlichen Projekt. Vielleicht f{\"u}r Schulen, k{\"o}nnte ich mir das noch ganz gut vorstellen. Dass die irgendwie vielleicht Projektwochen zu einem Thema und dann sehen "`Oh da gibts ja schon was"' und dann sich darauf st{\"u}tzen k{\"o}nnen. Noch eine andere M{\"o}glichkeit die ich gerade noch im Kopf hatte, war, dass sich politische Initiativen dadurch ein bisschen organisieren k{\"o}nnten. Ich glaube wenn man l{\"a}nger dar{\"u}ber nach denkt, k{\"o}nnte man sicherlich auch noch viel mehr Anwendungsm{\"o}glichkeiten finden. Oder auch noch ein wichtiger Punkt, den ich eben schon im Kopf hatte, dass soziale Bewegungen die Karte zur Selbstreflexion benutzen k{\"o}nnten. Dass man erstmal {\"u}berhaupt seine Wirksamkeit sieht. Und auch dass es der Bewegung noch einen Schub gibt, von wegen in Richtung Transparenz.
    \item[I:] Welche Gr{\"u}nde sprechen f{\"u}r den Einsatz dieser L{\"o}sung gegen{\"u}ber anderen L{\"o}sungen?
    \item[P2:] In Bezug auf B{\"u}rgerbeteiligung? Oder in Bezug auf was jetzt?
    \item[I:] Konkret jetzt in diesem Nachhaltigkeitskontext
    \item[P2:] Was daf{\"u}r spricht, ist erstmal dass es jetzt da ist (lacht). Das ist nat{\"u}rlich ein wichtiges Argument. Das andere was daf{\"u}r spricht, ist sicherlich einfach dass es online da ist, und relativ leicht das halt von jedem eingesehen werden kann. Also der Zugang ist einfach relativ offen dadurch dass es dann im Netz ist. Das ist sicherlich ein Plus. Was noch daf{\"u}r spricht, ist sicherlich auch die graphische Darstellung, denke ich. Es erlaubt ja das ganze erstmal so digital zu sehen. Das ist vielleicht nochmal anders, als nur ne Liste von Projekten zu haben oder so. (\dots) Was daf{\"u}r spricht, ist dass die meisten Leute inzwischen Internet benutzen, denke ich. Also dass es relativ breit gestreut ist. Was vielleicht dagegen spricht, ist dass einige Gruppen dadurch nat{\"u}rlich sich auch wieder ausgeschlossen f{\"u}hlen werden. Also, {\"a}ltere zum Beispiel. Da k{\"o}nnte ich mir vorstellen, dass die schwierigen Zugang haben. Das w{\"a}re vielleicht nochmal so ein negatives Argument dass man dagegen setzen k{\"o}nnte oder so.
    \item[I:] Gibt es denn irgendwelche Alternativen die in Betracht gezogen wurden?
    \item[P2:] Wir hatten uns mal gedacht, dass manuell quasi Landkarten ausdrucken k{\"o}nnte und die dann so auf dem Tag der Nachhaltigkeit an verschiedenen St{\"a}nden oder so platzieren k{\"o}nnte. Und dann zum Beispiel so mit Pinnadeln oder so sowas machen k{\"o}nnte. Und dass da kleine Zettel liegen, auf die die Leute draufschreiben k{\"o}nnen welche Initiative das ist. Mit den Pinnen k{\"o}nnten die dann das ganze auf der manuellen Karte machen. Und das k{\"o}nnte man dann sp{\"a}ter sogar vielleicht kombinieren, so dass man die Aspekte dann in die Karte eintr{\"a}gt oder sowas in der Art. Das waren so noch die {\"U}berlegungen die wir hatten.
    \item[I:] Was f{\"u}r Eigenschaften oder Bedingungen w{\"u}rden Sie abhalten diese L{\"o}sung einzusetzen?
    \item[P2:] (\dots) Abhalten (\dots) Wei{\ss} ich jetzt nicht so genau, vielleicht dass es sich in der Praxis sich einfach nicht als praktikabel ergibt. Dass man sieht, die M{\"o}glichkeit ist da, die Leute nutzen es aber nicht. Auch wenn man ihnen die Zug{\"a}nge eigentlich legt. Das w{\"a}re eher so im Nachhinein, dass man nachher nochmal schaut. Sonst w{\"a}re jetzt eigentlich konkret nichts was mir so einfallen w{\"u}rde.
    \item[I:] Was w{\"a}ren aus ihrer Sicht Gr{\"u}nde die B{\"u}rger davon abhalten w{\"u}rden sich nicht durch die Anwendung zu beteiligen?
    \item[P2:] Ich denke das sind insbesondere Fragen der Nutzbarkeit. Das hei{\ss}t, ob es kompliziert sich anzumelden, ob es gut bedienbar ist, oder ob das als gut bedienbar empfinden. Sage ich jetzt erstmal so die Wahrnehmung davon, ob man sich da einfach beteiligen kann, oder nicht. Fehlerfreiheit ist sicherlich ein Punkt. Ich glaube wenn Fehler auftauchen oder etwas nicht funktioniert, dann ist das sehr schnell demotivierend. Das ist vielleicht noch so ein Punkt. (\dots) Was also auch noch abschreckend wirken kann, ist nicht klar kommuniziert ist, in welchem Kontext das ganze steht, wo das herkommt, wer da verantwortlich ist, wie dann mit den Daten umgegangen wird. Also in Richtung Datensicherheit und Datenschutz. (\dots) Das w{\"a}ren jetzt so spontan die Sachen die mir einfallen w{\"u}rden.
\end{itemize}

\textbf{Teil 3 -- Abschlie{\ss}ende Fragen}
\begin{itemize}
    \item[I:] Dann jetzt noch ein paar abschlie{\ss}ende Fragen. Kennen Sie Beispiele f{\"u}r die Verkn{\"u}pfung geographischer Daten mit Diskussionsbeitr{\"a}gen?
    \item[P2:] (\dots) Von Nexthamburg, die haben da ja auch so eine Verkn{\"u}pfung von Karte und den Beitr{\"a}gen. Es gibt noch so ne Karte vom Tag des guten Lebens in K{\"o}ln. Ich wei{\ss} aber nicht ob da Diskussionsbeitr{\"a}ge bei waren, oder ob es nur eine reine Darstellung war. Das wei{\ss} ich nicht mehr so genau. Aber ich denke dass es in diesem Kontext sicherlich noch mehr Beispiele gibt, wenn man mal recherchiert. Da wei{\ss} man dann aber auch nicht wie erprobt oder ausgereift sind.
    \item[I:] Haben Sie sich dann bei einem Projekt beteiligt?
    \item[P2:] Nein.
    \item[I:] Kennen Sie Werkzeuge um interaktive mit Karten mit eigenen Inhalten zu erstellen?
    \item[P2:] Nein. Au{\ss}er jetzt dass was du da jetzt programmiert hast. 
    \item[I:] Ja okay. Dann gibt es noch Anmerkungen oder Fragen von ihrer Seite aus?
    \item[P2:] Nein, konkret eigentlich jetzt nicht.
    \item[I:] Dann vielen Dank f{\"u}r das Interview.
\end{itemize}