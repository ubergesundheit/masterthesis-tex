The interview guideline was developed following rules of Helfferich \cite{helfferich2005}. It is in german as the interviews were held in german. Participants were shown the developed application prior to the interview (\hyperref[demo]{Appendix \ref{demo}}).
\begin{adjustwidth}{-8em}{-8em}


\begin{longtable}{|p{6.45cm}|p{6.45cm}|p{6.45cm}|p{6.45cm}|}
 \hline
 \textbf{Leitfrage (Erz{\"a}hlaufforderung)}&\textbf{Check -- Wurde das erw{\"a}hnt? Memo f{\"u}r m{\"o}gliche Nachfragen -- nur stellen wenn nicht von allein angesprochen! Formulierung anpassen}&\textbf{Konkrete Fragen -- bitte an passender Stelle (auch am Ende m{\"o}glich) in dieser Formulierung stellen}&\textbf{Aufrechterhaltungs- und Steuerungsfragen}\\
 \hline

 \multicolumn{4}{|l|}{\textbf{Teil 1 -- B{\"u}rgerbeteiligung}}\\
 \hline
 
 Erz{\"a}hlen Sie mir {\"u}ber ihre Rolle und Aufgaben in B{\"u}rgerbeteiligung & Wie lange aktiv (Befragter, Projekt)\newline "`Organisator"' oder "`an der Basis"' & & Erz{\"a}hlen Sie noch mehr {\"u}ber\dots \\
 \hline
 
 
 Bitte beschreiben Sie mir die aus ihrer Sicht wichtigsten Aspekte von B{\"u}rgerbeteiligung. & Ziele \newline Nutzen & & \\
 \hline
 
 Bitte geben Sie mir eine Einf{\"u}hrung in ein(e) laufende(s)/ abgeschlossene(s) Initiative/Projekt (spontan entscheiden welches mehr "`dialogische"' Interaktion zwischen B{\"u}rgern und Aktion erfordert)& Methoden f{\"u}r B{\"u}rgerbefragung \newline Wie erfolgreich/Probleme? \newline "`Moderne"' (Social media) methoden angedacht? \newline Form von Beitr{\"a}gen die B{\"u}rger gebracht haben \newline Wie wurden die Aspekte ber{\"u}cksichtigt?
   & Welchen Wert wurde auf Dialoge zwischen den Akteuren gelegt? & Wie ist das ganze dann abgelaufen?\\
 \hline
 
 \multicolumn{4}{|l|}{\textbf{Teil 2 -- Einsatz der Anwendung}}\\
 \hline
 
 Bitte geben Sie mir eine Einf{\"u}hrung in das Projekt in dem Sie die Anwendung einsetzen wollen. & Zielgruppe (Bev{\"o}lkerungsgruppen, Geographisch) \newline redaktionelle Inhalte \newline erwartete Inhalte \newline Anreize zu Dialogen/Austausch mit B{\"u}rgern? & K{\"o}nnen Sie sich weitere Anwendungsf{\"a}lle f{\"u}r die Verkn{\"u}pfung von Texten mit Karten neben B{\"u}rgerbeteiligung vorstellen? & Erz{\"a}hlen Sie noch mehr {\"u}ber\dots \\
 \hline
 
 Welche Gr{\"u}nde sprechen f{\"u}r den Einsatz dieser L{\"o}sung gegen{\"u}ber anderen L{\"o}sungen. & Bedingungen (technisch, funktional) \newline angedachte Alternativen und deren Defizite \newline B{\"u}rgerbeteiligungsaspekte ber{\"u}cksichtigt? & Welche Eigenschaften w{\"u}rden Sie davon abhalten solch eine Anwendung einzusetzen? \newline Was k{\"o}nnte B{\"u}rger davon abhalten sich durch die Anwendung zu beteiligen? & \\
 \hline

 \multicolumn{4}{|l|}{\textbf{Teil 3 -- Abschlie{\ss}ende Fragen}}\\
 \hline
 
 Kennen Sie Beispiele f{\"u}r die Verkn{\"u}pfung geographischer Daten mit Diskussionsbeitr{\"a}gen? & Next Kassel/Hamburg \newline Frankfurt Gestalten \newline Shareabouts \newline collaborativemap.org & & \\
 \hline
 
 Haben Sie sich dort beteiligt? & In welcher Form & & Wie ist das ganze dann abgelaufen? \\
 \hline
 
 Kennen Sie Werkzeuge um interaktive Karten mit eigenen Inhalten zu erzeugen? & Google Map Maker \newline Here Map Creator \newline Wikimapia \newline Unclemap & & \\
 \hline
 
 Haben Sie schonmal ein solches Werkzeug eingesetzt? & Wie? & &\\
 \hline
 
 \end{longtable}

\end{adjustwidth}