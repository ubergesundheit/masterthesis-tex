\textbf{Teil 1 -- B{\"u}rgerbeteiligung}
\begin{itemize}
    \item[I:] Bitte erz{\"a}hlen Sie mir {\"u}ber Ihre Rolle und Aufgaben in der B{\"u}rgerbeteilung.
    \item[P7:] Ich arbeite halt am Institut f{\"u}r Soziologie als wissenschaftliche Hilfskraft seit jetzt April. Vorher als studentische Hilfskraft und habe dabei unter anderem f{\"u}r [Dozent 1] und [Dozent 2]. [Dozent 2] hat halt letztes Jahr diese Tagung organisiert "`H{\"o}her schneller weiter. Erfolgsfaktoren einer nachhaltigen Stadtentwicklung"' aus der dann auch die Initiative entsprungen ist, den Prozess zu verstetigen. Also diesen Ansto{\ss} der Tagung zu nutzen. "`Nachhaltige Stadtentwicklung, das ist irgendwie was gutes"'. Da wurden dann viele theoretisch tiefgr{\"u}ndige Analysen vorgestellt. Und die Idee war aber dann auch nat{\"u}rlich zu sagen, "`Okay, irgendwas muss daraus jetzt auch folgen"'. Wir reden da ja {\"u}ber nachhaltige Stadtentwicklung, das ist ja ein Prozess, der auch irgendwie von den wichtigen Akteuren in der Stadt weitergetragen werden muss. Was jetzt in unserer Perspektive wahrscheinlich erstmal (\dots). Also f{\"u}r die Stadt, die Zivilgesellschaft und die Universit{\"a}t, die in M{\"u}nster sehr bedeutsam ist. Jetzt aus der Organisation der Tagung heraus. Und eventuell die lokale Wirtschaft. Das sind alles sozusagen Akteure, die da relevant sind. Die man da zumindest ber{\"u}cksichtigen muss. Dabei war dann die Idee irgendeine Form von weiterer Verstetigung dieses Prozesses (\dots) oder eines Ansto{\ss}es dieses Prozesses hin zu einer nachhaltigen Stadtentwicklung in M{\"u}nster erstmal vorzunehmen, und deswegen gab es im Anschluss Treffen an denen ich auch teilgenommen habe. Einfach auch in meiner (\dots) Ja in so einer {\"u}berschnittenen Funktion einerseits sozusagen als interessierter B{\"u}rger und Student, andererseits aber auch als Angestellter des Instituts. Da haben wir dann viel vorbereitet. Wir haben die R{\"a}ume gestellt, wir haben irgendwie ein bisschen Input vorbereitet, wir haben ein bisschen so schonmal versucht, dass zu strukturieren. Und da war dann sozusagen, aus diesem Ansto{\ss} heraus, haben wir {\"u}berlegt, "`Was kann man machen um so eine Verstetigung zu erreichen lang oder mittelfristig?"'. Und dann kam halt die Idee auf, den Tag der Nachhaltigkeit zu veranstalten. Weil das etwas ist, wo sich sozusagen alle gut drauf einigen k{\"o}nnen, wo man auch die Heterogenit{\"a}t der Bewegung, oder dieser verschiedenen Akteure, irgendwie auch durchaus darstellen kann, ohne jetzt sofort sich die K{\"o}pfe einzuhauen {\"u}ber vielleicht ideologische Grenzen, die auch innerhalb dieser Gruppe sicherlich auch in dieser Form da sein werden. Also "`Wie weit soll das gehen? Soll das einfach nur (\dots) Wie soll das Wachstum (\dots) Dann m{\"u}ssen wir das Wachstum komplett einstellen. K{\"o}nnen wir das auf Staatsebene {\"u}berhaupt beeinflussen? Wie wollen wir das beeinflussen?"' und so weiter. Und dementsprechend bin ich eigentlich seit dem Anfang dann dabei und habe auch relativ regelm{\"a}{\ss}ig an den Treffen teilgenommen. Ich habe vielleicht so zwei oder drei ausgesetzt, weil ich da einfach pers{\"o}nlich keine Zeit hatte, war aber immer dar{\"u}ber informiert wie es weiter ging, weil wir halt auch die Dokumentation des Prozesses gr{\"o}{\ss}tenteils gemacht haben. Das hei{\ss}, wir haben dann auch immer die Protokolle der einzelnen Treffen verschriftlicht und hochgeladen, und so was. Auf unsere Internetseite. Wir haben da eine Internetplattform, schon ein Jahr vorher, gegr{\"u}ndet auf Initiative von [Dozent 1]. Bei dieser geht es darum, dass eine st{\"a}rkere Vernetzung zwischen Zivilgesellschaft und Wissenschaftlichkeit geschaffen werden soll. Wo wir halt schon einige Grundlagen dann auch gelegt haben. Da gibt es dann zum Beispiel Informationsmaterial, da gibt es das Transition Wiki in dem regionale Akteure schon aufgez{\"a}hlt sind, die durchaus eine Rolle spielen k{\"o}nnen. Also einzelne Gruppen, die kurz beschrieben. Aber auch Konzepte einf{\"u}hrend erl{\"a}utert. Wie "`Postwachstum"' oder "`Nachhaltigkeit - Was hei{\ss}t das eigentlich?"'.
    \item[I:] Waren Sie davor in irgendwelchen Initiativen oder Projekten t{\"a}tig?
    \item[P7:] In M{\"u}nster vorher noch nicht, weil ich leider lange Zeit nichts gefunden habe, was f{\"u}r mich wirklich gepasst hat. In K{\"o}ln war ich w{\"a}hrend meiner Studienzeit an der Uni in K{\"o}ln, ich glaube drei oder vier Jahre Mitglied bei "`Campus Gr{\"u}n"'. Das war eine studentische Hochschulgruppe der Gr{\"u}nen. Wobei die aber parteiunabh{\"a}nig agiert. Da habe ich dann sozusagen sicherlich Erfahrungen in dem Bereich sammeln k{\"o}nnen. Also da war ich dann auch im Studierendenparlament und war gleichzeitig auch im Rahmen des Bildungsstreiks aktiv. Die Erfahrungen dann im M{\"u}nster (\dots) Ich finde, das Problem ist, damit ich mich engagiere, da muss es auch wirklich passen. Es muss mir auch wirklich Spa{\ss} machen und erf{\"u}llend sein. Neben allem ideologischen Hintergrund dass das alles hat, brauch es halt auch schon irgendwie so eine Gruppe, mit der man sich gut versteht. Dann war ich da mal bei ein, zwei Gruppen, aber da habe ich mich dann einfach nicht wohl gef{\"u}hlt. Und dann bin jetzt erst wieder durch die Arbeit und das Studium wieder an das Thema gelangt.
    \item[I:] Bitte beschreiben Sie mir die aus ihrer Sicht wichtigsten Aspekte der B{\"u}rgerbeteiligung.
    \item[P7:] Also erstmal w{\"u}rde ich sagen, ist B{\"u}rgerbeteiligung sehr wichtig. Da es meiner Meinung nach, eine Demokratie nur funktionieren kann, wenn sie auch aktive B{\"u}rger hat, die sich aktiv beteiligen, einbringen und interessieren. Dann hat B{\"u}rgerbeteiligung verschiede Ebenen sozusagen. Die erste ist dann vielleicht einfach sich zu Informieren. An {\"o}ffentlichen Diskursen in irgend einer Form teilzunehmen, w{\"a}hlen zu gehen. Sonst alle formellen Rechte die man da hat zu nutzen. Und dann dar{\"u}ber hinaus gehend ist nat{\"u}rlich eine B{\"u}rgerbeteiligung die die Zivilgesellschaft st{\"a}rkt. Das hat dann einerseits eine gro{\ss}e Beteutung f{\"u}r das Funktionieren von Demokratie an sich, gleichzeitig aber auch gro{\ss}e Bedeutung, einfach als Gegenpol zu Machtinteressen die jenseits des B{\"u}rgers liegen. Also gro{\ss}e Wirtschaftsverb{\"a}nde oder {\"a}hnliches. Die sind einfach in einer sehr starken Position, auch durch Lobbyarbeit, und ich denke, dass die Zivilgesellschaft da einfach ein legitimer Gegenpol ist. Und auch eigentlich noch st{\"a}rker werden muss. Deswegen ist B{\"u}rgerbeteiligung erstmal wichtig. Jetzt f{\"u}r B{\"u}rgerbeteiligung ist das Problem, dass B{\"u}rgerbeteiligung sozial sehr selektiv ist. Das ist ein Problem von B{\"u}rgerbeteiligung. Und da muss man auch noch einiges machen. Da ist auch die Aufgabe, meiner Meinung nach des Staates oder auch der lokalen Stellen vor Ort, vielleicht auch durch die Implementierung von einzelnen Stellen. Das reicht h{\"a}ufig schon. Das f{\"a}nde ich schonmal einen guten Schritt, wenn die Stadt einen B{\"u}rgerbeteiligungsbeauftragten h{\"a}tte, der ganz klar irgendwie im Internet zu finden ist, und an den sich B{\"u}rger wenden k{\"o}nnen mit ihren Problemen. Und der sich wirklich nur darum k{\"u}mmert, B{\"u}rgerbeteiligung vorran zu treiben oder irgendwie publik zu machen, oder die M{\"o}glichkeiten zu erl{\"a}utern. Die zu bewerben und {\"a}hnliches. Gerade auch in Stadtteilen oder in sozialen Schichten, die normalerweise nicht so nah der B{\"u}rgerbeteiligung sind. F{\"a}nde ich schonmal einen wichtigen Schritt. Das ist halt ein gro{\ss}es Problem von B{\"u}rgerbeteiligung. Sie ist halt sehr sozial selektiv und behandelt dadurch nat{\"u}rlich auch eher die Probleme von den vielleicht gebildeteren und besser situierten Menschen in der Gesellschaft, als von Menschen, die keinen Zugang zu diesen B{\"u}rgerbeteiligungsforen haben. Oder die didaktischen F{\"a}higkeiten die man da braucht um sich zu integrieren. 
\end{itemize}

\textbf{Teil 2 -- Einsatz der Anwendung}
\begin{itemize}
    \item[I:] Kommen wir zu Fragen zum Einsatz der Anwendung. Bitte geben Sie mir eine Einf{\"u}hrung in das Projekt in dem die Anwendung eingesetzt werden soll.
    \item[P7:] Also eigentlich ist die Karte ja unabh{\"a}ngig vom Nachhaltigkeitstag entstanden. Also die Karte war erstmal ein Projekt zwischen der Geoinformatik und der Soziologie. Auch im Anschluss an die Tagung allerdings. Die Idee war eigentlich, durch moderne Medien, zu kartographieren was es in M{\"u}nster so an Initiativen und Projekten {\"u}berhaupt gibt. Wir hatten halt das Gef{\"u}hl, dass h{\"a}ufig doch mehr existiert als man so mitkriegt. Es wird viel parallel gearbeitet. Und da erstmal irgendwie Synergieeffekte zu verst{\"a}rken, dadurch dass die Leute erstmal voneinander wissen. Da glaube ich, hat die Karte sehr viel potential. Weil es auch eine sehr direkte und sehr haptische Darstellung davon ist, was existiert. Da kann man direkt sch{\"o}n auf dieser Karte sehen "`Ah da ist was, und da ist was"'. Und da entstand erstmal diese Idee {\"u}berhaupt so eine Karte zu machen. Und dann kam ja vor allem von euch der Input, dass euch aber auch daran gelegen ist, mit der Karte mehr zu machen als nur zu informieren. Wir hatten das erstmal gar nicht auf dem Schirm dass man da {\"u}berhaupt machen k{\"o}nnte, so auf einer Karte die Leute diskutieren zu lassen. Wir hatten ja erstmal nur an eine recht statische Karte gedacht. Ja und dann haben wir das ja zusammen weiterentwickelt, diesen Gedanken. Auch dann durch den Input von euch, und dann dem Input von uns. Und halt immer so ein bisschen hin und her {\"u}berlegt. Dass man ja auch Beispielsweise die M{\"o}glichkeit wie in Hamburg zu nutzen. Bei Nexthamburg. Sozusagen, dass B{\"u}rger aktiv die M{\"o}glichkeit bekommen, auch an dieser Karte direkt teilzunehmen. Nicht nur zu konsumieren, sondern auch zu partizipieren. Und zum einen vielleicht als Diskussionsforum, um erstmal Stellen aufzuzeigen die sehr wenig nachhaltig oder sehr stark nachhaltig sind. Also gute und negative Beispiele die man dann diskutieren kann. Und dann auch {\"u}ber diesen Weg an die Stadt (\dots) das muss man sich vielleicht nochmal {\"u}berlegen. Im Hamburg wurde dass dann so gemacht, da wurden aus den tausenden Beitr{\"a}gen von Experten 250 Vorschl{\"a}ge ausgew{\"a}hlt und dann der Stadt vorgestellt. Und daraus wurden dann einige wiederrum dann umgesetzt. Wie dass dann halt so ist in einem b{\"u}rokratischen Ablauf. Erstmal so dass die B{\"u}rger die M{\"o}glichkeit haben aktiv sich zu beteiligen. Und dann auch direkt {\"u}ber eine Diskussionsfunktion auf der Karte direkt zu diskutieren. Und vielleicht auch schon Argumente auszutauschen. Wieso dass jetzt vielleicht gut oder schlecht ist. Vielleicht wenn jemand jetzt vorschlagen w{\"u}rde den ganzen inneren Ring in M{\"u}nster f{\"u}r private Autos zu sperren, da k{\"o}nnte man dann auch dar{\"u}ber diskutieren. Da gibts dann ja auch Kontra-Argumente. Also w{\"a}re es ja sch{\"o}n, wenn es einer auf der Karte vorschl{\"a}gt, und dass es dann Kritik gibt und aber auch Unterst{\"u}tzer sich zu Wort melden. Daraus kann sich ja dann auch weitere Erkenntnisgewinn entwickeln. (\dots) Achso, und was mir gerade noch einf{\"a}llt. Auf den konkreten Tag bezogen, da ist die Karte nat{\"u}rlich auch sehr sch{\"o}n, um Aktionen die nur an dem Tag stattfinden hervorzuheben. Vielleicht dass man da noch eine weitere Farbe einf{\"u}gt f{\"u}r Aktionen an dem Tag. Weil die Idee ist ja auch an dem Tag, dass dieser nicht nur zentral organisiert wird, sondern auch da ist die Idee, dass die B{\"u}rger direkt partizipieren und selbst Aktionen anbieten. Oft l{\"a}uft das ja {\"u}ber so eine nachbarschaftliche Verbindung, dass da sich Beispielsweise in einem Hinterhof irgendwer etwas {\"u}berlegt hat. Und das sollen die dann da auch in die Karte mit eintragen. Das w{\"a}re dann ja eine M{\"o}glichkeit f{\"u}r die B{\"u}rger diesen Tag, oder den Ablauf des Tages, dezentral zu planen. Das ist auch so eine Sache, die ich mir da f{\"u}r den konkreten Tag verspreche. Das f{\"a}nde ich auch eine gute Funktion.
    \item[I:] Welche Gr{\"u}nde sprechen f{\"u}r den Einsatz dieser L{\"o}sung gegen{\"u}ber anderen L{\"o}sungen?
    \item[P7:] Ich glaube so ganz platt erstmal, finde ich dass eine Karte einfach eine viel sch{\"o}nerer und angenehmerer Zugang ist, als eine Liste. Also wenn ich jetzt eine Liste mit Initiativen habe, dann habe ich eine ellenlange Liste und kann mir die alle durchlesen. So sehe ich nat{\"u}rlich viel sch{\"o}ner so direkt Konzentrationen. Wo ist denn vielleicht viel oder wo fehlt denn noch was? Also wenn ich mich in meiner Freizeit hinsetze, um mich f{\"u}r solche Dinge zu engagieren, dann ist durch diese Karte dieser Zugang niedrigschwellig. Die Schwelle des Zugangs ist verringert. Sowohl auf individueller Ebene, also dass die Benutzung ziemlich einfach ist (\dots) Wobei wir da nat{\"u}rlich dann die genaueren Ergebnisse der Evaluation abwarten m{\"u}ssen. Also vielleicht denken wir auch nur das ist einfach. Aber naja. Also niedrigschwellig in dem Sinne, dass es relativ wenig Zeit kostet, sich erst einmal da in so eine Diskussion einzuklinken. Und vielleicht kommt man dar{\"u}ber auch tiefer in das Thema rein und kriegt vielleicht da auch dann den Schub, sich weiter mit solchen Aspekten zu besch{\"a}ftigen. Oder vielleicht mal ein eine Gruppe zu gehen, oder vielleicht an einer Sitzung teilzunehmen.
    \item[I:] Welche Eigenschaften w{\"u}rden Sie davon abhalten diese L{\"o}sung einzusetzen?
    \item[P7:] Also da ich ja an der Entwicklung der Karte direkt mitgearbeitet habe, und eigentlich auch sehr zufrieden bin, wie das gelaufen ist, w{\"u}rde ich sagen mich h{\"a}lt erstmal nichts ab. Grunds{\"a}tzlich ist bei so einer Karte immer das Problem, dass es doch viele Kleinigkeiten geben kann, die einem dann den Spa{\ss} an der Karte nehmen k{\"o}nnen. Also einfach eine schlechte Anwendbarkeit in verschiedenen Bereichen. Dass das nicht gut aussieht, dass das nicht gut lesbar ist, dass das chaotisch strukturiert ist, dass man nichts findet. Dass vielleicht die Anmeldung zu kompliziert ist. Wobei ich da auch ein Freund von E-Mailanmeldung bin. Da gibts zwar auch datenrechtlich Einw{\"a}nde, aber da kann sich ja jeder neue E-Mailadressen machen mit nem Witznamen und sich dar{\"u}ber dann anmelden. Ich hab halt die Erfahrung gemacht mit anderen Online-M{\"o}glichkeiten, dass es einfach schon gut ist, wenn man diese Schwelle setzt. Weil darunter wir dann einfach auch viel Mist geschrieben. Das ist dann auch f{\"u}r so eine innere Schwelle. Die h{\"a}lt dann nicht nur Bots auf, sondern auch die Leute, dass da kein Unsinn geschrieben, oder Beleidigungen geschrieben wird. Dass die Leute da einfach ein bisschen ernster an die Sache gehen.
    \item[I:] Und welche Eigenschaften w{\"u}rden B{\"u}rger davon abhalten, sich zu beteiligen?
    \item[P7:] Nat{\"u}rlich wenn das Thema nicht weit genug bekannt ist. Wenn vielleicht die Karte nicht weit genug bekannt ist. Auch bei solchen Versuchen ist es wichtig, dass es eine breite Partizipation gibt. Die stehen und fallen halt so ein bisschen mit der Beteiligung. Bei der Karte ist das Risiko vielleicht ein bisschen geringer, da man die ja auch nur angucken kann und dann ist schon ein Teil des Sinns gegeben. Die Karte ist ja auch so gestaltet, dass sie auch ohne die Partizipation funktioniert. Das ist ja bei so einem Forum anders. Aber es könnte ja auch sein, dass es dann auf lange Sicht gesehen dann doch nicht mit der Karte funktioniert. Da fehlen dann die neuen Informationen. Man brauch halt in der B{\"u}rgerbeteiligung auch B{\"u}rger, die sich beteiligen. Ist dann aber {\"u}berall so.
    \item[I:] K{\"o}nnen Sie sich weiter Anwendungsf{\"a}lle neben der B{\"u}rgerbeteiligung f{\"u}r die Anwendung vorstellen?
    \item[P7:] Naja. Dieses Nachhaltigkeitsthema ist ja erstmal ein sehr spezielles Thema. Man kann sicherlich diese Karte dann auch f{\"u}r andere Aktionen umr{\"u}sten. Also wo man erstmal informiert. Im Sinne von Open Data. Wo man dann erstmal alles drauf packt, was so an Daten von der Regierung oder den L{\"a}ndern aufgenommen werden und verf{\"u}gbar sind. Da w{\"a}re das dann viel einfacher zug{\"a}nglich. Also es gibt ja auch so diese Anwendungen, wo die S{\"u}ddeutsche gerne mit arbeitet. Dass Daten einfach auf einer Karte visualisiert werden. Durch so eine Karte werden ja r{\"a}umliche Zusammenh{\"a}nge sehr schnell und einfach sichtbar. Das k{\"o}nnte ich mir aber auch f{\"u}r vieles vorstellen. Sei es die M{\"u}llabfuhr oder die Gewalt, oder so. K{\"o}nnte man ja erstmal einfach darstellen um dann vielleicht auf Zusammenh{\"a}nge zu kommen. Das w{\"a}re dann auch was, was die B{\"u}rger ganz konkret h{\"a}tten um zu ihren Volksvertretern zu gehen und zu sagen "`Hier, da gibts die Probleme"'. Sicherlich gibt es da sehr breite Anwendungsm{\"o}glichkeiten. Es darf aber auch dann nicht {\"u}berladen sein. Sonst k{\"o}nnen normale B{\"u}rger auch nichts damit anfangen. Das ist auch wieder so eine Gefahr.
\end{itemize}

\textbf{Teil 3 -- Abschlie{\ss}ende Fragen}
\begin{itemize}
    \item[I:] Dann noch ein paar abschlie{\ss}ende Fragen. Kennen Sie noch andere Beispiele oder Systeme, in denen Diskussionsbeitr{\"a}ge mit Geodaten verkn{\"u}pft worden sind?
    \item[P7:] Also mal abgesehen von Geocaching, habe ich neulich geh{\"o}rt, es gibt da noch weitergehende Spiele. Bei denen muss man da an bestimmten Punkten irgendwie so sich melden und dann hat man das Gebiet eingenommen und dann gehts weiter. So mit mehreren Fraktionen und so. Ansonsten mit Diskussionsbeitr{\"a}gen. Da f{\"a}llt mir nichts ein so jetzt. Ah ja doch Nexthamburg. Das gibts sicherlich auch in anderen St{\"a}dten.
    \item[I:] Haben Sie sich dann bei irgendeiner Sache wahrscheinlich auch nicht beteiligt?
    \item[P7:] Nein. Also ich gucke mir gerne Karten an, aber beteiligt noch nicht.
    \item[I:] Kennen Sie Werkzeuge um interaktive Karten mit eigenen Inhalten zu erzeugen?
    \item[P7:] Nein
    \item[I:] Okay, gibt es dann noch Fragen oder Anmerkungen von Ihrer Seite? 
    \item[P7:] Nein, au{\ss}er dass ich die Karte erstmal super finde. Da ist echt was draus geworden. Das h{\"a}tte ich so nicht gedacht. Ich war zwar ganz optimistisch, aber ist echt sch{\"o}n geworden. Vielen Dank daf{\"u}r.
    \item[I:] Dann sage ich auch vielen Dank!
\end{itemize}