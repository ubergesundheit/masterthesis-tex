\textbf{Teil 1 -- B{\"u}rgerbeteiligung}
\begin{itemize}
    \item[I:] Erz{\"a}hlen Sie mir {\"u}ber Ihre Rolle und Aufgaben in der B{\"u}rgerbeteiligung
    \item[P3:] Ja. Also, ich bin in verschiedenen B{\"u}rgerbeteiligungen und Initiativen aktiv. Also, das ist einmal Transition Town. Da bin ich seit fast vier Jahren aktiv im Bereich der Gemeinschaftsg{\"a}rten, der Kerngruppe, Filme im Cinema, (\dots) und alles das was die Sichtbarkeit von Transition Town betrifft. Barackentage, also Initiativen, so, dann bin ich da dann dabei. Und moderiere von Transition Town aus jetzt auch f{\"u}r die Soziologen bei dem Tag der Nachhaltigkeit. So, das ist mein Job da. Dann bin ich neu dazu gekommen bei der B{\"u}rgerstiftung M{\"u}nster. Da bin ich noch nicht Mitglied, aber bin mit der Leitung da schon verbandelt. Wo wir uns auch vernetzen und da gibts ein Projekt, was jetzt im Herbst starten soll, das nennt sich "`Essen retten"'. Das l{\"a}uft am Schiller-Gymnasium in M{\"u}nster, wo es darum geht, dass Obst was so auf den B{\"a}umen in der Innenstadt w{\"a}chst, speziell im Kreuzviertel, eingesammelt wird und Saft draus gemacht wird. Die Sch{\"u}ler, zur Zeit sind das schon zwanzig Sch{\"u}ler, {\"u}ber ich glaube sechs Jahrgangsstufen. Also geht richtig von ganz klein bis ganz gro{\ss}. Finde ich auch toll, dass da auch generations{\"u}bergreifendes Lernen stattfindet innerhalb einer Schule. Die Kleinen mit den Gro{\ss}en. Die lernen im Prinzip, wie man so ein kleines Wirtschaftsunternehmen aufbaut. So als Projekt. Und lernen aber auch Themen (\dots) wie kommuniziert man dann richtig. Das ist ja so ein Steckenpferd von mir, also wie "`was geh{\"o}rt zum gut gelebten Leben"'. Und da geh{\"o}rt also auch Kommunikation dazu. Wertsch{\"a}tzende Kommunikation auf Augenh{\"o}he. Das ist so ein ganz wichtiges Thema. Das lernen die dabei. Die lernen zu planen, die lernen einen Businessplan zu machen, die lernen herum zu gehen und die Leute und B{\"u}rger zu begeistern daf{\"u}r, ihren Apfel abzugeben. Die m{\"u}ssen sich um eine Presse k{\"u}mmern, die m{\"u}ssen das steril abf{\"u}llen. Und letztendlich m{\"u}ssen sie es auf dem Markt verscherbeln. So, dabei lernen die, wie man miteinander umgeht, die lernen wie eine Firma funktioniert und was nicht funktioniert. Also Druck erzeugt Gegendruck und alle diese ganzen sch{\"o}nen Sachen. Und die lernen auch, wie viel Arbeit in einem Glas Apfelsaft drin steckt. Das ist das zum meinem Engagement zur B{\"u}rgerstiftung. Dann das Kulturquartier. Da geht es darum, ganz viele Facetten miteinander zu kombinieren. Weil Kultur ist nicht nur das, was im Theater l{\"a}uft. Also findet nicht auf der B{\"u}hne statt. Kultur ist auch, wie wir beide miteinander umgehen. Kultur ist, wie ich jemandem auf der Stra{\ss}e begegne. Kultur ist, wie ich mein Haus gestalte, damit das nachhaltiger ist, damit auch folgende Generationen noch was (\dots). Also ist so "`Wie wollen Menschen miteinander leben"' und das an einem Ort, wo man nicht so viel denkt, sondern mehr tut. Mehr ausprobiert. Das ist dieses Kulturquartier was gerade neu entsteht. So das sind so drei Dinge, mit denen ich mich hier in M{\"u}nster besch{\"a}ftige. Da werden aber noch andere Sachen dazukommen. Also, eine Sache wird noch dazukommen, dass ich f{\"u}r M{\"u}nsteraner B{\"u}rger Themen anbiete, wie zum Beispiel "`Wie funktioniert intrinsisch motivierte Arbeit"'. Das geht so ein bisschen dass das Thema Burnout ein bisschen der Vergangenheit angeh{\"o}rt. Das Thema "`Happy working people"' und "`Sinnerf{\"u}lltes Arbeiten -- Wie geht das in M{\"u}nster"'. Das sind so die vier Themen.
    \item[I:] Dann k{\"o}nnten Sir mir bitte die aus ihrer Sicht wichtigsten Aspekte von B{\"u}rgerbeteiligung nennen?
    \item[P3:] Also ich glaube (\dots) dass Leben zutiefst menschlich ist, dass immer Menschen mit Menschen f{\"u}r Menschen etwas arbeiten. Das, was du jetzt hier machst, ist auch f{\"u}r Menschen, die sich treffen. Und B{\"u}rger sind einfach Menschen. Ich glaube nicht, dass es irgendwo einen Schlauen gibt, ganz oben, der wei{\ss} wie es geht. Politiker oder sowas. Sondern, dass vielmehr es in bestimmten Situationen sinnvoll sein kann, dass einer mal sagt wo es langgeht. Auf einem Schiff zum Beispiel. "`Jetzt m{\"u}ssen wir das Segel mal nach links oder rechts r{\"u}berholen, sonst kentern wir."' Dann sollte man das tun, und nicht anfangen zu diskutieren. Aber, im menschlichen Zusammenleben in so ner Stadt wie M{\"u}nster, finde ich es einfach toll, wenn Freir{\"a}ume von den B{\"u}rgern gestaltet werden k{\"o}nnen, so wie sie es gerne m{\"o}chten. Und dass nicht jemand anders wei{\ss} was f{\"u}r uns gut ist. Sondern die B{\"u}rger wissen schon selbst was f{\"u}r sie gut ist. Und zu einem m{\"u}ndigen B{\"u}rger geh{\"o}rt, dass er sich ausdr{\"u}cken kann. Deshalb finde ich das wichtig, dass es sowas gibt.
    \item[I:] Bitte geben Sie mir eine kurze Einf{\"u}hrung in ein laufendes oder abgeschlossenes Projekt oder Initiative, bei dem Sie denken, dass es da besonders auf die Kommunikation angekommen ist.
    \item[P3:] Ja, da gibts so viele.
    \item[I:] Dann von ihrem Liebsten.
    \item[P3:] Ja mein Liebstes. Nehmen wir mal das Kulturquartier. Das Kulturquartier hat sehr viele Akteure. Es sind schon allein acht Gesellschafter, dann gibt es jede Menge Mieter, dann gibt es die Politik, die Wirtschaftsf{\"o}rderung, dann gibt es Bankengeldgeberstiftungen, die uns unterst{\"u}tzen wollen. Dann gibts ein Programm "`1000x100"' wo uns tausend B{\"u}rger mit jeweils einhundert Euro im Jahr unterst{\"u}tzen, damit das passiert. Also das ist so eine spezielle Art von Crowdfunding. Sodass im Prinzip, die vielen, wenn sie gut informiert sind, und in einer guten Interaktion sind, und sehen was da gerade passiert an dem Ort und wie es weitergeht, dass sie mal in Dialog treten k{\"o}nnen. Ideen mit einbringen k{\"o}nnen ohne Anspruch dadrauf, dass es auch passiert, weil das entscheidet letztendlich so ein Gremium innerhalb des Kulturquartieres. Sonst ist es ein Debattierclub bis ans Ende der Zeit. Aber da k{\"o}nnten sich viele beteiligen, Ideen reinbringen, damit die Leute, die dann letztendlich entscheiden, basierend auf diesem tollen, vielen Ideen, ganz inspiriert sagen: "`Hey, das machen wir jetzt, da w{\"a}ren wir selbst nie drauf gekommen. Das w{\"a}re ein riesen Vorteil."' Und andere Akteure wie eine Bank oder eine Stiftung sieht, "`das ist total lebendig. Da wird diskutiert, und da wird auch nicht jeder Schrott genommen, sondern da werden gute Ideen auch wirklich aufgenommen, und es entwickelt einen Speed, eine Geschwindigkeit in eine Richtung zum besser werden, besser leben. Das ist ja der Hammer"'
    \item[I:] Wie l{\"a}uft dann dort meistens die Kommunikation ab?
    \item[P3:] Seit neustem haben wir ein tolles Tool, das hei{\ss}t Trello. (lacht) Da machen wir relativ viel jetzt mit seit einer Woche. Und vorher lief es einfach so, es gibt zu bestimmten Terminen sogennante Tischgespr{\"a}che. Da setzen wir uns hin, da gibts was zu futtern, aber nichts gro{\ss}es. So was zu knabbern oder was zum dippen. Und dann treffen sich die Leute, die in der Planung sind. Es gibt mittlerweile Teams zum Thema Bau oder Stiftungsgelder oder Gr{\"u}ndung oder wei{\ss} der Kuckuck was. Die dann bei den Tischgespr{\"a}chen die anderen informieren. Dadurch ist nat{\"u}rlich durchaus eine gewisse Zeitverz{\"o}gerung bei der Information dabei. Und wenn die anderen Bescheid wissen, was jetzt gerade abgeht (\dots) Vor allem, wenn die wissen was neu ist, das ist immer ganz wichtig. Das gibts auch bei Trello so. Ich wei{\ss} nicht ob es jetzt bei der Applikation drin ist, also wirklich "`Hey, was ist neu"'. Dass die Alarmglocke bimmelt. Bei den Themen, dass ich vielleicht sogar die Themen tagge. (\dots) Bei Trello ist das so, dass ich sage: "`Das sind meine Boards"' und "`Was ist an den Boards neu"'. Dass ich da wei{\ss}, was so neu ist. Ich muss nicht alles wissen, aber dass ich da bei denen ich subscribed bin, dass ich wei{\ss} was los ist. Ja bisher war es sehr viel pers{\"o}nlicher. Jedes Treffen, telefonieren, E-Mail. Und das geht gerade in so ein co-kreatives, kooperatives, IT-unterst{\"u}tztes Umfeld rein. Und ich k{\"o}nnte mir vorstellen, dass	das gut ist.
    \item[I:] Und die Beitr{\"a}ge sind dann welcher Natur?
    \item[P3:] Es ist viel Information. Dann gibts Ideen. Aus den Ideen, da wird eine Menge verworfen. Es gibt eigentlich einen riesigen Ideenparkplatz. Also es gibt auch in den Teams auch immer Leute, die sprudeln vor Ideen, die setzen aber nichts um. Die m{\"u}ssen Platz haben wo sie es loswerden k{\"o}nnen. Es muss Platz geben, wo man Ideen aufgreifen kann. "`Tackatacka, die nehm ich"' Und dann gibts die Umsetzer, die das umsetzen, die vielleicht nicht so kreativ sind. Das hei{\ss}t, man brauch Informationen, wie man es umsetzen kann. Man braucht vielleicht Verlinkungen zu anderen die es vielleicht auch schon gemacht haben. Und es sollte auch dokumentiert werden, welche Entscheidungen jetzt getroffen worden sind. Also am Ende einer Diskussion oder so, dass man sagt, "`Ja, vielen Dank. Ich schlie{\ss} jetzt diesen Track und die Entscheidung ist so"'. Man kann einen neuen aufmachen, aber der ist jetzt erstmal abgeschlossen. Dass man einen Zyklus abschlie{\ss}t. Das finde ich auch immer ganz wichtig. Das machen wir dann auch.
\end{itemize}

\textbf{Teil 2 -- Einsatz der Anwendung}
\begin{itemize}
    \item[I:] Dann geht es jetzt weiter mit Fragen konkret zur Anwendung. Bitte geben Sie mir eine Einf{\"u}hrung in das Projekt in dem Sie die Anwendung einsetzen wollen.
    \item[P3:] Ich kann mir einfach vorstellen, wenn es um den Nachhaltigkeitstag geht, dass man die Anwendung sehr sch{\"o}n f{\"u}r die Vorbereitung nutzen kann. Dass wir, wenn wir die verschiedenen Stationen haben wo in M{\"u}nster was stattfindet, dass man sagt, "`Hey, da und da und da"' und dass man dann rund um diesen Platz sich das vielleicht auch noch ein bisschen auff{\"a}chern kann, "`Da ist die Hauptaktion, da mache ich eben den Stand"'. Ich kann das ja ganz ganz kleinteilig machen. Ich wei{\ss} nicht, geht das bis auf f{\"u}nf Meter oder so? (Ja ganz klein) Ja also dass ich sage, ich baue da wirklich den Stand hin. Ich kann ja praktisch m{\"o}blieren. Und das k{\"o}nnte ich mir auch vorstellen, also wenn wir wirklich ein Projekt aufmachen kann, so wie es bei Trello mit den Firmen geht. Ich mach ein Projekt auf. Dann kann ich wirklich, wenn ich ein Sommerfest habe im Kindergarten, oder ich mach da eine Station am Tag der Nachhaltigkeit, kann ich wirklich super geilomat planen, was wo sein soll. Wo kommt die Leinwand hin, wo kommt der Beamer hin. Also ich kann ganz ganz kleinteilig (\dots) Und jeder wei{\ss} genau wo es hinkommt. Und dann w{\"a}re es nat{\"u}rlich schon, wenn man dadr{\"u}ber diskutiert. Und dann f{\"a}nde ich es gut, wenn man auch die M{\"o}glichkeit h{\"a}tte, wenn es so Aufgaben gibt, oder sowas. Oder man m{\"u}sste ein anderes Tool nehmen. Dass man sagt "`Hey, die Aufgabe ist jetzt erledigt"'. Also da soll der Beamer hin. Der ist auf einmal gr{\"u}n. Erledigt. Da k{\"o}nnte ich das sch{\"o}n best{\"u}cken und dann sehe ich auf einen Blick auf der Landkarte wenn ich reinzoome, da ist alles gr{\"u}n aber der Teil, der ist noch irgendwie gelb oder orange. Orange weil es jetzt bald zu tun ist. Also irgendwie in einer Woche, aber da ist immer noch nichts passiert. Und dann kann jeder sehen. Und das w{\"u}rde auch so eine Gruppe unterst{\"u}tzen. Auch den der verantwortlich ist, der muss da zwar draufgucken, aber auch ein anderer sieht das vielleicht mal und sagt "`Hey ich bring jetzt noch einen Beamer mit von zuhause. Ich hab das gesehen, da ist keiner. Zumindest nicht eingetragen."' Also wir haben dann einen {\"U}berblick, {\"u}ber all das was los ist an den verschiedenen Standorten. B{\"u}rger, die man einl{\"a}dt, ich wei{\ss} nicht ob das auch der Gedanke ist, die wissen auch was da passiert. Man {\"o}ffnet das ja dann f{\"u}r die B{\"u}rger. Und da ist dann die Frage (\dots) Die k{\"o}nnen ja mitdiskutieren. Ich wei{\ss} jetzt nicht wie es da mit den Rechten ist. Also mitdiskutieren ja, aber ob die dann auch Objekte verschieben d{\"u}rfen, weil dann sonst schmei{\ss}en die uns den ganzen Raum wieder durcheinander. Das w{\"a}re voll grottenschlecht. So k{\"o}nnte ich mir das aber vorstellen. Wenn es keine graphischen Objekte gibt, dass man eben sagt an dem Ort, da muss das und das und das passieren. Und dann eben textlich.
    \item[I:] Was f{\"u}r Anreize f{\"u}r B{\"u}rger sich mit der Anwendung dialogisch auszutauschen k{\"o}nnten Sie sich vorstellen?
    \item[P3:] Es gibt B{\"u}rger, die wollen in einen Dialog gehen und das ist sch{\"o}n, wenn die sehen bei mir in der Nachbarschaft passiert was, oder das Thema interessiert mich. Also einmal A: es betrifft mich weil es hier um die Ecke ist und es betrifft mich, ich fahr auch f{\"u}nf Kilometer daf{\"u}r weil es mich interessiert. Und dann m{\"o}chte ich da einmal vielleicht genau wissen was ist da los, das f{\"a}nde ich spannend. Und dann hab ich vielleicht noch eine Idee. Also sieht er hier da vorne an dem Ort, da k{\"o}nnte ich euch noch meinen Starkstromanschluss anbieten. Der ist hier in meiner Garage. Dann k{\"o}nnt ihr mehr machen, oder irgendsowas. Das hei{\ss}t, ich kann meine eigenen Ideen oder Unterst{\"u}tzungsleistungen besser anbieten. Und ich werde besser informiert, {\"u}ber das was da ist. Und diskutieren, wenn ich diskutieren will. Aber ich glaube, in erster Instanz f{\"a}nde ich das Informieren wichtig. Was geht denn da ab. F{\"u}r mich jetzt. Aber andere sind da vielleicht anders gestrickt.
    \item[I:] K{\"o}nnen Sie sich weitere Anwendungsf{\"a}lle f{\"u}r die Verkn{\"u}pfung von Texten mit Karten neben der B{\"u}rgerbeteiligung vorstellen?
    \item[P3:] Also zum Beispiel wenn man da so gut reinzoomen kann, kann ich mir das f{\"u}r jede Art von Gro{\ss}veranstaltung vorstellen. Als Planungstool. Dass man sieht, was geht hier jetzt gerade ab. Ich kann es mir sogar vorstellen f{\"u}r Institutionen wie zum Beispiel die Polizei. Wenn Gro{\ss}demos sind oder wenn, was wei{\ss} ich, eine Riesenveranstaltung ist wie M{\"u}nster gegen Bayern. Bayern gegen Preu{\ss}en. Also das l{\"a}uft ja, aber wenn so Neuland ist, oder ein gigantisches Konzert, f{\"u}r Sicherheitskr{\"a}fte, f{\"u}r Gro{\ss}veranstaltungen wie Rock am Ring, keine Ahnung. Also gro{\ss}e Veranstaltungen, wo das Gel{\"a}nde weitr{\"a}umig ist, k{\"o}nnte ich mir vorstellen. Dann gibt es sehr gro{\ss}e Unternehmen. Zum Beispiel eine BASF, ein Flughafen. Also Unternehmen, die {\"u}ber sehr weitr{\"a}umiges Gel{\"a}nde verf{\"u}gen und mal schnell irgendwie was auch lokal ver{\"a}ndern wollen, und sagen "`Hey, an der Stelle, da ist dies und jenes"' Wo man sein Ideenmanagement vielleicht auch zusammen mit der Lokalisierung macht. Das ist ja Kartenmaterial, vielleicht noch eine Frage: Ist das von Google? (Nein das ist von Openstreetmap.) Okay, das hei{\ss}t also weltweit? (Ja) Das hei{\ss}t, ich kann auch mit Unternehmen arbeiten, die weltweit operieren (Richtig.) Genau. Ja das w{\"a}re auch gut. Wenn man dann Ideen hat, "`Hey, ich arbeite hier in M{\"u}nster, aber in der amerikanischen Niederlassung. Ich komm gerade wieder. Da habe ich gesehen, in der Einheit, da funktioniert irgendwas nicht. Und ich mache da irgendwie ein Todo-Marker rein an die Stelle"' Und dann guckt irgendwie so ein Global-Todo-Manager drauf und sagt "`Hey, wo sind denn so Open Issues"' und sieht die. An einer Stelle, wo irgendwas ist, was getan werden muss. Das finde ich gut.
    \item[I:] Welche Gr{\"u}nde sprechen f{\"u}r den Einsatz dieser L{\"o}sung gegen{\"u}ber anderen, angedachten L{\"o}sungen?
    \item[P3:] Nein, mit Kartenmaterial, hab ich jetzt keine Idee. Mit Kartendialog, habe ich mir jetzt noch keine angeschaut. Ich h{\"a}tte ganz andere Projektmanagementtools eingesetzt. Ganz klassische. Aber die sind nicht verkn{\"u}pft mit der Karte.
    \item[I:] Welche Eigenschaften w{\"u}rden Sie davon abhalten, diese Anwendung einzusetzen?
    \item[P3:] (\dots) Was mich zur{\"u}ckhalten w{\"u}rde, w{\"a}re, wenn Daten auf einmal weg sind. F{\"a}nde ich irgendwie uncool. Da hat man sich dann viel M{\"u}he gegeben irgendwas einzupflegen. Auf einmal, ist das dann weg. Ich find das Thema Backup sehr wichtig. In Zusammenhang mit der Rechteverwaltung. Wenn jeder alles machen darf, und {\"u}berall drin rum schreiben darf, das f{\"a}nde ich nicht gut. Also so wie es hier auch schon gel{\"o}st ist, also dass jeder seine eigene ID hat f{\"u}r die Antworten. Dass es eigene Eintr{\"a}ge sind finde ich gut. Was w{\"u}rde mich noch abhalten? Nein, ich w{\"u}rde es einfach so nutzen.
    \item[I:] Was k{\"o}nnten Gr{\"u}nde f{\"u}r B{\"u}rger sein, sich nicht zu beteiligen?
    \item[P3:] Es k{\"o}nnte sein, dass er Einstieg zu schwierig ist. Also ich hab jetzt bei Trello festgestellt, dieses Thema anmelden bei Trello, halte ich pers{\"o}nlich f{\"u}r total easy, wenn man das erste mal dabei ist. Aber wenn man mal auf die Webseite guckt. Also bei dem achter Team waren f{\"u}nf Leute dabei, die haben es nicht gerafft. Woran lag das? Es gab oben einen dicken Button Login. Okay Loginname und Passwort oder "`Login with Google"'. Die haben immer drauf geklickt sind immer auf Fehler gekommen. Und unten drunter steht ganz klein unterstrichen halt auch als Link "`If you don't have an account, please create here"' oder so {\"a}hnlich. So {\"a}hnlich wie Allgemeine Gesch{\"a}ftsbedingungen. Aber das haben die nicht gefunden. Das war nervig. Das habe ich denen geschrieben, hab ich einen Screenshot gemacht. Das war voll bl{\"o}d, weil die das immer noch nicht gerafft hatten. Wir mussten uns wirklich treffen, oder am Telefon erkl{\"a}ren "`Ach da, da ist das ja, okay"'. Also die Schwelle sich da anzumelden, deutlich zu machen wo man draufklicken muss, wenn man noch nicht registriert ist. Das finde ich wichtig dass man {\"u}berhaupt rein kommt. Und dann finde ich sehr gut, wenn es eine einfache Erkl{\"a}rung gibt. Zum Beispiel das mit den Videos finde ich super.
\end{itemize}

\textbf{Teil 3 -- Abschlie{\ss}ende Fragen}
\begin{itemize}
    \item[I:] Kennen Sie Beispiele f{\"u}r die Verkn{\"u}pfung Geographischer Daten mit Diskussionsbeitr{\"a}gen?
    \item[P3:] Also wenn ich mich richtig entsinne, geht das ja bei Google. Da kann ich das machen. So irgendwie, da ist ein Hotel, das fande ich jetzt gut, oder fande ich nicht so gut und dann kommt da dann das Rating. Da kenne ich es her. Sonst nicht.
    \item[I:] Haben Sie sich dann da auch schon einmal beteiligt?
    \item[P3:] Nein. Geographisch noch nicht. Das war mir auch zu weit weg. Also das muss mit meinem Thema zu tun haben. Sonst tu ich das nicht. Also ich schreib normalerweise nicht rein "`Das Hotel war jetzt geil"' oder so. Das mache ich nicht.
    \item[I:] Kennen Sie Werkzeuge um interaktive Karten mit eigenen Inhalten zu erzeugen?
    \item[P3:] Nein, kenne ich nicht.
    \item[I:] Also haben Sie sowas dann auch nicht benutzt?
    \item[P3:] Hm also nein. Aber es ist ganz lange her, da hab ich dann damals mit Powerpoint mir Karten gemacht. Ich nehm eine Karte und dann mach ich da einen Hotspot drauf. Also sowas hab ich schonmal gemacht. Aber das ist ja wirklich sehr Laienhaft. Das w{\"u}rde ich jetzt nicht als Geoinformationssystem bezeichnen. Sondern das ist mehr Pr{\"a}sentations (\dots) Hyperlink oder sowas. Mehr ist das nicht.
    \item[I:] Gibt es noch noch Fragen oder Anmerkungen von Ihrer Seite?
    \item[P3:] Ich finds geil. Ich finds richtig gut. Und ich freu mich dadrauf, wenns dann da ist. Finde ich auch spannend dann in welcher Form das kommuniziert wird. Das w{\"a}re noch ein Punkt zu den B{\"u}rgern, was k{\"o}nnte die davon abhalten. Davon abhalten k{\"o}nnte sie, dass sie es nicht wissen, dass es das gibt. Das w{\"a}re noch ein wichtiger Punkt.
    \item[I:] Dann bedanke ich mich recht Herzlich.
\end{itemize}