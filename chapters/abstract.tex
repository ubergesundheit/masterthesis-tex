\begin{abstract}
Public deliberation conducted by citizen initiatives is an important part of the democratic foundation of our society. Lists on Internet pages, meetings and newspaper advertisements are used as medium for informing the public. Giving citizens the opportunity to actively participate through discussing public matters is a step into a more profound deliberation. Although research of spatial discussion platforms exist, very few investigate the use of dialogs or even spatially enhanced dialogs. This thesis will explore the support of public deliberation performed by citizen initiatives through spatially enhanced dialogs. For this, a prototypical spatial discussion platform was developed, which explicitly enables participants to engage each other in dialogs. The concept of spatially enhanced dialogs, the developed prototype and how spatially enhanced dialogs could support public deliberation performed by citizen initiatives were tested through semi-structured interviews, a focus group and expert interviews. Results show general understanding of the concept but conveying spatial characteristics seems to have more importance among evaluation participants.
\end{abstract}
% Die wollen das ding haben, aber aus den falschen Gründen. Die sehen nur die Möglichkeit damit ihre Projekte auf ner Karte sichtbar zu machen. Die haben verstanden, was spatially enhanced dialogs sind, enthusiasm ist aber definitiv auf der informing stage