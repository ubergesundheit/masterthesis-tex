\begin{abstract}
Public deliberation conducted by citizen initiatives is an important part of the democratic foundation of our society. Mostly Internet pages, meetings and newspaper advertisements are used as medium for informing the public. Giving citizens 


This thesis tries to answer how the use of spatially enhanced dialogs could support deliberation performed by citizen initiatives. For this, a prototypical spatial discussion platform was developed, which will be provided to a local citizen initiative for a sustainability project. Then, semi-structured interviews, expert interviews and a focus group were performed to answer how public deliberation performed by citizen initiatives could be supported tho suppor could manifest itself. 


Results show some impediments but general helpfulness for supporting public deliberation performed by citizen initiatives through spatially enhanced dialogs.
\end{abstract}


This thesis features three work packages. 


