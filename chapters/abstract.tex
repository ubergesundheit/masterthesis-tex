\begin{abstract}
Public deliberation conducted by citizen initiatives is an important part of the democratic foundation of our society. Lists on Internet pages, meetings and newspaper advertisements are used as medium for informing the public. Giving citizens the opportunity to actively participate through discussing public matters is a step into a more profound deliberation. Although research of spatial discussion platforms exist, very few enable participants to engage each other in dialogs. This thesis will explore the support of public deliberation performed by citizen initiatives through spatially enhanced dialogs. For this, a prototypical spatial discussion platform was developed, which will be provided to a local citizen initiative for a sustainability project. Semi-structured interviews, expert interviews and a focus group were performed to answer how public deliberation performed by citizen initiatives could be supported through the use of spatially enhanced dialogs. Results show some impediments but general helpfulness for supporting public deliberation performed by citizen initiatives.
\end{abstract}
