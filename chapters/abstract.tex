\begin{abstract}
Public deliberation conducted by citizen initiatives is an important part of the democratic foundation of our society. Lists on Internet pages, meetings and newspaper advertisements are used as medium for informing the public. A more profound deliberation can be achieved through giving citizens the opportunity to actively participate. Although research of spatial discussion platforms exist, only few investigate the use of dialogs or even spatially enhanced dialogs. This thesis explores the support of public deliberation performed by citizen initiatives through spatially enhanced dialogs. In order to enable citizen initiatives to engage in dialogs, a prototypical spatial discussion platform was developed. Semi-structured and expert interviews, as well as a focus group, helped to evaluate how spatially enhanced dialogs support deliberation performed by citizen initatives. In this context, the concept of spatially enhanced dialogs and the developed prototype were tested. The results show general understanding of the respondents for the concept. However, the conveyance of projects' spatial characteristics appear to have a higher level of importance for evaluation participants.
\end{abstract}
% Die wollen das ding haben, aber aus den falschen Gründen. Die sehen nur die Möglichkeit damit ihre Projekte auf ner Karte sichtbar zu machen. Die haben verstanden, was spatially enhanced dialogs sind, enthusiasm ist aber definitiv auf der informing stage
