\section{Methodology}
\label{chap:methodology}
This section will give insight into the development process and chosen evaluation methods (\hyperref[subchap:ev_methodology]{section \ref{subchap:ev_methodology}}).

\subsection{Development Methodology}
As already stated in \hyperref[sub:dialogmap]{section \ref{sub:dialogmap}}, the development of the DialogMap prototype was performed in an agile and iterative manner, meaning that working prototypes could be presented and tried out by potential users. These potential users were a small group within a scientific citizen initiative. The cooperation was established through the second supervisor of this thesis. A vague concept of a map based information delivery emerged during the organization of a Münster-based, sustainability project within the citizens' initiative. According to this concept, citizens should be enabled to inform themselves about the different activities offered by the sustainability project through the map. This concept was then mutually transformed to match both the requirements of this thesis and the initial idea of the scientific citizen initiative. Not only should citizens be able to inform themselves through the map, but also engage each other in dialogs about the activities. Additionally, an agreement to participate in the evaluation of the research question was made. Through the cooperation, the opportunity to answer the research question with real-world users arised.\\
The requirements derived from the literature (\hyperref[sub:dialogmap]{section \ref{sub:dialogmap}}) formed a general framework for a spatial discussion platform, which guided the development of the DialogMap prototype. During the development phase, the small group of citizen activists contributed real-world suggestions and requirements according to their and the modified concept. The goal of the iterations was to deliver working prototypes in a monthly manner. These monthly prototypes were presented to the members of the scientific citizen initiative at monthly meetings. Through these monthly meetings it was possible to incorporate the reactions and suggestions to the current iteration into the next iterations. Additional communication was conducted via e-mail. Throughout the development process, a working instance of the current iteration was accessible to the members of the initiative.\\

\subsection{Evaluation methodology}
\label{subchap:ev_methodology}
For the assessment of the research question posed in the \hyperref[chap:introduction]{introduction} of this thesis, different evaluation methods were considered. Previous research in the field of participatory GIS and argumentation mapping suggested usability assessment through user studies with quantitative evaluation. As the focus of this thesis is on how the use of geospatially enhanced dialogs supports public deliberation, qualitative evaluation methods were chosen instead. Subsequently, semi-structured interviews with both 


As the project of the scientific citizen initative 

Semi-structured interviews in respect to the future use

\subsubsection{Semi-structured interviews}

\subsubsection{Expert interviews}

\subsubsection{Focus group}

