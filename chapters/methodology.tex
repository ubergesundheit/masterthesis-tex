\section{Methodology}
\label{chap:methodology}
This section will give insight into the development process of the \textit{DialogMap} concept and the evaluation techniques chosen (\hyperref[subchap:ev_methodology]{section \ref{subchap:ev_methodology}}).

\subsection{Development Methodology}
As already stated in \hyperref[sub:dialogmap]{section \ref{sub:dialogmap}}, the development of the \textit{DialogMap} prototype was performed in an agile and iterative manner, meaning that working prototypes could be presented and tried out by potential users. These potential users were a small group within a scientific citizen initiative \todo{wie vorhin: citizen initative launched by scientists oder so} with which \todo{which? whom? oder so..} a cooperation was established. During the organization of a Münster-based, initiative-spanning sustainability project, an idea of an interactive map visualization of the planned activities emerged. It was agreed upon applying the \textit{DialogMap} concept described in \hyperref[sub:dialogmap]{sub-section \ref{sub:dialogmap}} to the sustainability project. Not only should citizens be able to inform themselves through the map, but also engage each other in dialogs about the activities. Additionally, an agreement to participate in the evaluation of the research question was made. Through the cooperation, the opportunity to answer the research question with real-world users arised.\\
The requirements derived from the literature (\hyperref[sub:dialogmap]{section \ref{sub:dialogmap}}) formed a general framework for a spatial discussion platform, which guided the development of the \textit{DialogMap} prototype. During the development phase, the small group of citizen activists contributed real-world suggestions and requirements according to their idea and the mutually modified concept \todo{Wenn du sagst das es modifiziert wurde.. musst du auch sagen wie es sich von sec. 3.1 unterscheidet.. oder modified raus du hast in den sourcen unten ja schon was auskommentiert.}. The goal of the iterations was to deliver working prototypes in a monthly manner. These monthly prototypes were presented to the members of the scientific citizen initiative at monthly meetings. Through these reoccurring meetings, it was possible to incorporate the reactions and suggestions to the current iteration into the next iterations. Additional communication was conducted via e-mail. Throughout the development process, a working instance of the current iteration was accessible to the members of the initiative. %According to this concept, citizens should be enabled to inform themselves about the different activities offered by the sustainability project through the map. This concept was then mutually transformed to match both the requirements of this thesis and the initial idea of the scientific citizen initiative.

\subsection{Evaluation Methodology}
\label{subchap:ev_methodology}
For the assessment of the research question posed in the \hyperref[chap:introduction]{introduction} of this thesis, different evaluation methods were considered. Previous research in the field of participatory GIS and argumentation mapping suggested usability assessment through user studies with quantitative evaluation. As the focus of this thesis is on how the use of geospatially enhanced dialogs supports public deliberation, qualitative evaluation methods were chosen instead. Following the cooperation with the scientific citizen initiative for their sustainability project use case, evaluation methods related to this real-world use case were selected. Subsequently, semi-structured interviews with both members of the participating initiatives and other citizen initiatives, expert interviews and a focus group were adopted to answer how the use of spatially enhanced dialogs supports public deliberation performed by citizen initiatives.\\
Through the cooperation with the scientific citizen initiative, contact to other citizen initiatives could be established. An initial request for participation at interviews for the developed prototype was sent to participating initiatives at the sustainability project via e-mail. Following this request, eight semi-structured interviews and one focus group with three participants were conducted. Experts were also contacted via e-mail. Except one expert interview, which was performed via video chat, all interviews were carried out in quiet and private places chosen by the respondents. After all legal requirements for recording interviews were met, each interview was recorded and transcribed afterwards using modified transcription rules by Kuckartz \cite{kuckartz2007} (\ref{transcriptionrules}). Recording and transcription was also done for the focus group. Both the transcription of interviews and focus group were anonymized afterwards. After the transcription, the responses were categorized in order to singularize central statements and recurring information \cite{naderer2007auswertung}. Interviews and transcription were conducted in German.

\todo{Guter Abschnitt!}
\subsubsection{Semi-structured Interviews}
According to Helfferich \cite{helfferich2005}, semi-structured interviews enables the interviewer to explore emergent thoughts of respondents during the interviews thus allowing to modify the questioning accordingly. She also gives suggestions on question design and structure of interviews. Following the advice of Hellferich, all questions were narration prompts. Leading questions were avoided. Questions were structured in three parts. The functions and usage of the developed prototype were showcased before the questions. This was done either with a local installation or with a video recorded beforehand. The respondents were informed about the cooperation and the development of the prototype in advance by the scientific citizen initiative.
\todo{Hier noch ein Satz über das was du gleich auslistet: Aka Parts of the interview. Hast Du demographische Angaben? Alter, Geschlecht etc.? Wenn nicht - kannst Du an die noch rankommen? Falls vorhanden hier erwähnen, das sie beforhand abgefragt wurden.}

\textbf{Part I -- Public Participation}\\
General questions about the understanding of public participation were asked at the beginning of the interviews. Involvement in their respective initiative was established through asking about the respondents role and function. This included past and current activities in public participation. As public participation is an important part of public deliberation, important personal aspects of public participation were retrieved. This also included the personal opinion on benefits and purposes of public participation. Lastly, respondents were asked to give an introduction to an ongoing or past project where communication between actors played an important role. This should reveal how dialogs were incorporated and fostered in the past.

\textbf{Part II -- Planned Usage of the Prototype}\\
Respondents were asked to give an introduction to the project where the developed prototype will be deployed in the future to gain insights to target groups, editorial and anticipated contributions and incentives for dialogic contributions. These questions were adapted from Walker and Rinner \cite{Walker2013Qualitative}. Reasons for and against the deployment of the developed prototype in their respective projects were retrieved to find non-technical requirements, as well as personal expectations related to dialogic discussions with citizens. Other use cases apart from public deliberation for a spatially enhanced discussion platform were asked at last in part II.

\textbf{Part III -- Supplementary Questions}\\
Supplementary questions comprised of questions to assess knowledge of the respondent in the fields of spatial media, interactive mapping applications and spatially enhanced discussions. Respondents were asked about knowledge of applications and past usage respectively.

\subsubsection{Expert Interviews}
Apart from the semi-structured interviews described above, two expert interviews were conducted. Expert interviews allow to gather insights, opinions and assessments from domain experts \cite{hopf20045}. The two experts were Carsten Ke{\ss}ler, author of multiple publications in the argumentation mapping field \cite{Kessler2005_ArgumentationMapPrototype,kessler_argumap,Kessler2005_Conflict_Resolution}, including the development of the first argumentation map prototype and Tobias Heide, domain expert of usability of web applications. The focus of the expert interviews was the assessment of the implementation, usability of the \textit{DialogMap} prototype, related methods for public participation and similar systems. As experts can handle more direct questions than lay people \cite{helfferich2005}, the questions for the domain experts were more direct and focused on only one aspect.\\
After a presentation of the developed prototype and a general explanation of the background where the prototype will be deployed, 18 questions were asked \todo{Sind die im Anhang? Referenzieren}. At first, knowledge of applications with spatially enhanced discussions were retrieved. Subsequently, prior usage of such systems and pros and cons of the applications were asked. These questions should assess and gather applications and systems, which are not described in the literature. Next, use cases for geospatially enhanced discussions and especially dialogs apart from public participation were sought. Then, solutions to engage participation between citizens, initiatives, governments and politicians and the way communication is conducted there were enquired.\\
Before a block of questions aimed directly at the functionalities of \textit{DialogMap}, a general assessment of the idea of spatially enhanced dialogs in the context of public participation was requested. The functionalities, to which the experts comments were sought were hiding and showing of the spatial features related to a thread, two-way highlighting of spatial features and textual representations, filter and sorting functions, composing of contributions, creation and referencing of spatial features and hyperlinks, favoring of contributions, sign in and registration and social login. After the block related to the functionalities, missing characteristics were inquired.\\
Then experts were asked for their opinion if dialogs will be supported through \textit{DialogMap}. Finally the experts should list reasons that, in their opinion, could keep citizens away from contributing.

\subsubsection{Focus Group}
A technique to gather unspecific and quantitative feedback from a group of people is called focus group discussion \cite{morgan1996_focus_groups}. It allows to harness group discussions about a topic with an active guidance of the researcher.\\
In order to gather quantitative feedback about the usability and user friendliness of the developed prototype in a semi real-world scenario, a focus group with three participants was conducted. Small focus groups give each participant more time to express their thoughts \cite{morgan1996_focus_groups}. The preparation and design of the focus group was done using recommendations of Asbury \cite{asbury1995overview}.\\
As an introduction, participants were given a short walkthrough that showcased functionalities and the concept of \textit{DialogMap} (similar to the introduction used in the semi-structured interviews. See \ref{demo}). After the introduction, a simple task, which required to use all functionalities of the \textit{DialogMap} prototype, was then given to the participants. Focus of the task was the engagement of the participants in dialogs. After about 30 minutes, a discussion was started, in which the participants were asked about their opinions and impressions about the concept and actual implementation of the \textit{DialogMap} prototype.
\todo{Gute Sektion.. kleinigkeiten weghaun und dann wenig anfassen.. nur für Sprache / Rechtschreibung :)}
