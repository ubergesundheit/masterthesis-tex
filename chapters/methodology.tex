\section{Methodology}
\label{chap:methodology}
This section will give insight into the development process and chosen evaluation methods (\hyperref[subchap:ev_methodology]{section \ref{subchap:ev_methodology}}).


\subsection{Development Methodology}
As already stated in \hyperref[sub:dialogmap]{section \ref{sub:dialogmap}}, the development of the DialogMap prototype was performed in an agile and iterative manner, meaning that working prototypes could be presented and tried out by potential users. These potential users were a scientific citizen initiative. The cooperation was established through the second supervisor of this thesis. They had a vague idea about a map application to transport the idea and locations related to a planned project. This idea was then mutually transformed to match both the requirements of this thesis and the initial idea of the scientific citizen initiative.\\
The requirements derived from the literature (\hyperref[sub:dialogmap]{section \ref{sub:dialogmap}}) formed a general framework for a spatial discussion platform, which guided the development of the DialogMap prototype. The scientific citizen initiative contributed real-world suggestions and requirements to the development direction of the prototype according to their idea.\\
The goal of the iterations was to deliver working prototypes in a monthly manner. These monthly prototypes were presented to the members of the scientific citizen initiative at monthly meetings. Through these monthly meetings it was possible to incorporate the reactions and suggestions to the current iteration into the next iterations. Additional communication was conducted via e-mail.\\



Monthly meetings. Presentation. Input.

Planned future use is the advertisement of and invitation for participation at an event.

Semi-structured interviews in respect to the future use

\subsection{Evaluation methodology}
\label{subchap:ev_methodology}

\subsubsection{Semi-structured interviews}

\subsubsection{Expert interviews}

\subsubsection{Focus group}

