\section{Methodology}
\label{chap:methodology}
This section will give insight in chosen evaluation methods (\hyperref[subchap:ev_methodology]{section \ref{subchap:ev_methodology}}) and development practices.


\subsection{Development Methodology}
As already stated in \hyperref[sub:dialogmap]{section \ref{sub:dialogmap}}, the development of the DialogMap prototype was performed in an agile and iterative manner, meaning that working prototypes could be presented and tried out by potential users. These potential users were a scientific citizen initiative. The cooperation was established through the second supervisor of this thesis.   In an early conceptual phase of this thesis, 


The requirements derived from the literature (\hyperref[sub:dialogmap]{section \ref{sub:dialogmap}}) formed a general framework for a spatial discussion platform, which guided the development of the DialogMap prototype. In addition to that, the scientific citizen initiative contributed real-world suggestions and requirements to the development direction of the prototype.


The suggestions to each iteration of the scientific citizen initiative were incorporated in the development of the next iterations. 



The goal of the iterations was to deliver working prototypes in a monthly manner. The monthly iterations were presented A group of scientific citizen initiative members 


Monthly meetings. Presentation. Input.

Planned future use is the advertisement of and invitation for participation at an event.

Semi-structured interviews in respect to the future use

\subsection{Evaluation methodology}
\label{subchap:ev_methodology}

\subsubsection{Semi-structured interviews}

\subsubsection{Expert interviews}

\subsubsection{Focus group}

