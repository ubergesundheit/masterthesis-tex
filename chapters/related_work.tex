\section{Related Work}
\label{chap:related_work}
This section explores previous research in the fields of participation, e-Participation, participatory GIS and argumentation mapping. It is structured as following: After establishing the background of public participation, deliberation through the use of modern Information technologies is investigated. Finally, geographic information systems in decision-making, geographic discussion tools and argumentation mapping research is explored.
%\todo{Van Eemeren and Grootendorst (1996) and Tweed 1998 for definition of argumentation or discussion} 
%\todo{\cite{Elwood2013_NewSpatialMedia}}


\subsection{History of Public Participation Research}
%%% Ladder
\todo{Kommentare beziehen sich immer auf die / den Satz vorher.}
In 1969, Arnstein \cite{Arnstein1969_citizen_participation} defined eight levels of increasing citizen participation. Since then, her ``ladder of citizen participation'' has been adapted and modernized by several authors \cite{Connor1988_new_ladder,carver2003future,Collins2009_social_learning,you2009_participatory_map_based,Cai2009_spatial_annotation_deliberation,Macintosh2004_eParticipation_characterization,Schlossberg2005_PPGIS}. She claimed that citizen participation is a term for the redistribution of power from the government among citizens. While the first two rungs belong to the category ``Nonparticipation'', involvement of citizens starts with the third level ``Informing''. All of the following, ``Consultation'', ``Placation'', as well as ``Informing'' belong to the category ``Tokenism''. Furthermore, Arnstein defined the category of ``Citizen Power'' with the levels of ``Partnership'', ``Delegated Power'' and ``Citizen Control''.\\
\todo{Vielleicht sagt ein Bild hier mehr als 1000 Worte?}
Wiedemann and Femers \cite{Wiedemann1993355} proposed a similarly structured ladder, which is based on the premise that information given to the public and the amount of possible participation is collateral. By requiring the implementation of the previous steps before the higher levels can be reached, each step on their ladder leads to more empowerment of citizens. In their research, Wiedemann and Femers concluded that general understanding of an issue is a first step towards public participation.\\
As a reaction to the ladder proposed by Arnstein, Connor \cite{Connor1988_new_ladder} constructed his version of a new ladder ``whose elements have a cumulative effect''. It is designed to help decision makers to apply techniques ``to prevent and resolve public controversy about various proposals''. The ``participation'' step on the ladders of Arnstein and her colleagues serves as a premise for the participation through spatially enhanced dialogs. Deliberation performed by citizen initiatives also roots in these levels. They act upon the same assumption that public deliberation empowers citizens and redistributes power from the government.

\subsection{The use of Internet Communication Technologies in Participation, Government and Democracy}
As stated by Arnstein \cite{Arnstein1969_citizen_participation}, involvement of the public begins with the supply of sufficient information to the general population. This includes the publishing of information as well as contacting officials with suggestions and questions. Traditional contacting means were transformed to be used through modern information technologies. Reddick \cite{Reddick2005_Citizen_interaction_with_egovernment} found out that these modern ways of citizen interactions are an improvement over the traditional means of contacting their government. Although the Internet often leads to a lack of personal interactions by only serving the information needs of citizens. However, the Internet enables citizens to skip ``street-level bureaucrats''. The author recommends that the focus should be on ease of use, user friendliness and marketing of online services. Additionally, Reddick also warns of the digital divide by excluding non-technical population groups.\\
Due to the fact that after voting, contacting officials is the most common act of political participation, Bimber \cite{Bimber1999_Citizen_communication_with_government} conducted a comparison of traditional contacting methods with Internet based communication means. Specifically, he executed a survey with 2021 participants to answer the question if the contact medium for interacting with officials matters. He found out that the Internet is ``just'' another option for citizens and officials to connect, and not to ``revolutionize'' their way of communicating.\\
S{\ae}b{\o} et al. \cite{Saebo_eParticipation} conducted a thorough review of 131 ``e-Participation'' articles. On a technical level, e-Participation is generally understood as ``joining in'', taking part or taking role and in general, it is associated with political deliberation or decision-making. It can occur in- and outside of political processes by using Internet communication technologies. The work of S{\ae}b{\o} et al. was continued by Medaglia in 2012 \cite{Medaglia2012_eParticipation}. He reviewed 122 e-Participation articles published between 2006 and 2011. Both, S{\ae}b{\o} et al., and Medaglia classified the e-Participation research domain as follows: ``Actors'' conduct ``activities'', which result in ``effects''. ``Activities'' are influenced by other ``activities''. Finally, the ``evaluation'' of the ``effects'' is used to improve future ``activities''. Susha et al. \cite{Susha2012_eParticipation} did an analog classification of e-Participation research with the categories of ``stakeholders'', ``environment'' and ``applications and tools''.\\
In 2004, Macintosh \cite{Macintosh2004_eParticipation_characterization} defined ``e-Democracy'' as ``the use of Internet communication technologies to support the democratic decision-making processes''. In order to better understand this definition, she described several key dimensions of e-Participation. Similar to the participation ladders of Arnstein, Wiedemann, and Connor \cite{Arnstein1969_citizen_participation,Wiedemann1993355,Connor1988_new_ladder}, the level of participation is one of these key dimensions. Additionally according to Macintosh, the stage of policy-making in which the citizens are engaged should be considered carefully. Other important key dimensions are ``Who should be engaged by whom'', ``which technologies are used'' and ``rules and duration of engagement''. Finally, evaluation and outcomes should be reflected thoroughly.\\
Above the one-way ``Informing'' stage is the bilateral communication in form of discussions and dialogs, which serves as additional base for the so called ``e-Participation'', ``e-Government'' and ``e-Democracy''. In 1998, Kent and Taylor \cite{Kent1998_dialogic_relationships_through_www} proposed a theoretical framework for building dialogic relationships through the Internet. They define dialogic communication as ``any negotiated exchange of ideas and opinions''. Participants in a dialog do not necessarily have to agree with each other, but participate in order to reach a mutually satisfying position. Dialogs can help to share subjective views and to focus on the attitude towards each other. According to Kent and Taylor, the Internet aids this process by allowing both synchronous and asynchronous means of communication.\\
Renn et al. \cite{Renn1993_participation} proposed in 1993 a three step procedure to apply and organize public participation above the ``Informing'' stage. They recommend to structure values and concerns of citizens and stakeholders into a hierarchy, which should then be discussed and judged by experts. The results should be evaluated in citizen panels. Renn et al. divided participants into three groups: Firstly, ``stakeholders'', which bring concerns and interests, as well as metrics for evaluation. Next, ``experts'' who analyze and provide related data and help to uncover functional relationships between options and their impacts. And finally, ``citizens'', as potential victims and benefactors, who assess the results and outcomes of the proposals of the other groups.\\
Wright and Street \cite{Wright2007_deliberation_design} evaluated online forums as a tool for mass deliberation. They found out that current online forums are not designed for social interaction. Furthermore both, proponents and opponents of mass deliberation through online forums, do not comprehend the role played by design in facilitating or thwarting deliberation and therefore tend to tread information technology as given and determinant. Hence, Wright and Street conclude that technology is both shaped by and also shaping political discussions on the Internet and as a result, recommend to focus on moderation of discussions in online forums.\\
Jaeger \cite{Jaeger2005_deliberate_democracy_and_egovernment} explored potential social impediments of the increasing use of Internet communication technologies in e-Government and e-Participation. Through a survey of existing information studies in public policy, law, and governance, he came to the conclusion that the Internet ``poses real dangers of creating or fostering social fragmentation''. According to Jaeger, the asynchronous nature of interaction makes it easy to avoid people with contrary opinions. At the same time, finding people with same opinions is easier than in real world situations such as public hearings. Settings like these tend to facilitate group polarization. Jaeger stated that members of groups with shared beliefs or identity are more likely to be inclined to extreme opinions. Anonymity of Internet discussions further increases tendencies to take extreme positions. Apart from these concerns, he clearly saw an advantage in applying Internet communication technologies, which ``could potentially benefit the health of the entire democracy''.\\
Kriplean et al. \cite{Kriplean2012_Considerit} explored public deliberation through decision support with pro/con lists. Their system allows to create public pro/con lists in which one could include other participants pro/con points. Then, the lists showed an overview of all stances and contributions. This indirect discussion mitigates ``political identity and flaming'' by forcing the users to reflect on their standpoints and disallowing any portrayal of political affiliations.\\
In 2009, Collins and Ison \cite{Collins2009_social_learning} suggested that the traditional ladders of Arnstein and her colleagues are too dominant in citizen participation in policy discourses. They propose to focus more on the social learning aspect of participation, which builds upon convergence of goals, co-creation of knowledge, and change of behavior and actions. If all participants and stakeholders apply this ``social learning'', the understanding of discussion matters would be further increased.%\todo{mehr!}
\todo{Guter Abschnitt, bitte kurz zu den Diskussion Maps linken. Z.B. das der und der Aspekt bei Dir berücksichtig wurde / special attention paid.}
\subsection{Geographic Information Systems in Decision-Making}
\label{subchap:gis_stuff}
Since the development of geographic information systems (GIS), they were used to support experts in making decisions. In 1991, Densham \cite{densham_sdss} created an early overview over spatial decision support systems (SDSS). Standard decision support systems were developed to help experts in solving ``ill-structured'' issues with difficult problem definitions. By combining existing data with statistical models, solution space can be explored. Multiple decision-making styles are supported through the ability to give different factors different weights. The introduction of spatial capabilities into decision support systems brings several benefits and allows to solve semi-structured spatial problems. Additional features of spatial decision support systems are the ability to store and illustrate spatial relations, the analysis through statistical methods, and to create maps as output. Therefore, Densham proposes a ``SDSS generator'' for future development, which combines analysis tools, GIS functionalities, and database management systems. Furthermore, the differentiation between ``Objective'' and ``Map'' space is considered very important. Users must be able to view both spaces simultaneously, which update each other when changes are made to either one.\\
Following the developments in spatial decision support systems, the concept of Public Participatory GIS (PPGIS) and Participatory GIS (PGIS) emerged. Both focus on enabling non-expert groups to apply GIS technologies to strengthen involvement in decision-making.\\
In 2006, Sieber \cite{Sieber2006_PublicParticipationGIS} traced the social history of PPGIS and lists major themes found in PPGIS research: ``Place and people'' relate to the question ``who should be participating in PPGIS projects''. ``Technology and Data'' cover representation of knowledge, accessibility of data, and appropriateness of information. ``Process'' focuses on decision-making structures and processes, participation and communication in the policy making process, and system implementation and sustainability. The theme ``Outcomes and evaluation'' measures goals and results. Siebers' themes are similar and related to Macintosh's \cite{Macintosh2004_eParticipation_characterization} key dimensions of e-Democracy. Sieber also notes that although PPGIS have been constructed and practiced by a broad range of groups and agents in multiple research disciplines, the enhancement of public participation and deliberation is controversially attributed to GIS. Similar observations were made by Obermeyer \cite{obermeyer1998evolution}, Craig et al. \cite{Weiner2002_Participation_and_GIS_eigentlich_Craig} and Blaschke \cite{Blaschke2004_PGIS_critically_revised}. According to Blaschke ,the distinction of PPGIS from SDSS is the general participation of the public. He also claims that public participation does not automatically lead to better decisions.\\
Schlossberg and Shuford \cite{Schlossberg2005_PPGIS} tried to delineate the terms ``public'' and ``participation'' in PPGIS through a literature review. They defined a matrix with the axes ``Domain of Participation'' and ``Domain of Public''. The ``Domain of Participation'' axis contains participation techniques, which the actors on the ``Domain of Public'' axis perform. Both axes range from ``simple'' to ``complex''. The ``Domain of Participation'' dimension is based upon the various participation ladders defined by Arnstein and her colleagues \cite{Arnstein1969_citizen_participation,Wiedemann1993355,Connor1988_new_ladder}.\\% They populate their matrix with four scenarios. \todo{Warum erwähnst Du das es 4 szenarien gibt, wenn du sie nicht erläuterst?}\\
Voss et al. \cite{Voss2004_Evolution_PGIS} describe the combination of a structured argumentation tool, Dito, and a spatial decision support system, CommonGIS. The researchers integrated the systems gradually over the course of three experiments exploring different aspects of spatial discussions. Through these experiments, Voss et al. identified multiple conceptual, technical, and user interface requirements. Due to their approach to combine two existing systems, several technical issues emerged. \todo{Gehst Du auf die später ein? Was sind requirenments / issues? Sehr interessant!}\\
Alongside with an overview of PGIS systems, Jankowski \cite{Jankowski2005_community_based_pgis} analyzed two studies in PGIS water resource decision-making. In both cases, participants had to make suggestions for water source protection sites. The second case featured shared displays to communicate spatial data. Jankowski found out that the citizens' trust in state supplied data was not high and that the technology sometimes reduced creativity, especially, if it was difficult to use. He also uncovered the fact that citizens need to have real interest and insight in information in order to support decisions effectively. Subsequently, Jankowski recommended future systems should focus on how technology can be used without reducing the creativity of participants.

\subsection{Geographic Discussion Tools}
\label{zweivier}
Rinner \cite{Rinner_ArgumentationMaps} picked up on the idea of PPGIS but focused on the combination of e-Participation principles (the use of Internet communication technologies to involve citizens in discussions about decision-making processes) with geographic information systems. After a review of discussion and collaboration tools, he found out that asynchronous discussions were not considered during planning procedures. As a result, Rinner proposed the concept of ``Argumentation mapping'' for which he described four use cases, which outline the design of argumentation mapping. The GIS functionality of spatial data presentation is used for navigation in argumentation maps. Similar to Goodchild's \cite{goodchild2007citizens} concept of volunteered geographic information, Rinner facilitated user created geographic data to be used in public participation in planning procedures. Retrieval and analysis functions of GIS can be used for exploration and evaluation of suggestions and solutions in argumentation maps. According to Rinner, geographically referenced argumentation can only be used successfully, if an object-based model of geographically referenced argumentation is applied.\\
Since then, multiple implementations of argumentation mapping and PPGIS systems with multiple research goals and outcomes have emerged.\\
In 1999, Kingston et al. \cite{kingston1999gis} developed a PPGIS system to enable a digital version of annotated map pins. The system allowed to create comments with a spatial reference, but it lacked a structured discussion support and only allowed one spatial reference per comment.\\
A first implementation of the argumentation map idea was realized by Ke{\ss}ler et al. \cite{Kessler2005_ArgumentationMapPrototype}. They proposed a set of requirements and design guidelines for argumentation maps and analyzed different options for linking maps and discussion contributions. The implementation was supposed to address the two main issues of the analyzed systems: User friendliness and support of semi open standards. Ke{\ss}ler et al.'s implementation featured separate discussion and map components. According to the authors, requirements for the discussion component are an ``Integrated user interface'', which supports ``Structured discussion'', an ``Integrated database, [which supports] many-to-many relationships'' and ``Access control, security''. Requirements for the map component are ``Integrated user interface'', ``Common (Web)mapping functions'', ``Integrated database, many-to-many relationships'' and ``Customization by Provider''. An important aspect was the use of open standards throughout all system parts to ensure re-usability. The system allowed users to upload ESRI shapefiles\footnote{\url{http://www.esri.com/library/whitepapers/pdfs/shapefile.pdf}} and to create point features in a map. These spatial features could be annotated with texts and labels. Sidlar and Rinner \cite{sidlar_argumap_2007} conducted an evaluation of the Argumentation map prototype implemented by Ke{\ss}ler et al. with HCI principles. They specifically investigated learnability, memorability and user satisfaction. The evaluation of the implementation in a user study revealed generally positive results. Finally, Sidlar and Rinner gave a list of recommendations for improvements of the argumentation map prototype. Furthermore, they conducted an utility assessment \cite{Sidlar2009-AssessmentMapGeocollaborationTool} of the Argumentation Map prototype. The authors state that the utility of application is often understood as given, thus, they developed a framework for investigating participatory GIS utility. Utility was measured by dividing actual use of argumentation mapping functions by the potential use of those functions. After applying their framework to the prototype, Sidlar and Rinner reached the conclusion that every aspect of the argumentation map prototype was used but ``not always to its fullest''. \todo{Wichtig - reflektierst / nutzt Du diese Erkentniss später bei deinen Ergebnissen?}\\
Ke{\ss}ler et al. \cite{Kessler2005_Conflict_Resolution} discussed a combination of a spatial data infrastructure as extension of the argumentation map prototype from 2005. Due to the fact that PPGIS applications naturally make extensive use of geographic information, the authors argue that participatory discussions in PPGIS could benefit from readily available geospatial data. They also implemented basic analysis functionalities and applied their idea in a conflict resolution context. Ke{\ss}ler et al. found out that spatial data infrastructures can ease and simplify the setup of a PPGIS because geospatial objects can be discussed from the beginning.\\
Two systems in PPGIS context were developed by Carver et al. \cite{Carver2001_PPGIS_Cyberdemocracy} to find out how the Internet and GIS can be used together in order to involve the general public more in environmental decisions. While the first system only conveyed information about a natural reserve, the other was used for collecting citizen ideas to improve a small town. They identified technological issues with implications for decision-making processes with citizen involvement. Furthermore, access to geospatial (and non-geospatial) data and tools may encourage the general public to contribute more to decision-making processes. Unfortunately, participation is held back by inequalities of citizens in their computer literacy. The authors also list principles for future implementations of web based PPGIS applications.\\
In their article ``Design of a GIS Enabled Online Discussion Forum for Participatory Planning'', Tang et al. \cite{Tang2005_PPGIS_discussion_forum} focus on effective communication and mutual understanding. They reviewed eleven PPGIS applications for participatory planning. Among these eleven were systems of Carver \cite{Carver2001_PPGIS_Cyberdemocracy} and Rinner \cite{Rinner_ArgumentationMaps,Kessler2005_ArgumentationMapPrototype}. The PPGIS applications were evaluated under the following criteria: First of all, experts should be able to play the facilitators role. Exchange of views must be supported as well as the documentation and sharing of the evolution of ideas. Furthermore, made decisions should be shown in the context of related decisions, and finally the effectiveness of communication about spatial context was evaluated. The authors revealed several shortcomings of the reviewed applications and discussed the development of GeoDF, which is a GIS-enabled online discussion forum to enable citizens of a small town in Canada to provide in-depth feedback to the government. Being the co-authors of Tang et al. \cite{Tang2005_PPGIS_discussion_forum}, Zhao and Coleman \cite{zhao2006geodf} summarized the process and lessons learned in implementing the GeoDF prototype in an follow-up article: In GeoDF, textual components and spatial context have a one-to-one relationship. The spatial context consists of map extent, visible layers, annotations, and sketches from both, the user and other contributors. Due to the use of proprietary technology, issues of data availability, licensing, maintenance, re-usability, and interoperability emerged.\\
Multi-criteria evaluation (MCE) is a computation method, which helps to put alternative solutions of a problem into order by applying metrics to the different parameters of the problem. MCE is seen as an alternative to ``hard'' boolean filters used in SDSS. In order to answer if geovisualization in multi-criteria evaluations can support spatial decision-making, Rinner \cite{Rinner2007_geovis_decisionsupport} conducted two studies, in which users could evaluate different outcomes of a problem by manipulating parameters with sliders. The result of the parameter manipulation was immediately visible on a map and diagrams. Rinner found out that his solution with sliders supported the decision-making process.\\
The development of a multi-criteria decision support system in combination with an argumentation map is described by Sim\~{a}o et al. \cite{Simao2009Webbased}. They used their system to educate their users about all outcomes of a collaborative planning process. Their application consist of a three tier architecture, which the users are navigated through. After an information area, the user enters the actual MC-SDSS application (multi-criteria spatial decision support system) to explore the solution space. The last step consists of a map centric communication tool, which records opinions about the explored solutions. The most discussed solutions are assessed by experts. During the development, the authors identified problems of planning processes. Often, the problems have many dimensions making it extremely difficult to define the problem beforehand. As well as expert knowledge, communication is key in finding the best solution.

\subsection{Argumentation Mapping with Web 2.0 Principles}
\label{zweifuenf}
Rinner \cite{Rinner2009_Web2_argumap} et al. assessed the use of Web 2.0 principles and technologies for collaborative spatial decision-making by re-implementing Ke{\ss}ler et al.'s \cite{Kessler2005_ArgumentationMapPrototype} original Argumentation map. The re-implementation called ``ArgooMap'' allows users to submit place based comments and to respond to other comments. Spatial references could only be point-based. The authors evaluated their thread based online map discussion forum with a simulation. Existing discussions were re-enacted in a sandbox with no user interaction.\\
An implementation of a multi-criteria decision analysis (MCDA) tool was described by Boroushaki and Malczewski \cite{Boroushaki2010_ParticipatoryGIS}. They took the ArgooMap prototype of Rinner et al. \cite{Rinner2009_Web2_argumap} and extended it by adding multi-criteria decision support. Automatically generated problem solution alternatives could be discussed through the ArgooMap part. A follow up paper of Boroushaki and Malczewski evaluated their MCDA implementation through a study with citizens \cite{Boroushaki2010_Consensus_measurement}. They identified bringing together experts and laypeople as a main challenge of GIS-based spatial decision-making tools. The goal should always be to reach a high consensus among decision-makers and citizens. Meng and Malczewski \cite{Meng2010_ArgooMap_evaluation} conducted another evaluation of the MCDA implementation of Boroushaki and Malczewski. They tested the usability of the front-end ArgooMap with a user study where participants had to choose and discuss possible parking facility sites. The authors measured usability through perceived user effectiveness, efficiency, and satisfaction. They found that user effectiveness has a strong influence on how long one stays on the site. Efficiency impacts user visit numbers, page views, and interaction with others. Meng and Malczewski suggested that user testing should be considered in system-design processes of web-PPGIS. \todo{Wichtig - Eine der Anforderungen die du umgesetzt hast. Wird das später erwähnt?}\\
General concepts and methods for designing Web 2.0 community-based geoportals are presented by Longueville \cite{Longueville2010_community_based_geoportals_web20}. He sees community-based geoportals as advanced spatial data infrastructures, which should allow users to gather and share resources, organize themselves into groups, and to create resources collaboratively. Each resource should also include meta-resources like popularity, tags, and comments. Longueville recommends to modularize applications to create both human and machine readable interfaces.\\
Cherubini and Dillenbourg \cite{Cherubini2007_shared_maps} tested a chat system that allows users to create spatial references either through displaying the chat contributions directly on a map or referencing map features through symbols in the text. In a user study, they found that the references provided a tool to reinforce the reference frame of the conversation. Also, explicit referencing made communication more efficient. Participants used fewer sentences with fewer words. Consequently Hopfer and MacEachren \cite{Hopfer2007_Communication} applied a group communication theory, the ``Collective Information Sharing bias'' (CIS), to a geospatial annotation tool. The authors found that the ability to annotate map-based displays can ease spatial communication tasks and enable participants of a discussion to move beyond ``what are we talking about'' to ``who knows what about an area and how this could be of use''. Therefore, visualizing all information spatially will allow users to see gaps and thoroughly covered areas and to reduce repetition of known information. ``Information that is made visually explicit is more likely to be recalled and, therefore, discussed'' \cite{Hopfer2007_Communication}. The authors give design recommendations in applying CIS to the design of a geospatial collaboration tool. Cai and Yu \cite{Cai2009_spatial_annotation_deliberation} discussed the challenges of spatially enabled public deliberation. In their opinion, PPGIS merges analytical capabilities of GIS with deliberative capabilities of discussion forums. They distinguished two deliberation types, cognitive and practical deliberation. Similarly to Arnsteins' ladder, they define five phases of geodeliberative dialogs. After the first stage of ``Briefing and introduction'', the purpose of geodeliberative dialogs is to ``Elaborate on problems and issues''. Next, it allows to ``Contribute personal knowledge, experiences, and stories'', which then aids the ``Development of public judgment and creation of a common ground''. The final stage is ``Generating and evaluating alternative courses of actions''.\\
A set of non-functional requirements of scalable, reliable, and easy-to-maintain applications in a cloud computing context are given by Sani and Rinner \cite{Sani2011_Scalable_Argumap}. They state that system security, performance and scalability, robustness, availability and fault tolerance, maintainability and portability, modifiability, and testability are the key non-functional requirements when developing a scalable argumentation mapping tool.\\
Although described in their respective literature, very few systems are readily available for an examination of their functionality and mode of operation. However, several non-scientific implementations of spatially enhanced discussion tools exist.\\
The project ``Nexthamburg''\footnote{\url{http://www.nexthamburg.de/}, Kulus et al. \cite{Kulus_nexthamburg}} enabled citizens of Hamburg to suggest ideas for its urban development. The ideas were located on a map and other citizen could comment on the suggestions and ideas. Then, experts and decision-makers evaluated the ideas for implementation. The concept spawned a similar project in Kassel\footnote{\url{http://www.nextkassel.de/}}. Though, the suggestions and ideas could only be loosely tied to a location.\\
The ``Shareabouts''\footnote{\url{http://openplans.org/shareabouts/}} and ``CollaborativeMap.org''\footnote{\url{http://www.collaborativemap.org/home/}} applications use a spatial approach for gathering input of citizens. Although contributions are directly related to a spatial location, only one location is allowed per suggestion.
\todo{Du hast wirklich gute Arbeit mit der Related Work gemacht. Sie würde wahrscheinlich noch etwas stärker sein, wenn Du in jedem Abschnitt jeweils kurz sage würdest was das für deine Arbeit bedeutet und welche Aspekte für Sie am relevantesteten sind, aber es ist ja immer etwas Luft nach oben. Chris würde wahrscheinlich ähnliches sagen, vielleicht auch, dass man das ein und andere kürzen könnte - aber ich finds mit leichten Abstrichen (Relevanz zu eigner Arbeit, manchmal Sprache) quantitativ und qualitativ wirklich gut.}

%\subsection{Evaluation Methodology}
%This sub-section will explore the 
%Usability of PPGIS Haklay and Tobón \cite{Haklay2003_ppgis_usability_evaluation}


 %\cite{Pocewicz2012_Paper_vs_PPGIS} \cite{Aggett2006_evaluation_dss_ppgis}
 %\cite{Damianos1999_evaluation_collaborative}
 
%semi-structured Interviews: 
%\cite{helfferich2005} Transcription rules: \cite{kuckartz2007}  
%Focus group conduction: 
%\cite{asbury1995overview} 
    %What is a focus group: \cite{carey1994capturing} 
%\cite{morgan1996_focus_groups}
