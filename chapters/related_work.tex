\section{Related Work}
\label{chap:related_work}
This section explores previous research in the fields of participation, e-Participation, participatory GIS and argumentation mapping.% and evaluation techniques to provide an overview of the fields.

%\todo{Van Eemeren and Grootendorst (1996) and Tweed 1998 for definition of argumentation or discussion} 
%\todo{\cite{Elwood2013_NewSpatialMedia}}


\subsection{History of Participation Research}
%%% Ladder
By defining eight levels of increasing citizen participation, Arnstein \cite{Arnstein1969_citizen_participation} coined the term ``ladder of citizen participation'' which, since then, has been adapted and modernized by several authors \cite{Connor1988_new_ladder,carver2003future,Collins2009_social_learning,you2009_participatory_map_based,Cai2009_spatial_annotation_deliberation,Macintosh2004_eParticipation_characterization,Schlossberg2005_PPGIS}. She claimed that citizen participation is a term for the redistribution of power from the government among citizens. While the first two rungs belong to the category ``Nonparticipation'', involvement of citizens starts with the third level ``Informing''. All of the following, ``Consultation'', ``Placation'', as well as ``Informing'' belong to the category ``Tokenism''. Furthermore, Arnstein defined the category of ``Citizen Power'' with the levels of ``Partnership'', ``Delegated Power'' and ``Citizen Control''.\\
Wiedemann and Femers \cite{Wiedemann1993355} proposed a similarly structured ladder, which is based on the premise that information given to the public and the amount of possible participation is collateral. By requiring the implementation of the previous steps before the higher levels can be reached, each step on their ladder leads to more empowerment of citizens. In their research, Wiedemann and Femers concluded that general understanding of an issue is a first step towards public participation.\\
As a reaction to the ladder proposed by Arnstein, Connor \cite{Connor1988_new_ladder} constructed his version of a new ladder ``whose elements have a cumulative effect''. It is designed to help decision makers to apply techniques ``to prevent and resolve public controversy about various proposals''. The ``participation'' step on the ladders of Arnstein and her colleagues serves as a premise for the participation through spatially enhanced dialogs.%\todo{HIER NOCH WAS SCHREIBEN}

\subsection{The use of Internet communication technologies in Participation, Government and Democracy}
As stated by Arnstein \cite{Arnstein1969_citizen_participation}, involvement of the public begins with the supply of sufficient information to the general population. This includes the publishing of information as well as contacting officials with suggestions and questions. Traditional contacting means were transformed to be used through modern information technologies. Reddick \cite{Reddick2005_Citizen_interaction_with_egovernment} found out that these modern ways of citizen interactions are an improvement over the traditional means of contacting their government. Although the Internet often leads to a lack of personal interactions by only serving the information needs of citizens. However, the Internet enables citizens to skip ``street-level bureaucrats''. The author recommends that the focus should be on ease of use, user friendliness and marketing of online services. Additionally, Reddick also warns of the digital divide by excluding non-technical population groups.\\
Due to the fact that after voting, contacting officials is the most common act of political participation, Bimber \cite{Bimber1999_Citizen_communication_with_government} conducted a comparison of traditional contacting methods with Internet based contact methods. Specifically, he tried to answer if the medium of contacting officials matters by conducting a survey with 2021 participants. He found that the Internet is ``just'' incrementing the connection between citizens and officials, not ``revolutionizing'' it.\\
A thorough review of 131 ``e-Participation'' articles was conducted by S{\ae}b{\o} et al. \cite{Saebo_eParticipation}. E-Participation is generally understood as ``joining in'', taking part or taking role and is normally associated with political deliberation or decision making. It can take place in and outside of political processes and makes use of Internet communication technologies. The work of S{\ae}b{\o} et al. was continued by Medaglia in 2012 \cite{Medaglia2012_eParticipation}. He reviewed 122 e-Participation articles published between 2006 and 2011. Both S{\ae}b{\o} et al. and Medaglia classified the e-Participation research domain as follows:  ``Actors'' conduct ``activities'', which have results called ``effects''. ``Activities'' are influenced by other ``activities''. Finally ``evaluation'' of the ``effects'' is used to improve future ``activities''. Susha et al. \cite{Susha2012_eParticipation} did an analog classification of e-Participation research. Their categories are ``stakeholders'', ``environment'' and ``applications and tools''.\\
In 2004, Macintosh \cite{Macintosh2004_eParticipation_characterization} defined ``e-Democracy'' as ``the use of Internet communication technologies to support the democratic decision-making processes''. In order to better understand this definition, she described several key dimensions of e-Participation. Analogous to the participation ladders of Arnstein, Wiedemann, and Connor \cite{Arnstein1969_citizen_participation,Wiedemann1993355,Connor1988_new_ladder}, the level of participation was one of these key dimensions. Additionally, the stage of policy-making in which the citizens are engaged should be considered carefully according to Macintosh. Who should be engaged by whom, which technologies are used, rules and duration of engagement were other important key dimensions. Finally, evaluation and outcomes should be reflected thoroughly.\\
Above the one-directional ``Informing'' stage lies bilateral communication in form of discussions and dialogs, which also serves as base for the so called ``e-Participation'', ``e-Government'' and ``e-Democracy''. Kent and Taylor \cite{Kent1998_dialogic_relationships_through_www} proposed a theoretical framework for building dialogic relationships through the Internet in 1998. In their understanding, dialogic communication is ``any negotiated exchange of ideas and opinions''. Participants in a dialog do not necessarily have to agree, but participate to reach a mutually satisfying position. Dialogs are about creating shared subjective views and focuses on the attitude towards each other. According to Kent and Taylor, the Internet aids this process by allowing both synchronous and asynchronous means of communication.\\
A three step procedure to apply and organize public participation above the ``Informing'' stage was proposed by Renn et al. in 1993 \cite{Renn1993_participation}. Values and concerns of citizens and stakeholders are structured into an hierarchy, which are discussed and judged by experts. The results are evaluated in citizen panels. The authors divided participants into three groups. Stakeholders, which bring concerns and interests, as well as metrics for evaluation. Experts analyze and provide related data and help to uncover functional relationships between options and their impacts. Finally, citizens, as potential victims and benefactors, assess the results and outcomes of the proposals of the other groups.\\
Online forums as a tool for mass deliberation were evaluated by Wright and Street \cite{Wright2007_deliberation_design}. They found that current online forums are not designed for social interaction. Furthermore, both proponents and opponents of mass deliberation through online forums miss the role played by design in facilitating or thwarting deliberation and tend to tread information technology as given and determinant. Wright and Street conclude that technology is both shaped by and also shaping political discussions on the Internet and recommend to focus on moderation of discussions in online forums.\\
Jaeger \cite{Jaeger2005_deliberate_democracy_and_egovernment} explored potential social impediments of the increasing use of Internet communication technologies in e-Government and e-Participation. Through a survey of existing information studies in public policy, law and governance, he found that the Internet ``poses real dangers of creating or fostering social fragmentation''. According to Jaeger, the asynchronous nature of interaction makes it easy to avoid people with contrary opinions. Finding people with same opinions is easier than in real world situations, like public hearings. Set-ups like these tend to facilitate group polarization. Jaeger stated, that members of such groups with shared beliefs or shared identity are more likely to be inclined to extreme opinions. Anonymity of Internet-discussions further increases tendencies to take extreme positions. Apart from these concerns, he clearly saw an advantage in applying Internet communication technologies, which ``could potentially benefit the health of the entire democracy''.\\
Public deliberation through decision support with pro/con lists were evaluated by Kriplean et al. \cite{Kriplean2012_Considerit}. Their system allows to create public pro/con lists in which other participants pro/con points could be included. The lists then created an overview of all stances and contributions. This indirect discussion mitigated ``political identity and flaming'' by forcing the users to reflect on their standpoints and disallowing any portrayal of political affiliations.\\
In 2009, Collins and Ison \cite{Collins2009_social_learning} suggested that the traditional ladders of Arnstein and colleagues are too dominating in citizen participation in policy discourses. They propose to focus more on the social learning aspect of participation, which builds upon convergence of goals, co-creation of knowledge and change of behavior and actions. If all participants and stakeholders would apply this ``social learning'', understanding would be supported.%\todo{mehr!}

\subsection{Geographic information systems in decision making}
\label{subchap:gis_stuff}
Since the development of geographic information systems (GIS), they were used to support experts in making decisions. An early overview over spatial decision support systems (SDSS) is made by Densham \cite{densham_sdss} in 1991. Standard decision support systems were developed to support experts in solving ``ill-structured'' problems where the definition of the problems is difficult. By combining existing data with statistical models, solution space can be explored. Through the ability to give different factors different weights, multiple decision-making styles are supported. The introduction of spatial capabilities into an decision support systems brings several benefits and allows to solve semi-structured spatial problems. Additional features of spatial decision support systems are the ability to store and illustrate spatial relations, the analysis through statistical methods and to create maps as output. Densham proposes a ``SDSS generator'' for future development, which combines analysis tools, GIS functionalities and database management systems. Furthermore, the differentiation between ``Objective'' and ``Map'' space is deemed important. Users must be able to view both spaces simultaneously, which update each other when changes are made to either one.\\
Following the developments in spatial decision support systems, the concept of Public Participatory GIS (PPGIS) and Participatory GIS (PGIS) emerged. Both focus on enabling non-expert groups to apply GIS technologies to strengthen involvement in decision making.\\
In 2006, Sieber \cite{Sieber2006_PublicParticipationGIS} traced the social history of PPGIS and lists mayor themes found in PPGIS research. ``Place and People'' relate to the question ``who should be participating in PPGIS projects''. ``Technology and Data'' cover representation of knowledge, accessibility of data and appropriateness of information. ``Process'' focuses on decision-making structures and processes, participation and communication in the policy making process and system implementation and sustainability. The ``Outcomes and Evaluation'' measures goals and results. The themes are similar and related to Macintosh's \cite{Macintosh2004_eParticipation_characterization} key dimensions of e-Democracy. Sieber also notes that, although PPGIS have been constructed and practised by a broad set of actors in multiple research disciplines, GIS alone is controversially attributed to enhance public participation and deliberation. Similar observations were made by Obermeyer \cite{obermeyer1998evolution}, Craig et al. \cite{Weiner2002_Participation_and_GIS_eigentlich_Craig} and Blaschke \cite{Blaschke2004_PGIS_critically_revised}. According to Blaschke, broad participation of the public is the distinction of PPGIS from SDSS. He also claims that public participation does not automatically lead to better decisions.\\
Schlossberg and Shuford tried to delineate the terms ``public'' and ``participation'' in PPGIS through a literature review \cite{Schlossberg2005_PPGIS}. They defined a matrix with ``Domain of Participation'' and ``Domain of Public'' as axes. The ``Domain of Participation'' axis contains participation techniques, which the actors on the ``Domain of Public'' axis perform. Axes are ordered from simple to complex. The ``Domain of Participation'' dimension is leaned against the various participation ladders defined by Arnstein and colleagues \cite{Arnstein1969_citizen_participation,Wiedemann1993355,Connor1988_new_ladder}. They populate their matrix with four scenarios.\\
Voss et al. \cite{Voss2004_Evolution_PGIS} describe the combination of a structured argumentation tool, Dito and a spatial decision support system, CommonGIS. The integration of the systems was made gradually over the course of three experiments exploring different aspects of spatial discussions. Through these, they identified multiple conceptual, technical and user interface requirements. Due to their approach to combine two existing systems, several technical issues emerged.\\
Alongside with an overview of PGIS systems, Jankowski \cite{Jankowski2005_community_based_pgis} analyzed two studies in PGIS water resource decision making. In both cases, participants had to make suggestions for water source protection sites. The second case featured shared displays for the conveying of spatial data. Jankowski found that trust in state supplied data was not high and that the technology sometimes reduced creativity, especially if it was hard to use. He also found that citizens need to have real interest and insight in information in order to support decisions effectively. Future systems should focus on how technology can be used without reducing the creativity of participants.

\subsection{Geographic discussion tools}
\label{zweivier}
Rinner \cite{Rinner_ArgumentationMaps} picked up on the idea of PPGIS but focused on the combination of e-Participation principles (use of Internet communication technologies to involve citizens in discussions about decision making processes) with geographic information systems. After a review of discussion and collaboration tools, he found that asynchronous discussions were not considered during planning procedures. Following this, Rinner proposed the concept of ``Argumentation mapping'', for which he described four use cases which outline the design of argumentation mapping. The GIS functionality of spatial data presentation is used for navigation in argumentation maps. Similar to Goodchild's concept of volunteered geographic information \cite{goodchild2007citizens}, Rinner translates input of geographic data to participation. Retrieval and analysis functions of GIS can be used for exploration and evaluation in argumentation maps. These use cases can only be used fully, if an object-based model of geographically referenced argumentation is used.\\
Since then, multiple implementations of argumentation mapping and PPGIS systems with multiple research goals and outcomes have emerged.\\
In 1999, Kingston et al. \cite{kingston1999gis} developed a PPGIS system to enable a digital version of annotated map pins. The system allowed to create comments with a spatial reference. It lacked a structured discussion support and allowed only one spatial reference per comment.\\
A first implementation of the argumentation map idea was made by Ke{\ss}ler et al. \cite{Kessler2005_ArgumentationMapPrototype}. He proposed a set of requirements and design guidelines for argumentation maps and analyzed different options for linking maps and discussion contributions. The implementation should address the two main issues of the analyzed systems: User friendliness and support of semi open standards. Ke{\ss}ler's implementation featured separate discussion and map components. According to him, requirements for the discussion component were an ``Integrated user interface'', which supports ``Structured discussion'', an ``Integrated database, [which supports] many-to-many relationships'' and ``Access control, security''. Requirements for the map component were ``Integrated user interface'', ``Common (Web)mapping functions'', ``Integrated database, many-to-many relationships'' and ``Customization by Provider''. An important aspect is the throughout use of open standards to ensure reusability. The system allowed users to upload ESRI shapefiles\footnote{\url{http://www.esri.com/library/whitepapers/pdfs/shapefile.pdf}} and to create point features in a map. These spatial features could be annotated with texts and labels. An evaluation of the Argumentation map prototype implemented by Ke{\ss}ler et al. with HCI principles was then conducted by Sidlar and Rinner \cite{sidlar_argumap_2007}. They specifically investigated learnability, memorability and user satisfaction. Evaluation yielded generally positive results. A list of recommendations for improvements were given by Sidlar and Rinner. They also conducted an utility assessment \cite{Sidlar2009-AssessmentMapGeocollaborationTool} of the Argumentation Map prototype. The authors state that the utility of application is often seen as given, thus developing a framework for investigating participatory GIS utility. Utility was measured by calculating ratios of actual use of argumentation mapping functions over the potential use of those functions. After applying their framework to the prototype, Sidlar and Rinner reached to the conclusion that every aspect of the Argumentation map prototype was used but ``not always to its fullest''.\\
A combination of a spatial data infrastructure as extension to the Argumentation map prototype from 2005 is discussed by Ke{\ss}ler et al. \cite{Kessler2005_Conflict_Resolution}. As PPGIS applications naturally make extensive use of geographic information, the authors argue that participatory discussions in PPGIS could benefit from readily available geospatial data. They also implemented basic analysis functionalities and applied their idea in a conflict resolution context. Ke{\ss}ler et al. found that spatial data infrastructures can ease and simplify the setup of a PPGIS as geospatial objects to discuss are readily available.\\
Two systems in PPGIS context were developed by Carver et al. \cite{Carver2001_PPGIS_Cyberdemocracy} to find out how the Internet and GIS can be used together in order to provide the public with becoming more involved in environmental decisions. While the first system only conveyed information about a natural reserve, the other was used for collecting citizen ideas to improve a small town. They identified technological issues with implications for decisions with citizen involvement. Although access to geospatial (and nongeospatial) data and tools may empower the general public to contribute to decision making processes, participation is held back by inequalities of citizens in their computer literacy. The authors also list principles for future implementations of web based PPGIS applications.\\
In their article ``Design of a GIS Enabled Online Discussion Forum for Participatory Planning'' Tang et al. focus on effective communication and mutual understanding \cite{Tang2005_PPGIS_discussion_forum}. They reviewed eleven PPGIS applications for participatory planning. Among these eleven were systems of Carver \cite{Carver2001_PPGIS_Cyberdemocracy} and Rinner \cite{Rinner_ArgumentationMaps,Kessler2005_ArgumentationMapPrototype}. The eleven PPGIS application were evaluated for the following criteria: experts should be able to play the facilitators role, exchange of views must be supported as well as the documentation and sharing of the evolution of ideas, made decisions should be shown in the context of related decisions and, the effectiveness of communication about spatial context. The authors revealed several shortcomings of the reviewed applications and discussed the development of GeoDF, which is a GIS-enabled online discussion forum to enable citizens of a small town in Canada to provide in-depth feedback to the government. Being the co-authors of Tang et al. \cite{Tang2005_PPGIS_discussion_forum} Zhao and Coleman \cite{zhao2006geodf} summarize the process and lessons learned in implementing the GeoDF prototype. In GeoDF, textual components and spatial context have a one to one relationship. The spatial context consists of map extent, visible layers, annotations and sketches from both the contributor and other contributors. Due to the use of proprietary technology, issues of data availability, licensing, maintenance, re-usability and interoperability emerged.\\
Multi-criteria evaluation (MCE) is a computation method to bring alternative solutions of a problem into an order by applying metrics to the different parameters of the problem. MCE is seen as an alternative to ``hard'' boolean filters used in SDSS. In order to answer if geovisualization in multi-criteria evaluations can support spatial decision making, Rinner \cite{Rinner2007_geovis_decisionsupport} conducted two studies where users could evaluate different outcomes of a problem by manipulating parameters with sliders. The result of the parameter manipulation was immediately visible on both a map and diagrams. Rinner found, that his solution with sliders suppported the decision making process.\\
The development of a multi-criteria decision support system in combination with an argumentation map is described by Sim\~{a}o et al. \cite{Simao2009Webbased}. They used their system to educate their users about all outcomes of a collaborative planning process. Their application consist of a three tier architecture which the users are navigated through. After an information area, the actual MC-SDSS (multi-criteria spatial decision support system) is entered where the solution space could be explored. The last step is a map centric communication tool to record opinions about the explored solutions. Finally, most discussed solutions are assessed by experts. During the development, the authors identified problems of planning processes. Often the problems have many dimensions, making the definition of the problem statement beforehand really difficult. As well as expert knowledge, communication is key in finding the best solution.

\subsection{Argumentation mapping with Web 2.0 principles}
\label{zweifuenf}
The use of Web 2.0 principles and technologies for collaborative spatial decision-making was assessed by Rinner \cite{Rinner2009_Web2_argumap} et al. by re-implementing the original Argumentation map by Ke{\ss}ler \cite{Kessler2005_ArgumentationMapPrototype} called ``ArgooMap''. It allows users to submit place based comments and to respond to other comments. Only marker as spatial reference were allowed. The authors evaluated their thread based online map discussion forum with a simulation. Existing discussions were re-enacted in a sandbox with no user interaction.\\
An implementation of a multi-criteria decision analysis (MCDA) tool was described by Boroushaki and Malczewski \cite{Boroushaki2010_ParticipatoryGIS}. They took the ArgooMap prototype of Rinner et al. \cite{Rinner2009_Web2_argumap} and extended it by adding multi-criteria decision support. Automatically generated problem solution alternatives could be discussed through the ArgooMap part. A follow up paper of Boroushaki and Malczewski evaluated their MCDA implementation through a study with citizens \cite{Boroushaki2010_Consensus_measurement}. They identified bringing together experts and laypeople as a main challenge of GIS-based spatial decision-making tools. The goal should always be to reach a high consensus among decision-makers and citizens. Another evaluation of the MCDA implementation of Boroushaki and Malczewski was conducted by Meng and Malczewski \cite{Meng2010_ArgooMap_evaluation}. They tested the usability of the front-end ArgooMap with a user study where participants had to choose and discuss possible parking facility sites. They measured usability through perceived user effectiveness, efficiency and satisfaction. They found that effectiveness has a strong influence on how long a user stays on the website and efficiency impacts on user visit numbers, page views and interaction with others. Meng and Malczewski suggested that user testing should be considered in system-design processes of web-PPGIS.\\
General concepts and methods for designing Web 2.0 community-based geoportals are presented by Longueville \cite{Longueville2010_community_based_geoportals_web20}. He sees community-based geoportals as advanced spatial data infrastructures which should allow users to gather and share resources, organize themselves into groups and to create resources collaboratively. Each resource should also include meta-resources like popularity, tags and comments. Longueville recommends to modularize applications to create both human and machine readable interfaces.\\
Cherubini and Dillenbourg \cite{Cherubini2007_shared_maps} tested a chat system that allows its users to create spatial references either through displaying the chat contributions directly on a map or referencing map features through symbols in the text. They found in an user study, that the references provided a tool to reinforce the reference frame of the conversation. Also, explicit referencing made communication more efficient. Participants used fewer sentences with fewer words. Hopfer and MacEachren \cite{Hopfer2007_Communication} applied a group communication theory, the Collective Information Sharing (CIS) bias, to a geospatial annotation tool. The authors found that the ability to annotate map-based displays can ease spatial communication tasks and enables participants of a discussion to move beyond ``what are we talking about'' to ``who knows what about an area and how this could be of use''. Therefore visualizing all information spatially will enable users to see gaps and thoroughly covered areas and reduce repeat of known information. ``Information that is made visually explicit is more likely to be recalled and, therefore, discussed'' \cite{Hopfer2007_Communication}. The authors give design recommendations in applying CIS to the design of a geospatial collaboration tool. Cai and Yu \cite{Cai2009_spatial_annotation_deliberation} discussed the challenges of spatially enabled public deliberation. In their opinion, PPGIS merges analytical capabilities of GIS with deliberative capabilities of discussion forums. They distinguished two deliberation types, cognitive and practical deliberation. Similarly to Arnsteins' ladder, they define five phases of geodeliberative dialogs. After a stage of ``Briefing and introduction'', geodeliberative dialogs' purpose is to ``Elaborate on problems and issues''. Next it allows to ``Contribute personal knowledge, experiences, and stories'', which then aids ``Development of public judgment and creation of a common ground''. The final stage is comprised of ``Generating and evaluating alternative courses of actions''.\\
A set of non-functional requirements of scalable, reliable and easy-to-maintain applications in a cloud computing context are given by Sani and Rinner \cite{Sani2011_Scalable_Argumap}. They state that system security, performance and scalability, robustness, availability and fault tolerance, maintainability and portability, modifiability and testability are the key non-functional requirements when developing a scalable argumentation mapping tool.\\
Although described in their respective literature, very few systems are readily available for an examination of their functionality and mode of operation. However, there are several non-scientific implementations of spatially enhanced discussion tools in existence.\\
The project ``Nexthamburg''\footnote{\url{http://www.nexthamburg.de/}, Kulus et al. \cite{Kulus_nexthamburg}} enabled citizens of Hamburg to suggest ideas for its urban development. The ideas were located on a map and other citizen could comment on the suggestions and ideas. Experts and decision makers then evaluated the ideas to implement them. The concept spawned a similar project in Kassel\footnote{\url{http://www.nextkassel.de/}}. The suggestions and ideas are only loosely tied to a location.\\
The ``Shareabouts''\footnote{\url{http://openplans.org/shareabouts/}} and ``CollaborativeMap.org''\footnote{\url{http://www.collaborativemap.org/home/}} applications use a spatial approach for gathering input of citizens. Although contributions are directly related to a spatial location, only one location per suggestions can be referred to.

%\subsection{Evaluation Methodology}
%This sub-section will explore the 
%Usability of PPGIS Haklay and Tobón \cite{Haklay2003_ppgis_usability_evaluation}


 %\cite{Pocewicz2012_Paper_vs_PPGIS} \cite{Aggett2006_evaluation_dss_ppgis}
 %\cite{Damianos1999_evaluation_collaborative}
 
%semi-structured Interviews: 
%\cite{helfferich2005} Transcription rules: \cite{kuckartz2007}  
%Focus group conduction: 
%\cite{asbury1995overview} 
    %What is a focus group: \cite{carey1994capturing} 
%\cite{morgan1996_focus_groups}
