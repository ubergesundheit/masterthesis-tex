\section{Related Work}
\label{chap:related_work}

By defining eight levels of increasing citizen participation, Arnstein \cite{Arnstein1969_citizen_participation} coined the term ``ladder of citizen participation'' which, since then, has been adapted and modernized \cite{Cai2009_spatial_annotation_deliberation,Connor1988_new_ladder,Collins2009_social_learning,carver2003future,you2009_participatory_map_based} by several authors. She claims that citizen participation is a term for the redistribution of power among citizen. Each step on Arnstein's ladder leads to more and more empowerment of citizens, by requiring the implementation of the previous steps before the higher levels can be reached.

In 2009, Collins and Ison \cite{Collins2009_social_learning} suggested that Arnstein's ladder is heavily dominating citizen participation in policy discourses. They proposed to focus more on the social learning aspect of participation which 

Participation as democratic process \cite{Collins2009_social_learning} \cite{Renn1993_participation}

\cite{Kent1998_dialogic_relationships_through_www}
\cite{Reddick2005_Citizen_interaction_with_egovernment}
 
\cite{Wright2007_deliberation_design}

eParticipation and egovernment \cite{Bimber1999_Citizen_communication_with_government} \cite{Jaeger2005_deliberate_democracy_and_egovernment} \cite{Macintosh2004_eParticipation_characterization}

Review and summarization in \cite{Saebo_eParticipation,Medaglia2012_eParticipation}


Participation with geographic information systems \cite{zhao2006geodf} \cite{Tang2005_PPGIS_discussion_forum} \cite{Rinner_ArgumentationMaps} \cite{Schlossberg2005_PPGIS} \cite{densham_sdss} \cite{Jankowski2005_community_based_pgis} \cite{Longueville2010_community_based_geoportals_web20} \cite{Rinner2009_Web2_argumap} \cite{sidlar_argumap_2007} \cite{Simao2009Webbased} \cite{Voss2004_Evolution_PGIS} \cite{Blaschke2004_PGIS_critically_revised} \cite{Sieber2006_PublicParticipationGIS}


Existing implementations\dots \cite{Rinner2007_geovis_decisionsupport} \cite{Boroushaki2010_ParticipatoryGIS} \cite{Kessler2005_ArgumentationMapPrototype} \cite{Kessler2005_Conflict_Resolution} \cite{Meng2010_ArgooMap_evaluation} \cite{Meng2010_WebPPGIS_Usability} \cite{Sani2011_Scalable_Argumap}


Frameworks for evaluation \cite{Walker2013Qualitative}

Evaluation techniques\dots 


\subsection{Public deliberation and eParticipation}

