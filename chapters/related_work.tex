\section{Related Work}
\label{chap:related_work}

By defining eight levels of increasing citizen participation, Arnstein \cite{Arnstein1969_citizen_participation} coined the term ``ladder of citizen participation'' which, since then, has been adapted and modernized \cite{Cai2009_spatial_annotation_deliberation,Connor1988_new_ladder,Collins2009_social_learning,carver2003future,you2009_participatory_map_based} by several authors. She claims that citizen participation is a term for the redistribution of power among citizen. Each step on Arnstein's ladder leads to more and more empowerment of citizens, by requiring the implementation of the previous steps before the higher levels can be reached.

Participation as democratic process \cite{Collins2009_social_learning}

\cite{Kent1998_dialogic_relationships_through_www}
\cite{Reddick2005_Citizen_interaction_with_egovernment}

eParticipation and egovernment \cite{Bimber1999_Citizen_communication_with_government} \cite{Jaeger2005_deliberate_democracy_and_egovernment}


Participation with geographic information systems \cite{Schlossberg2005} \cite{densham_sdss} \cite{Jankowski2005_community_based_pgis} \cite{Longueville2010_community_based_geoportals_web20}



Rinner\cite{Rinner_ArgumentationMaps}

Existing implementations\dots

Evaluation techniques\dots


\subsection{Public deliberation and eParticipation}

