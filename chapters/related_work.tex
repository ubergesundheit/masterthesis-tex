\section{Related Work}

%Ladder Arnstein + Wiedemann -> Information of citizen through egovernment (contacting etc) -> eParticipation (Reddick, zum beispiel auch Renn1993) -> dialogic stuff (Kent1988, new ladder from collins) -> Wright


\label{chap:related_work}
%%% Ladder
By defining eight levels of increasing citizen participation, Arnstein \cite{Arnstein1969_citizen_participation} coined the term ``ladder of citizen participation'' which, since then, has been adapted and modernized \cite{Connor1988_new_ladder,carver2003future,Collins2009_social_learning,you2009_participatory_map_based,Cai2009_spatial_annotation_deliberation,Macintosh2004_eParticipation_characterization} by several authors. She claims that citizen participation is a term for the redistribution of power among citizen. While the first two rungs on the ladder are denoted as ``Nonparticipation'', involvement of citizens begin at the third level ``Informing''. ``Consultation'' and ``Placation'' are also labeled as ``Tokenism''. The last three steps (``Partnership'', ``Delegated Power'', ``Citizen Control'') are then declared as ``Citizen Power'' by Arnstein.

Structured similarly, Wiedemann and Femers \cite{Wiedemann1993355} proposed a ladder which is based on the premise that information given to the public and amount of possible participation is collateral. By requiring the implementation of the previous steps before the higher levels can be reached, Each step on their ladder leads to more and more empowerment of citizens. Wiedemann and Femers conclude that general understanding of an issue is a first step towards public participation.

In reaction to the ladder proposed by Arnstein, Connor \cite{Connor1988_new_ladder} constructs his version of a new ladder ``whose elements have a cumulative effect''. It is designed lead decision makers to apply techniques ``to prevent and resolve public controversy about various proposals''. \todo{HIER NOCH WAS SCHREIBEN}
%%% Information of citizens
As stated by Arnstein \cite{Arnstein1969_citizen_participation}, involvement of the public begins with the supply of sufficient information. \todo{PAPER ZU INFORMING?} This includes the publishing of information as well as contacting officials with suggestions and questions. Traditional contacting means were transformed to use modern information technologies. Reddick \cite{Reddick2005_Citizen_interaction_with_egovernment} described this process in citizen interaction as an improvement over traditional means to contact their government. Although often lacking interaction by only serving the information needs of citizens, the Internet enables citizens to skip ``street-level bureaucrats''. The author gives the recommendations, that focus should be laid on ease of use, user friendliness and marketing of online services. Reddick also warns of digital divide by excluding non-tech savvy population groups.

Because contacting officials is the most common act of political participation after voting, a comparison of traditional contacting methods with Internet based contact methods was conducted by Bimber in 1999 \cite{Bimber1999_Citizen_communication_with_government}. Specifically, he tried to answer if the medium of contacting officials matters by conducting a survey with 2021 participants. He found that the Internet is ``just'' incrementing the connection between citizen and officials, not ``revolutionizing'' it.
%%% eparticipation
A thorough survey of ``eParticipation'' literature was conducted by S{\ae}b{\o} et al. \cite{Saebo_eParticipation}. Through their review of 131 articles, they formulated an outline of ``eParticipation''. Participation is generally understood as ``joining in'', taking part or taking role. 



In 2004, Macintosh \cite{Macintosh2004_eParticipation_characterization} defined ``eDemocracy'' as ``the use of Internet communication technologies to support the democratic decision-making processes''. In order to better understand this definition, she described several key dimensions of eParticipation. Analogous to the participation ladders of Arnstein, Wiedemann and Connor \cite{Arnstein1969_citizen_participation,Wiedemann1993355,Connor1988_new_ladder}, the level of participation is one of these key dimensions. Additionally, the stage of policy-making in which the citizens are engaged should be considered carefully according to Macintosh. Who should be engaged by whom, technologies used, rules and duration of engagement are other important key dimensions. Finally, evaluation and outcomes should be reflected thoroughly.

Above the one-directional ``Informing'' stage lies the bilateral communication in form of discussions and dialogs which also serves as base for the so called ``eParticipation'', ``eGovernment'' and ``eDemocracy''. Kent and Taylor proposed a theoretical framework for building dialogic relationships through the Internet in 1998 \cite{Kent1998_dialogic_relationships_through_www}. In their understanding, dialogic communication is ``any negotiated exchange of ideas and opinions''. Participants in a dialog not necessarily have to agree, but are in it to reach a mutually satisfying position. Dialog is about creating shared subjective views and focuses on the attitude towards each other. The Internet aids this process by allowing both synchronous and asynchronous means of communication.

A three step procedure to apply and organize public participation above the ``Informing'' stage was proposed by Renn et al. in 1993 \cite{Renn1993_participation}. Values and concerns of citizens and stakeholders are structured into an hierarchy which then are discussed and judged by experts. The results are then evaluated in citizen panels. Participants can be divided into three groups. Stakeholders which bring concerns and interests, as well as metrics for evaluation. Experts serve as base for related data and to uncover functional relationships between options and their impacts. Finally, citizens, as potential victims and benefactors, assess the results and outcomes of the proposals of the other groups.

Online forums as a tool for mass deliberation were evaluated by Wright and Street \cite{Wright2007_deliberation_design}. They found, that current online forums are not designed for social interaction. Furthermore, both proponents and opponents of mass deliberation through online forums miss the role played by design in facilitating or thwarting deliberation and tend to tread information technology as given and determinant. Wright and Street conclude that technology is both shaped by and shaping political discussion on the Internet and recommend to focus on moderation of discussions in online forums.

Jaeger \cite{Jaeger2005_deliberate_democracy_and_egovernment} explored potential social impediments of the increasing use of Internet communication technologies in eGovernment and eParticipation. Through survey of existing information studies in public policy, law and governance, he found that the Internet ``poses real dangers of creating or fostering social fragmentation''. Through the asynchronous nature of interaction, avoiding people with contrary opinions and finding people with same opinions is easier than in real world situations, like a public hearing. Set-ups like these tend to facilitate group polarization. Jaeger states, that members of such groups with shared beliefs or shared identity are more likely to tend to extreme opinions. Anonymity of Internet-discussions further increases tendencies to take extreme positions. Apart from these concerns, he clearly sees an advantage in applying Internet communication technologies and ``could could potentially benefit the health of the entire democracy''.




In 2009, Collins and Ison \cite{Collins2009_social_learning} suggested that the traditional ladders of Arnstein and colleagues are too dominating in citizen participation in policy discourses. They propose to focus more on the social learning aspect of participation which builds upon convergence of goals, co-creation of knowledge and change of behavior and actions. If all participants and stakeholders apply this ``social learning'', understanding is supported.









Review and summarization in \cite{Saebo_eParticipation,Medaglia2012_eParticipation}


Participation with geographic information systems \cite{zhao2006geodf} \cite{Tang2005_PPGIS_discussion_forum} \cite{Rinner_ArgumentationMaps} \cite{Schlossberg2005_PPGIS} \cite{densham_sdss} \cite{Jankowski2005_community_based_pgis} \cite{Longueville2010_community_based_geoportals_web20} \cite{Rinner2009_Web2_argumap} \cite{sidlar_argumap_2007} \cite{Simao2009Webbased} \cite{Voss2004_Evolution_PGIS} \cite{Blaschke2004_PGIS_critically_revised} \cite{Sieber2006_PublicParticipationGIS}


Existing implementations\dots \cite{Rinner2007_geovis_decisionsupport} \cite{Boroushaki2010_ParticipatoryGIS} \cite{Kessler2005_ArgumentationMapPrototype} \cite{Kessler2005_Conflict_Resolution} \cite{Meng2010_ArgooMap_evaluation} \cite{Meng2010_WebPPGIS_Usability} \cite{Sani2011_Scalable_Argumap}


Frameworks for evaluation \cite{Walker2013Qualitative}

Evaluation techniques\dots 


\subsection{Public deliberation and eParticipation}

