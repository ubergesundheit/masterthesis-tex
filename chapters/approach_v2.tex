\section{Approach}
\label{chap:approach}

As seen in \hyperref[chap:related_work]{section \ref{chap:related_work}}, research proposes various theoretical frameworks and implementations in the area of argumentation mapping. This section introduces the concept of \textit{DialogMap} and gives information about the context and background of the developed prototype (sub-section \ref{sub:dialogmap}), describes general concepts of the implemented prototype (\hyperref[sub:design]{sub-section \ref{sub:design}}) and lay out implementation details (\hyperref[sub:implementation]{sub-section \ref{sub:implementation}}).

\subsection{DialogMap}
\label{sub:dialogmap}


\begin{table*}
\centering
\caption{Main requirements for online spatial discussion platforms derived from research presented in sections \ref{subchap:gis_stuff}, \ref{zweivier} and \ref{zweifuenf}.}
\label{tab:requirements}
\begin{tabular}{|p{7cm}|p{7cm}|l|} \hline
\textbf{Requirement} & \textbf{Impact} & \textbf{Main Sources}\\ \hline

a) Explicit spatial references in discussion contributions \label{req:a} & Clarify denotation of spatial relationships and make communication more efficient & \cite{Rinner_ArgumentationMaps,Cherubini2007_shared_maps}\\ \hline

b) Multiple connections to locations and locations created in the context of other contributions in one contribution \label{req:b} & More fine grained references and further clarification. Many-to-many connections possible & \cite{Kessler2005_ArgumentationMapPrototype,Voss2004_Evolution_PGIS,you2009_participatory_map_based,Cai2009_spatial_annotation_deliberation}\\ \hline

c) Tags and categories alongside discussion contributions can be attached \label{req:c} & Support statements of contributions & \cite{Longueville2010_community_based_geoportals_web20,Kessler2005_ArgumentationMapPrototype,Kessler2005_Conflict_Resolution,Tang2005_PPGIS_discussion_forum,zhao2006geodf,you2009_participatory_map_based,Cai2009_spatial_annotation_deliberation}\\ \hline

d) Two-way highlighting of geo-features and contributions \label{req:d} & Visualization of relationships & \cite{Cai2009_spatial_annotation_deliberation,Sidlar2009-AssessmentMapGeocollaborationTool}\\ \hline

e) Filter and sort contributions by custom rules \label{req:e} & Custom overviews, visualization of gaps and thoroughly covered areas & \cite{Voss2004_Evolution_PGIS,you2009_participatory_map_based,Hopfer2007_Communication}\\ \hline

f) Easy user interface \label{req:f} & No impediments for non-technical users. No suppression of participation & \cite{Rinner2009_Web2_argumap,Jankowski2005_community_based_pgis,Tang2005_PPGIS_discussion_forum,zhao2006geodf,you2009_participatory_map_based,Carver2001_PPGIS_Cyberdemocracy}\\ \hline

\end{tabular}
\end{table*}


Following the concept of Rinner's Argumentation Map \cite{Rinner_ArgumentationMaps} and the idea of supporting public deliberation through spatially enhanced dialogs, the concept of \textit{DialogMap} was developed for this thesis. Following the recommendations and suggestions of the research presented in sections \ref{subchap:gis_stuff}, \ref{zweivier} and \ref{zweifuenf}, a set of functional and non-functional requirements were derived (see \hyperref[tab:requirements]{Table \ref{tab:requirements}}). The remainder of this sub-section describes functionalities of \textit{DialogMap} together with the suggestions and recommendations of the related work described in sections \ref{subchap:gis_stuff}, \ref{zweivier} and \ref{zweifuenf}.\\
\textit{DialogMap} is a spatial online discussion platform to support geo-deliberative dialogs \cite{Cai2009_spatial_annotation_deliberation} performed by citizen initiatives. It enables users to make explicit spatial references in their discussion contributions to clarify denotation of spatial relationships. This is achieved by allowing the interlinking of one or multiple words with a location or area on the map of the application \cite{Rinner_ArgumentationMaps}. It is possible to create multiple connections between locations in one contribution, as well as referring to locations created in the context of another contribution to allow more fine grained references. Although recommended by multiple authors, many-to-many connections between locations and words is only possible in very few \cite{Kessler2005_ArgumentationMapPrototype,Voss2004_Evolution_PGIS,you2009_participatory_map_based,Cai2009_spatial_annotation_deliberation} of the implemented systems. It is also possible to create multiple hyperlinks on words or multiple words in the contributions' text. The creation of the references and the text can occur in any order \cite{Voss2004_Evolution_PGIS}. It was found that through allowing explicit spatial references, exchange of information can be made more efficient \cite{Cherubini2007_shared_maps}.\\
Contributions comprise not only of a text with spatial references and hyperlinks, but is composed of multiple other attributes and properties \cite{Longueville2010_community_based_geoportals_web20,Kessler2005_ArgumentationMapPrototype,Kessler2005_Conflict_Resolution}. Users are able to specify tags and a category and to attach an image to each contribution \cite{Tang2005_PPGIS_discussion_forum,zhao2006geodf,you2009_participatory_map_based,Cai2009_spatial_annotation_deliberation} to support their statements in their contributions. The tags and category of the contribution affect the visual appearance of the geo-features created in its context.\\
\textit{DialogMap} structures contributions chronologically \cite{Cherubini2007_shared_maps,you2009_participatory_map_based}. Contributions can be created as a reply to a contribution or as a new topic. It is possible to edit contributions along with its properties and created references. Furthermore, it is possible to mark contributions as deleted, which results in a visual marking of the textual representation as well as the fading of the geo-features created in the context of the contribution \cite{Hopfer2007_Communication}.\\
Creation of contributions is allowed only after a successful authentication to the system. Users can either register a user account with an e-mail/password combination or authenticate themselves through the third party authentication providers Twitter, Facebook and Google \cite{Sani2011_Scalable_Argumap,chun2014usability}.\\
The user interface of \textit{DialogMap} features a map and an area to display the textual representations of a contribution. Spatial and textual representations of the contributions are interlinked by a two way highlighting to indicate the relationships between geo-features and contributions \cite{Cai2009_spatial_annotation_deliberation,Sidlar2009-AssessmentMapGeocollaborationTool}.\\
\textit{DialogMap} allows to filter and search for contributions by categories, tags and by free text. This enables users to create their own contribution overviews \cite{Voss2004_Evolution_PGIS,you2009_participatory_map_based}, and allows users to see gaps and thoroughly covered areas \cite{Hopfer2007_Communication}.\\
The importance of an easy user interface was mentioned by multiple authors \cite{Rinner2009_Web2_argumap,Jankowski2005_community_based_pgis,Tang2005_PPGIS_discussion_forum,zhao2006geodf,you2009_participatory_map_based}. As users are likely non-technical \cite{Cai2009_spatial_annotation_deliberation}, the user interface should not discriminate them, thus suppressing participation \cite{Carver2001_PPGIS_Cyberdemocracy}.\\
\hyperref[tab:requirements]{Table \ref{tab:requirements}} lists the most important requirements along with implications and sources.\\
The \textit{DialogMap} concept was developed first and foremost to support public participation on a discussion level. \hyperref[fig:my_ladder]{Figure \ref{fig:my_ladder}} shows the theoretical participation level capabilities of \textit{DialogMap} derived from Arnstein and colleagues \cite{Arnstein1969_citizen_participation,Wiedemann1993355,Connor1988_new_ladder}. Like Wiedemann and Femers \cite{Wiedemann1993355}, each level requires the implementation of the previous level. Informing and educating the public is the first step required for citizens to form an opinion. This then allows citizens to contribute information and proposals to the topic. Eventually, if the means for discussing a topic exists, consensus can be reached. Although \textit{DialogMap} does not contain functionalities for establishing a consensus, discussion participants are able to do this manually by writing contributions.\\
In order to test the initial idea of supporting public deliberation through spatially enhanced dialogs, the development of an online spatial discussion platform prototype was started. The development was done in an iterative and agile approach. Through the creation of working iterations of the \textit{DialogMap} concept, early feedback possible. The possibility to evaluate the concept in a real world use case with users of a citizen initiative started by a university group presented itself early in the development process. The further development was then conducted with practical advice from three members of the citizens initiative started by a university group. Their input ranged from general suggestions to opinions of specific features. Thus, the following sections describe the distinct outcome of their input to the developed prototype.

\begin{figure}[!h]
    \centering
    \includegraphics[width=1\columnwidth]{images/my_ladder}
    \caption{Levels of participation of the \textit{DialogMap} concept. The levels are adapted from research of Arnstein and her colleagues. The amount of participation raises from bottom to top. Each consecutive level is established on the level below it.}
    \label{fig:my_ladder}
\end{figure}


\subsection{Application Design}
\label{sub:design}

\begin{figure}[!h]
    \centering
    \includegraphics[width=1\columnwidth]{images/data_structure}
    \caption{Schema of the underlying data structure of the prototype. Time and id fields are omitted for brevity. \textit{Contributions} reference other \textit{contributions}, spatial \textit{features}, \textit{references}, and \textit{users}. \textit{References} can be either a hyperlink or refer to \textit{features} created in other \textit{contributions}.}
    \label{fig:data_structure}
\end{figure}

Internally, the prototype uses few data models. In the configuration used for this thesis, a contribution contains a title, description, two categories, a tags field, a favored counter, an optional time restriction field for start and ending times, an optional image, an optional reference to a parent contribution and optional references to child contributions. The parent and child contribution references create a simple parent-child connection between contributions, as children inherit the categories, tags, time restriction and title. A contribution serves both as a start of a topic and as response to a topic. It also contains references to features, references to feature references and references to URLs.\\
Features are geospatial entities with a location and a reference to its contribution.\\
Feature references contain a title, which can differ from the original title of the feature and a reference to a feature. URL references contain hyperlinks and a description of the hyperlink. Text typed for the description of a contribution is specially encoded to mark hyperlinks and references. Additionally, each contribution stores the ids of the users who favored it.\\
Registered users can create contributions in the manner of creating topics or writing responses to existing topics. The required credentials for registering are an e-mail address and a name.\\
\hyperref[fig:data_structure]{Figure \ref{fig:data_structure}} depicts entities and relations in a generalized data structure diagram.

\begin{figure}[!h]
    \centering
    \includegraphics[width=1\columnwidth]{images/screenshot}
    \caption{Screenshot of the front page of \textit{DialogMap} with active highlight of a contribution and spatial feature. Spatial features are highlighted through a distinct circle. Corresponding contributions are highlighted and scrolled to the top in the sidebar. The highlighting mechanism can be triggered by both pointing on markers and contributions with the mouse pointer.}
    \label{fig:screenshot}
\end{figure}

The user interface of the prototype consists of a map with a sidebar (See \hyperref[fig:screenshot]{Figure \ref{fig:screenshot}}). This allows the user to see both spatial and textual representation of contributions at one glance. On the right hand side sidebar is the input form for new contributions, filter options, sorting order selector and the list of contributions located. The input form consists of input fields for title, categories, time restriction, image and description. Furthermore, the description field allows the creation of spatial features and URL/feature references through connecting words with spatial representations or URLs (See \hyperref[fig:screenshot_create]{Figure \ref{fig:screenshot_create}}).\\
A text area for arbitrary text and multiple checkboxes allow to restrict the listed contribution as well as the geo-features displayed in the map. It is also possible to change the sorting order of the list of contribution through a drop down field. \hyperref[fig:screenshot_filter]{Figure \ref{fig:screenshot_filter}} depicts the expanded filter with several checkboxes enabled.

\begin{figure}[!h]
    \centering
    \includegraphics[width=1\columnwidth]{images/screenshot_create}
    \caption{Input form of \textit{DialogMap} for creating a new topic. Here, all input field are populated by the user. The user has created a spatial reference (green in the lowest box in the sidebar) and a hyperlink (blue box in the lowest box). The color as well as the icon of the marker is determined by the selected categories in the selection fields below the title (``Bürgerbrunch 2014'').}
    \label{fig:screenshot_create}
\end{figure}

\begin{figure}[!h]
    \centering
    \includegraphics[width=1\columnwidth]{images/screenshot_expanded}
    \caption{The \textit{DialogMap} frontpage with an expanded contribution rectangle. Through a click on the contribution rectangle, the description and image (if present) are displayed. Contributions do not show image and description by default to conserve visual space.}
    \label{fig:screenshot_expanded}
\end{figure}

The list of contributions contains colored rectangles representing the different topics. Each rectangle contains the title, time of writing, name of the author, categories, tags and the amount of times the contribution has been favored by users. It also contains a link, which navigates the user to the replies written to the topic. A click on the contribution rectangle expands it vertically, revealing the description and image of the current topic. This mechanism is implemented as description length can vary between contributions. By hiding description and image by default, even visual heights of the contribution rectangles are ensured. \hyperref[fig:screenshot_filter]{Figure \ref{fig:screenshot_expanded}} depicts an expanded contribution rectangle.\\
After clicking the ``reply'' link, only the selected topic and replies are shown in the sidebar in a chronological order. In this view, each contribution shows the description by default, as well as author and time and date of writing. The author of the contribution is able to edit and delete the contribution. Upon deletion, the user can enter a reason for deletion, which then will be displayed below the deleted contribution. The deletion is not destructive. The contribution as well as features created for the contribution are marked visually as deleted in order to retain both structure and meaning of the conversation. Other users are able to favor contribution to show interest or agreement.\\
The map view contains a base map and several markers and polygons in different colors and different icons in case of markers. These relate to the contributions and are connected through the references in the description of the contributions. Which spatial features are displayed is determined through the state of the sidebar. In the topics overview, only the features created for the starting contributions are displayed in order to prevent cluttering of the view-port. When only a topic and its replies are displayed in the sidebar, all features related to the topic and its replies are shown on the map.\\
To emphasize the relationship between a contribution and its spatial features, a two way highlighting has been implemented. Hovering over either a contribution-box, marked word or spatial feature on the map triggers visual highlighting on all related contributions, marked words and spatial features. This allows to quickly grasp the relationship between features and contributions. \hyperref[fig:screenshot]{Figure \ref{fig:screenshot}} shows an active highlighting initiated through a mouseover over a marker.\\
Users are able to use either traditional sing-up/sign-in methods or sign-in through different social log-in providers to authenticate to the system.

\begin{figure}[!h]
    \centering
    \includegraphics[width=1\columnwidth]{images/screenshot_filters}
    \caption{\textit{DialogMap} with expanded filter in the sidebar. Switching filter options on and off narrows down displayed markers on the map as well as displayed contributions in the sidebar. An input field allows to filter with arbitrary text.}
    \label{fig:screenshot_filter}
\end{figure}


\subsection{Implementation}
\label{sub:implementation}
\textit{DialogMap} has been implemented from scratch as a single-page web application using AngularJS\footnote{\url{http://angularjs.org/}} and Ruby on Rails\footnote{\url{http://rubyonrails.org/}}. The single-page structure was chosen in order to provide the user with a clear navigation between the overview and contribution answers. This also allows for a seamless browsing experience without full reloads of the page. AngularJS is a JavaScript framework with functionalities like templating, two-way data binding and DOM manipulation. It follows the model-view-controller pattern in order to bring server side paradigms to client-side development. AngularJS was chosen because of its popularity, extensibility and high number of available libraries. It also enables to wrap existing JavaScript libraries to be used in AngularJS context.\\
The mapping library Leaflet\footnote{\url{http://leafletjs.com/}} serves as base for displaying tiled web maps and geospatial data. User-facing web pages were developed using the programming languages CoffeeScript\footnote{\url{http://coffeescript.org/}}, Haml\footnote{\url{http://haml.info/}} and Sass\footnote{\url{http://sass-lang.com/}} to speed up the development. These languages compile to JavaScript, HTML and CSS respectively. Using such tools increases browser compatibility and correctness of the code, as the web page was developed with all major browsers in mind.\\
On the server side, components were developed using the Ruby on Rails framework with PostgreSQL\footnote{\url{http://www.postgresql.org/}}/PostGIS\footnote{\url{http://postgis.net/}} as data storage. PostGIS is a spatial database extension for PostgreSQL, which is at the time of writing this thesis, the best option for storing geospatial data. Ruby on Rails, a full-stack model-view-controller web framework, is used as a JSON serving application logic. It was chosen because of its maturity and high number of available libraries. Front- and backend of the prototype communicate in REST-API\footnote{Representational State Transfer Application programming interface} like manner. This allows for easily replaceable front- and backend application stacks.\\
Every software and libraries used for the development of the prototype are open source. The whole software is open sourced under the Apache License, Version 2.0 \footnote{\url{http://www.apache.org/licenses/LICENSE-2.0}} and is available through GitHub\footnote{\url{https://github.com/ubergesundheit/dialogmap}} for future examination and continuation.


