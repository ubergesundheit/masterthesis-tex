\section{Introduction}
Since their first appearance, Web 2.0 applications utilized their collaborative character to gather information and opinions from their users. Today, modern information technologies are ubiquitous in many aspects of daily life. Involving citizens in decision processes around public matters through such applications has formed the field of ``e-Participation''. Its premise is to strengthen democratic processes between citizens and its governments through said modern information technologies \cite{Saebo_eParticipation, Medaglia2012_eParticipation}. One of many aspects is public deliberation, which revolves around engaging citizens in dialogues about their surroundings and decision processes. In the past, these decision processes were performed by experts with specialized tools and domain-specific data. The increasing digitization of more and more spatial data led to the development of spatial decision support systems (SDSS) \cite{densham_sdss}. These systems were developed specifically to serve experts in public planning and often required professional training in order to be used effectively.

Following the idea to support decision processes using geographic information systems and involving citizens into decision processes, the research field of Participatory GIS were conceived \cite{Macintosh2004_eParticipation_characterization,Sieber2006_PublicParticipationGIS}. Rinner then proposed the idea of ``Argumentation Maps'' \cite{Rinner_ArgumentationMaps}. It features asynchronous discussions to support planning procedures and follows the assumption that locating arguments in a discussion spatially enhances the comprehensibility of the arguments. Since then, multiple implementations and extensions to Rinners ``Argumentation Map'' idea were developed and tested. User trust and representation of minorities \cite{Carver2001_PPGIS_Cyberdemocracy}, level of participation \cite{Steinmann2005_Combination_Ladder_GIS} and prevention of information duplication \cite{Hopfer2007_Communication} are some of the research topics addressed.

\subsection{Motivation}

SDSS, Participatory GIS as well as Argumentation Mapping focus on supporting spatial use cases in decision processes which often revolve around public matters or planning. Target groups of these applications are 