\section{Introduction}
\label{chap:introduction}
Public deliberation serves the purpose to give citizens the means to educate themselves about public matters \cite{page1996deliberates}. In this contex, democracy heavily relies on the ability and willingness of citizens to deliberate themselves. Deliberation is the process of considerate informing and weighing of options related to a topic. In additon to the distribution of information, every stakeholders' participation is required for a healthy relationship between citizens and their government \cite{Arnstein1969_citizen_participation}.\\
Since their first appearance, Web 2.0 applications are being used because of their collaborative character to gather information and opinions from users \cite{o2007web}. Today, modern information technologies and the Internet are ubiquitous in many aspects of daily life. Involving citizens in decision-making processes concerning public matters through such named applications and technologies led to the creation of the field ``e-Participation''. Its premise is to strengthen democratic processes between citizens and their governments through previously mentioned modern information technologies \cite{Saebo_eParticipation, Medaglia2012_eParticipation}. One of many aspects of e-Participation is public deliberation, which revolves around engaging citizens in dialogs about their surroundings and involving them in decision-making processes. In the past, these decision-making processes were performed by experts with specialized tools and domain-specific data at their respective agencies. The increasing digitization of more spatial data led to the development of spatial decision support systems (SDSS) \cite{densham_sdss}. These systems, which usually require professional training in order to be used effectively, were specifically developed to serve experts in public planning. Citizens could only participate if they attended public meetings or read notices at the departments. Subsequently, participation was limited to certain times and places.\\
Following the idea to involve citizens in decision-making processes using geographic information systems, the research field of participatory GIS was formed \cite{Macintosh2004_eParticipation_characterization,Sieber2006_PublicParticipationGIS}. Hence, the target group of spatial decision support systems shifted from experts to the general public. Participatory GIS allow citizens to partake in decision-making processes through Internet based systems. Rinner proposed the idea of ``Argumentation Maps'' \cite{Rinner_ArgumentationMaps}, which specifically foster discussions through spatial references. The idea is based on the assumption that locating arguments in a discussion spatially enhances the comprehensibility of the arguments and contributions. Since then, multiple implementations and extensions of Rinners ``Argumentation Map'' idea were developed and tested.\\
The use case of supporting decision-making processes and planning around public matters is being addressed extensively in participatory GIS research. Some of the research topics in question are user trust, representation of minorities \cite{Carver2001_PPGIS_Cyberdemocracy}, level of participation \cite{Steinmann2005_Combination_Ladder_GIS}, and prevention of information duplication \cite{Hopfer2007_Communication}. So far, research of enabling citizen initatives through spatially enhanced dialogs has been conducted very sparsely \cite{Cai2009_spatial_annotation_deliberation}. The contribution to citizen deliberation of citizen initatives as intermediaries between citizens and government should not be overlooked. Although some research about the use of ``new spatial media'' by citizen initiatives has been conducted \cite{Elwood2013_NewSpatialMedia}, the focus lies more on opportunities for social engagement than on participation or deliberation.\\
This thesis tries to answer the following research question:
\begin{itemize}
  \item[] \textbf{RQ:} How does the use of spatially enhanced dialogs support public deliberation performed by citizen initiatives?
\end{itemize}
It focuses on the ability to make explicit spatial references in discussion contributions, which enable initiatives and citizens to discuss and deliberate about public matters apart from public meetings and opening times of departments. Therefore, a spatial online discussion platform was developed, which allows participants of a discussion to reference geo-locations in their contributions. The results of the evaluation among citizen initiatives members as well as domain experts of argumentation mapping and web application usability showed that although finding the developed prototype suitable for spatially enhanced dialogs, it seems that making spatial characteristics, relations, and proximities visible is more important among citiizen initative members.The ability to engage in spatially enhanced dialogs about geo-features is considered a great addition, but it is feared that society will show a limited acceptance of spatially enhanced dialogs.
Furthermore, citizen initatives are discouraged with constantly managing the discussions.\\
In order to adequately describe the research perfomed in this thesis, it is structured as following: \hyperref[chap:related_work]{Section 2} explores related research in the fields of participation, e-Participation and participatory GIS. The concept of the developed prototype along with application design and implementation is described in \hyperref[chap:approach]{section 3}. Furthermore, \hyperref[chap:methodology]{Section \ref{chap:methodology}} gives insight into the evaluation methods and the approach for the development of the prototype. \hyperref[chap:methodology]{Section 5} shows the findings of the three evaluation steps, while \hyperref[chap:discussion]{section 6} discusses the evaluation results, methodology and limitations of this thesis. Finally, \hyperref[chap:conclusion]{section 7} summarizes the work in this thesis and gives recommendations for future work.
% Die wollen das ding haben, aber aus den falschen Gründen. Die sehen nur die Möglichkeit damit ihre Projekte auf ner Karte sichtbar zu machen. Die haben verstanden, was spatially enhanced dialogs sind, enthusiasm ist aber definitiv auf der informing stage