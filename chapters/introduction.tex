\section{Introduction}
\label{chap:introduction}
Public deliberation serves the purpose of giving citizens the means to educate themselves about public matters \cite{page1996deliberates}. In this sense, democracy heavily relies on the ability and willingness of citizens to deliberate themselves. Not only distribution of information, but also participation from every stakeholder is needed for a healthy relationship between citizen and their government \cite{Arnstein1969_citizen_participation}.\\
Since their first appearance, Web 2.0 applications utilized their collaborative character to gather information and opinions from their users. Today, modern information technologies and the Internet are ubiquitous in many aspects of daily life. Involving citizens in decision processes around public matters through such applications and technologies has formed the field of ``e-Participation''. Its premise is to strengthen democratic processes between citizens and its governments through said modern information technologies \cite{Saebo_eParticipation, Medaglia2012_eParticipation}. One of many aspects of e-Participation is public deliberation, which revolves around engaging citizens in dialogues about their surroundings and involving them in decision processes. In the past, these decision processes were performed by experts with specialized tools and domain-specific data at their respective agencies. The increasing digitization of more and more spatial data led to the development of spatial decision support systems (SDSS) \cite{densham_sdss}. These systems were developed specifically to serve experts in public planning and often required professional training in order to be used effectively. Citizens could only participate synchronously by attending meetings or reading public notices at the departments. 

Following the idea to support decision processes using geographic information systems and involving citizens into decision processes, the research field of participatory GIS was conceived \cite{Macintosh2004_eParticipation_characterization,Sieber2006_PublicParticipationGIS}. This shifted the target group of spatial decision support from experts to the broad public. Often, the participation through participatory GIS could happen asynchronously through the use of the Internet. Rinner then proposed the idea of ``Argumentation Maps'' \cite{Rinner_ArgumentationMaps} which specifically fosters discussions through spatial references. The idea follows the assumption that locating arguments in a discussion spatially enhances the comprehensibility of the arguments and contributions. Since then, multiple implementations and extensions to Rinners ``Argumentation Map'' idea were developed and tested.\\
The use case of supporting decision processes and planning around public matters is used extensively in participatory GIS research. User trust and representation of minorities \cite{Carver2001_PPGIS_Cyberdemocracy}, level of participation \cite{Steinmann2005_Combination_Ladder_GIS} and prevention of information duplication \cite{Hopfer2007_Communication} are some of the research topics addressed. The aspect of enabling citizen initiatives through spatially enhanced dialogues has been left out in research so far. As citizen initiatives can act as intermediary between citizens and government, their contribution to public deliberation of the public can not be overlooked. Although some research about the use of ``new spatial media'' by citizen initiatives has been conducted \cite{Elwood2013_NewSpatialMedia}, the focus lies more on opportunities for social engagement than on participation or deliberation.

This thesis tries to answer how the use of spatially enhanced dialogs supports public deliberation done by citizen initiatives. It focuses on how the ability to make spatial references in discussion contributions explicit, enables initiatives and citizens to discuss and deliberate about public matters. For this, a spatial online discussion platform was developed, enabling participants of a discussion to reference locations in their contributions. Results of semi-structured interviews with members of a citizen initiative showed, that \todo{yolo yolo yolo, swag swag swag}.


In order to adequately describe the work done in this thesis, it is structured as follows. \hyperref[chap:related_work]{Section 2} explores related research in the fields of participation, e-Participation, participatory GIS and evaluation methodology. The concept of the developed prototype along with application design and implementation is described in \hyperref[chap:approach]{section 3}. The methodology used for this thesis is reported in \hyperref[chap:methodology]{section 4}. The section gives insight about evaluation methods and approach to the development of the prototype. \hyperref[chap:methodology]{Section 5} reports on the findings on the semi-structured interviews, while \hyperref[chap:discussion]{section 6} discusses the evaluation results, methodology and limitations of this thesis. Finally, \hyperref[chap:conclusion]{section 7} reflects on the work done in this thesis and gives recommendations for future work.




