\section{Introduction}
Since their first appearance, Web 2.0 applications utilized their collaborative character to gather information and opinions from their users. Today, modern information technologies and the Internet are ubiquitous in many aspects of daily life. Involving citizens in decision processes around public matters through such applications and technologies has formed the field of ``e-Participation''. Its premise is to strengthen democratic processes between citizens and its governments through said modern information technologies \cite{Saebo_eParticipation, Medaglia2012_eParticipation}. One of many aspects of e-Participation is public deliberation, which revolves around engaging citizens in dialogues about their surroundings and involving them in decision processes. In the past, these decision processes were performed by experts with specialized tools and domain-specific data at their respective agencies. The increasing digitization of more and more spatial data led to the development of spatial decision support systems (SDSS) \cite{densham_sdss}. These systems were developed specifically to serve experts in public planning and often required professional training in order to be used effectively. Citizens could only participate synchronously by attending meetings or reading public notices at the departments. 

Following the idea to support decision processes using geographic information systems and involving citizens into decision processes, the research field of participatory GIS was conceived \cite{Macintosh2004_eParticipation_characterization,Sieber2006_PublicParticipationGIS}. This shifted the target group of spatial decision support from experts to the broad public. Often, the participation through participatory GIS could happen asynchronously through the use of the Internet. Rinner then proposed the idea of ``Argumentation Maps'' \cite{Rinner_ArgumentationMaps} which specifically fosters discussions with spatial references. The idea follows the assumption that locating arguments in a discussion spatially enhances the comprehensibility of the arguments and contributions.  Since then, multiple implementations and extensions to Rinners ``Argumentation Map'' idea were developed and tested. SDSS, participatory GIS and argumentation mapping all focus on supporting spatial use cases in decision processes which often revolve around public matters or planning. % Target groups of these applications are experts for SDSS and the broad public for participatory GIS and argumentation maps.

%These asynchronous discussions support planning procedures.

\subsection{Motivation}
The use case of supporting decision processes and planning around public matters is used extensively in participatory GIS research. User trust and representation of minorities \cite{Carver2001_PPGIS_Cyberdemocracy}, level of participation \cite{Steinmann2005_Combination_Ladder_GIS} and prevention of information duplication \cite{Hopfer2007_Communication} are some of the research topics addressed.
