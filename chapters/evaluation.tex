\section{Evaluation}
\label{chap:evaluation}
Subsequently to \hyperref[chap:methodology]{section \ref{chap:methodology}}, which covered evaluation techniques applied, this section will present findings inferred from the replies recorded in the semi-structured interviews (\hyperref[sub:ev_interviews]{sub-section \ref{sub:ev_interviews}}), expert interviews (\hyperref[sub:ev_expert_interviews]{sub-section \ref{sub:ev_expert_interviews}}) and focus group (\hyperref[sub:ev_focus]{sub-section \ref{sub:ev_focus}}).


\subsection{Semi-structured Interviews}
\label{sub:ev_interviews}
Two of the eight participating respondents were female. Except two, all respondents are involved in the organization of the sustainability project. The other two are members of a citizen foundation in M{\"u}nster. Because of the anonymization of the interviews, participants and will be denoted as P1 to P8 in the following sections.\\
As mentioned in \hyperref[chap:methodology]{section \ref{chap:methodology}}, the responses were categorized to emphasize and expose important information. The exposed categories are ``Benefits and goals of public deliberation'', ``Current citizen-initiative communication means'', ``Benefits gained through the deployment of the prototype'', ``Advantages of the prototype over other solutions considered'', ``Disadvantages of the prototype'' and ``Other use cases for the prototype''.\\

According to the respondents, public deliberation is a step to self-determined organization of society (P2, P3, P5) and a method to inform oneself (P1). One of the most important aspect is the ability of citizens toactively shape their environment through participation (P2, P3, P5, P6, P7). Public deliberation allows to pass on experience (P4, P8) and fosters discussions (P1, P2). P5 and P7 indicated that public deliberation strengthens thus allowing the public to become an opposition to powerful corporations and the government. Generally, public deliberation is understood as an act of doing useful things together.

Currently, communication in initiatives and with the public is done mostly through mailing lists, e-mails and meetings. P5, P7 and P8 additionally use Internet pages. A wiki application was mentioned by P7. As means to reach out specifically to the public, newspaper advertisements were named (P2, P


% Current communication means
    % E-Mail, Mailinglist (P1, P2, P3, P5, P6
    % Meetings (P1, P3, P5, P6, P8
    % Web page, Wiki (P5, P7, P8
    % Newspaper ads (P2, P6
    % Trello (P3
    % Telephone (P5


% P7: Public deliberation is socially selective
% Benefits and goals of public deliberation
    % participate and shape actively (P2, P3, P5, P6, P7
    % step to self-determined organisation (P2, P3, P5
    % pass on experience (P4, P8
    % power comes from the people (P5, P7
    % come together and discuss (P1, P2
    % inform oneself (P1
    % “Participation” (Beteiligung P2) (P2
    % doing useful things together (P6
    % strengthen the public (P7
    % Serve as opposite to the government or powerful corporations (P7


% Benefits gained through deployment of the prototype
    % get an idea of the topic, inform the public (P1, P3, P7
    % user can see projects near him (discussion -> make something of toblers first law) (P7, P8
    % get contact information (P1
    % public discussion about the topic (P2
    % better information mean than a list, spatial overview of sparse and dense spots (P7
    % decentral planning of the project (discussion -> dennis paper about same time same place MAYBE) (P7




% Target Group
    % everyone who is interested (P2

% Incentives for participation
    % visualisation of proximity (P3, P4, P5, P7
    % making sure the questions posted by users reach the right people (P8

% Advantages of the prototype over other solutions
    % online
        % access is easy through online (P2
        % internet use is ubiquitous (P2, P5
    % graphic representation (P2, P7 P8
    % direct connection of project with its location (P4
    % interactivity (P6
    % Other solutions considered instead of the prototype
        % Paper maps (P2
        % Lists (P2, P7
        % Project management software (P3
        % Brochure (P4, P8
        % Web-page (P4, P8
        % PDF (P6
        % Wiki (P7

% Advantages through the use of the prototype
    % Able to see what's there at one glance (\dots dass man auf einen Blick sehen kann, da und da und da gibt es \dots P1) (P1, P6, P7, P8
    % Able to see when somethings there (Man sieht [\dots] auch wann die da sind. P1) (P1, P6
    % digitalize what's there (data collection) (P2, P8
    % spark communication through the collection of data (P2, P5
    % spark acquaintances, ideas and execution of ideas (P5
    % Incentives for participation
        % visualisation of proximity (P3, P4, P5, P7
        % making sure the questions posted by users reach the right people (P8

% Other use cases imagined
    % Biking tour planner (P1
    % convey information (P2, P6
    % database (P2
    % scientific use (P2
    % organization tool for
        % schools (P2
        % politic initiatives (P2
        % detailed position of stalls and booths (P3
        % huge events (P3
        % companies with big area (P3
    % Police (P3
    % Searching (P4
    % To network the different initiatives around the city (P5, P6
    % Display of Open Data (P7

% Things keeping away citizens from participating
    % low usability (P2, P3, P5, P6, P7
    % unclear data protection and privacy (authorization) (P2, P3, P4
    % only online (P1, P2, Carver2001_PPGIS_Cyberdemocracy
        % exclusion of groups with no internet access (elderly,  (P2
    % impracticability because of low acceptance (P2, P5
    % errors (P2, P3
    % no knowledge about the existence of the application (P3, P7
    % registration (P4, P7
    % unclear context (P2
    % no initial content (P5
    % low visual appearance (P7
    % no structure of information and discussion (P7
    % no knowledge about the project (P7
    % cost (P4
    % maintenance (P4
    % Missing Feature: Tasks (P3



% Knowledge of tools which enable the connection of spatial data with discussions
    % Next-Hamburg (P2
    % Tag des guten Lebens Köln (P2
    % Google (P3, P4, P5
    % Powerpoint (P3
    % Foursquare (P4
    % Yelp (P4
    % Openstreetmap (P4, P5








\subsection{Expert Interviews}
\label{sub:ev_expert_interviews}


% Applications that support discussions through spatial features?
% E1:
% Emergency Response maps
% Benefits:
% bijectivity of features and objet makes immideately clear about what the discussion is about
% E2:
% ArguMap

% Use cases for the connection of spatial features with discussions?
% E1:
% Literally everything
% E2:
% public participation
% organization
% Solutions for bringing together citizens with initiatives/politicians?
% E1:
% Petition portals
% E2:
% Town Hall meetings
% public postings
% Connection of spatial features with discussions is helpful in the context of public deliberation?
% E1:
% yes
% E2:
% yes
% but: steep learning curve
% usability must be good
% easy to exclude non-tech-savy persons
% In comparison to applications you know, what do you think about
% Omitting the spatial features of the replies in the overview?
% E1:
% good for clarity
% but focus is on the creator of the topic
% E2:
% good idea to prevent cluttering of map
% Two Way- Highlighting of spatial features and contribution rectangles
% E1:
% Important aspect for comprehensibility
% E2:
% helps the user to see the relationships 
% Filter and sorting
% E1:
% Filter is less important with only a few contributions
% Many options to choose from, wishes more explanation
% E2:
% As expected
% Compose and reply forms
% E1:
% generally good
% E2:
% Categories are not good enough explained and multiple categories are confusing
% Connection of words with spatial features, links and references to existing spatial features
% E1:
% Not very intuitive.
% icons not self explanatory
% E2:
% Geocoding and automatic location suggestion would be helpful
% not very intuitive
% Favoring of contributions
% E1:
% good measure for popularity of a contribution
% but no incentives to use the function
% toggle button is not visible at plain sight
% E2:
% generally good idea
% no incentives to use
% Registration and Sign in
% E1:
% standard, everything as expected
% E2:
% as expected
% Social Login
% E1:
% state-of-the-art
% E2:
% good idea. makes it easier for people
% Do you think Dialogues will be supported?
% E1:
% Discussion with tools is generally difficult
% Depends on moderation tools
% Small sidebar leads focus more on the map, not on the discussion
% E2:
% Depends on type of dialogues
% Needs more evaluation
% Missing Features
% E1:
% List of favored contributions for the user as bookmarking tool
% List of contributions the user has contributed to
% unread topics
% E2:
% spatial search
% geocoding of words
% Reasons not to participate through this tool
% E1:
% Registration
% E2:
% Learning curve





\subsection{Focus Group}
\label{sub:ev_focus}
analysis following \cite{asbury1995overview}