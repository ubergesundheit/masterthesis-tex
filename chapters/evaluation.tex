\section{Evaluation}
\label{chap:evaluation}
Subsequently to \hyperref[chap:methodology]{section \ref{chap:methodology}}, which covered evaluation techniques applied, this section will present findings inferred from the replies recorded in the semi-structured interviews (\hyperref[sub:ev_interviews]{sub-section \ref{sub:ev_interviews}}), expert interviews (\hyperref[sub:ev_expert_interviews]{sub-section \ref{sub:ev_expert_interviews}}) and focus group (\hyperref[sub:ev_focus]{sub-section \ref{sub:ev_focus}}).


\subsection{Semi-structured Interviews}
\label{sub:ev_interviews}
Two of the eight participating respondents were female. Except two, all respondents are involved in the organization of the sustainability project. The other two are members of a citizen foundation in M{\"u}nster. As the interviews were anonymized, participants and will be denoted as P1 to P8 in the following sections.\\
As mentioned in \hyperref[chap:methodology]{section \ref{chap:methodology}}, the responses were categorized to emphasize and expose important information. The exposed categories are ``Benefits and goals of public deliberation'', ``Current citizen-initiative communication means'', ``Attributes of spatial discussion platforms which support dialogs and public deliberation'' and ``Other use cases for spatial discussion platforms''.\\ % ``Attributes disadvantageous for dialogs and public deliberation''

According to the respondents, public deliberation is a step to self-determined organization of society (P2, P3, P5) and a method to inform oneself (P1). One of the most important aspect is the ability of citizens to actively shape their environment through participation (P2, P3, P5, P6, P7). Public deliberation allows to pass on experience (P4, P8) and fosters discussions (P1, P2). P5 and P7 indicated that public deliberation strengthens thus allowing the public to become an opposition to powerful corporations and the government. Generally, public deliberation is understood as an act of doing useful things together.

Currently, internal communication in initiatives is done mostly through mailing lists, e-mails and meetings. P2, P5, P7 and P8 name the use of Internet pages with simple lists and texts to transport information to the public. A wiki web application was mentioned by P7. Newspaper advertisements as medium were named by P2 and P6. In addition to that, brochures and flyers were used (P4, P8). According to P6 online PDF documents are also used. Telephone meetings and Trello\footnote{\url{https://trello.com/}} or other project management software were named by P5 and P3 respectively. P2 added the use of paper maps as mean to transport information. Knowledge of online tools for creating interactive maps was generally low.

Nearly all respondents mentioned the informing of the public through spatial discussion platforms as useful attribute. The ability to gain a spatial overview of the topic (P1, P3, P4, P5, P7) and a sense of proximity to projects (P7, P8) were also mentioned as advantageous attributes for public deliberation. Visualization of proximity was also named as an incentive for users to participate (P3, P4, P5, P7). Discouraging factors for participation through spatial discussion platforms were low usability, application errors and unstructured discussions according to most of the respondents (P2, P3, P5, P6, P7). P7 also mentioned no initial content as impeding factor. Communication, as well as acquaintances, ideas and execution of ideas are believed to be sparked through the creation of spatial references in discussions (P2, P5). The same was said about data collection. Digitization of projects is said to inspire more contributions and discussions (P2, P8). The ubiquitousness of Internet use is believed to allow the general public to engage others in discussions (P2, P5). On the other hand, concerns about excluding groups without Internet access were raised by P1, P2 and P7. According to P7, even without the use of Internet technologies, public deliberation is highly socially selective. In addition to that, P2, P3 and P4 raised privacy concerns and unclear data protection. P4 and P7 believed the requirement to create user accounts could dampen spontaneous contributions. If the context, in which the application is deployed and reuse of contributions are not clearly communicated, acceptance will be generally low (P2, P8). Low acceptance induced by low publicity were also named by P2, P3, P5 and P7. Generally, graphic representation of objects in the discussions are considered useful (P2, P7, P8). Displaying markers and polygons on a map allows users to easily grasp the density of projects in different areas (P7). P7 also mentioned that unpleasing visual appearance could deter users from contributing. Additionally, interactivity enables users to understand the connection between a project and its location (P4, P6). Possible monetary costs of deployment and maintenance as negative factors for online spatial discussion platforms were mentioned by P4. 

The use of a spatial discussion platform as an organizational tool mentioned the most by respondents. Conveyance of information of all sort was named by P2, P5 and P6. P7 came up with the idea to display open data produced by governments through a spatial discussion platform.

% Knowledge of tools which enable the connection of spatial data with discussions
    % Next-Hamburg (P2
    % Tag des guten Lebens Köln (P2
    % Google (P3, P4, P5
    % Powerpoint (P3
    % Foursquare (P4
    % Yelp (P4
    % Openstreetmap (P4, P5

% Other use cases imagined
    % Biking tour planner (P1
    % convey information (P2, P6
    % database (P2
    % scientific use (P2
    % organization tool for
        % schools (P2
        % politic initiatives (P2
        % detailed position of stalls and booths (P3
        % huge events (P3
        % companies with big area (P3
    % Police (P3
    % Searching (P4
    % To network the different initiatives around the city (P5, P6
    % Display of Open Data (P7

% Things keeping away citizens from participating
    % low usability (P2, P3, P5, P6, P7
    % unclear data protection and privacy (authorization) (P2, P3, P4
    % only online (P1, P2, Carver2001_PPGIS_Cyberdemocracy
        % exclusion of groups with no internet access (elderly,  (P2
    % impracticability because of low acceptance (P2, P5
    % errors (P2, P3
    % no knowledge about the existence of the application (P3, P7
    % registration (P4, P7
    % unclear context (P2
    % no initial content (P5
    % low visual appearance (P7
    % no structure of information and discussion (P7
    % no knowledge about the project (P7
    % cost (P4
    % maintenance (P4
    % Missing Feature: Tasks (P3

% Benefits gained through deployment of the prototype
    % get an idea of the topic, inform the public (P1, P3, P7
    % user can see projects near him (discussion -> make something of toblers first law) (P7, P8
    % get contact information (P1
    % public discussion about the topic (P2
    % better information mean than a list, spatial overview of sparse and dense spots (P7
    % decentral planning of the project (discussion -> dennis paper about same time same place MAYBE) (P7
    % online
        % access is easy through online (P2
        % internet use is ubiquitous (P2, P5
    % graphic representation (P2, P7 P8
    % direct connection of project with its location (P4
    % interactivity (P6
    
% Target Group
    % everyone who is interested (P2

% Incentives for participation
    % visualisation of proximity (P3, P4, P5, P7
    % making sure the questions posted by users reach the right people (P8

% Advantages through the use of the prototype
    % Able to see what's there at one glance (\dots dass man auf einen Blick sehen kann, da und da und da gibt es \dots P1) (P1, P6, P7, P8
    % Able to see when somethings there (Man sieht [\dots] auch wann die da sind. P1) (P1, P6
    % digitalize what's there (data collection) (P2, P8
    % spark communication through the collection of data (P2, P5
    % spark acquaintances, ideas and execution of ideas (P5
    % Incentives for participation
        % visualisation of proximity (P3, P4, P5, P7
        % making sure the questions posted by users reach the right people (P8

% Advantages of the prototype over other solutions
    % Other solutions considered instead of the prototype
        % Paper maps (P2
        % Lists (P2, P7
        % Project management software (P3
        % Brochure (P4, P8
        % Web-page (P4, P8
        % PDF (P6
        % Wiki (P7

% Current communication means
    % E-Mail, Mailinglist (P1, P2, P3, P5, P6
    % Meetings (P1, P3, P5, P6, P8
    % Web page, Wiki (P5, P7, P8
    % Newspaper ads (P2, P6
    % Trello (P3
    % Telephone (P5

% P7: Public deliberation is socially selective
% Benefits and goals of public deliberation
    % participate and shape actively (P2, P3, P5, P6, P7
    % step to self-determined organisation (P2, P3, P5
    % pass on experience (P4, P8
    % power comes from the people (P5, P7
    % come together and discuss (P1, P2
    % inform oneself (P1
    % “Participation” (Beteiligung P2) (P2
    % doing useful things together (P6
    % strengthen the public (P7
    % Serve as opposite to the government or powerful corporations (P7

\subsection{Expert Interviews}
\label{sub:ev_expert_interviews}
As mentioned earlier, expert interviews were conducted with Carsten Ke{\ss}ler and Tobias Heide. Statements will be denoted E1 and E2 respectively.


To the question to knowledge about spatial discussion platforms, E2 mentioned ArguMap\footnote{short for argumentation map}. E1 counted emergency response maps to the class of spatial discussion platforms. He states that bijectivity between features and objects supports mutual understanding of what discussions are about.\\
``Literally everything''\footnote{E1: Ich k{\"o}nnte mir das relativ grob strukturiert eigentlich {\"u}berall vorstellen.} could be the use case for linking discussions with spatial features according to E1. Public participation and involving citizens in planning processes are named by E2 for use cases.\\
The idea to include spatial references in discussion contributions in public deliberation is received positively by both experts. Applications that use this concept could be used in conjunction with petition portals (E1), town hall meetings and public postings (E2). E2 remarked that the use of online applications could backfire so that non-tech-savy persons or groups could be excluded. In his experience, usability should be prioritized in the design of spatial discussion platforms to avoid a steep learning curve.\\
The presentation of the developed prototype was generally well received by both experts. The design decision to omit spatial features created as replies, was described as good for clarity (E1) and as an good idea to prevent visual clutter (E2). E1 added that through this, the focus is on the creator of the the first contribution.\\
According to E1, the two way highlighting is an important aspect for comprehensibility of relationships between spatial features and textual contributions. E2 adds that it helps users to see and understand the relationships.\\
The ability to filter and sort contributions was deemed as less important with only few contributions by E1. The many options users can choose from, could confuse users. E2 indicates that the functions are implemented as he was expecting it. He misses the feature to filter spatially.\\
The compose and reply forms were conceived as generally good and working as expected. E2 denoted that the categories are not explained well enough and found multiple categories confusing. In the same sense, the creation of spatial references was found unintuitive by both experts. The icons for enable the referencing actions are not self-explanatory. E2 adds that suggesting locations by geocoding would be helpful for users.\\
Although the feature for favoring contributions is a good measure of the popularity of a contribution, no incentives for using it exist (E1, E2). Additionally, the button is not accessible immediately and not self explanatory (E1). E2 thinks the ability to favor contributions is a good idea.\\
Registration and authentication works as expected for both experts. The social login is called ``state-of the-art'' by E1. E2 believes third-party authentication mechanisms lowers participation barriers for users.\\
In the opinion of both experts, spatial dialogs will be aided through the use of the prototype. E1 points out that a discussion tool is only as good as its moderation tools. In his opinion, a real dynamic discussion is generally difficult through text-based technology (also E2). With the small sidebar, the prototype leads the focus more to the map, not to the discussion. E2 adds that more evaluation is needed to answer this question thoroughly.\\
To create incentives to use the favor function, E1 suggests to implement user accessible favorite-lists. He adds that there is no measure for users to only display unread contributions or find all contributions of him or herself. E2 misses the ability for searching spatially.\\


\subsection{Focus Group}
\label{sub:ev_focus}
analysis following \cite{asbury1995overview}