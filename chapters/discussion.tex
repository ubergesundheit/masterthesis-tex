\section{Discussion}
\label{chap:discussion}
Following the descriptive evaluation of the semi-structured interviews, expert interviews and focus group in \hyperref[chap:evaluation]{section \ref{chap:evaluation}}, \hyperref[sub:evaluation-results]{sub-section \ref{sub:evaluation-results}} will do an interpretation of the results of the evaluation steps. \hyperref[sub:method-discussion]{Sub-section \ref{sub:method-discussion}} will critically revisit the methodology described in \hyperref[chap:methodology]{section \ref{chap:methodology}}. Limitations of both methodology and evaluation techniques, which emerged over the course of this thesis, will be described in \hyperref[sub:limitations]{sub-section \ref{sub:limitations}}.

\subsection{Evaluation Results}
\label{sub:evaluation-results}
According to the respondents of the semi-structured interviews, public deliberation enables citizens to actively influence their environment through self-determined organization. Easy access of information is seen as first step to initiate interest and spark discussions. Means currently used for this are generally unspecialized. Generally, tools are mainly used to inform and educate citizens. Although many respondents mentioned the use of Internet pages, only one interviewee addressed the use of a wiki web application for collaboratively collecting information. Online based participation for engaging citizens beyond informing or collecting arbitrary information was generally not considered by any of the respondents. Respondents noted they have no experience in using Internet based applications and that they had no access to such applications. ``Nexthamburg'' was the only similar system in the knowledge of participants.\\
Benefits of using spatially enhanced dialogs in public deliberation were immediately visible for all participants of the evaluation. The ability to grasp proximities and spatial densities of discussion subjects was among the most useful aspects according to the respondents. Interactivity helped to recognize relationships between spatial features and contributions. Namely, the desired impact of \hyperref[req:d]{requirement d}, was visible.\\
User interface functions were received generally well by experts and participants of the focus group. Experts listed spatial filtering and a user accessible favorite list as missing functionalities. One focus group member noted that the button for editing top-level contributions was too hidden. Except for the placement of the favorite and edit buttons, all functions worked as expected and were found easily. This could be a direct result from the agile approach to the development process with ideas and suggestions of future users. As \hyperref[req:f]{requirement f} intended to ease the use of the application by an easy user interface, this requirement can also be considered to be fulfilled.\\
Reactions of both experts were positive. In their opinion, the developed prototype enables supporting public deliberation though spatially enhanced dialogs. Suggestions, revealed during the evaluation, like geocoding the text of a contribution and user accessible favorite-lists could further enhance the usefulness of the prototype.\\
Despite the well reception of the concept of spatially referenced discussions, participants of the focus group evaluation showed doubts about broad acceptance of spatial discussion platforms. The relatively high mean age of participants and background cloud be the reason for this reluctance. Proximity to projects was named as incentive for using discussion functionalities by respondents. As respondents of the semi-structured interviews mentioned informing the public as one of the most useful aspects, focus of interest of the respondents lies in the ability to visualize spatial relationships of their activities. Spatial relationships were found to be clarified (\hyperref[req:a]{Requirement a}). A possible reason for this could be the low experience with spatial online discussion platforms mentioned by the respondents.\\
Potential exclusion of groups or persons without Internet access, concerns about data privacy, low public acceptance, unclear context and reuse of contributions could be reasons for this reluctance. Especially citizens above the age of 50 were named as vigilant against data privacy. Low acceptance could be evoked through no initial content or low usability. Respondents of younger age thought that Internet based systems are suitable for the use case. Despite recognising benefits of spatially enhanced dialogs and understanding the concept, participants suggested to use the developed platform to convey information with spatial aspects. Of the participation levels possible with the \textit{DialogMap} concept (\hyperref[fig:my_ladder]{Figure \ref{fig:my_ladder}}) ``Collect Information'' and ``Collect Proposals'' levels are aspired by respondents.\\
With recourse to the research question, public deliberation conducted by citizen initiatives could benefit through the deployment of spatially enhanced dialogs if they feel comfortable to deploy it. Projects like ``Nexthamburg'' and ``Nextkassel'' are good examples for similar systems with broad participation. Further research to determine most important factors for the deployment of spatial discussion platforms in context of public deliberation performed by citizen initiatives has to be conducted.\\
Impacts of the requirements \hyperref[req:c]{c} and \hyperref[req:e]{e} played only minor roles for evaluation participants. Although experts generally found filtering and sorting important, a spatial filter was missed by E2. Tags and categories were noticed, but not commented. 

\subsection{Methodology}
\label{sub:method-discussion}
In order to answer the research question posed in \hyperref[chap:introduction]{section \ref{chap:introduction}}, several steps were taken. A spatial discussion platform prototype was developed to demonstrate the concept of spatially enhanced dialogs, which was then evaluated with overall 13 participants distributed over semi structured interviews, expert interviews and a focus group.\\
In opposition to methods like paper prototyping or interactive mock ups, the development of a full working prototype was chosen. A working prototype, which can be used and tested has several advantages over methods mentioned before. Questions of participants could be answered directly on it. Also, the release of the source code to the public allows future researchers not only to learn from this thesis, but also from the implementation details. The fact that development was performed in an agile manner, with direct feedback of members of a scientific citizen initiative, allowed to further refine the outcome of the development.\\
Qualitative over quantitative evaluation techniques were chosen to explore reasons for acceptance and opinions to the use of spatially enhanced dialogs in a fine grained matter. Also, interviews allow to dynamically react to the respondent to further explore aspects previously not considered by evaluation design. As P7 noted, public participation and deliberation already is highly socially selective. Remarks like this could not have been come up with quantitative evaluation techniques like usability questionnaires.\\

\subsection{Limitations}
\label{sub:limitations}
The development of the prototype was conducted with input of members of only one citizen initiative started by a university group. Their influence resulted in a software which serves first and foremost their needs specialized to the sustainability project. For example the multiple categories for the contributions were confusing for people unfamiliar with the background of the sustainability project.\\
Although introduction texts to concept and functionalities as well as videos are included, a real tutorial was not implemented due to limited time available and focus on other last-minute additions, like the ability to upload pictures. Functionalities were sufficiently explained before the interviews.\\
The prototype implements all functionalities to support dialogs. Users are able to create spatial references and hyperlinks and relate to other spatial references. Contrary to the findings of the focus group, experts noted the spatial reference creation workflow was not self-explanatory. Here, interview questions to determine a better procedure cloud have been asked.\\
Discussion contributions are organized chronologically and users can express support by favoring contributions. The prototype does not feature functions found in traditional forum applications like voting, merging of threads, drafting contributions, private messaging to other users or direct moderation. It is also not possible to upload geo-data. These functionalities were not part of the requirements posed in \hyperref[sub:dialogmap]{section \ref{sub:dialogmap}} as they were not considered by related research considered for this thesis. Effects of implementing and promoting such functionalities could yield interesting results. Also, functionalities to form consensus among discussion participants, as the final level of \hyperref[fig:my_ladder]{figure \ref{fig:my_ladder}}, could be implemented and investigated.\\
The effects of different manifestations of the user interface to focus more on the discussion or map could also be object of future work.\\
The evaluation conducted in this thesis focuses on benefits and support of public deliberation conducted by citizen initiatives. Despite considering user friendliness during the development of the prototype, user friendliness was only directly tested by one expert. For this, a quantitative user study focusing on measuring user experience could be conducted. For this thesis, user friendliness was not the focus of research.\\
Participants for both the semi-structured interviews as well as the focus group were recruited over mailing-lists of the scientific citizen initiative. This was done to get insights and opinions from real-world citizen activists. Participants mean age was 44.63 with a standard deviation of 16.68. A more diverse selection of participants or more interviews and focus groups with more participants over longer periods of time could have led to other results. The outcome of respondents favoring the informing ability of spatial discussion platforms could be a direct result of the heterogeneous age distribution. A deployment in a long running study to record types of contributions, contribution habits, user types and general suitability in the citizen initiative context could be a potential research direction for future research. The three evaluation methods deployed in this thesis scratched only on the surface of possible use cases.\\
Additionally, legal implication of running such an online platform have to be explored.
