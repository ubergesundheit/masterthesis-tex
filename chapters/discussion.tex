\section{Discussion}
\label{chap:discussion}
Following the descriptive evaluation of the semi-structured interviews, expert interviews and focus group in \hyperref[chap:evaluation]{section \ref{chap:evaluation}}, \hyperref[sub:evaluation-results]{sub-section \ref{sub:evaluation-results}} will do an interpretation of the results of the evaluation steps. \hyperref[sub:method-discussion]{Sub-section \ref{sub:method-discussion}} will critically revisit the methodology described in \hyperref[chap:methodology]{section \ref{chap:methodology}}. Limitations of both methodology and evaluation techniques which emerged over the course of this thesis, will be described in \hyperref[sub:limitations]{sub-section \ref{sub:limitations}}.

\subsection{Evaluation Results}
\label{sub:evaluation-results}
The general tone of results of the evaluation 


While both respondents of the semi-structured interviews and participants of the focus group evaluation were generally receptive to the concept of spatially enhanced dialogs, participants of the focus group evaluation showed doubts about general acceptance of spatial discussion platforms.

Although \todo{seeing} benefits in spatially enhanced dialogs and understanding the concept, participants generally prefer 

the idea to support dialogs through spatially referenced discussions is generally well understood by the participants of the evaluation steps.


Focus group analysis following \cite{asbury1995overview}

\subsection{Methodology}
\label{sub:method-discussion}

\subsection{Limitations}
\label{sub:limitations}

Limitations of Methodology, Implementation, Limited time available.

%In his ``Twenty years of progress: GIScience in 2010'', Goodchild devotes a section to ``The role of the citizen'' \cite{goodchild2014twenty}. He states

%Following the famous quote of Garson and Biggs \cite{Garson1992_80Percent} that