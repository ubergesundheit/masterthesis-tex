\section{Discussion}
\label{chap:discussion}
Following the descriptive evaluation of the semi-structured interviews, expert interviews and focus group in \hyperref[chap:evaluation]{section \ref{chap:evaluation}}, \hyperref[sub:evaluation-results]{sub-section \ref{sub:evaluation-results}} will do an interpretation of the results of the evaluation steps. \hyperref[sub:method-discussion]{Sub-section \ref{sub:method-discussion}} will critically revisit the methodology described in \hyperref[chap:methodology]{section \ref{chap:methodology}}. Limitations of both methodology and evaluation techniques which emerged over the course of this thesis, will be described in \hyperref[sub:limitations]{sub-section \ref{sub:limitations}}.

\subsection{Evaluation Results}
\label{sub:evaluation-results}
According to the respondents of the semi-structured interviews, public deliberation enables citizens to actively influence their environment through self-determined organization. Easy access of information is seen as first step to initiate interest and spark discussions. The means currently used for this are generally unspecialized. Generally, tools are used to inform and educate citizens. Although many respondents mentioned the use of Internet pages, only one interviewee addressed the use of a wiki web application for collaboratively collecting information. Online based participation for engaging citizens beyond informing or collecting arbitrary information was generally not considered by any of the respondents.\\
Benefits of using spatially enhanced dialogs in public deliberation were immediately visible for all participants of the evaluation. The ability to grasp proximities and spatial densities of discussion subjects was among the most useful attributes according to the respondents. Interactivity helped to recognize relationships between spatial features and contributions.\\
Despite the well reception the concept of spatially referenced discussions, participants of the focus group evaluation showed doubts about broad acceptance of spatial discussion platforms. As respondents of the semi-structured interviews mentioned informing the public as one of the most useful attribute, it becomes apparent that either the need for specialized spatial discussion platforms is not seen by evaluation participants or they apprehend attributes of either the developed prototype, the concept of spatially enhanced dialogs or online systems in general. 

F


Being an Internet based web application, 

Possible reasons could be the potential exclusion of groups or persons without Internet access,
being an online system with implications like data privacy, or the fact that low usability and errors were dreaded the most.  \todo{Sachen aufzäheln die die nicht so gut gefunden haben, was sie abhalten wuerde}  %Here, The answer for the reasons cannot be given without more research.




Despite recognising benefits of spatially enhanced dialogs and understanding the concept, participants suggested to use the developed platform to convey information with spatial aspects.



No need for spatial discussion platforms is seen.



Spatial discussion platforms deployed in context of public deliberation




Focus group analysis following \cite{asbury1995overview}

\subsection{Methodology}
\label{sub:method-discussion}

\subsection{Limitations}
\label{sub:limitations}

Limitations of Methodology, Implementation, Limited time available.

%In his ``Twenty years of progress: GIScience in 2010'', Goodchild devotes a section to ``The role of the citizen'' \cite{goodchild2014twenty}. He states

%Following the famous quote of Garson and Biggs \cite{Garson1992_80Percent} that