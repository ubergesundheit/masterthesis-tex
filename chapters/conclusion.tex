\section{Conclusion}
\label{chap:conclusion}
The research goal of this thesis was to find out how the use of spatially enhanced dialogs supports public deliberation conducted by citizen initiatives. In order to accomplish this task, a spatial discussion platform prototype was developed. Three qualitative evaluation methods were used to explore opinions and thoughts to spatially enhanced dialogs of citizen initiative activists.\\
The spatial discussion platform was developed using recommendations and suggestions taken from past argumentation mapping, participatory GIS and SDSS research as well as practical advice of members of a scientific citizen initiative. It supports structured discussions with creation of multiple spatial features, hyperlinks and references to other spatial features per contribution. Contributions can be filtered by categories, tags and free text. Users are able to favor contributions in order to express support for the written statement. An administrative interface allows simple moderation tasks.\\
Eight semi-structured interviews with citizen initiative members were conducted in order to gain insight into how they conduct public deliberation and how the deployment spatially enhanced dialogs could support them in deliberating citizens. The respondents were recruited in the context of a sustainability project of the scientific citizen initiative that provided input to the development of the prototype.\\
Then two experts of the fields argumentation mapping and usability were interviewed in order to test the suitability of the prototype to the field respectively. Expert for argumentation mapping was Carsten Keßler, who authored and developed several argumentation mapping papers and platforms. Tobias Heide was chosen because of his expertise in the field of usability of web applications.\\
Finally, a focus group discussion was conducted with three additional members of different citizen initiatives related to the sustainability project. They were given an introduction into the prototype and then tested feasibility of a discussion related to the organization of the sustainability project. Lastly, their impressions to the prototype and opinions to the concept of spatially enhanced dialogs were recorded in a group discussion.\\
Results showed that although the concept is understood and deemed as useful, evaluation participants did not see the need to use spatially enhanced dialogs for their public deliberation activities. Most respondents are more interested in the ability of the prototype to convey spatial information. The ability to convey proximities and densities are seen as important attributes to transport information to citizens. Additionally Internet based systems received mixed responses ranging from being named easily accessible by everyone to excluding persons and groups without Internet access. Data privacy and unclear context and reuse was also mentioned in context to Internet based systems. Expert reactions to implementation of the concept and usability of the prototype were generally good. In addition to feature suggestions and recommendations, some minor inconsistencies in the usage of the prototype were found.\\
Evaluation also sparked ideas and feature suggestions at every evaluation step. The use of spatial discussion platforms for planning and organization was mentioned by most respondents. Features like creating tasks with spatial references, alternative chronological visualizations, geocoding written text to support the creation of spatial references or prevent repetition of information, user-accessible favorite-lists and spatial searching and filtering were suggested over the course of the evaluation steps. Features like allowing the upload of arbitrary geo-data or offering the option to include open government data for discussions could be other methods to support public deliberation with spatially enhanced dialogs. Different sizes of the sidebar to focus more on the discussion could also be object of future work.


The results of this thesis suggest that spatially enhanced dialogs could support public deliberation conducted by citizen initiatives. Map based information display as a first step seems promising. Future research could pick up on the suggestions made by evaluation participants and implement a long-running user study. Exploring legal implication of running such an online platform could be another direction for future research.
