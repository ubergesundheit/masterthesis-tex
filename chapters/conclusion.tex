\section{Conclusion}
\label{chap:conclusion}
The research goal of this thesis was to find out how the use of spatially enhanced dialogs supports public deliberation conducted by citizen initiatives. In order to accomplish this task, a spatial discussion platform prototype was developed. Three qualitative evaluation methods were used to explore opinions and thoughts to spatially enhanced dialogs of citizen initiative activists.\\
The spatial discussion platform was developed using recommendations and suggestions taken from past argumentation mapping, PPGIS and SDSS research as well as practical advice of members of a scientific citizen initiative. It supports structured discussions with creation of multiple spatial features, hyperlinks and references to other spatial features per contribution. Contributions can be filtered by categories, tags and free text. A favori\\





Mention recommendations from interviewers (Geocoder, list of favorites of each user, tasks)

Feature idea: let users upload geodata.

Maybe pick up unimplemented things from literature

Legal implications of running such a website have to be explored.

Copyright Issues \cite{Carver2001_PPGIS_Cyberdemocracy}


Future direction \textit{Combining Social and Government Open Data for Participatory Decision-Making}