\documentclass{sigchi}

% Use this command to override the default ACM copyright statement (e.g. for preprints). 
% Consult the conference website for the camera-ready copyright statement.


%% EXAMPLE BEGIN -- HOW TO OVERRIDE THE DEFAULT COPYRIGHT STRIP -- (July 22, 2013 - Paul Baumann)
% \toappear{Permission to make digital or hard copies of all or part of this work for personal or classroom use is 	granted without fee provided that copies are not made or distributed for profit or commercial advantage and that copies bear this notice and the full citation on the first page. Copyrights for components of this work owned by others than ACM must be honored. Abstracting with credit is permitted. To copy otherwise, or republish, to post on servers or to redistribute to lists, requires prior specific permission and/or a fee. Request permissions from permissions@acm.org. \\
% {\emph{CHI'14}}, April 26--May 1, 2014, Toronto, Canada. \\
% Copyright \copyright~2014 ACM ISBN/14/04...\$15.00. \\
% DOI string from ACM form confirmation}
%% EXAMPLE END -- HOW TO OVERRIDE THE DEFAULT COPYRIGHT STRIP -- (July 22, 2013 - Paul Baumann)


% Arabic page numbers for submission. 
% Remove this line to eliminate page numbers for the camera ready copy
\pagenumbering{arabic}


% Load basic packages
\usepackage{balance}  % to better equalize the last page
\usepackage{graphics} % for EPS, load graphicx instead
\usepackage{times}    % comment if you want LaTeX's default font
\usepackage{url}      % llt: nicely formatted URLs
\usepackage{color}

\usepackage{pdflscape}
\usepackage{longtable}
% \usepackage{geometry}
\usepackage[strict]{changepage}
\usepackage{german}

% llt: Define a global style for URLs, rather that the default one
\makeatletter
\def\url@leostyle{%
  \@ifundefined{selectfont}{\def\UrlFont{\sf}}{\def\UrlFont{\small\bf\ttfamily}}}
\makeatother
\urlstyle{leo}


% To make various LaTeX processors do the right thing with page size.
\def\pprw{210mm}
\def\pprh{297mm}
\special{papersize=\pprw,\pprh}
\setlength{\paperwidth}{\pprw}
\setlength{\paperheight}{\pprh}
\setlength{\pdfpagewidth}{\pprw}
\setlength{\pdfpageheight}{\pprh}

% Make sure hyperref comes last of your loaded packages, 
% to give it a fighting chance of not being over-written, 
% since its job is to redefine many LaTeX commands.
\usepackage[pdftex]{hyperref}
\hypersetup{
pdftitle={Supporting public deliberation through spatially enhanced dialogues},
pdfauthor={LaTeX},
bookmarksnumbered,
pdfstartview={FitH},
colorlinks,
citecolor=black,
filecolor=black,
linkcolor=black,
urlcolor=black,
breaklinks=true,
}

% create a shortcut to typeset table headings
\newcommand\tabhead[1]{\small\textbf{#1}}

% End of preamble. Here it comes the document.
\begin{document}

\title{Supporting public deliberation\\through spatially enhanced dialogues}
\subtitle{Master thesis}

\numberofauthors{1}
\author{
  \alignauthor Gerald Pape\\
    \affaddr{Institute for Geoinformatics}\\
    \email{g.pape@uni-muenster.de}\\
%   \alignauthor 2nd Author Name\\
%     \affaddr{Affiliation}\\
%     \affaddr{Address}\\
%     \email{e-mail address}\\
%     \affaddr{Optional phone number}    
%   \alignauthor 3rd Author Name\\
%     \affaddr{Affiliation}\\
%     \affaddr{Address}\\
%     \email{e-mail address}\\
%     \affaddr{Optional phone number}
}

\maketitle

\begin{abstract}
swaghetti yolonaise
\end{abstract}

% \keywords{
% 	Guides; instructions; author's kit; conference publications;
% 	keywords should be separated by a semi-colon.
% 	\textcolor{red}{Mandatory section to be included in your final version.}
% }

% \category{H.5.m.}{Information Interfaces and Presentation (e.g. HCI)}{Miscellaneous}

% See: \url{http://www.acm.org/about/class/1998/}
% for more information and the full list of ACM classifiers
% and descriptors. 
% \textcolor{red}{Mandatory section to be included in your
% final version. On the submission page only the classifiers'
% letter-number combination will need to be entered.}

\section{Introduction}
Since their first appearance, Web 2.0 applications utilized their collaborative character to gather information and opinions from their users. Today, modern information technologies are ubiquitous in many aspects of daily life. Involving citizens in decision processes around public matters through such applications has formed the field of ``eParticipation''. Its premise is to strengthen democratic processes between citizens and its governments through said modern information technologies \cite{Saebo_eParticipation, Medaglia2012_eParticipation}. One of many aspects is public deliberation, which revolves around engaging citizens in dialogues about their surroundings and decision processes.

Following this idea, Rinner proposed the idea of ``Argumentation Maps'' \cite{Rinner_ArgumentationMaps}. 


\section{Related Work}

\subsubsection{Argumentation mapping}
Rinner\cite{Rinner_ArgumentationMaps}\dots

Existing implementations\dots

Evaluation\dots
\subsubsection{Public deliberation and eParticipation}


\section{Approach}



\subsubsection{DialogMap}

In order to test the initial idea of supporting public deliberation through spatially enhanced dialogues, a working prototype had to be developed. Starting from an initial survey of existing research, a first prototypical application was developed. This prototype was then extended and refined with practical advice from members of a scientific citizens' initiative. Their input ranged from general suggestions to opinions of specific features. This chapter will give some details to design and implementation of the developed prototype.


\subsubsection{Design} % and features

As seen in X,Y and Z, important aspects of A are\dots

Internally, the prototype uses few data models. Contributions contain a title, description, two categories, a tags field, a favored counter, an optional time restriction field for start and ending times, an optional image, an optional reference to a parent contribution and optional references to child contributions. The parent and child contribution references create a simple parent-child connection between contributions, as children inherit the categories, tags, time restriction and title. A contribution serves both as a topic and as response to a topic. A contribution also contains references to features, references to feature references and references to URLs.\\
Features are geospatial entities with a location, a reference to its contribution and properties for styling\footnote{\url{https://github.com/mapbox/simplestyle-spec}}.\\
Feature references contain a title, which can differ from the original title of the feature and a reference to a feature. URL references contain hyperlinks and a description of the hyperlink. The description of a contribution contains the text typed by a user with specially encoded references to features, URL references and feature references. Additionally, each contribution stores the ids of the users who favorited it.\\
Users can create contributions in the manner of creating topics or writing responses to existing topics. Users have an e-mail address and a name. Figure \ref{fig:data_structure} depicts a generalized data structure diagram.

\begin{figure}[!h]
    \centering
    \includegraphics[width=1\columnwidth]{data_structure}
    \caption{Schema of the underlying data structure of the prototype. Time fields and ids are omitted for brevity.}
    \label{fig:data_structure}
\end{figure}

The front page of the prototype puts a map side by side with a sidebar. The right hand sidebar contains the input form for new contributions, filter options, sorting order selector and the list of contributions. The input form consists of input fields for title, categories, time restriction, image and description. The description field allows the creation of spatial features and URL/feature references through connecting words with spatial representations or URLs.\\
A text area for arbitrary text and multiple checkboxes allow to restrict the listed contribution as well as the geo-features displayed in the map. It is also possible to change the sorting order of the list of contribution through a drop down field.\\
The list of contributions contains colored rectangles representing the different topics. Each rectangle contains the title, time of writing, name of the author, categories, tags and the amount of times the contribution has been favored by users. It also contains a link which navigates to the replies written to the topic. A click on the contribution rectangle expands it vertically, revealing the description of the current topic.\\
After clicking the ``reply'' link, only the selected topic and replies are shown in the sidebar in a chronological order. In this view, each contribution shows the description by default as well as author and time and date of writing. The author of the contribution is able to edit and delete the contribution. Upon deletion, the user can enter a reason for deletion, which then will be displayed below the deleted contribution. The deletion is not destructive. The contribution as well as features created for the contribution are marked visually as deleted. Other users are able to favor contribution to show interest or agreement.\\
The map view contains a base map and several markers and polygons in different colors and different icons in case of markers. These relate to the contributions and are connected through the references in the description of the contributions. Which spatial features are displayed is determined through the state of the sidebar. In the topics overview, only the features created for the starting contributions are displayed in order to prevent cluttering of the view-port. When only a topic and its replies are displayed in the sidebar, all features related to the topic and its replies are shown on the map.\\
To emphasize the relationship between a contribution and its spatial features, a two way highlighting has been implemented. Hovering over either a contribution-box, marked word or spatial feature on the map triggers visual highlighting on all related contributions, marked words and spatial features. This allows to quickly grasp the relationship between features and contributions.\\
Users are able to use either traditional sing-up/sign-in methods or sign-in through different social log-in providers to authenticate to the system.

% \begin{itemize}
% \item Map view with sidebar on right hand side
% \item Two way highlighting between contributions in sidebar and features on map
% \item Creation of Topic with \begin{itemize}
%     \item Title
%     \item Category/ies \begin{itemize}
%         \item in this specific case for two dimensions \begin{itemize}
%             \item Color
%             \item Icon
%             \end{itemize}
%         \end{itemize}
%     \item Tags
%     \item Time limit
%     \item Image
%     \item Special Description field which allows to create \begin{itemize}
%         \item Points and Polygons
%         \item References to existing Points and Polygons
%         \item Hyperlinks
%         \end{itemize}
%     \end{itemize}
% \item Sorting
% \item Filter \begin{itemize}
%     \item Fulltext
%     \item Categories
%     \item Tags
%     \item Time
% \end{itemize}
% \item Favorites
% \item Register/Sign in \begin{itemize}
%     \item with Google,Facebook, Twitter
%     \end{itemize}
% \end{itemize}


\subsubsection{Implementation}
\textit{DialogMap} has been implemented as a single-page web application using AngularJS\footnote{\url{http://angularjs.org/}} and Ruby on Rails\footnote{\url{http://rubyonrails.org/}}. The single-page structure was chosen in order to provide the user with a clear navigation between the overview and contribution answers. This also allows for a seamless browsing experience without full reloads of the page. AngularJS is a JavaScript framework with features like templating, two-way binding and DOM manipulation. It follows the model-view-controller pattern in order to bring server side paradigms to client-side development. AngularJS was chosen because of its popularity, extensibility and high number of available libraries. It also enables to wrap existing JavaScript libraries to be used in AngularJS context.\\
The mapping library Leaflet\footnote{\url{http://leafletjs.com/}} serves as base for displaying base maps and geospatial data. The user-facing web page was developed using tools like CoffeeScript\footnote{\url{http://coffeescript.org/}}, Haml\footnote{\url{http://haml.info/}} and Sass\footnote{\url{http://sass-lang.com/}} to speed up the development. The web page was developed with all major browsers in mind.\\
On the server side, components were developed using the Ruby on Rails framework with PostgreSQL\footnote{\url{http://www.postgresql.org/}}/PostGIS\footnote{\url{http://postgis.net/}} as data storage. Ruby on Rails, originally a full-stack model-view-controller web framework, is used as a JSON serving application logic. It was chosen because of its maturity and high number of available libraries. Front- and backend of the prototype communicate in REST\footnote{Representational State Transfer}-API\footnote{Application programming interface} like manner. This allows for easily replaceable front- and backend application stacks.\\
Figure \ref{fig:screenshot} shows the front page of the prototype with an active two way highlight.\\
Without the extensive use of open source software and code, development would have taken much longer. It is planned to release the source code through github\footnote{\url{https://github.com/ubergesundheit/dialogmap}}.

\begin{figure}[!h]
    \centering
    \includegraphics[width=0.9\columnwidth]{screenshot}
    \caption{Screenshot of the front page of \textit{DialogMap} with active highlight of a contribution and spatial feature.}
    \label{fig:screenshot}
\end{figure}


\section{Evaluation}

Interviews

Utility evaluation

Types of questions and why chosen

Results

\section{Conclusion}

This work discusses the implementation and quantitative evaluation of an prototype to support public deliberation through spatially enhanced dialogues in a specifc context.

\subsubsection{Future Work}
Pick up shortcomings emerged during evaluation. Point to solutions...

Legal implications of running such a website have to be explored.

% \section{Figure and Table Example}

% \begin{figure}[!h]
% \centering
% \includegraphics[width=0.9\columnwidth]{Figure1}
% \caption{With Caption Below, be sure to have a good resolution image
%   (see item D within the preparation instructions).}
% \label{fig:figure1}
% \end{figure}

% \begin{table}
%   \centering
%   \begin{tabular}{|c|c|c|}
%     \hline
%     \tabhead{Objects} &
%     \multicolumn{1}{|p{0.3\columnwidth}|}{\centering\tabhead{Caption --- pre-2002}} &
%     \multicolumn{1}{|p{0.4\columnwidth}|}{\centering\tabhead{Caption --- 2003 and afterwards}} \\
%     \hline
%     Tables & Above & Below \\
%     \hline
%     Figures & Below & Below \\
%     \hline
%   \end{tabular}
%   \caption{Table captions should be placed below the table.}
%   \label{tab:table1}
% \end{table}



% Balancing columns in a ref list is a bit of a pain because you
% either use a hack like flushend or balance, or manually insert
% a column break.  http://www.tex.ac.uk/cgi-bin/texfaq2html?label=balance
% multicols doesn't work because we're already in two-column mode,
% and flushend isn't awesome, so I choose balance.  See this
% for more info: http://cs.brown.edu/system/software/latex/doc/balance.pdf
%
% Note that in a perfect world balance wants to be in the first
% column of the last page.
%
% If balance doesn't work for you, you can remove that and
% hard-code a column break into the bbl file right before you
% submit:
%
% http://stackoverflow.com/questions/2149854/how-to-manually-equalize-columns-
% in-an-ieee-paper-if-using-bibtex
%
% Or, just remove \balance and give up on balancing the last page.
%
\newpage
\balance

% REFERENCES FORMAT
% References must be the same font size as other body text.

\bibliographystyle{acm-sigchi}
\bibliography{masterthesis-pape}


\clearpage
\newpage
\onecolumn

\begin{landscape}


\section{Appendix A. Semi-Structured Interview and Expert Interview Guidelines}
\subsubsection{Appendix A.1. Semi-Structured Interview Guideline (in German)}
The interview guideline was developed following rules of Helfferich \cite{helfferich2005}. It is in german as the interviews were held in german. Participants were shown the developed application prior to the interview.
\begin{adjustwidth}{-8em}{-8em}

\begin{longtable}{|p{6.45cm}|p{6.45cm}|p{6.45cm}|p{6.45cm}|}
 \hline
 \textbf{Leitfrage (Erz{\"a}hlaufforderung)}&\textbf{Check -- Wurde das erw{\"a}hnt? Memo f{\"u}r m{\"o}gliche Nachfragen -- nur stellen wenn nicht von allein angesprochen! Formulierung anpassen}&\textbf{Konkrete Fragen -- bitte an passender Stelle (auch am Ende m{\"o}glich) in dieser Formulierung stellen}&\textbf{Aufrechterhaltungs- und Steuerungsfragen}\\
 \hline

 \multicolumn{4}{|l|}{\textbf{Teil 1 -- B{\"u}rgerbeteiligung}}\\
 \hline
 
 Erz{\"a}hlen Sie mir {\"u}ber ihre Rolle und Aufgaben in B{\"u}rgerbeteiligung & Wie lange aktiv (Befragter, Projekt)\newline "`Organisator"' oder "`an der Basis"' & & Erz{\"a}hlen Sie noch mehr {\"u}ber\dots \\
 \hline
 
 
 Bitte beschreiben Sie mir die aus ihrer Sicht wichtigsten Aspekte von B{\"u}rgerbeteiligung. & Ziele \newline Nutzen & & \\
 \hline
 
 Bitte geben Sie mir eine Einf{\"u}hrung in ein(e) laufende(s)/ abgeschlossene(s) Initiative/Projekt (spontan entscheiden welches mehr "`dialogische"' Interaktion zwischen B{\"u}rgern und Aktion erfordert)& Methoden f{\"u}r B{\"u}rgerbefragung \newline Wie erfolgreich/Probleme? \newline "`Moderne"' (Social media) methoden angedacht? \newline Form von Beitr{\"a}gen die B{\"u}rger gebracht haben \newline Wie wurden die Aspekte ber{\"u}cksichtigt?
   & Welchen Wert wurde auf Dialoge zwischen den Akteuren gelegt? & Wie ist das ganze dann abgelaufen?\\
 \hline
 
 \multicolumn{4}{|l|}{\textbf{Teil 2 -- Einsatz der Anwendung}}\\
 \hline
 
 Bitte geben Sie mir eine Einf{\"u}hrung in das Projekt in dem Sie die Anwendung einsetzen wollen. & Zielgruppe (Bev{\"o}lkerungsgruppen, Geographisch) \newline redaktionelle Inhalte \newline erwartete Inhalte \newline Anreize zu Dialogen/Austausch mit B{\"u}rgern? & K{\"o}nnen Sie sich weitere Anwendungsf{\"a}lle f{\"u}r die Verkn{\"u}pfung von Texten mit Karten neben B{\"u}rgerbeteiligung vorstellen? & Erz{\"a}hlen Sie noch mehr {\"u}ber\dots \\
 \hline
 
 Welche Gr{\"u}nde sprechen f{\"u}r den Einsatz dieser L{\"o}sung gegen{\"u}ber anderen L{\"o}sungen. & Bedingungen (technisch, funktional) \newline angedachte Alternativen und deren Defizite \newline B{\"u}rgerbeteiligungsaspekte ber{\"u}cksichtigt? & Welche Eigenschaften w{\"u}rden Sie davon abhalten solch eine Anwendung einzusetzen? \newline Was k{\"o}nnte B{\"u}rger davon abhalten sich durch die Anwendung zu beteiligen? & \\
 \hline

 \multicolumn{4}{|l|}{\textbf{Teil 3 -- Abschlie{\ss}ende Fragen}}\\
 \hline
 
 Kennen Sie Beispiele f{\"u}r die Verkn{\"u}pfung geographischer Daten mit Diskussionsbeitr{\"a}gen? & Next Kassel/Hamburg \newline Frankfurt Gestalten \newline Shareabouts \newline collaborativemap.org & & \\
 \hline
 
 Haben Sie sich dort beteiligt? & In welcher Form & & Wie ist das ganze dann abgelaufen? \\
 \hline
 
 Kennen Sie Werkzeuge um interaktive Karten mit eigenen Inhalten zu erzeugen? & Google Map Maker \newline Here Map Creator \newline Wikimapia \newline Unclemap & & \\
 \hline
 
 Haben Sie schonmal ein solches Werkzeug eingesetzt? & Wie? & &\\
 \hline
 
 \end{longtable}
\end{adjustwidth}
\end{landscape}
\newpage

\subsubsection{Appendix A.1. Expert Interview Guideline}

As mentioned by Helfferich \cite{helfferich2005}, can handle more direct questions. Therefore these questions are much more straightforward. Because the interviews were held in german, the questions are also in german. Prior to the interview, the developed application was demoed.
\begin{itemize}
    \item Kennen Sie Anwendungen die Diskussionen durch Geoobjekte unterst{\"u}tzen?
    \item Welche davon haben Sie in der Vergangenheit schon einmal benutzt?
    \item Z{\"a}hlen Sie bitte die Vor- und Nachteile dieser Anwendungen auf
    \item Welche Anwendungsf{\"a}lle f{\"u}r die Verkn{\"u}pfung von Diskussionen und Geoobjekten k{\"o}nnen Sie sich außerhalb des B{\"u}rgerbeteiligungskontextes vorstellen?
    \item Welche L{\"o}sungen um B{\"u}rger mit Initiativen/Politik zusammenzubringen kennen Sie?
    \item Wie l{\"a}uft die Kommunikation zwischen den B{\"u}rgern und Initiativen/Politik bei diesen L{\"o}sungen ab?
    \item Denken Sie die explizite Verkn{\"u}pfung von Geoobjekten mit Diskussionsgegenst{\"a}nden ist generell hilfreich im B{\"u}rgerbeteiligungskontext / bei Dialogen?
    \item Im Vergleich zu den Anwendungen die Sie kennen, was denken Sie {\"u}ber die folgenden Funktionen der eben vorgestellten Anwendung?
        \begin{itemize}
            \item Verstecken von Geoobjekten die zu Antworten erstellt worden sind; In der Themenansicht nur die Geoobjekte der initialen Beitr{\"a}ge auf der Karte
            \item Zwei Wege Highlights von Geoobjekten und Beitr{\"a}gsboxen
            \item Filter und Sortierung
            \item Verfassen/Antworten
            \item Verkn{\"u}pfen von W{\"o}rtern mit neuen Geoobjekten, bestehenden Geoobjekten und Links
            \item Favorisierung von Beitr{\"a}gen
            \item Benutzerregistrierung/Anmeldung (und Social Login)
        \end{itemize}
    \item Werden ihrer Meinung nach Dialoge vereinfacht oder unterst{\"u}tzt?
    \item Welche Funktionen haben Sie vermisst?
    \item Was k{\"o}nnte B{\"u}rger davon abhalten sich durch die Anwendung zu beteiligen?

\end{itemize}

\twocolumn
\section{Appendix B. Transcribed Interviews}

\subsubsection{Appendix B.1. Transcription System}
The interviews and the focus group were transcribed the following these rules (Rules from Kuckarz \cite{kuckartz2007} with modifications):
\begin{enumerate}
    \item The transcription is literal. Dialects are not transcribed.
    \item Punctuation and language are modified to match grammar and syntax of the german language.
    \item All personal details and mentions are removed and anonymized to prevent re-identification.
    \item Pauses and breaks are marked with ellipses (\dots).
    \item Agreeing sounds like ``Mhms'', ``Ahas'', etc. of the interviewer are not transcribed if they did not interrupt the interviewee.
    \item Interjections of the other person are in brackets.
    \item Supporting or clarifying sounds of the interviewee like laughing or sighing are noted in brackets.
    \item Passages of the interviewing person are denoted with ``I:'', passages of the interviewed person with a distinct abbreviation like ``P1:''.
\end{enumerate}

\subsubsection{Appendix B.2. Demo and Introduction to the Application}

In each interview, the interviewer introduced and demoed the application to the interviewed person. This exemplary introduction was transcribed from the interview with participant 1. In order to retain brevity, it is omitted in the other transcriptions.

\label{demo}
In each interview, the interviewer introduced and demoed the application to the interviewed person. This exemplary introduction was transcribed from the interview with participant 1. In order to retain brevity, it is omitted in the other transcriptions.

\textbf{I:}\\
Hallo, erstmal vielen Dank, dass Sie sich hier Zeit f{\"u}r mich und meine Masterarbeit nehmen. Es soll jetzt gleich hier um die von mir entwickelte Anwendung gehen. Danach werde ich Ihnen noch ein paar Fragen stellen. Also, meine Masterarbeit hat das Thema "`Supporting public deliberation through spatially enhanced dialogs"'. Das bedeutet grob, dass ich herausfinden will, wie man Dialoge in der B{\"u}rgerbeteiligung durch kartenbasierte Anwendungen unterst{\"u}tzen kann.\\
Also, ich fange dann einfach mal mit der Demo der Anwendung an. (ruft die Anwendung auf) Wenn man als Benutzer auf die Webseite kommt, dann wird man beim ersten Besuch erstmal hier mit so einem Text begr{\"u}{\ss}t, der erstmal ein bisschen das Thema vom Nachhaltigkeitstag erkl{\"a}rt. Bei der Entwicklung der Karte wurde ich von Mitgliedern des "`Arbeitskreis Gemeinsam Nachhaltig"' unterst{\"u}tzt. Die sind vom Institut f{\"u}r Soziologie hier in M{\"u}nster. Das sah dann so aus, dass wir uns an mehreren Treffen {\"u}ber die Entwicklung der Anwendung unterhalten haben. Die Soziologen haben mir dann ihre Meinung und Ideen gesagt, die ich dann versucht habe umzusetzen. Also letztendlich soll wohl wahrscheinlich die Anwendung dann n{\"a}chstes Jahr in dem geplanten Nachhaltigkeitstag eingesetzt werden. (klickt auf weiter) Hier kriegt man so kleine Einf{\"u}hrungsvideos, aber das mache ich ja jetzt m{\"u}ndlich, da brauchen wir uns die nicht anschauen. (klickt weiter) Dann hier die Zeichenerkl{\"a}rung f{\"u}r die Marker in der Karte. Man sieht ja hier, dass es zwei Dimensionen gibt. Das sind hier die Akteure, gekennzeichnet durch die Farbe und dann die Aktivit{\"a}t durch den Buchstaben im Marker. (f{\"a}hrt mit der Maus {\"u}ber die Tabellenzellen) Hier kann man sich dann noch kleine Erkl{\"a}rungen zu den Akteuren und Aktivit{\"a}ten durchlesen. (klickt auf weiter) Dann gibts hier noch einen Text zum Kontext der Anwendung. (klickt auf "`Alles klar, ich will loslegen!"')\\
So, dann sind wir hier jetzt in der {\"U}bersicht der Anwendung. Man sieht direkt die Karte und am rechten Rand gibt es noch so eine Seitenleiste. In der Seitenleiste finden sich die Beitr{\"a}ge, ein Filter und die Eingabemaske f{\"u}r das Einstellen von neuen Themen. (f{\"a}hrt mit der Maus {\"u}ber einen Marker in der Karte) Hier, wenn man jetzt mit der Maus {\"u}ber so einen Marker f{\"a}hrt, dann sieht man direkt, dass der Marker visuell hervorgehoben wird. Also der orangene Ring und das Popup. Gleichzeitig wird der zugeh{\"o}rige Beitrag in der Seitenleiste hervorgehoben, damit man direkt sehen kann, zu welchem Beitrag der Marker geh{\"o}rt. (f{\"a}hrt mit der Maus {\"u}ber einen Beitrag in der Seitenleiste) Das ganze funktioniert dann auch in der anderen Richtung. Man kann direkt dann sehen welche Marker zu dem Beitrag geh{\"o}ren. (klickt auf einen Beitrag in der Seitenleiste) So, dann kann man die Beitr{\"a}ge hier noch so ausklappen um dann auch den Beschreibungstext lesen zu k{\"o}nnen. Ja, also, neben dem Beschreibungstext kann man dann hier auch lesen von wann der Beitrag ist und wer ihn geschrieben hat. Sonst sind hier noch der Akteur, Aktivit{\"a}t und Inhalte zu lesen. Dann gibts hier noch dieses Herzchen mit der Zahl davor. Das zeigt an, wie oft der Beitrag von den Benutzern favorisiert worden ist. Das ist so {\"a}hnlich wie ein Facebook "`Gef{\"a}llt mir"'. (tippt ein paar Buchstaben in den Filter) So, dann gibts hier noch den Filter. Hier kann man entweder direkt nach einem Wort suchen oder dann mit den Filteroptionen (klickt auf "`Filter einblenden"') entweder nach Akteur, Aktivit{\"a}t oder Inhalt filtern. Hier, ganz unten, gibt es noch die Filteroption "`Zeitraum unbegrenzt"'. Das ist noch eine Besonderheit. Die Beitr{\"a}ge k{\"o}nnen ein Ablaufdatum bekommen. Dann werden die nach Ablauf des Ablaufdatums dann auch nicht mehr auf der Karte und in der Seitenleiste angezeigt. Hier, mit der Option, kann man sie dann wieder einblenden. (klickt auf "`alle Filter zur{\"u}cksetzen"' und dann auf "`Filter ausblenden"') So und dann kann man noch hier noch die Sortierreihenfolge in der Seitenleiste ver{\"a}ndern. Da kann man dann sortieren, wie man die Beitr{\"a}ge hier gerne h{\"a}tte. Achso, hier, in den Beitr{\"a}gen, kann man dann auch schon sehen wie viele Antworten das Thema schon hat. (klickt auf "`Antworten anzeigen"') Wie Ihnen wahrscheinlich schon aufgefallen ist, hat sich nicht nur der Kartenausschnitt ver{\"a}ndert, sondern es werden jetzt auch andere Marker als gerade angezeigt. Das ist auch so eine Besonderheit. In der {\"U}bersicht werden nur die Marker angezeigt, die zu den initialen Beitr{\"a}gen der Themen erstellt worden sind. Das ist, damit die Karte nicht zu schnell zu voll wird, und man noch den {\"U}berblick hat. Hier in den Antworten k{\"o}nnen Sie ja sehen, da sind so unterschiedlich farbige Boxen. Die Gr{\"u}nen zeigen an, dass da ein Marker verlinkt worden ist, die Braunen zeigen, dass ein bestehender Marker referenziert worden ist, und blau bedeutet, dass dort eine Webseite verlinkt worden ist.\\
Also schreiben wir jetzt einfach mal eine Antwort. (klickt auf "`Antwort verfassen"') Hier hat sich jetzt die Eingabemaske f{\"u}r Antworten ausgeklappt. Da kann man nur eine Beschreibung mit Geoobjekten, referenzierten Geoobjekten und Link angeben und ein Bild anh{\"a}ngen. Der Rest Attribute wie Akteur und so werden vom Thema geerbt. (tippt eine Beschreibung) So nachdem man nun seinen Text verfasst hat, kann man noch W{\"o}rter verlinken. Das geht, wie gesagt mit einem neuen Marker, einem bestehenden Marker oder einem Link zu einer Webseite. Dazu muss man hier ein Wort oder halt mehrere W{\"o}rter markieren. (markiert ein Wort mit der Maus) Dann sieht man hier so ein Kontextmen{\"u} mit drei Buttons. Der erste ist f{\"u}r einen neuen Marker, der zweite um einen bestehenden Marker zu verkn{\"u}pfen und der dritte um eine Webseite zu verlinken. (klickt auf den "`Marker"'-Button) Ich mach jetzt hier mal einen neuen Marker. (klickt in die Karte) So jetzt wurde das Wort und der Ort, an dem ich den Marker gesetzt habe, verkn{\"u}pft. (markiert ein anderes Wort und klickt auf den "`Verkn{\"u}pfen"'-Button) Genauso funktioniert das dann auch mit dem Verkn{\"u}pfen. (klickt auf einen bestehenden Marker in der Karte). Und dann nochmal mit der Webseite (markiert ein drittes Wort und klickt auf den "`Link"'-Button) Hier beim Webseiten-verlinken {\"o}ffnet sich dann so ein kleines Eingabefeld, in dem man dann den Link reinschreiben kann. (schreibt einen Link in das Feld und dr{\"u}ckt die Enter-Taste auf der Tastatur) So lange man die Antwort noch nicht abgeschickt hat, kann man auch noch alles l{\"o}schen und dann ist es auch weg. Sp{\"a}ter beim Editieren geht das nicht mehr. So, wenn man jetzt noch ein Bild hat, kann man das hier unten anh{\"a}ngen noch. Das funktioniert aber so wie man es erwartet, ich hab jetzt auch keines gerade. (klickt auf "`Abschicken"') So, da man noch nicht eingeloggt ist, kann man nat{\"u}rlich den Beitrag jetzt noch nicht abschicken. Also Verfassen geht, Abschicken aber nicht. Da kommt man dann hier zu dem Login-Dialog. Hier kann man sich entweder mit Facebook, Twitter oder Google einloggen oder halt auch ganz traditionell mit E-Mail und Passwort, wenn man sich vorher hier auch registriert hat. (loggt sich ein). Dann kann man jetzt auch die Antwort abschicken.\\
So, wenn man jetzt merkt, dass man sich verschrieben hat oder dass der Marker falsch ist, kann man jetzt den Fehler korrigieren. Dazu muss man hier auf den kleinen Stift dr{\"u}cken, (klickt auf den Stift) und kann dann den Beitrag bearbeiten. ({\"a}ndert einen Buchstaben, verschiebt den Marker und {\"a}ndert den Link) Man kann das hier durch draufklicken auf den kleinen Kasten ausl{\"o}sen, das {\"A}ndern des Links. (klickt auf "`Abschicken"') So, jetzt sind die {\"A}nderungen gespeichert und man kann direkt sehen, dass jetzt hier auch "`ge{\"a}ndert am"' steht. So dann gibts hier noch den Button mit der kleinen Tonne. Damit kann man den Beitrag l{\"o}schen. Editieren und l{\"o}schen geht nat{\"u}rlich nur bei Beitr{\"a}gen, von denen man selber der Autor ist. Das L{\"o}schen ist dann auch kein richtiges L{\"o}schen, sondern da kann man dann einen Grund angeben, warum der Beitrag gel{\"o}scht werden soll. (klickt auf die kleine Tonne) Also hier kann man den Grund angeben (tippt Grund ein und klickt auf "`Ja, Beitrag l{\"o}schen"') So, dann sieht man direkt, dass der Beitrag ausgegraut wird, durchgestrichen und dann auch noch die Marker in hellerer Farbe dargestellt werden. Man kann nicht komplett l{\"o}schen, weil sonst der Sinn von den Diskussionen verloren gehen k{\"o}nnte. Die Themenstarter kann dann auch nichtmal der Autor l{\"o}schen. So, wenn man denn nun jetzt nicht der Autor eines Beitrages ist, dann kann man hier mit dem kleinen Herzchen den Beitrag favorisieren. (klickt auf das Herzchen) Dann sieht man auch direkt, die Zahl hier unten neben dem anderen Herzchen erh{\"o}ht sich und das Herzchen wird ausgef{\"u}llt. Das bedeutet, dass man selbst den Beitrag favorisiert hat. Das ganze kann man dann nat{\"u}rlich auch wieder entfavorisieren, wenn man wieder auf das kleine Herzchen klickt. (klickt auf den "`zur{\"u}ck"'-Button) So okay, dann wollen wir auch nochmal ein neues Thema erstellen. (klickt in das "`Titel"'-Feld) So, hier hat sich jetzt die Eingabemaske f{\"u}r ein neues Thema ausgeklappt. Hier kann man dann den Titel, einen Akteur, eine Aktivit{\"a}t und mehrere Inhalte ausw{\"a}hlen. Dann kann man hier einen Start- und Endzeitpunkt ausw{\"a}hlen. Darunter, das kennen Sie ja schon von eben, kann man eine Beschreibung eingeben und darunter noch ein Bild anh{\"a}ngen. (f{\"u}llt die Felder aus und klickt "`Abschicken"') So und dann hat man hier ein neues Thema.\\
Hier unten auf der Seite kommt man dann auch nochmal zu der Zeichenerkl{\"a}rung und hier oben nochmal zu den Erkl{\"a}rungsvideos. Ja, also, das war es jetzt erstmal zur Anwendung. Wie man schon erahnen kann, l{\"a}sst sich die Anwendung jetzt super benutzen, um damit Dialoge zu f{\"u}hren oder Ideen und Meinungen einzuholen. Man k{\"o}nnte also zum Beispiel hier ein neues Thema erstellen der sich mit dem Ort des Nachhaltigkeitstages auseinandersetzt. Jemand erstellt dann ein neues Thema, sagt "`Hier soll der Nachhaltigkeitstag dann stattfinden. Was haltet ihr davon? Hat jemand eine andere Idee?"', markiert einen Ort in der Karte und sendet ab. Andere Benutzer k{\"o}nnen sich dann dar{\"u}ber austauschen und auch neue Orte vorschlagen, die dann auch diskutiert werden k{\"o}nnen.


\subsubsection{Appendix B.3. Participant 1}

\textbf{Teil 1 -- B{\"u}rgerbeteiligung}
\begin{itemize}
    \item[I:] Erz{\"a}hlen Sie mir {\"u}ber ihre Rolle und Aufgaben in B{\"u}rgerbeteiligung
    \item[P1:] Ich kann nur sagen was f{\"u}r mich wichtig ist
    \item[I:] Ja, das ist auch eine Sache die Sie mir erz{\"a}hlen k{\"o}nnen. Dann beschreiben Sie mir bitte die aus ihrer Sicht wichtigsten Aspekte der B{\"u}rgerbeteiligung
    \item[P1:] Die B{\"u}rgerbeteiligung f{\"u}hrt dazu, dass erstmal Leute sich informieren, dass sie mehr wissen, als nur {\"u}ber Zeitung. Dann k{\"o}nnen sie sich auch zusammenschlie{\ss}en und diskutieren und Aktionen besprechen. Ja und auch entsprechend Aktionen machen. Das st{\"a}rkt auch im Grunde eine Stadt.
    \item[I:] An welchen B{\"u}rgerbeteiligungsaktionen haben Sie dann schonmal teilgenommen?
	\item[P1:] Ja, die Frage ist jetzt, was alles unter B{\"u}rgerbeteiligung f{\"a}llt?
	\item[I:] Da kann alles drunter fallen, was f{\"u}r die {\"O}ffentlichkeit geschieht.
	\item[P1:] Also, ich habe zum Beispiel mit Leuten zusammen einen Gemeinschaftsgarten, der ist gegr{\"u}ndet worden und da treffen wir uns, da wird Gem{\"u}se angebaut, da gibts Bienen und da ist auch gedacht, dass man sich noch in Zukunft wenn das mal l{\"a}uft sich vernetzt mit anderen G{\"a}rten. Zum Beispiel zum Thema Bienen haben wir einen Nachmittag gehabt. Aber das kann man nat{\"u}rlich auch in einem gr{\"o}{\ss}eren Ma{\ss}stab machen.
	\item[I:] Und w{\"u}rden Sie denken, dass in diesem Kontext so eine Art von geographischer Diskussions-Anwendung dann sinnvoll einzusetzen w{\"a}re um das ganze bekannter zu machen und die Inhalte nach au{\ss}en zu kommunizieren?
	\item[P1:] Ja, erstmal stell ich mir das so vor, dass jemand, der meinetwegen neu ist oder keine Kontakte hat, sich mit Hilfe der Karte {\"u}berhaupt mal ein Bild machen kann was es f{\"u}r M{\"o}glichkeiten gibt. Und dann geht es ja in die Feindifferenzierung. Da w{\"u}rde er sagen: "`Gut, ich interessiere mich f{\"u}r Umwelt. Wer ist zust{\"a}ndig f{\"u}r Umwelt? Naja Greenpeace kann ich mal anklicken. Wo treffen die sich? Wann treffen die sich? Was haben die f{\"u}r Aktivit{\"a}ten zum Beispiel. Oder Transition Town. Was machen die eigentlich? Muss ich mal lesen was das {\"u}berhaupt ist. Ich wei{\ss} gar nicht genau was das ist. Also kann ich das mal lesen und vielleicht auch Kontakt aufnehmen."' Ich muss mir jetzt nicht m{\"u}hsam diese ganzen Adressen zusammensuchen, diese WWW-Adressen, sondern die sind ja auf deiner Karte schon angegeben. Das ist nat{\"u}rlich schon auch erleichternd. Denn manchmal scheitert es an solchen Sachen. Auch an Bequemlichkeit.
	\item[I:] Wie l{\"a}uft dann im Moment die Kommunikation intern f{\"u}r diesen Garten ab?
	\item[P1:] {\"U}ber E-Mail und {\"u}ber Treffen.
	\item[I:] Wie oft treffen Sie sich da?
	\item[P1:] Ja, da gibt es dann Einladungen. Aber das ist unterschiedlich. Alle zwei Monate, wenn was ansteht. Jetzt, wo das Wasser da ist, da trifft man sich mal um aufzur{\"a}umen oder um Projekte zu besprechen.	
\end{itemize}

\textbf{Teil 2 -- Einsatz der Anwendung}
\begin{itemize}
	\item[I:] Wie soll dann die Beteiligung von Transition Town oder dem Garten auf dem Nachhaltigkeitstag aussehen?
	\item[P1:] Naja, das kann ich jetzt nur erfinden. Letzten Endes m{\"u}ssen wir das ja als Gruppe besprechen.
	\item[I:] Also, am Besten wie Sie sich das vorstellen
	\item[P1:] Themen, die Transition Town wichtig sind, w{\"u}rden da einen Raum finden und den Rahmen m{\"u}ssen die sich dann geben. Ob das jetzt in Form von (\dots) dass man gesundes Essen anbietet oder mal so eine Karte entwirft, wo Transition Town ist. Es gibt ja auch einen Film {\"u}ber Transition Town. Da gibt es ja vielf{\"a}ltige M{\"o}glichkeiten. (\dots) Zum Beispiel in unserem Garten, da hat die Bienen-Frau einen Vortrag gehalten, {\"u}ber die Bienen und das soziale Miteinander. Das ist ja hoch differenziert. Zum Beispiel die Drohnen, die treffen sich an ganz bestimmten Pl{\"a}tzen, vierzig Meter {\"u}ber der Erde. Solche Detailinformationen, die kein Mensch eigentlich wei{\ss}, die k{\"o}nnte man dann geben, indem diese Bienen-Frau vielleicht was mitbringt und dann dar{\"u}ber redet und das dann auch Kindern zeigt, wie so ein Bienenstock aussieht und mal Honig probieren l{\"a}sst. Und dann auch zum Engagieren auffordert. Oder, hab ich heute in der Zeitung gelesen, dass es einen Jungen gibt, der hat ein Bienenhaus gebaut, f{\"u}r den Balkon. Der w{\"u}rde dann eingeladen und w{\"u}rde das vorstellen.
	\item[I:] Wer w{\"a}re dann die Zielgruppe? 
	\item[P1:] Wie meinst du, die Zielgruppe?
	\item[I:] Ich meine damit die Personenkreise, die man ansprechen m{\"o}chte
	\item[P1:] Naja an dem Tag werden ja viele Menschen da sein. Und das w{\"a}re ja dann ein wichtiger Aspekt zum Nachhaltigkeitsthema. Ich meine, da dass ja wahrscheinlich drau{\ss}en stattfindet, kann man ja von Zielgruppe nicht so unbedingt sprechen, oder? Wer will, der kommt.
	\item[I:] Gibt es andere Ans{\"a}tze, die Sie zur Kommunikation bez{\"u}glich des Nachhaltigkeitstages in Betracht gezogen haben?
	\item[P1:] Nein, im Moment nicht.
	\item[I:] Was f{\"u}r Gr{\"u}nde w{\"u}rden f{\"u}r den Einsatz der Karte sprechen? 
    \item[P1:] Also du meinst, was f{\"u}r Vorteile es f{\"u}r uns h{\"a}tte den Garten in deine Karte einzutragen?
    \item[I:] Richtig.
	\item[P1:] Das h{\"a}tte den Vorteil, dass man auf einen Blick sehen kann, da und da und da gibt es einen freien Garten. Man sieht, welche Adresse das sind. Man sieht vielleicht auch, wann die da sind. Und dann ist das nat{\"u}rlich sehr {\"u}bersichtlich. Mit einem Klick hat man sozusagen die Information. Es gibt ja noch mehrere G{\"a}rten. Es gibt da unseren Paradies-Garten, dann gibt es am Campus noch einen Garten, dann gibts noch an der Gasselstiege einen Garten. Ja. Die w{\"u}rde man dann da sehen und dann k{\"o}nnte man auch Leute, die das wollen, meinetwegen eine Fahhradtour machen lassen und die G{\"a}rten angucken. Man k{\"o}nnte die Karte auch benutzen um die Fahrradtour zu organisieren. Dass man Start, Ziel und Zwischenhalte markiert.
	\item[I:] Was k{\"o}nnten Sachen sein, die B{\"u}rger davon abhalten w{\"u}rden diese Karte zu benutzen?
	\item[P1:] (\dots) Ja, also, die Karte ist ja elektronisch. Geht ja nur {\"u}ber das Internet. Also sofern man einen Internetanschluss hat und einen Laptop oder einen Computer, gibt es da nichts, was dagegen spricht.
\end{itemize}
\textbf{Teil 3 -- Abschlie{\ss}ende Fragen}
\begin{itemize}
    \item[I:] Kennen Sie Beispiele f{\"u}r die Verkn{\"u}pfung geographischer Daten mit Diskussionsbeitr{\"a}gen?
	\item[P1:] Sag mir nochmal, was man alles unter geographische Daten fasst.
	\item[I:] Orte und Objekte mit einem Ort
	\item[P1:] Naja, ich war jetzt auf dem Jakobsweg, da hat man auch Karten. Aber die nutzt man nicht so oft. Da hat man B{\"u}cher, in denen die Adressen drin stehen. Und die Zeichen sind an den B{\"a}umen.
	\item[I:] Ja, ist auch eine M{\"o}glichkeit. Ich ziele mit der Frage eher ab auf elektronische Anwendungen.
	\item[P1:] Nein, ich bin da nicht da auch nicht so firm. Ich mag das auch nicht.
	\item[I:] Also haben Sie sowas auch noch nie benutzt?
	\item[P1:] Nein.
	\item[I:] Kennen Sie Werkzeuge um interaktive Karten mit eigenen Inhalten zu erzeugen?
	\item[P1:] Nein. Es gibt ja viele Leute, die nicht so interessiert sind mit den neuen Medien. 
	\item[I:] Ja. Alles klar. Das waren dann die Fragen von meiner Seite. Gibt es noch Fragen von Ihrer Seite? 
    \item[P1:] Nein. Eigentlich nicht. Gute Sache.
    \item[I:] Vielen Dank f{\"u}r Ihre Zeit.
\end{itemize}

\subsubsection{Appendix B.4. Participant 2}

\textbf{Teil 1 -- B{\"u}rgerbeteiligung}
\begin{itemize}
    \item[I:] Erz{\"a}hlen Sie mir {\"u}ber ihre Aufgaben und Rollen in der B{\"u}rgerbeteiligung
    \item[P2:] Also das ist ne schwierige Frage weil ich da gar nicht so Aufgaben oder Rollen habe sondern mir sie eher selbst suche. Das hei{\ss}t, ich mach meistens dass was ich interessant finde. Und jetzt in dem Kontext halt zum Beispiel diese Organisation des Nachhaltigkeitstages und in dem Kontext dann auch mit der Arbeit mit der Karte.
    \item[I:] Und wie lange sind Sie da jetzt schon aktiv? 
    \item[P2:] Es hat denke ich so angefangen mit der Organisation der Tagung. Vor allem in letzter Zeit wieder mehr. (War das letztes Jahr?) Ja letztes Jahr. Oder es hat angefangen 2012. Wahrscheinlich sogar eher Ende 2011 oder so. Aber davor war ich aber auch schonmal so in dem Bereich w{\"a}hrend des Studiums unterwegs. So NGOs und Entwicklungszusammenarbeit und sowas.
    \item[I:] Bitte beschreiben Sie mir aus ihrer Sicht wichtige Aspekte der B{\"u}rgerbeteiliung.
    \item[P2:] Beteiligung. (lacht) Also das ist nat{\"u}rlich erstmal ein theoretisches Konzept, aber in der Praxis w{\"u}rde ich sagen, ist das wichtige dass die Leute wirklich mitmachen und mitgestalten. Das hei{\ss}t nicht nur passiv so da sitzen, sondern halt eine aktive Rolle haben.
    \item[I:] Und die Ziele und Nutzen davon?
    \item[P2:] Ja die Ziele und Nutzen sind erstmal so ne Art von Legitimation von Ma{\ss}nahmen w{\"u}rde ich sagen. Dabei w{\"u}rde ich nichtmal sagen dass das der Hauptpunkt ist, sondern eigentlich dass den Menschen die M{\"o}glichkeit gegeben wird ihr eigenes Umfeld so zu gestalten, wie sie es gerne wollen. Und dass sie in dem Umfeld wo sie leben nicht so beschr{\"a}nkt sind von {\"a}u{\ss}eren Sachen. Sondern halt eher selbstbestimmt alles zu organisieren.
    \item[I:] Bitte geben Sie mir eine Einf{\"u}hrung in das Nachhaltigkeitsprojekt.
    \item[P2:] Also im Prinzip hat es wie gesagt schon begonnen mit der Tagung letztes Jahr. Das war der Startschuss, dass wir uns gedacht haben, nach dieser Tagung muss es eigentlich irgendwie weitergehen. Das haben damals auch alle gesagt dass man danach einen Nachfolgeprozess organisieren wollte. Dazu haben wir dann ein paar Treffen nach der Tagung gemacht. Auf diesen Treffen haben wir dann entschieden, dass wir so einen Tag der Nachhaltigkeit organisieren, was dahin f{\"u}hren soll, dass n{\"a}chstes Jahr, am 15. Juni 2015 ein Tag in M{\"u}nster stattfindet, an dem an verschiedenen Orten Nachhaltigkeitsprojekte vorgestellt werden und {\"o}ffentlich {\"u}ber das Thema diskutiert werden kann. Dazu soll dann auch vorher ein bisschen {\"O}ffentlichkeitsarbeit in den Medien gemacht werden. Das hei{\ss}t dass man versucht den Diskurs somit weiter zu befeuern um das Projekt im Bewusstsein zu halten. Das ist auch dann das Hauptanliegen im Moment.
    \item[I:] Wieviel Wert wurde da im Vorfeld auf Dialoge gelegt?
    \item[P2:] Es kommt drauf an. Wir haben auf dieser Tagung ne Mailingliste angelegt, so dass wir jetzt verschiedene Verteiler haben. Das ist jetzt dann aber nicht f{\"u}r die gesamte {\"O}ffentlichkeit, sondern eher f{\"u}r diesen Kreis der auch da auf der Tagung war. Gleichzeitig haben wir auch {\"u}ber Zeitungsartikel und Einladungen in den Medien versucht Leute zu mobilisieren au{\ss}erhalb des Kreises. Das war allerdings nicht sehr erfolgreich. Insofern versuchen wir das demn{\"a}chst auch nochmal im Oktober oder so aber haben jetzt noch nicht gezielt auf Breitenbeteiligung geschaut.
    \item[I:] Habt ihr euch im Vorfeld schon nur auf Zeitung festgelegt, oder gab es auch in Richtung Social Media vorschl{\"a}ge?
    \item[P2:] Nein, das wurde eher spontan entschieden. Da war nur die Entscheidung dass wir unsere eigenen Netzwerke aktivieren. Das war der eine Pfad und der andere w{\"a}re halt {\"u}ber Medien die breitere {\"O}ffentlichkeit zu erreichen. Also {\"u}ber Social Media haben wir glaube ich gar nicht geworben. Nur halt E-Mail-m{\"a}{\ss}ig {\"u}ber die Listen.
\end{itemize}

\textbf{Teil 2 -- Einsatz der Anwendung}
\begin{itemize}
    \item[I:] Dann geht es jetzt weiter konkret zur Anwendung. Wer ist die Zielgruppe f{\"u}r die Anwendung?
    \item[P2:] Also die Zielgruppe sind potentiell eigentlich erstmal alle Interessierten. Ich w{\"u}rde das gar nicht so eingrenzen wollen. Nat{\"u}rlich in der Praxis sind das dann meistens die jenigen, die sowie so engagiert sind und in dem Bereich arbeiten.
    \item[I:] Was f{\"u}r Inhalte erwarten Sie?
    \item[P2:] Ich geh mal erstmal vom Idealfall aus. Ideal w{\"a}re es, wenn sich die Leute die sowieso schon aktiv in ihren Bereichen sind, der Karte annehmen w{\"u}rden und dar{\"u}ber die Strukturen der Beteiligung in M{\"u}nster im Bereich Nachhaltigkeit einfach digital sichtbar machen w{\"u}rden von allein. Das w{\"a}re dann so eine Form der Datenerhebung in einer gewissen Art und Weise. Das w{\"a}re dann so der Idealfall, wenn dann {\"u}ber diese Prozesse dann Kommunikation in Gang kommt. Das sollte dann auch m{\"o}glichst weit streuen {\"u}ber m{\"o}glichst viele Schichten und Stadtbezirke und so weiter. Das hei{\ss}t dass sich dann so einen Synergieeffekt ergibt. Realistisch gesehen wird es wahrscheinlich nicht ganz so weit gehen, denke ich. Deswegen w{\"a}re vielleicht so der erste Punkt dass das anl{\"a}uft, dass sich andere beteiligen. Das w{\"a}re schon ein erster kleiner Schritt dass es {\"u}ber den Kreis, die da sowieso schon mitmachen, hinaus bekannter wird.
    \item[I:] K{\"o}nnen Sie sich weitere Anwendungen f{\"u}r die Verkn{\"u}pfung von Geoobjekten mit Karten neben der B{\"u}rgerbeteiligung vorstellen?
    \item[P2:] Sicherlich. Da k{\"o}nnte man einfach erstmal Informationen vermitteln (\dots) Hm, es gibt sicherlich noch (\dots) Ja es ist immer die Frage wie man solche Begriffe definiert. Wie eng oder wie breit. Also ob man informieren schon dazu z{\"a}hlt oder ob man sagt, das ist was ganz anderes und so weiter. Oder auch in dem Sinne von so Web 2.0 Anwendungen. Das halt jemand was dazu beitragen kann, ob das schon ne Form von B{\"u}rgerbeteiligung ist, oder ob dazu wirklich eben nur was konkretes in der Stadt, ein Projekt oder so, z{\"a}hlt. Ich k{\"o}nnte mir vorstellen, dass das durchaus auch ne wissenschaftliche Frage ist. Also zum Beispiel was f{\"u}r Projekte sich da eintragen. Ich k{\"o}nnte mir durchaus vorstellen dass man das ganze als Datenbank verwenden k{\"o}nnte in einem wissenschaftlichen Projekt. Vielleicht f{\"u}r Schulen, k{\"o}nnte ich mir das noch ganz gut vorstellen. Dass die irgendwie vielleicht Projektwochen zu einem Thema und dann sehen "`Oh da gibts ja schon was"' und dann sich darauf st{\"u}tzen k{\"o}nnen. Noch eine andere M{\"o}glichkeit die ich gerade noch im Kopf hatte, war, dass sich politische Initiativen dadurch ein bisschen organisieren k{\"o}nnten. Ich glaube wenn man l{\"a}nger dar{\"u}ber nach denkt, k{\"o}nnte man sicherlich auch noch viel mehr Anwendungsm{\"o}glichkeiten finden. Oder auch noch ein wichtiger Punkt, den ich eben schon im Kopf hatte, dass soziale Bewegungen die Karte zur Selbstreflexion benutzen k{\"o}nnten. Dass man erstmal {\"u}berhaupt seine Wirksamkeit sieht. Und auch dass es der Bewegung noch einen Schub gibt, von wegen in Richtung Transparenz.
    \item[I:] Welche Gr{\"u}nde sprechen f{\"u}r den Einsatz dieser L{\"o}sung gegen{\"u}ber anderen L{\"o}sungen?
    \item[P2:] In Bezug auf B{\"u}rgerbeteiligung? Oder in Bezug auf was jetzt?
    \item[I:] Konkret jetzt in diesem Nachhaltigkeitskontext
    \item[P2:] Was daf{\"u}r spricht, ist erstmal dass es jetzt da ist (lacht). Das ist nat{\"u}rlich ein wichtiges Argument. Das andere was daf{\"u}r spricht, ist sicherlich einfach dass es online da ist, und relativ leicht das halt von jedem eingesehen werden kann. Also der Zugang ist einfach relativ offen dadurch dass es dann im Netz ist. Das ist sicherlich ein Plus. Was noch daf{\"u}r spricht, ist sicherlich auch die graphische Darstellung, denke ich. Es erlaubt ja das ganze erstmal so digital zu sehen. Das ist vielleicht nochmal anders, als nur ne Liste von Projekten zu haben oder so. (\dots) Was daf{\"u}r spricht, ist dass die meisten Leute inzwischen Internet benutzen, denke ich. Also dass es relativ breit gestreut ist. Was vielleicht dagegen spricht, ist dass einige Gruppen dadurch nat{\"u}rlich sich auch wieder ausgeschlossen f{\"u}hlen werden. Also, {\"a}ltere zum Beispiel. Da k{\"o}nnte ich mir vorstellen, dass die schwierigen Zugang haben. Das w{\"a}re vielleicht nochmal so ein negatives Argument dass man dagegen setzen k{\"o}nnte oder so.
    \item[I:] Gibt es denn irgendwelche Alternativen die in Betracht gezogen wurden?
    \item[P2:] Wir hatten uns mal gedacht, dass manuell quasi Landkarten ausdrucken k{\"o}nnte und die dann so auf dem Tag der Nachhaltigkeit an verschiedenen St{\"a}nden oder so platzieren k{\"o}nnte. Und dann zum Beispiel so mit Pinnadeln oder so sowas machen k{\"o}nnte. Und dass da kleine Zettel liegen, auf die die Leute draufschreiben k{\"o}nnen welche Initiative das ist. Mit den Pinnen k{\"o}nnten die dann das ganze auf der manuellen Karte machen. Und das k{\"o}nnte man dann sp{\"a}ter sogar vielleicht kombinieren, so dass man die Aspekte dann in die Karte eintr{\"a}gt oder sowas in der Art. Das waren so noch die {\"U}berlegungen die wir hatten.
    \item[I:] Was f{\"u}r Eigenschaften oder Bedingungen w{\"u}rden Sie abhalten diese L{\"o}sung einzusetzen?
    \item[P2:] (\dots) Abhalten (\dots) Wei{\ss} ich jetzt nicht so genau, vielleicht dass es sich in der Praxis sich einfach nicht als praktikabel ergibt. Dass man sieht, die M{\"o}glichkeit ist da, die Leute nutzen es aber nicht. Auch wenn man ihnen die Zug{\"a}nge eigentlich legt. Das w{\"a}re eher so im Nachhinein, dass man nachher nochmal schaut. Sonst w{\"a}re jetzt eigentlich konkret nichts was mir so einfallen w{\"u}rde.
    \item[I:] Was w{\"a}ren aus ihrer Sicht Gr{\"u}nde die B{\"u}rger davon abhalten w{\"u}rden sich nicht durch die Anwendung zu beteiligen?
    \item[P2:] Ich denke das sind insbesondere Fragen der Nutzbarkeit. Das hei{\ss}t, ob es kompliziert sich anzumelden, ob es gut bedienbar ist, oder ob das als gut bedienbar empfinden. Sage ich jetzt erstmal so die Wahrnehmung davon, ob man sich da einfach beteiligen kann, oder nicht. Fehlerfreiheit ist sicherlich ein Punkt. Ich glaube wenn Fehler auftauchen oder etwas nicht funktioniert, dann ist das sehr schnell demotivierend. Das ist vielleicht noch so ein Punkt. (\dots) Was also auch noch abschreckend wirken kann, ist nicht klar kommuniziert ist, in welchem Kontext das ganze steht, wo das herkommt, wer da verantwortlich ist, wie dann mit den Daten umgegangen wird. Also in Richtung Datensicherheit und Datenschutz. (\dots) Das w{\"a}ren jetzt so spontan die Sachen die mir einfallen w{\"u}rden.
\end{itemize}

\textbf{Teil 3 -- Abschlie{\ss}ende Fragen}
\begin{itemize}
    \item[I:] Dann jetzt noch ein paar abschlie{\ss}ende Fragen. Kennen Sie Beispiele f{\"u}r die Verkn{\"u}pfung geographischer Daten mit Diskussionsbeitr{\"a}gen?
    \item[P2:] (\dots) Von Nexthamburg, die haben da ja auch so eine Verkn{\"u}pfung von Karte und den Beitr{\"a}gen. Es gibt noch so ne Karte vom Tag des guten Lebens in K{\"o}ln. Ich wei{\ss} aber nicht ob da Diskussionsbeitr{\"a}ge bei waren, oder ob es nur eine reine Darstellung war. Das wei{\ss} ich nicht mehr so genau. Aber ich denke dass es in diesem Kontext sicherlich noch mehr Beispiele gibt, wenn man mal recherchiert. Da wei{\ss} man dann aber auch nicht wie erprobt oder ausgereift sind.
    \item[I:] Haben Sie sich dann bei einem Projekt beteiligt?
    \item[P2:] Nein.
    \item[I:] Kennen Sie Werkzeuge um interaktive mit Karten mit eigenen Inhalten zu erstellen?
    \item[P2:] Nein. Au{\ss}er jetzt dass was du da jetzt programmiert hast. 
    \item[I:] Ja okay. Dann gibt es noch Anmerkungen oder Fragen von ihrer Seite aus?
    \item[P2:] Nein, konkret eigentlich jetzt nicht.
    \item[I:] Dann vielen Dank f{\"u}r das Interview.
\end{itemize}

\subsubsection{Appendix B.5. Participant 3}

\textbf{Teil 1 -- B{\"u}rgerbeteiligung}
\begin{itemize}
    \item[I:] Erz{\"a}hlen Sie mir {\"u}ber Ihre Rolle und Aufgaben in der B{\"u}rgerbeteiligung
    \item[P3:] Ja. Also ich bin in verschiedenen B{\"u}rgerbeteiligungen und Initiativen aktiv. Also das ist einmal Transition Town. Da bin ich seit fast vier Jahren aktiv im Bereich der Gemeinschaftsg{\"a}rten, der Kerngruppe, Filme im Cinema, (\dots) und alles das was die Sichtbarkeit von Transition Town betrifft. Barackentage, also Initiativen so dann bin ich da dann dabei. Und moderiere von Transition Town aus jetzt auch den Soziologen bei dem Tag der Nachhaltigkeit. So das ist mein Job da. Dann bin ich neu dazu gekommen bei der B{\"u}rgerstiftung M{\"u}nster. Da bin ich noch nicht Mitglied, aber bin mit der Leitung da schon verbandelt. Wo wir uns auch vernetzen, und da gibts ein Projekt was jetzt im Herbst starten soll, das nennt sich "`Essen retten"'. Das l{\"a}uft am Schiller-Gymnasium in M{\"u}nster wo es darum geht, dass Obst was so auf den B{\"a}umen in der Innenstadt w{\"a}chst, speziell im Kreuzviertel, eingesammelt wird und Saft drau{\ss} gemacht wird. Die Sch{\"u}ler, zur Zeit sind das schon zwanzig Sch{\"u}ler {\"u}ber ich glaube sechs Jahrgangsstufen. Also geht richtig von ganz klein bis ganz gro{\ss}. Finde ich auch toll, dass da auch generations{\"u}bergreifendes Lernen stattfindet innerhalb einer Schule. Die kleinen mit den gro{\ss}en. Die lernen im Prinzip wie man so ein kleines Wirtschaftsunternehmen aufbaut. So als Projekt. Und lernen aber auch Themen (\dots) wie kommuniziert man dann richtig. Das ist ja so ein Steckenpferd von mir, also wie "`was geh{\"o}rt zum gut gelebten Leben"'. Und da geh{\"o}rt also auch Kommunikation dazu. Wertsch{\"a}tzende Kommunikation auf Augenh{\"o}he. Das ist so ein ganz wichtiges Thema. Das lernen die dabei. Die lernen zu planen, die lernen Buisnessplan zu machen, die lernen rum zu gehen und die Leute und B{\"u}rger zu begeistern daf{\"u}r ihren Apfel abzugeben. Die m{\"u}ssen sich um ne Presse k{\"u}mmern, die m{\"u}ssen das steril abf{\"u}llen. Und letztendlich m{\"u}ssen sie es auf dem Markt verscherbeln. So, dabei lernen die wie man miteinander umgeht, die lernen wie ne Firma funktioniert und was nicht funktioniert. Also Druck erzeugt Gegendruck und all diese ganzen sch{\"o}nen Sachen. Und die lernen auch wie viel Arbeit in einem Glas Apfelsaft drin steckt. Das ist das zum meinem Engagement zur B{\"u}rgerstiftung. Dann das Kulturquartier. Da geht es darum, ganz viele Facetten miteinander zu kombinieren. Weil Kultur ist nicht nur das was im Theater l{\"a}uft. Also findet nicht auf der B{\"u}hne statt. Kultur ist auch wie wir beide miteinander umgehen. Kultur ist wie ich jemandem auf der Stra{\ss}e begegne. Kultur ist wie ich mein Haus gestalte, damit das nachhaltiger ist, damit auch folgende Generationen noch was (\dots). Also ist so "`Wie wollen Menschen miteinander leben"' und das an einem Ort, wo man nicht so viel denkt, sondern mehr tut. Mehr ausprobiert. Das ist dieses Kulturquarier was gerade neu entsteht. So das sind so drei Dinge, mit denen ich mich hier in M{\"u}nster besch{\"a}ftige. Da werden aber noch andere Sachen dazukommen. Also eine Sache wird noch dazukommen, dass ich f{\"u}r M{\"u}nsteraner B{\"u}rger Themen anbiete, wie zum Beispiel "`wie funktioniert intrinsisch motivierte Arbeit"'. Das geht so ein bisschen dass das Thema Burnout ein bisschen der Vergangenheit angeh{\"o}rt. Das Thema "`Happy working people"' und "`Sinnerf{\"u}lltes Arbeiten -- Wie geht das in M{\"u}nster"'. Das sind so die vier Themen.
    \item[I:] Dann k{\"o}nnten Sir mir bitte die aus ihrer Sicht wichtigsten Aspekte von B{\"u}rgerbeteiligung nennen?
    \item[P3:] Also ich glaube (\dots) dass leben zutiefst menschlich ist. Dass immer Menschen mit Menschen f{\"u}r Menschen etwas arbeiten. Das was du jetzt hier machst, ist auch f{\"u}r Menschen die sich Treffen. Und B{\"u}rger sind einfach Menschen. Ich glaube nicht, dass es irgendwo einen Schlauen gibt, ganz oben, der wei{\ss} wie es geht. Politiker oder sowas. Sondern, dass vielmehr es in bestimmten Situationen sinnvoll sein kann, dass einer mal sagt wo es langgeht. Auf einem Schiff zum Beispiel. "`Jetzt m{\"u}ssen wir das Segel mal nach links oder rechts r{\"u}berholen, sonst kentern wir."' Dann sollte man das tun, und nicht anfangen zu diskutieren. Aber, im menschlichen Zusammenleben in so ner Stadt wie M{\"u}nster, finde ich es einfach toll, wenn Freir{\"a}ume von den B{\"u}rgern gestaltet werden k{\"o}nnen, so wie sie es gerne m{\"o}chten. Und dass nicht jemand anders wei{\ss} was f{\"u}r uns gut ist. Sondern die B{\"u}rger wissen schon selbst was f{\"u}r sie gut ist. Und zu einem m{\"u}ndigen B{\"u}rger geh{\"o}rt, dass er sich ausdr{\"u}cken kann. Deshalb finde ich das wichtig, dass es sowas gibt.
    \item[I:] Bitte geben Sie mir eine kurze Einf{\"u}hrung in ein laufendes oder abgeschlossenes Projekt oder Initiative, bei dem Sie denken, dass es da besonders auf die Kommunikation angekommen ist.
    \item[P3:] Ja, da gibts so viele.
    \item[I:] Dann von ihrem liebsten.
    \item[P3:] Ja mein liebstes. Nehmen wir mal das Kulturquartier. Das Kulturquartier hat sehr viele Akteure. Es sind schon allein acht Gesellschafter, dann gibt es jede Menge Mieter, dann gibt es die Politik, die Wirtschaftsf{\"o}rderung, dann gibt es Bankengeldgeberstiftungen, die uns unterst{\"u}tzen wollen. Dann gibts ein Programm "`1000x100"' wo uns tausend B{\"u}rger mit jeweils einhundert Euro im Jahr unterst{\"u}tzen, damit das passiert. Also das ist so ne spezielle Art von Crowdfunding. Sodass im Prinzip, die vielen, wenn sie gut informiert sind, und in einer guten Interaktion sind, und sehen was da gerade passiert an dem Ort und wie es weitergeht, dass sie mal in Dialog treten k{\"o}nnen. Ideen mit einbringen k{\"o}nnen ohne Anspruch dadrauf, dass es auch passiert, weil das entscheidet letztendlich so ein Gremium innerhalb des Kulturquartieres. Sonst ist es ein Debattierclub bis ans Ende der Zeit. Aber da k{\"o}nnten sich viele beteiligen, Ideen reinbringen, damit die Leute, die dann letztendlich entscheiden, basierend auf diesem tollen, vielen Ideen, ganz inspiriert sagen: "`Hey, das machen wir jetzt, da w{\"a}ren wir selbst nie drauf gekommen. Das w{\"a}re ein riesen Vorteil."' Und andere Akteure wie eine Bank oder ne Stiftung sieht, "`das ist total lebendig. Da wird diskutiert, und da wird auch nicht jeder Schrott genommen, sondern da werden gute Ideen auch wirklich aufgenommen, und es entwickelt einen Speed, eine Geschwindigkeit in eine Richtung zum besser werden, besser leben. Das ist ja der Hammer"'
    \item[I:] Wie l{\"a}uft dann dort meistens die Kommunikation ab?
    \item[P3:] Seit neustem haben wir ein tolles Tool, das hei{\ss}t Trello. (lacht) Da machen wir relativ viel jetzt mit seit ner Woche. Und vorher lief es einfach so, es gibt zu bestimmten Terminen sogennante Tischgespr{\"a}che. Da setzen wir uns hin, da gibts was zu futtern, aber nichts gro{\ss}es. So was zu knabbern oder was zum dippen. Und dann treffen sich die Leute, die in der Planung sind. Es gibt mittlerweile Teams zum Thema Bau oder Stiftungsgelder oder Gr{\"u}ndung oder wei{\ss} der Kuckuck was. Die dann bei den Tischgespr{\"a}chen die anderen informieren. Dadurch ist nat{\"u}rlich durchaus eine gewisse Zeitverz{\"o}gerung bei der Information dabei. Und wenn die anderen Bescheid wissen, was jetzt gerade abgeht (\dots) Vor allem, wenn die wissen was neu ist, das ist immer ganz wichtig. Das gibts auch bei Trello so. Ich wei{\ss} nicht ob es jetzt bei der Applikation drin ist, also wirklich "`Hey, was ist neu"'. Dass die Alarmglocke bimmelt. Bei den Themen, dass ich vielleicht sogar die Themen tagge. (\dots) Bei Trello ist das so, dass ich sage: "`Das sind meine Boards"' und "`Was ist an den Boards neu"'. Dass ich da wei{\ss}, was so neu ist. Ich muss nicht alles wissen, aber dass ich da bei denen ich subscribed bin, dass ich wei{\ss} was los ist. Ja bisher war es sehr viel pers{\"o}nlicher. Jedes Treffen, telefonieren, E-Mail. Und das geht gerade in so ein co-kreatives, kooperatives, IT-unterst{\"u}tztes Umfeld rein. Und ich k{\"o}nnte mir vorstellen, dass	das gut ist.
    \item[I:] Und die Beitr{\"a}ge sind dann welcher Natur?
    \item[P3:] Es ist viel Information. Dann gibts Ideen. Aus den Ideen, da wird ne Menge verworfen. Es gibt eigentlich einen riesigen Ideenparkplatz. Also es gibt auch in den Teams auch immer Leute, die sprudeln vor Ideen, die setzen aber nichts um. Die m{\"u}ssen Platz haben wo sie es loswerden k{\"o}nnen. Es muss Platz geben, wo man Ideen aufgreifen kann. "`Tackatacka, die nehm ich"' Und dann gibts die Umsetzer, die das umsetzen, die vielleicht nicht so kreativ sind. Das hei{\ss}t, man brauch Informationen, wie man es umsetzen kann. Man braucht vielleicht Verlinkungen zu anderen die es vielleicht auch schon gemacht haben. Und es sollte auch dokumentiert werden, welche Entscheidungen jetzt getroffen worden sind. Also am Ende einer Diskussion oder so, dass man sagt, "`Ja, vielen Dank. Ich schlie{\ss} jetzt diesen Track und die Entscheidung ist so"'. Man kann einen neuen aufmachen, aber der ist jetzt erstmal abgeschlossen. Dass man einen Zyklus abschlie{\ss}t. Das finde ich auch immer ganz wichtig. Das machen wir dann auch.
\end{itemize}

\textbf{Teil 2 -- Einsatz der Anwendung}
\begin{itemize}
    \item[I:] Dann geht es jetzt weiter mit Fragen konkret zur Anwendung. Bitte geben Sie mir eine Einf{\"u}hrung in das Projekt in dem Sie die Anwendung einsetzen wollen.
    \item[P3:] Ich kann mir einfach vorstellen, wenn es um den Nachhaltigkeitstag geht, dass man die Anwendung sehr sch{\"o}n f{\"u}r die Vorbereitung nutzen kann. Dass wir, wenn wir die verschiedenen Stationen haben wo in M{\"u}nster was stattfindet, dass man sagt, "`Hey, da und da und da"' und dass man dann rund um diesen Platz sich das vielleicht auch noch ein bisschen auff{\"a}chern kann, "`Da ist die Hauptaktion, da mache ich eben den Stand"'. Ich kann das ja ganz ganz kleinteilig machen. Ich wei{\ss} nicht, geht das bis auf f{\"u}nf Meter oder so? (Ja ganz klein) Ja also dass ich sage, ich bau da wirklich den Stand hin. Ich kann ja praktisch m{\"o}blieren. Und das k{\"o}nnte ich mir auch vorstellen, also wenn wir wirklich ein Projekt aufmachen kann, so wie es bei Trello mit den Firmen geht. Ich mach ein Projekt auf. Dann kann ich wirklich, wenn ich ein Sommerfest habe im Kindergarten, oder ich mach da ne Station am Tag der Nachhaltigkeit, kann ich wirklich super geilomat planen, was wo sein soll. Wo kommt die Leinwand hin, wo kommt der Beamer hin. Also ich kann ganz ganz kleinteilig (\dots) Und jeder wei{\ss} genau wo es hinkommt. Und dann w{\"a}re es nat{\"u}rlich schon, wenn man dadr{\"u}ber diskutiert. Und dann f{\"a}nde ich es gut, wenn man auch die M{\"o}glichkeit h{\"a}tte, wenn es so Aufgaben gibt, oder sowas. Oder man m{\"u}sste ein anderes Tool nehmen. Dass man sagt "`Hey, die Aufgabe ist jetzt erledigt"'. Also da soll der Beamer hin. Der ist auf einmal gr{\"u}n. Erledigt. Da k{\"o}nnte ich das sch{\"o}n best{\"u}cken und dann sehe ich auf einen Blick auf der Landkarte wenn ich reinzoome, da ist alles gr{\"u}n aber der Teil, der ist noch irgendwie gelb oder orange. Orange weil es jetzt bald zu tun ist. Also irgendwie in einer Woche, aber da ist immer noch nichts passiert. Und dann kann jeder sehen. Und das w{\"u}rde auch so eine Gruppe unterst{\"u}tzen. Auch den der verantwortlich ist, der muss da zwar draufgucken, aber auch ein anderer sieht das vielleicht mal und sagt "`Hey ich bring jetzt noch einen Beamer mit von zuhause. Ich hab das gesehen, da ist keiner. Zumindest nicht eingetragen."' Also wir haben dann einen {\"U}berblick, {\"u}ber all das was los ist an den verschiedenen Standorten. B{\"u}rger, die man einl{\"a}dt, ich wei{\ss} nicht ob das auch der Gedanke ist, die wissen auch was da passiert. Man {\"o}ffnet das ja dann f{\"u}r die B{\"u}rger. Und da ist dann die Frage (\dots) Die k{\"o}nnen ja mitdiskutieren. Ich wei{\ss} jetzt nicht wie es da mit den Rechten ist. Also mitdiskutieren ja, aber ob die dann auch Objekte verschieben d{\"u}rfen, weil dann sonst schmei{\ss}en die uns den ganzen Raum wieder durcheinander. Das w{\"a}re voll grottenschlecht. So k{\"o}nnte ich mir das aber vorstellen. Wenn es keine graphischen Objekte gibt, dass man eben sagt an dem Ort, da muss das und das und das passieren. Und dann eben textlich.
    \item[I:] Was f{\"u}r Anreize f{\"u}r B{\"u}rger sich mit der Anwendung dialogisch auszutauschen k{\"o}nnten Sie sich vorstellen?
    \item[P3:] Es gibt B{\"u}rger, die wollen in einen Dialog gehen und das ist sch{\"o}n, wenn die sehen bei mir in der Nachbarschaft passiert was, oder das Thema interessiert mich. Also einmal A: es betrifft mich weil es hier um die Ecke ist und es betrifft mich, ich fahr auch f{\"u}nf Kilometer daf{\"u}r weil es mich interessiert. Und dann m{\"o}chte ich da einmal vielleicht genau wissen was ist da los, das f{\"a}nde ich spannend. Und dann hab ich vielleicht noch ne Idee. Also sieht er hier da vorne an dem Ort, da k{\"o}nnte ich euch noch meinen Starkstromanschluss anbieten. Der ist hier in meiner Garage. Dann k{\"o}nnt ihr mehr machen, oder irgendsowas. Das hei{\ss}t, ich kann meine eigenen Ideen oder Unterst{\"u}tzungsleistungen besser anbieten. Und ich werd besser informiert, {\"u}ber das was da ist. Und diskutieren, wenn ich diskutieren will. Aber ich glaube, in erster Instanz f{\"a}nde ich das Informieren wichtig. Was geht denn da ab. F{\"u}r mich jetzt. Aber andere sind da vielleicht anders gestrickt.
    \item[I:] K{\"o}nnen Sie sich weitere Anwendungsf{\"a}lle f{\"u}r die Verkn{\"u}pfung von Texten mit Karten neben der B{\"u}rgerbeteiligung vorstellen?
    \item[P3:] Also zum Beispiel wenn man da so gut reinzoomen kann, kann ich mir das f{\"u}r jede Art von Gro{\ss}veranstaltung vorstellen. Als Planungstool. Dass man sieht, was geht hier jetzt gerade ab. Ich kann es mir sogar vorstellen f{\"u}r Institutionen wie zum Beispiel die Polizei. Wenn Gro{\ss}demos sind oder wenn, was wei{\ss} ich, eine Riesenveranstaltung ist wie M{\"u}nster gegen Bayern. Bayern gegen Preu{\ss}en. Also das l{\"a}uft ja, aber wenn so Neuland ist, oder ein gigantisches Konzert, f{\"u}r Sicherheitskr{\"a}fte, f{\"u}r Gro{\ss}veranstaltungen wie Rock am Ring, keine Ahnung. Also gro{\ss}e Veranstaltungen, wo das Gel{\"a}nde weitr{\"a}umig ist, k{\"o}nnte ich mir vorstellen. Dann gibt es sehr gro{\ss}e Unternehmen. Zum Beispiel eine BASF, ein Flughafen. Also Unternehmen, die {\"u}ber sehr weitr{\"a}umiges Gel{\"a}nde verf{\"u}gen und mal schnell irgendwie was auch lokal ver{\"a}ndern wollen, und sagen "`Hey, an der Stelle, da ist dies und jenes"' Wo man sein Ideenmanagement vielleicht auch zusammen mit der Lokalisierung macht. Das ist ja Kartenmaterial, vielleicht noch ne Frage: Ist das von Google? (Nein das ist von Openstreetmap.) Okay, das hei{\ss}t also weltweit? (Ja) Das hei{\ss}t, ich kann auch mit Unternehmen arbeiten, die weltweit operieren (Richtig.) Genau. Ja das w{\"a}re auch gut. Wenn man dann Ideen hat, "`Hey, ich arbeite hier in M{\"u}nster, aber in der amerikanischen Niederlassung. Ich komm gerade wieder. Da habe ich gesehen, in der Einheit, da funktioniert irgendwas nicht. Und ich mache da irgendwie ein Todo-Marker rein an die Stelle"' Und dann guckt irgendwie so ein Global-Todo-Manager drauf und sagt "`Hey, wo sind denn so Open Issues"' und sieht die. An einer Stelle, wo irgendwas ist, was getan werden muss. Das finde ich gut.
    \item[I:] Welche Gr{\"u}nde sprechen f{\"u}r den Einsatz dieser L{\"o}sung gegen{\"u}ber anderen, angedachten L{\"o}sungen?
    \item[P3:] Nein, mit Kartenmaterial, hab ich jetzt keine Idee. Mit Kartendialog, habe ich mir jetzt noch keine angeschaut. Ich h{\"a}tte ganz andere Projektmanagementtools eingesetzt. Ganz klassische. Aber die sind nicht verkn{\"u}pft mit der Karte.
    \item[I:] Welche Eigenschaften w{\"u}rden Sie davon abhalten, diese Anwendung einzusetzen?
    \item[P3:] (\dots) Was mich zur{\"u}ckhalten w{\"u}rde, w{\"a}re, wenn Daten auf einmal weg sind. F{\"a}nde ich irgendwie uncool. Da hat man sich dann viel M{\"u}he gegeben irgendwas einzupflegen. Auf einmal, ist das dann weg. Ich find das Thema backup sehr wichtig. In Zusammenhang mit der Rechteverwaltung. Wenn jeder alles machen darf, und {\"u}berall drin rum schreiben darf, das f{\"a}nde ich nicht gut. Also so wie es hier auch schon gel{\"o}st ist, also dass jeder seine eigene ID hat f{\"u}r die Antworten. Dass es eigene Eintr{\"a}ge sind finde ich gut. Was w{\"u}rde mich noch abhalten? Nein, ich w{\"u}rde es einfach so nutzen.
    \item[I:] Was k{\"o}nnten Gr{\"u}nde f{\"u}r B{\"u}rger sein, sich nicht zu beteiligen?
    \item[P3:] Es k{\"o}nnte sein, dass er Einstieg zu schwierig ist. Also ich hab jetzt bei Trello festgestellt, dieses Thema anmelden bei Trello, halt ich pers{\"o}nlich f{\"u}r total easy, wenn man das erste mal dabei ist. Aber wenn man mal auf die Webseite guckt. Also bei dem achter Team waren f{\"u}nf Leute dabei, die haben es nicht gerafft. Woran lag das? Es gab oben einen dicken Button Login. Okay Loginname und Passwort oder Login with Google. Die haben immer drauf geklickt sind immer auf Fehler gekommen. Und unten drunter steht ganz klein unterstrichen halt auch als Link "`If you don't have an account, please create here"' oder so {\"a}hnlich. So {\"a}hnlich wie Allgemeine Gesch{\"a}ftsbedingungen. Aber das haben die nicht gefunden. Das war nervig. Das habe ich denen geschrieben, hab ich einen Screenshot gemacht. Das war voll bl{\"o}d, weil die das immer noch nicht gerafft hatten. Wir mussten uns wirklich treffen, oder am Telefon erkl{\"a}ren "`Ach da, da ist das ja, okay"'. Also die Schwelle sich da anzumelden, deutlich zu machen wo man draufklicken muss, wenn man noch nicht registriert ist. Das finde ich wichtig dass man {\"u}berhaupt rein kommt. Und dann finde ich sehr gut, wenn es eine einfache Erkl{\"a}rung gibt. Zum Beispiel das mit den Videos finde ich super.
\end{itemize}

\textbf{Teil 3 -- Abschlie{\ss}ende Fragen}
\begin{itemize}
    \item[I:] Kennen Sie Beispiele f{\"u}r die Verkn{\"u}pfung Geographischer Daten mit Diskussionsbeitr{\"a}gen?
    \item[P3:] Also wenn ich mich richtig entsinne, geht das ja bei Google. Da kann ich das machen. So irgendwie, da ist ein Hotel, das fande ich jetzt gut, oder fande ich nicht so gut und dann kommt da dann das Rating. Da kenne ich es her. Sonst nicht.
    \item[I:] Haben Sie sich dann da auch schon einmal beteiligt?
    \item[P3:] Nein. Geographisch noch nicht. Das war mir auch zu weit weg. Also das muss mit meinem Thema zu tun haben. Sonst tu ich das nicht. Also ich schreib normalerweise nicht rein "`Das Hotel war jetzt geil"' oder so. Das mache ich nicht.
    \item[I:] Kennen Sie Werkzeuge um interaktive Karten mit eigenen Inhalten zu erzeugen?
    \item[P3:] Nein, kenne ich nicht.
    \item[I:] Also haben Sie sowas dann auch nicht benutzt?
    \item[P3:] Hm also nein. Aber es ist ganz lange her, da hab ich dann damals mit Powerpoint mir Karten gemacht. Ich nehm eine Karte und dann mach ich da einen Hotspot drauf. Also sowas hab ich schonmal gemacht. Aber das ist ja wirklich sehr Laienhaft. Das w{\"u}rde ich jetzt nicht als Geoinformationssystem bezeichnen. Sondern das ist mehr Pr{\"a}sentations (\dots) Hyperlink oder sowas. Mehr ist das nicht.
    \item[I:] Gibt es noch noch Fragen oder Anmerkungen von Ihrer Seite?
    \item[P3:] Ich finds geil. Ich finds richtig gut. Und ich freu mich dadrauf, wenns dann da ist. Finde ich auch spannend dann in welcher Form das kommuniziert wird. Das w{\"a}re noch ein Punkt zu den B{\"u}rgern, was k{\"o}nnte die davon abhalten. Davon abhalten k{\"o}nnte sie, dass sie es nicht wissen, dass es das gibt. Das w{\"a}re noch ein wichtiger Punkt.
    \item[I:] Dann bedanke ich mich recht Herzlich.
\end{itemize}

\subsubsection{Appendix B.6. Participant 4}

\textbf{Teil 1 -- B{\"u}rgerbeteiligung}
\begin{itemize}
    \item[I:] Erz{\"a}hlen Sie mir {\"u}ber ihre Rolle und Aufgaben in der B{\"u}rgerbeteiligung.
    \item[P4:] Durch mein FSJ in der B{\"u}rgerstiftung stehe ich mit vielen Menschen in Kontakt die sich engagieren m{\"o}chten, die sich engagieren in verschiedenen Projekten der B{\"u}rgerstiftung. Und meine Rolle ist da konkret dass ich denen halt weiterhelfe und die in diese Projekte vermittele und daf{\"u}r sorge, dass die da das machen k{\"o}nnen, was sie gerne m{\"o}chten. Und dann einen Platz finden wie sie sich engagieren k{\"o}nnen.
    \item[I:] Und sind Sie da dann eher "`Organisator"' oder "`an der Basis"'?
    \item[P4:] Ich bin schon mehr Organisator, also f{\"u}r den administrativen Teil zust{\"a}ndig. In der Stiftung sind die Projekte halt so aufgebaut, dass jemand aus dem Vorstand ein Projekt betreut. Und da drunter gibts dann jeweils Projektleiter. Wenn man sich jetzt mal die Mentoren oder Lesepatenprojekte anschaut, gibts eine Projektleiterin, dadrunter sind dann jeweils an den Schulen noch Projektleiter, die sich halt selber auch noch engagieren als Mentor oder Lesepate, aber die k{\"u}mmern sich halt darum, dass Leute, die neu einsteigen m{\"o}chten, da einen Platz finden, eine Zeit bekommen, Sch{\"u}ler bekommen die sie betreuen. Genau. Und ich bin im B{\"u}ro f{\"u}r die Verwaltung entsprechend zust{\"a}ndig. Das hei{\ss}t, wenn Mails an die Leute verschickt werden m{\"u}ssen. Wenn Briefe fertig gemacht werden. Pflege von Listen. Und dann halt Telefonkontakte zu denen l{\"a}uft dann halt viel {\"u}ber das B{\"u}ro, {\"u}ber mich, weil wir die ganzen Daten haben. Und Verwaltung von F{\"u}hrungszeugnissen, dass die dann halt auch wirklich anfangen k{\"o}nnen. So der administrative Teil.
    \item[I:] Bitte beschreiben Sie mir die aus ihrer Sicht wichtigsten Aspekte der B{\"u}rgerbeteiligung.
    \item[P4:] (\dots) Der Nutzen ist halt, finde ich, (\dots). B{\"u}rger haben ja verschiedene Stellungen, haben ja verschiedene Erfahrungen, und k{\"o}nnen damit Dinge weitergeben. K{\"o}nnen anderen ziemlich simpel helfen und anderen Leuten einfach etwas gutes tun, ohne dass da viel Aufwand betrieben werden muss. Man geht halt selbst einfach irgendwo nur in die Schule und lie{\ss}t einfach ein bisschen vor. Und das Kind profitiert dadurch einfach dass es in der Schule dadurch bessere Chancen hat, im Deutschunterricht klar zu kommen. Man erm{\"o}glicht den Kindern dadurch einfach, (\dots) ein besseres Leben. Das ist jetzt vielleicht ein bisschen hoch gegriffen, aber man kann dadurch einfach ein bisschen was verbessern. Ja und dieses selbstlose finde ich da sehr wichtig, dass man da einfach sagt "`Ich habe Sachen die ich machen kann"' oder "`Ich habe freie Resourcen und damit setze ich mich f{\"u}r andere einfach ein, die eben nicht den Luxus haben, dass man eben selbst die freie Zeit hat, dass man da Mittel zur Verf{\"u}gung hat."'
    \item[I:] Bitte geben Sie mir eine Einf{\"u}hrung in ein laufendes oder abgeschlossenes Projekt oder Initiative bei der Sie denken, dass dort Dialoge zwischen den Akteuren am wichtigsten war oder ist.
    \item[P4:] (\dots) Das sind nat{\"u}rlich solche Projekte wie der B{\"u}rgerpreis. Das ist ein Preis, der vergeben wird, wo halt da wirklich dass aus der B{\"u}rgerschaft gew{\"u}rdigt wird. Das sorgt nat{\"u}rlich f{\"u}r entsprechende Gespr{\"a}chsstoffe, wenn da jedes Jahr wieder andere Projekte und Themen gew{\"u}rdigt werden. Und dann da auch verschiedene Leute ins Gespr{\"a}ch kommen durch ihre eigenen Bewerbungen. Dass der {\"O}ffentlichkeit da einfach viel vorgestellt wird. Das sind dann ja auch Akteure aus allen m{\"o}glichen Schichten dabei, die komplett verschieden auch sind. Also ansonsten bei den anderen Projekten der B{\"u}rgerstiftung nicht direkt Dialog zwischen den Akteuren und B{\"u}rgern. In den Projekten intern, da finden eigentlich Dialoge statt zwischen Akteuren und Sch{\"u}lern oder Jugendlichen. Aber nicht Dialog nach au{\ss}en hin.
    \item[I:] Bleiben wir einfach bei dem B{\"u}rgerpreis. Wie findet dort dann das Vorschlagen statt?
    \item[P4:] Der Preis wird von der Stiftung ausgeschrieben zu einem bestimmten Thema. Und da es ja die "`B{\"u}rgerstiftung -- B{\"u}rger f{\"u}r M{\"u}nster"' hei{\ss}t, ist ganz klar, dass ein gewisser M{\"u}nsterbezug da sein muss. Das hei{\ss}t, Bewerbungen werden nur der Jury weitergegeben, wenn sie aus M{\"u}nster wirklich kommen, also der M{\"u}nsterbezug da ist. Und es muss komplett rein ehrenamtlich sein. Das hei{\ss}t im Prinzip keine Verg{\"u}tung f{\"u}r die Arbeit. Und es muss entsprechend zum Thema passen. Ja. Man bewirbt sich normal als Verein oder Initiative dort selber. Dass Leute vorgeschlagen werden, ist eher selten, da der Preis eher auf Gruppen abzielt, anstatt auf Einzelpersonen. Da halt breites b{\"u}rgerschaftliches Engagement sichtbar gemacht werden soll. Es gibt zwar viele Einzelpersonen die sich engagieren. Das ist auch gut so, aber es soll mehr das gezeigt werden, wo sich B{\"u}rger zusammentun und zusammen was gutes machen, daran Spa{\ss} haben und dann dabei auch noch was f{\"u}r die Stadt tun.
\end{itemize}

\textbf{Teil 2 -- Einsatz der Anwendung}
\begin{itemize}
    \item[I:] Bitte geben Sie mir eine Einf{\"u}hrung in das Projekt in dem die Anwendung eingesetzt werden soll.
    \item[P4:] Das Gutscheinheft tr{\"a}gt den Beititel "`1000 Stunden f{\"u}r M{\"u}nster"' und soll so sein, dass dort f{\"u}nfundzwanzig Vereine und Initiativen dabei sind, die jeweils einen Gutschein {\"u}ber ein bis zwei Stunden b{\"u}rgerschaftliches Engagement anbieten. Das hei{\ss}t dass man da einfach hingehen kann, und sich dort kurzfristig engagiert. So sollen dann insgesamt eintausend Stunden von dem B{\"u}rgerengagement f{\"u}r die Stadt zusammenkommen. Und dort ist es dann nat{\"u}rlich so, dass die Vereine ja {\"u}ber die ganze Stadt verteilt sind. Die stellen sich in dem Heft kurz vor. Also eine Projektbeschreibung und Allgemein was die Einrichtung oder Vereine machen. Das wird dann so da drin sein. Allerdings ist das ja ein Unterschied, ob man jetzt sich eine Karte anschaut, und da einfach mit der Maus {\"u}ber ein paar Punkte geht und einem dass dann eingeblendet wird, als wenn man ein Heft vor sich liegen hat und da dann jede Adresse anschaut und dann guckt, wie weit das jetzt von sich entfernt ist. "`Wie weit ist das von mir weg, kann ich da nicht eben mal so gleich vorbeigehen?"'. Dementsprechend k{\"o}nnte ich mir das gut vorstellen, dass die Anwendung sich da sehr gut erg{\"a}nzen. Weil das ja im Prinzip ja nur eine digitale Form davon ist. Das dass einfach einem die M{\"o}glichkeit gibt, dass man Fokus auf seinen Standort gerade legt, und dann schaut wo ist in der N{\"a}he denn was, wo ich mit einer kurzen Strecke schnell hinkomme. Und das ist halt bei einem Heft (\dots) ist halt keine interaktive Karte mit drin.
    \item[I:] Welche Anreize k{\"o}nnte man geben, damit sich B{\"u}rger {\"u}ber die eingestellten Projekte austauschen?
    \item[P4:] Anreize zum Austauschen. (\dots) Also Leute berichten ja gerne {\"u}ber Sachen die entweder sehr schlecht waren oder sehr gut waren. (\dots) Aber wie man einen konkreten Anreiz erschafft, da f{\"a}llt mir konkret nichts ein. Wie man da ja Beitr{\"a}ge bekommen w{\"u}rde, wenn die Leute da sind, dass man denen dann direkt da vor Ort einmal die M{\"o}glichkeit gibt da direkt was einzutragen. Das w{\"a}re auf jeden Fall was sinnvolles, wie man da dann dazu kommt, dass da erstmal was steht, und ich denke wenn da was steht, und andere Leute das lesen, dass es da sehr gut war, dann ist das nat{\"u}rlich auch eine Motivation f{\"u}r die "`Ah, der hat da was spannendes erlebt, vom Titel her reizt mich das auch schon, nach den Sachen was der da schreibt, w{\"u}rde ich da glaube ich auch gerne hin"'. Aber direkt den Anreiz zu schaffen, w{\"u}sste ich gerade keinen.
    \item[I:] Gegen{\"u}ber anderen angedachten L{\"o}sungen, welche L{\"o}sungen sprechen f{\"u}r den Einsatz dieser L{\"o}sung?
    \item[P4:] Die andere L{\"o}sung, oder die andere M{\"o}glichkeit wie man das publik machen k{\"o}nnte, w{\"a}re {\"u}ber eine Internet-Seite. Wahrscheinlich auch erstmal nur in Listenform. Und f{\"u}r mich so was interaktives nat{\"u}rlich ein Schritt auf die Leute zu. Und dass man auch zeigt, dass man eben auch mit den neuen Medien gut arbeitet. Und eben nicht halt nur steife Listen dem Benutzer vorsetzt. Da gef{\"a}llt mir das Interaktive sehr gut, weil es halt einfacher f{\"u}r die Leute ist. Und viele M{\"o}glichkeiten eben bietet.
    \item[I:] Welche Eigenschaften w{\"u}rden Sie davon abhalten diese L{\"o}sung einzusetzen?
    \item[P4:] Ein eventueller Grund k{\"o}nnte sein, wenn damit noch h{\"o}here Kosten verbunden sind. Da man als Stiftung eigentlich eher das Geld f{\"u}r andere Vereine und zur Unterst{\"u}tzung verwenden will. Und der Verwaltungsaufwand und Werbeaufwand immer sehr gering gehalten werden soll. Das k{\"o}nnte ich mir vorstellen, dass das ein Grund w{\"a}re sich dagegen zu entscheiden. Und ansonsten wenn damit noch Verwaltungsaufwand verbunden ist. Dass man sein kleines Profil dass da sozusagen drin ist, dass man das sehr sehr viel pflegen muss. Weil das dann immer noch zus{\"a}tzliche Aufgaben sind, die dann anfallen, die halt bei einer einfachen Karte oder bei einer einfachen Liste wo die Sachen drin stehen halt nicht anfallen w{\"u}rden, da dass ja nur reines Informationsmaterial da f{\"u}r die Personen w{\"a}re.
    \item[I:] Was k{\"o}nnte Ihrer Meinung nach B{\"u}rger davon abhalten sich zu beteiligen?
    \item[P4:] Zum einen k{\"o}nnte dass die Anmelung eventuell sein. Dass da vielleicht eine kleine Hemmschwelle ist, oder dass das einen davon abh{\"a}lt, dass man sich da auch noch anmelden muss. Das w{\"u}rde ja aber nicht hei{\ss}en, dass die Leute das nicht benutzen k{\"o}nnten um sich nur zu informieren. Das w{\"u}rde ja so gehen. Ja die Anmeldung f{\"u}r junge Leute, die Facebook oder Twitter haben, ist das glaube ich kein Problem das zu nutzen, Ja aber wenn man halt die Generation f{\"u}nfzig plus oder sechzig plus ist, die ja da doch auf die Daten doch sehr viel mehr achten, dass die da dann eventuell zur{\"u}ckschrecken, und das dann nur zur Information nutzen.
    \item[I:] K{\"o}nnen Sie sich weitere Anwendungsf{\"a}lle f{\"u}r die Verkn{\"u}pfung von Texten mit Karten neben der B{\"u}rgerbeteiligung vorstellen?
    \item[P4:] (\dots) Also das Programm noch f{\"u}r andere Zwecke? (Ja) Ja im Prinzip gibts das ja schon f{\"u}r alle m{\"o}glichen Suchen. Ob es jetzt einfach (\dots) Ja ich denke gerade an Foursquare. Da wird mir ja auch angezeigt, was in der N{\"a}he f{\"u}r L{\"a}den, Gesch{\"a}fte und {\"a}hnliches gibt. Das k{\"o}nnte man nat{\"u}rlich allgemein auf Vereine ausweiten wenn man in dem B{\"u}rgerschaftlichen bleiben m{\"o}chte. Dass man Sportvereine und vielleicht noch sonstige Aktivit{\"a}ten einflechtet, und damit eine Karte h{\"a}tte, wo allgemein was drin ist, wie man sich (\dots) ja was man wo in der Stadt machen kann. Was f{\"u}r Angebote es gibt.
\end{itemize}

\textbf{Teil 3 -- Abschlie{\ss}ende Fragen}
\begin{itemize}
    \item[I:] Kennen Sie Beispiele f{\"u}r die Verkn{\"u}pfung geographischer Daten mit Diskussionsbeitr{\"a}gen?
    \item[P4:] Ja hatte ich ja gerade schon gesagt. Foursquare oder wie hei{\ss}t das. Yelp. Da kann man ja auch so Dinge eintragen kann. Bei Google gibts die Funktion ja glaube ich auch. Dass in den Karten was angezeigt wird. Dass ist dann ja immer noch {\"u}ber verschiedene Seiten verkn{\"u}pft. Was ich selber auch nur zur Information nutze. Mir ist die Anmeldung da halt (\dots) Dass man sich ja daf{\"u}r wieder anmelden muss, das hat mich bisher gehindert da selber mal aktiv zu werden.
    \item[I:] Also auch noch nicht beteiligt?
    \item[P4:] Nur bei Foursquare mal.
    \item[I:] Kennen Sie Werkzeuge um interaktive Karten mit eigenen Inhalten zu erzeugen?
    \item[P4:] Ja ich habe mal f{\"u}r eine Internetseite (\dots) Da kann man halt bei Google Routen erstellen, Routen anzeigen. Oder mit Openstreetmap gibts ja glaube ich auch. Das habe ich mal genutzt. Ansonsten nichts.
    \item[I:] Gut. Gibt es dann noch Anmerkungen oder Fragen von Ihrer Seite aus?
    \item[P4:] Ne h{\"o}chstens zur Umsetzung, ob das dann wirklich dann bald ins Netz gehen soll?
    \item[I:] Ja online ist das schon. Aber {\"u}ber die genauen Rahmenbedingungen muss da nochmal an anderer Stelle gesprochen werden. Die Seite wird wohl n{\"a}chstes Jahr f{\"u}r den Tag der Nachhaltigkeit eingesetzt werden.
    \item[P4:] Das hei{\ss}t, man platziert dann im Prinzip auch so ein Kartentool auf auf so einer Seite? Und ansonsten wie findet man das?
    \item[I:] Ja genau. Die Adresse kann ich Ihnen gleich auch nochmal aufschreiben.
    \item[P4:] Ja das w{\"a}re sch{\"o}n.
    \item[I:] Alles klar Dann auf jeden Fall vielen Dank.
\end{itemize}

\subsubsection{Appendix B.7. Participant 5}

\textbf{Teil 1 -- B{\"u}rgerbeteiligung}
\begin{itemize}
    \item[I:] Erz{\"a}hlen Sie mir {\"u}ber ihre Rolle und Aufgaben in der B{\"u}rgerbeteiliung.
    \item[P5:] Also generell halte ich das Thema B{\"u}rgerbeteiligung, das ist ja ein sehr weiter Begriff, f{\"u}r erstmal eine wichtige Sache. Auch wenn man sich das jetzt so unter Demokratieaspekten anguckt. Wenn man sich da anschaut, was es da ja prinzipiell darum gehen sollte, Macht vom Volke ausgehen zu lassen. Dann halte ich so ein Konzept "`Nur alle vier Jahre w{\"a}hlen gehen, oder halt wenn Wahlen sind"' eigentlich f{\"u}r zu wenig. Und deshalb freue ich mich auch, dass solche Themen, oder das Thema B{\"u}rgerbeteiligung in den letzten Jahren so meiner Wahrnehmenung nach auch gr{\"o}{\ss}er geworden ist. Dass das auch in der etablierten Politik zunehmend Anh{\"a}ngerinnen findet. Und dass das halt, wenn sich das durchsetzt, da dann das auch schon passiert, zumindes gef{\"u}hlt ein besseres Ergebnis entwickelt. Das muss ja nichtmal hinterher sein, dass das da automatisch ein besseres Ergebnis steht, aber dadurch, dass die Leute mitmachen k{\"o}nnen, und sich einbringen k{\"o}nnen, bleibt zumindest nicht so das Gef{\"u}hl "`{\"U}ber mich wird potentiell hinwegentschieden."'. Ist ja nicht so dass da alle mitmachen und es beschweren sich immer noch genug Leute. Und unter solchen Aspekten, denke ich, dass das auch auszuweiten ist, und solche M{\"o}glichkeiten, dass das die Leute m{\"o}glichst viel was sie Betrifft auch mitbestimmen k{\"o}nnen. Ja das halte ich f{\"u}r erstrebenswert und deshalb hat es mich dann auch	gefreut, dass ich jetzt hier im Rahmen meines Hiwi-Jobs auch die M{\"o}glichkeit hatte, da so ein St{\"u}ck weit das mit zumachen.
    \item[I:] Au{\ss}erhalb dieses Rahmen haben Sie dann noch an keiner Initiative oder Projekt teilgenommen?
    \item[P5:] Vielleicht nicht unbedingt unter dem Aspekt B{\"u}rgerbeteiligung, aber ich bin selber sehr stark engagiert in so politischen Projekten. So "`von unten"'. Also, wo habe ich in den letzten Jahren mitgemacht? In Stadtteilinitiativen oder generell in Osnabr{\"u}ck hat sich jetzt da auch so ein "`Recht auf Stadt"'-B{\"u}ndnis entwickelt, wo ich mitgemacht habe. Ich bin seit es das gibt in Osnabr{\"u}ck, seit 2008 glaube ich jetzt, mache ich mit bei so einem selbstverwalteten Zentrum f{\"u}r politische und kulturelle Veranstaltungen. In der Klima-Bewegung oder bei so Klima-Aktionen. Zum Beispiel halt im Klima-Camp im Rheinland war ich die letzten Jahre immer so mit dabei. Also halt solche Sachen. Das ist jetzt nicht direkt so klassisch B{\"u}rgerbeteiligung, aber letztlich geht es ja auch darum, dass das Leute, die betroffen sind, mitmachen k{\"o}nnen und sich f{\"u}r ihre Belange auch selber einsetzen.
    \item[I:] Wie lief dann in in diesen Projekten die Kommunikation zwischen den Beteiligten ab?
    \item[P5:] Also wenn man das einmal so pauschal sagen wollte, ist da halt schon immer so das Ziel m{\"o}glichste alle die da hinkommen, alle die irgenwie mitmachen (\dots) also es gibt dann ja verschiedene Sachen, direkte Treffen, Mailinglisten, Telefonkonferenzen und was nicht alles. Das ist schon immer das Ziel, dass sich die Leute nach M{\"o}glichkeit einigen, dass sie sich austauschen. Ich kann mich da gut dran erinnern, dass da bei Entscheidungen dann so ein Satz war "`Okay, wir haben das jetzt durchdiskutiert, wenn wir das jetzt entscheiden mit dieser Entscheidung, wer hat dann noch Bauchschmerzen?"` Das ist dann halt immer so, dass es auf Konsens zielt. Wo bei das bei manchen nat{\"u}rlich auch nicht m{\"o}glich ist, und trotzdem was entscheiden will, dann versucht man da halt dann doch noch vielleicht andere, meinetwegen andere Abstimmungen hinzukriegen. Aber dann auch immer so dass da niemand quasi mitentscheidet, wo er gar nicht daf{\"u}r ist. So ich glaube das zieht sich relativ durch. Das ist nat{\"u}rlich nicht immer ganz idealtypisch, dass das klappt, oder dass es auch mal ganz anders l{\"a}uft. Dass Leute mal nicht da waren und dann hinterher sagen, eigentlich wollte ich das doch nicht. Dann waren Sie halt nicht da. Kommt halt vor, aber im gro{\ss}en und ganzen zielt das immer darauf, dass das Leute, die irgendwie davon betroffen sind, und die mitentscheiden wollen, auch mitentscheiden k{\"o}nnen. 
\end{itemize}

\textbf{Teil 2 -- Einsatz der Anwendung}
\begin{itemize}
    \item[I:] Bitte geben Sie mir eine kurze Einf{\"u}hrung in das Projekt in dem Sie die Anwenung einsetzen wollen.
    \item[P5:] Genau. Diese Karte, die Idee kam uns ja schon ein bisschen fr{\"u}her. Seit dem ich jetzt hier mitmache bei diesem Hiwi-Job, seit Ende 2012, glaube ich, haben wir ja angefangen so ein Projekt zu starten, dass bisschen die Themen, die halt die beiden Profs hier an diesem Lehrstuhl mitbesch{\"a}ftigt sind, schon vorher l{\"a}ngere Zeit besch{\"a}ftigt haben, also so ein bisschen Stadtforschung und auf der anderen Seite Sozialisation und Gemeinschaftsforschung, Bildung ist da auch noch ein Aspekt der da drin noch mit herumwabert, das wollten die zusammenbringen. Und dann ging es eben darum, eine Plattform zu schaffen, um das "`Soziologische Wissen"', nenn ich es jetzt mal, einmal verf{\"u}gbar zu machen, auch nach au{\ss}en, dass das nicht immer nur etwas ist, was im Institut bleibt, halt eine Homepage zu gestalten. Und halt solches Soziologisches Wissen, dass halt entstanden ist, dort verf{\"u}gbar zu machen und auch gleichzeitig, das war dann eigentlich immer das Ziel, Leute die in der Stadt sind, und sich mit {\"a}hnlichen Themen besch{\"a}ftigen, und da gibt es hier in M{\"u}nster ja wohl eine ganze Menge, die aber oft so ein bisschen f{\"u}r sich und und unter sich bleiben, die haben halt ihr Spezialthema. Und da ging es ein bisschen darum zu gucken "`Okay, Spezialthemen sind auch wichtig, das ist auch gut so dass die das machen, aber vielleicht w{\"a}re es auch eine M{\"o}glichkeit eine st{\"a}rkere Wirkung zu erzielen, einmal diesem ganzen Initiativen, sofern das nicht sowieso schon der Fall war, bewusst zu machen, dass es noch andere Stellen gibt, die zu {\"a}hnlichen Themen arbeiten, die auch {\"a}hnliche Methoden haben, und {\"a}hnliche Ziele"' Und das ganze also dann auch untereinander zu vernetzen. Und halt auch den Austausch untereinander. Damit Diskussionen vielleicht mal woanders ankommen und dort aufgegriffen und mit verarbeitet werden k{\"o}nnen. Das war eigentlich so das Ziel. Und daraufhin haben wir dann auch hier eine Tagung veranstaltet. Beziehungsweise im Schloss veranstaltet. Zusammen auch mit der Stadt M{\"u}nster, und da dann auch eben viele der Initiativen und Parteien hier im Rat eingeladen. Auch nat{\"u}rlich die "`normalen"' Leute aus der B{\"u}rgerschaft. Das sind dann halt so Versuche solche Themen einzubringen, auch genau solche Vernetzungsprozesse zu starten und die Leute tats{\"a}chlich auch zusammen zu bringen, dass die sich auch kennenlernen. Ja dann im Zuge dessen kam dann eben auch die Idee auf, so eine Karte zu machen. Eine digitale Karte, die eben f{\"u}r alle Leute, zumindest wenn sie Internet haben, aber das ist ja mittlerweile eigentlich doch sehr sehr weit verbreitet, (\dots) Wo alle drauf zugreifen k{\"o}nnen, wo sie selber Vorschl{\"a}ge machen k{\"o}nnen. Vorschl{\"a}ge die schon dastehen kommentieren k{\"o}nnen und darauf antworten k{\"o}nnen. Und einfach sehen k{\"o}nnen, ja was passiert denn in meiner Stadt, kann ich irgendwo was mitmachen. Also da ist dann ein bisschen so der Gedanke, dass viele Leute vielleicht gute Ideen haben, viele Leute auch was machen und auch andere Leute das interessieren k{\"o}nnte oder auch interessiert, das nicht mitkriegen. Das ist ja nicht unbedingt einfach auch in einer gr{\"o}{\ss}eren Stadt, wo es oft so ein bisschen anonym zugeht, und da also die Initiativen von einzelnen Leuten sichtbar zu machen. Das war glaube ich so ein bisschen die Idee die hinter so einer Karte stand. Und genau. Daraufhin gab es dann diese Kooperation mit dem Institut f{\"u}r Geoinformatik. Und da haben wir dann versucht, diese Ideen dann quasi praktisch umzusetzen, mit der programmiertechnischen Hilfe. Und genau. Dann wollen wir halt jetzt sehen wie dass das jetzt anl{\"a}uft. Und gucken, wir sind ja auch gespannt wie das dann funktioniert. Ich wei{\ss} nicht ob ich das jetzt schon Vorweg sagen soll, aber ich fands immer noch super. Dieses Tool, und allen Leuten denen ich das gezeigt habe, waren auch relativ begeistert von den M{\"o}glichkeiten.
    \item[I:] Dann k{\"o}nnen wir gleich weitermachen. Welche Gr{\"u}nde sprechen f{\"u}r den Einsatz dieser L{\"o}sung gegen{\"u}ber anderen angedachten L{\"o}sungen?
    \item[P5:] Uh das ist ja eine schwere Frage. Ich wei{\ss} gar nicht, ob wir uns andere L{\"o}sungen {\"u}berlegt hatten. Ich kann mich jetzt gar nicht dran erinnern, dass wir noch andere Sachen hatten. Zumindest haben wir die wenn, nicht l{\"a}nger verfolgt. Das w{\"a}re dann halt so ein Brainstoring gewesen, und dann ist es glaube ich bei der Karte h{\"a}ngen geblieben, soweit ich mich jetzt erinnern kann. Ja also was halt daf{\"u}r spricht, ist halt, dass einmal die wie gesagt relativ allgemeine Verf{\"u}gbarkeit, so auch jederzeit. Das ist ja nicht darauf angewiesen, dass Leute zeitgleich auch irgendwo zusammenkommen oder zeitgleich mitmachen. Weil ja auch Tagesabl{\"a}ufe und Rythmen, doch sehr unterschiedlich sein k{\"o}nnen. Und da ist also die M{\"o}glichkeit eine Plattform zu haben, wo Leute, wenn sie Zeit haben, dann auch zugreifen k{\"o}nnen, sehen k{\"o}nnen, was ist passiert. Selber was machen k{\"o}nnen, und sich dar{\"u}ber dann vielleicht auch absprechen k{\"o}nnen und sich tats{\"a}chlich dann auch zu treffen. Das ist f{\"u}r mich halt immer noch so der gro{\ss}e Vorteil, das gilt ja generell f{\"u}r digitale Medien, (\dots) Es steht da, und es ist verf{\"u}gbar und verschwindet dann nicht sofort wieder. Und gleichzeitig macht es eben in der Form dieser Karte ja direkt sichtbar. Das ist ja quasi die Verkn{\"u}pfung mit dem echten Raum, mit der sozialen Welt, wo die Leute auch sehen k{\"o}nnen, was passiert denn hier bei uns in der Nachbarschaft, in meinem Viertel oder bei mir einfach in der N{\"a}he oder generell in der Stadt, wo ich vielleicht Lust h{\"a}tte vielleicht mitzumachen. Und genau. Da ist es dann, wenn man so will, vielleicht ein Vehikel, das ist ja so ein bisschen unsere Hoffnung, dass {\"u}ber die Vorschl{\"a}ge, die dann in dieser digitalen Karte stehen, dann daraus auch Bekanntschaften, Ideenaustausch und letztlich auch ganz handfeste Sachen entstehen, wo die Leute und B{\"u}rgerinnen sich selber Ideen geben k{\"o}nnen und Idenn umsetzen k{\"o}nnen. Manche Sachen vielleicht direkt vor Ort, manche Sachen dann geb{\"u}ndelt, dann geht man vielleicht, ich wei{\ss} nicht, zu Stadtteilb{\"u}ros oder wendet sich an die Stadt, je nachdem was es halt ist. Aber halt diesen Austausch der Leute und Ideen der Leute und Vernetzung von der digitalen Welt in die echte Welt zu {\"u}bertragen. Und da halt so ein Werkzeug zu haben. Deswegen ist halt eine Karte tats{\"a}chlich eine gute M{\"o}glichkeit um da so eine {\"U}bertragung zwischen Echt und Virtuell hinzubekommen.
    \item[I:] Welche Eigenschaften w{\"u}rden Sie davon habhalten diese Anwendung einzusetzen?
    \item[P5:] Ja so nach meiner Erfahrung (\dots) Ja einmal ist es kompliziert ist, also wenn die Sachen, die da passieren, die Funktionen die da vielleicht drin stecken, wenn die nicht gefunden werden, wenn die zu kompliziert wirken, wenn also Leute die da vielleicht nicht sowieso schon Interesse an so einem Thema haben, und auch bereit sind, sich ein bisschen Zeit daf{\"u}r zu investieren sich einzuarbeiten. Also wenn Leute, die mal da so drauf gesto{\ss}en sind, oder vielleicht zuf{\"a}llig drauf kommen oder das irgendwie mitbekommen und sich das angucken. Wenn es dann also zu frustrierend oder kompliziert ist, die Sachen die da M{\"o}glich sind, auch zu machen. Das k{\"o}nnte ich mir vorstellen, k{\"o}nnte einige Leute dann davon abhalten, dass weiter zu nutzen. Und was mir sonst jetzt noch spontan einf{\"a}llt, ist vielleicht auch eine Sache, wenn da wenig los ist. Das kenn ich ja auch von mir selber, wenn ich in Foren bin, in denen nicht viel passiert, da klickt man ein paar mal rein. Aber wenn man merkt, da schreibt sowieso kaum wer, und das was ich geschrieben hat, hat jetzt auch noch keiner drauf geantwortet, dann lasse ich es vielleicht nach einer kurzen Zeit wieder da mitzumachen. Das hei{\ss}t also wenn bei dieser Karte da keine Sachen stehen, und die Leute selber vielleicht auch noch nicht selber Vorschl{\"a}ge machen, sondern erst einmal gucken was da passiert. Wenn da dann also nicht viel passiert, k{\"o}nnte das nat{\"u}rlich auch dazu f{\"u}hren, dass die dann erstmal wieder von ablassen. Also d{\"u}rfte es da dann auch darum gehen, gerade dann am Start zumindest ein paar Sachen zu haben, und das dann auch selber, auch von organisatorischer Seite auch selber Sachen einzutragen, auf der einen Seite dann auch vielleicht Werbung zu machen. Wir sind ja jetzt schon ein St{\"u}ck weit vernetzt mit vielen Initiativen hier aus der Stadt. Da dann auch nochmal sagen, dass die auch selber Sachen reinsetzen, so dass da dann auch was passiert, dass die Leute auch vielleicht im besten Fall sehen, "`Ja, tats{\"a}chlich auch bei mir um die Ecke sind Sachen"'. Dass Internet kann ja manchmal das Gef{\"u}hl geben, dass das irgenwo stattfindet. Und es ist nicht direkt sp{\"u}rbar. Wenn aber so dieser Impuls vielleicht dann kommt. "`Das ist bei mir um die Ecke, ich kann da hingehen, und tats{\"a}chlich auch etwas machen, das was mich wirklich auch ber{\"u}hrt"'. Dass das einfach daf{\"u}r sorgt, dass Leute selber auch Motivationen haben, mitzumachen und selber dann damit auch bereit sind Erfahrungen zu machen. Und dann auch selber bereit sind Sachen da rein zu stellen. Dass das dann irgendwie so ein bisschen ein Selbstl{\"a}ufer wird. Aber es k{\"o}nnte auch da sein, dass dort erst so eine kritische Masse erreicht werden muss, das k{\"o}nnte ich mir am Anfang als Schwierigkeit vorstellen.
    \item[I:] Ja muss man dann schauen dass man die Seite dann in Gang h{\"a}lt. (Ja dass das auch gepflegt wird)
\end{itemize}

\textbf{Teil 3 -- Abschlie{\ss}ende Fragen}
\begin{itemize}
    \item[I:] Kennen Sie Beispiele f{\"u}r die Verkn{\"u}pfung geographischer Daten mit Diskussionsbeitr{\"a}gen?
    \item[P5:] Ja so ein bisschen. Ich kenne das so aus politischen Zusammenh{\"a}ngen. Zum Beispiel im Wendland. Da sind ja manchmal Castortransporte und das ist ja ein relativ gro{\ss}es Gebiet, sehr weitl{\"a}ufig. Wald, Felder. Und halt immer rund um diese Schiene, wo der Castor dann langl{\"a}uft, da kenne ich das durchaus, dass dann eben Karten, auch noch aus Papier aber bei dem letzten dann auch schon mit digitalen Karten, dass da dann auch Diskussionen gef{\"u}hrt wurde, {\"u}ber Punkte, die vielleicht gut zu erreichen sind, wo es gut w{\"a}re, sich da auf die Schienen zu setzen, oder so {\"a}hnliche Sachen. Dass da dann die Karten genutzt worden sind, um dar{\"u}ber zu diskutieren was man wo, wie machen k{\"o}nnte. Wo Infopunkte hin sollten.
    \item[I:] Und wenn das dann digitale Diskussionen waren, mit welchen Tools wurden die dort dann durchgef{\"u}hrt?
    \item[P5:] Ich glaube wir haben damals noch eine (\dots) So einen IRC\footnote{Internet relay chat}-Channel benutzt. Genau dar{\"u}ber.
    \item[I:] Da konnte man dann aber nicht direkt auf einer Karte diskutieren?
    \item[P5:] Nein die Karte war dann woanders. Die war halt irgendwo hochgeladen. Die hatte man sich dann irgendwie angeguckt. Und dann haben wir {\"u}ber IRC dar{\"u}ber dann uns ausgetauscht. Das war dann aber halt zwei Sachen die man getrennt hatte. Das war nicht verkn{\"u}pft.
    \item[I:] Kennen Sie Werkzeuge um interaktive Karten mit eigenen Inhalten zu erzeugen?
    \item[P5:] Ich wei{\ss} nicht, ob es das genau trifft, ich wei{\ss} dass es bei Google Maps so M{\"o}glichkeiten irgendwie Marker zu setzen, das habe ich aber selber noch nie gemacht. Und ich glaube bei Openstreetmap ist es auch m{\"o}glich. Da irgendwie Punkte in die Karte zu machen, oder sich selber Routen oder sowas anzulegen f{\"u}r eigene Zwecke. Habe ich selber aber auch noch nicht gemacht.
    \item[I:] Gut. Dann war es das jetzt auch von meiner Seite. Gibt es noch Fragen oder Anmerkungen von Ihrer Seite?
    \item[P5:] Nein, mir f{\"a}llt jetzt auch gerade nichts mehr ein. Au{\ss}er, dass ich mich halt freue, dass sich das so gefunden hat. Ich fande das auch insgesamt so ziemlich so nett. Also angenehm, so mit dir das zusammen zu machen. Diesen Austausch. Wobei du nat{\"u}rlich auch die meiste Arbeit hattest, und wir immer nur kluge Vorschl{\"a}ge hatten, oder so. Aber ich fand das war generell eine sch{\"o}ne Zusammenarbeit. Und das hat mir auch so Spa{\ss} gemacht.
    \item[I:] Ja vielen Dank!
\end{itemize}

\subsubsection{Appendix B.8. Participant 6}

\textbf{Teil 1 -- B{\"u}rgerbeteiligung}
\begin{itemize}
    \item[I:] Erz{\"a}hlen Sie mir {\"u}ber ihre Rolle und Aufgaben in der B{\"u}rgerbeteiligung.
    \item[P6:] Also wo ich mich sehe? (Genau) Ich glaube einfach als aktive M{\"u}nsteranerin, die schon an ganz verschiedenen Stellen sich ehrenamtlich engagiert hat, und Spa{\ss} daran hat, bestimmte Projekte auf die Beine zu stellen und es auch ein St{\"u}ck weit als Verantwortung sieht, sich Gesellschaftlich zu engagieren und sich einzubringen.
    \item[I:] Wie lange sind Sie dann jetzt schon aktiv?
    \item[P6:] Also irgendwie schon seit Schulzeiten. Schon bestimmt zwanzig Jahre oder so.	Also immer wieder mal ganz unterschiedliche Stellen. Auch mal Pause wieder zwischendurch.
    \item[I:] Welche Aspekte der B{\"u}rgerbeteiligung sind die aus Ihrer Sicht die Wichtigsten?
    \item[P6:] Ich glaube es geht mir ganz wichtig darum ein gutes Zusammensein zu haben. Und ein lebenswertes Leben wo man als Mensch mit dabei ist. Wo man sich einbringen kann, f{\"u}r Sachen die einem selber wichtig sind. Wo man anderen Menschen begegnen kann. Wo man was gemeinsam miteinander macht, und irgendwie das Gef{\"u}hl hat, es ist etwas sinnvolles. Man lernt neue Leute kennen. Man lernt neue Sichtweisen kennen. Das Miteinander ist mir wichtig vor allen Dingen. Und dann sicherlich auch eine gute Arbeit zu machen.
    \item[I:] Bitte geben Sie mir eine Einf{\"u}hrung in ein Projekt bei dem Sie denken dass es besonders auf eine gute Kommunikation und Diskussion zwischen den Akteuren angekommen ist.
    \item[P6:] Ich habe mich vor Jahren in der Queer-Gemeinde engagiert. War im Leitungsteam und Sprecherin der Gruppe. Ich habe vor Jahren Theologie studiert. Und das war eine Gruppe von Leuten, mit sehr vielen Schwulen und Lesben. Aber auch Leute die sich irgendwie nicht so ganz mit der Kirche identifizieren k{\"o}nnen. Da war es schon sehr wichtig, dass wir uns vernetzt haben, weil es erstens am Anfang eine sehr kleine Gruppe war. Und um so das Projekt am laufen zu halten, und auch in der Au{\ss}einandersetzung mit dem Bistum und der offiziellen Bistumspolitik, war schon auch wichtig, Kontakt zu anderen Gruppen zu haben. Also der Hook, der der anderen Queer-Initiativen in ganz Deutschland oder anderen Gruppierungen, Gespr{\"a}chskreisen, Universit{\"a}tskreisen, Gruppierungen innerhalb des Bistums. Und da war das sehr sehr wichtig, und auch bereichernd und hilfreich.
    \item[I:] Wie ist dann die Kommunikation abgelaufen?
    \item[P6:] Also so ein Internettool hatten wir noch nicht. Wir hatten selber einen Internetauftritt, und haben uns als Gruppe vorgestellt. So offen wie das mit dieser Thematik halt m{\"o}glich ist. Da ist nat{\"u}rlich eine heikle Sache, wenn sich schwule Priester oder schwule Theologen (\dots) schwule Theologen k{\"o}nnen sich nicht unbedingt outen. Lesbische eben auch nicht. Und in sofern eben tats{\"a}chlich eben E-Mail-Verteiler, Linklisten im Internet auf andere Initiativen. Und nat{\"u}rlich haben wir uns als Gruppe regelm{\"a}{\ss}ig mindestens einmal im Monat getroffen und {\"u}berlegt und geplant. Zumindest das Orga-Team. Und dar{\"u}ber hinaus eben noch andere Veranstaltungen. Das waren dann Gottesdienste f{\"u}r diese Randgruppe. Dann haben wir zum Beispiel Gemeindefeste oder Wochenendbegegnungen organisiert. Und da sind dann teilweise auch Leute aus ganz Deutschland dazu gekommen.
    \item[I:] Die Kommunikation ist dann eher in der Gruppe geblieben, oder wurde auch aktiv nach au{\ss}en kommuniziert?
    \item[P6:] Wir haben Anfragen von au{\ss}en bekommen. Es gab immer mal wieder Radiosender oder Journalisten, die was von uns wissen wollten. Selber so ganz aktiv haben wir durch die Internetseite oder durch bekannt machen in bestimmten Publikationen, es gab Na-Dann zum Beispiel, oder andere Printmedien die es glaube ich mittlerweile gar nicht mehr gibt, und eben {\"u}ber den universit{\"a}ren Kontext bekannt gemacht, dass es uns gibt. Aber eben auch nicht zu offensiv, weil es ist halt auch ein heikles Thema f{\"u}r viele.
\end{itemize}

\textbf{Teil 2 -- Einsatz der Anwendung}
\begin{itemize}
    \item[I:] Bitte geben Sie mir eine Einf{\"u}hrung in das Projekt, in dem die Anwendung eingesetzt werden soll.
    \item[P6:] Im n{\"a}chsten Jahr ist so ein Nachhaltigkeitstag geplant in M{\"u}nster, und da soll es darum gehen, diese Thematik "`Nachhaltigkeit, nachhaltiges Leben"' einer breiteren Stadtgemeinschaft bekannt zu machen, und auch einzuladen, sich damit auseinanderzusetzen. Die Zielgruppe ist unglaublich heterogen. Vom Kindergarten bis Universit{\"a}t. Von Wissenschaftlern bis Kreative. Da gibt es eine Menge Ideen. Ein wichtiger Punkt ist eben sicherlich auch eine M{\"o}glichkeit zu haben, dass man erstmal sagt was {\"u}berhaupt schon da ist. Weil ich glaube, dass es eine Menge Initiativen und M{\"o}glichkeiten hier in M{\"u}nster gibt, die aber noch nicht so optimal vernetzt sind, dass man wirklich von einander wei{\ss}, oder sich auch schnell und komplikationslos {\"u}ber deren Aktivit{\"a}ten informieren kann. Und daf{\"u}r k{\"o}nnte ich mir vorstellen, dass das gut w{\"a}re.
    \item[I:] Was f{\"u}r redaktionelle Inhalte wollen Sie einstellen?
    \item[P6:] Also wenn ich es mir einfach so als Nutzerin vorstelle, dann finde ich das mit der Karte gut. Dass man sich sagen kann, "`Ich m{\"o}chte gerne nach bestimmten Kategorien sortieren. Beispielsweise welche Gesch{\"a}fte nachhaltige Produkte verkaufen"'. Ich glaube es w{\"a}re mir wichtig, dass es aktuell ist. Ich glaube dass ich die klassischen Fragen habe "`Wo steht es, wie kann ich Kontakt aufnehmen, was machen die Inhaltlich"'. Also eine Verlinkung wenn m{\"o}glich zu deren Internetseite, dass man fragen kann "`Was machen die so? Passt das zu mir?"' Ich glaube das w{\"a}ren erstmal so die Grundinformationen, die erstmal wichtig w{\"a}ren.
    \item[I:] Welche Anreize f{\"u}r B{\"u}rger sich dann in Dialogen {\"u}ber die Anwendung auszutauschen sehen sie hier dann?
    \item[P6:] Ich glaube jetzt eher weniger durch die Internetseite, sondern durch die Aktivit{\"a}ten der Leute selber.
    \item[I:] Welche Gr{\"u}nde sprechen f{\"u}r den Einsatz dieser Anwendung gegen{\"u}ber anderen Anwendungen?
    \item[P6:] Also das einzige, was ich mal mitgekriegt habe, dass wohl der Asta mal vor zwei oder drei Jahren ein PDF-Dokument ver{\"o}ffentlicht hat, wo eben auch eine relativ ausf{\"u}hrliche Liste mit Nachhaltigkeitsinitiativen drin war. Die hat nat{\"u}rlich erstmal das Problem, dass sie erstmal m{\"u}hsam suchen muss im Internet und erst durch Gl{\"u}ck findet. Und dass nat{\"u}lich so ein PDF-Dokument innerhalb von ein bis zwei Jahren sp{\"a}testens veraltet ist. Dass das eine M{\"o}glichkeit ist, Daten und Informationen auch aktuell zu halten. Und wenn es m{\"o}glich w{\"a}re, sie eben sehr prominent zu platzieren. Meintetwegen auf der Seite der Stadt M{\"u}nster. Eben auch sehr gut zug{\"a}nglich und mit der Chance dann auch vielleicht Leute anzusprechen, die, wei{\ss} ich nicht, neu sind, und so einfach mal gucken wollen was in dieser Stadt los ist und dann dazu sto{\ss}en k{\"o}nnen.
    \item[I:] Was w{\"a}ren Dinge, von denen Sie denken, dass sie B{\"u}rger davon abhalten k{\"o}nnten, sich {\"u}ber die Anwendung zu beteiligen?
    \item[P6:] Es muss funktional sein. Also es m{\"u}sste relativ selbsterkl{\"a}rend sein. Ich finde es nicht gut, wenn man unglaublich lang durchklicken muss. Es muss ziemlich schnell sein. Also wie ist es aufgebaut, wie schnell komme ich an meine Informationen? Die wichtigsten Sachen m{\"u}ssten da sein. Also wie, wo, was, wann. Vielleicht noch die M{\"o}glichkeit Kontakt aufzunehmen. Das w{\"a}re glaube ich so das wichtigste. So als Ersteinstieg in diese Thematik.
    \item[I:] K{\"o}nnen Sie sich weiter Anwendungsf{\"a}lle f{\"u}r die Verkn{\"u}pfung von Texten mit Karten au{\ss}erhalb der B{\"u}rgerbeteiligung vorstellen?
    \item[P6:] Also das w{\"a}re jetzt so erst mal meine erste Idee gewesen. Was ich mir nat{\"u}tlich vorstellen k{\"o}nnte, dass solche Karten auch in anderen St{\"a}dten programmiert werden. Und man von daher dann eben auch Verkn{\"u}pfungen zwischen diesen St{\"a}dten herstellt. Dass man also vielleicht guckt, in welchen St{\"a}dten gibt es beispielsweise so eine Givebox, die wir da eben programmiert haben. Oder wo gibt es bestimmte regionale Untertreffen von beispielsweise Greenpeace oder irgendwelchen anderen Initiativen. Vielleicht auch die M{\"o}glichkeit, dass bestimmte Gruppen oder gr{\"o}{\ss}ere, auch {\"u}bergreifende Initiativen, auch Daten	oder so in diese Seite mit einbauen, dass man gucken kann, wo findet vielleicht so etwas wie der Nachhaltigkeitstag statt. Ist das nicht nur bei uns in M{\"u}nster, oder in Bremen gibt es etwas anderes. Wobei die glaube ich auch ein anderes Konzept haben. Gibt es das in K{\"o}ln, gibt es das in Aachen oder anderen gr{\"o}{\ss}eren St{\"a}dten. Das f{\"a}nde ich auch vielleicht ganz gut, dass da sich nicht nur aussenstehende B{\"u}rger und B{\"u}rgerinnen informieren k{\"o}nnen. Sondern auch Initiativen selber.
\end{itemize}

\textbf{Teil 3 -- Abschlie{\ss}ende Fragen}
\begin{itemize}
    \item[I:] Kennen Sie Beispiele f{\"u}r die Verkn{\"u}pfung geographischer Daten mit Diskussionsbeitr{\"a}gen?
    \item[P6:] Nein. Kenne ich selber noch nicht.
    \item[I:] Und Werkzeuge im interaktive Karten mit eigenen Inhalten zu erzeugen?
    \item[P6:] Habe ich bisher auch noch nicht benutzt, finde ich aber ziemlich gut, dass man von anderen Seiten auf diese Karte zugreifen k{\"o}nnte. Das meinen Sie doch, dass ich diese Karte jetzt in meine eigene einbinden k{\"o}nnte?
    \item[I:] Nein, die Frage zielt konkret auf Werkzeuge die es erlauben eigene Karten zu erzeugen. Bestes Beispiel war jetzt vor kurzem die Karte zur {\"U}berschwemmungshilfe.
    \item[P6:] Achso. Ja das habe ich nur am Rande mitbekommen. Aber ich kenne das aus anderen Bereichen. Habe ich auch selbst aber nicht genutzt.
    \item[I:] Okay. Gibt es dann noch abschlie{\ss}ende Fragen oder Anmerkungen von Ihrer Seite?
    \item[P6:] Nein. Ich denke h{\"o}chstens dann wenn es konkret mit der Umsetzung (\dots) Und, wird das angewandt, wer wird dann zust{\"a}ndig sein. Die Idee finde ich schon wirklich gut. Und dann auch immer ausgedr{\"u}ckt, mit der praktischen Erfahrung, dass solche Ideen am Anfang immer sehr gut sind, aber eben auch die langfristige Perspektive brauchen. Dass dann auch da anzubinden, wo das auch m{\"o}glichst gesichert ist.
    \item[I:] Ja das wird dann wohl noch im Detail entschieden werden. Dann vielen Dank!
\end{itemize}

\subsubsection{Appendix B.X Expert 1}

\begin{itemize}
    \item[I:] Kennen Sie Anwendungen die Diskussionen durch Geoobjekte unterst{\"u}tzen?
    \item[P1:] (\dots) Ich {\"u}berlege gerade. Es gibt so Emergency-Response-Maps so in Krisenregionen um Hilfseins{\"a}tze zu planen. So was habe ich mal gesehen. Das k{\"o}nnte so in die Richtung gehen weil da eben auch so bestehende Diskussionen mit irgendwie Geoobjekten angereichert werden. Ansonsten Diskussionen nicht wirklich.
    \item[I:] Also dann auch nicht benutzt?
    \item[P1:] Nicht dass ich w{\"u}sste. Also ich meine nat{\"u}rlich habe ich total viele Geo-Anwendungen irgendwie. Also nehmen wir an wie (\dots) Google Maps oder irgendwelche Tank Apps oder Navigations Apps und Apps um Einkaufszentren zu finden oder {\"a}hnliches. Aber das hat ja alles nichts mit einer Diskussion zu tun. Also ich denke nein.
    \item[I:] Also so richtig Problem oder Vorteile davon kennen Sie dann auch nicht?
    \item[P1:] Nein nicht wirklich. Also Vorteile k{\"o}nnte ich mir halt vorstellen, dass es eben Eineindeutig ist dass ich {\"u}ber einen bestimmten Ort schreibe den ich halt gleichzeitg dann noch mit einem Ort auf der Karte verkn{\"u}pfen kann. Dann ist es eben klar, {\"u}ber was ich spreche. Und das ganze ist eben eineindeutig. Und, also nehmen wir an, ich spreche jetzt {\"u}ber M{\"u}nster und es gibt zwei M{\"u}nster in Deutschland und dann ist klar welches M{\"u}nster ich meine.
    \item[I:] Das haben Sie ja gerade schon ein bisschen gesagt, aber welche Anwendungsf{\"a}lle zur Verkn{\"u}pfung von Diskussionen und Geoobjekten k{\"o}nnen Sie sich au{\ss}erhalb des B{\"u}rgerbeteiligungskontextes vorstellen?
    \item[P1:] Ich k{\"o}nnte mir das relativ grob strukturiert eigentlich {\"u}berall vorstellen. Sei es dass ich einen Foreneintrag verfasse oder jedem Diskussionsobjekt eben die M{\"o}glichkeit habe, Orte direkt zu verkn{\"u}pfen und mit dem Vorteil eben dann direkt verweisen kann auf irgendwelche Geoinformationsanwendungen. Und als konkreten Einsatzzweck f{\"a}llt mir eben nur dieses Krisenmanagementsystem ein von dem ich eben schon erz{\"a}hlt habe.
    \item[I:] Welche L{\"o}sungen um B{\"u}rger, Initiativen und die Politik zusammenzubringen kennen Sie?
    \item[P1:] B{\"u}rgerinitiativen k{\"o}nnte man sagen. B{\"u}rgerstammtische. (\dots) So offiziell organisierte Treffen wo dann irgenwie Informationsaustausch stattfindet. Also zum Beispiel zu irgendwelchen lokalen Initiativen. So als Beispiel "`Tagebau in M{\"u}nster"' und dann w{\"u}rde informiert werden. Oder Stuttgart 21 und dann findet irgendwie eine Informationsveranstaltung dazu statt. 
    \item[I:] Und so in Richtung Software-L{\"o}sungen?
    \item[P1:] Also es gibt ja auf jeden Fall so Petitionsportale wo ich jetzt sagen w{\"u}rde das ist ja eher vom B{\"u}rger initiiert. Also vom B{\"u}rger in Richtung Politik. (\dots) Ich glaube die Bundesregierung hat auch irgendwie so ein Ding wo man mitdiskutieren kann. Ich muss gestehen, ich wei{\ss} gerade nicht genau wie das hei{\ss}t. Komme ich irgendwann mal wieder drauf. Also ich glaube aber auch dass es von der Politik Portale gibt, die sich an die B{\"u}rger wendet und dann auch zu aktiver Mitarbeit aufruft. Hab ich aber aktiv noch nicht benutzt. Und wei{\ss} gerade nicht genau wie das hei{\ss}t. 
    \item[I:] Denken Sie die explizite Verkn{\"u}pfung von Geoobjekten mit Diskussionsgegenst{\"a}nden ist generell hilfreich im B{\"u}rgerbeteiligungskontext?
    \item[P1:] Ja absolut! Ich habe mal von sowas geh{\"o}rt, da konnte man, glaube ich, Stra{\ss}ensch{\"a}den melden. Das f{\"a}llt mir gerade auch noch so ein. Quasi zur Frage vorher. Und das ist ja schon ganz interessant, wenn ich mir sage "`Okay, hier ist irgendwie an folgender Stelle die Stra{\ss}e kaputt"' Und dann kann ich das direkt auf ner Karte markieren. Oder ich hab das glaube ich auch mal geh{\"o}rt, dass man zu so einem Blitzmarathon Vorschl{\"a}ge machen an welchen Stellen Gefahrenstellen sind und an welchen Stellen geblitzt werden soll. Da konnte man direkt auf so ner Karte markieren was die Stelle ist die ich konkret vorschlage. Ich denke schon dass das sinnvoll ist. 
    \item[I:] Dann konkret zur Anwendung im Vergleich zu bestehenden Anwendungen die Sie schon kennen. Was denken Sie speziell zu der Gegebenheit dass in der {\"u}bersicht nur die Geoobjekte des ersten Beitrages zum Thema angezeigt werden und dass in der Themendetailansicht nur die Geoobjekte zu dem Thema angezeigt werden?
    \item[P1:] Ich glaube das ist gut f{\"u}r die {\"u}bersichtlichkeit. Also nat{\"u}rlich hat das jetzt einen starken Fokus auf den Ersteller. Es scheint mir weniger Community-fokussiert, sondern hat eher einen Autorenfokus k{\"o}nnte man sagen. Aber auf der anderen Seite ist eben so dass ich nicht wei{\ss} wie gro{\ss} so diese Diskussionen werden k{\"o}nnen. Also wenn man sich jetzt wirklich vorstellt, zum Beispiel hat irgendjemand eine Frage gestellt und dann kommen da sagen wir mal zwanzig Antworten die jeweils auch Geoobjekte referenzieren. Dann w{\"u}rde dieses eine Projekt oder Beitrag sehr dominierend auf der Karte sein. Und deshalb denke ich schon, dass es eine sinnvolle Entscheidung ist nur ein Objekt pro Thema anzuzeigen. Zumindest jetzt gerade in diesem Kontext wie ich es einsch{\"a}tzen kann.
    \item[I:] Dann die Zwei-Wege Highlights bei Mausinteraktion?
    \item[P1:] Absolut sinnvoll! Also sonst w{\"u}sste ich ja gar nicht was wo zu geh{\"o}rt. Nat{\"u}rlich k{\"o}nnte ich das vermutlich auch hier anklicken und komm dann direkt in die Detailansicht. Theoretisch ist die Verlinkung von der Karte zu dem Objekt nicht so wichtig. Weil ich kann ja sowieso in die Detailansicht rein gehen. Aber dass ich, wenn ich nur das Objekt auf der rechten Seite habe, direkt sehe wo es auf der Karte verortet ist, das halte ich f{\"u}r sehr wichtig.
    \item[I:] Die Filter- und Sortierfunktion?
    \item[P1:] Gut da h{\"a}ngts immer davon ab, wie viele Eintr{\"a}ge ich hab. Also momentan mit zehn Eintr{\"a}gen ist nat{\"u}rlich so ein Filter noch nicht so wichtig, wenn aber das ganze mal deutlich mehr werden, ist so ein Filter auf jeden Fall wichtig. Da h{\"a}ngt es glaube ich dann davon ab dass die Funktionen die der Filter bietet, sinnvoll gew{\"a}hlt sind. Und dass sie verst{\"a}ndlich sind finde ich. Also dass ich jetzt hier zum Beispiel bei "`diskutieren"', "`mitmachen"', "`vorschlagen"' wirklich genau wei{\ss}, was ich anklicke. Das hier zum Beispiel bei "`vorschlagen"' w{\"u}rde ich mir noch ein bisschen zus{\"a}tzliche Erkl{\"a}rung was ich genau ich hier jetzt filtern kann w{\"u}nschen oder {\"a}hnliches. Ich denke bei den Akteuren ist das relativ eindeutig. Also "`Bildung"', "`B{\"u}rger"', "`Stadt"', "`Wirtschaft"', das versteht jeder. Bei den Inhalten ist das nat{\"u}rlich so ein bisschen (\dots) Ja, ob das jetzt wirklich trennscharf ist, und ob das alles abdeckt; insbesondere so Dinge wie "`Sonstiges"', da landet h{\"a}ufig viel zu viel in solchen Kategorien. Aber ansonsten, ja Filter eindeutig gut.
    \item[I:] Dann diese Verfassen- und Antworten- Funktion und speziell das Verkn{\"u}pfen von den W{\"o}rtern mit den Geoobjekten, bestehenden Geoobjekten und Hyperlinks?
    \item[P1:] Finde ich gut. Ich wei{\ss} allerdings nicht ob es einhundert Prozent intuitiv ist. Also es ist ja schon so wenn ich hier jetzt irgendwie drauf antworte, dann wird mir erstmal (\dots) Also erstmal h{\"a}tte ich mich hier intiuitv wahrscheinlich, wenn das jetzt hier gerade im Vorfeld nicht so eindeutig erkl{\"a}rt worden w{\"a}re, nicht mit diesen Icons besch{\"a}ftigt. Die h{\"a}tten mir erstmal nichts gesagt. Also ich glaube dieses "`Link"'-Icon, das erkenne ich und eigentlich diesen Geo-Marker erkenne ich auch. Das kenne ich schon aus einem anderen Kontext. Der ist ja schon sehr stark an Google Maps Objekt orientiert. Und so einen Link, kennt man aus jeden Online-Editor oder Text-Editor irgendwie. Aber jetzt zum Beispiel dass ich irgendwas eingetippt h{\"a}tte, dann das Wort markiere, dann dieser Kontextdialog. Das ist eben etwas da w{\"a}re ich glaube ich selber als Bediener nicht drauf gekommen. Und unter dem Standard "`Antwort verfassen"' habe ich auch nicht im Hinterkopf dass meine Antwort eben Geodaten enthalten kann. Nichtsdestotrotz glaube ich dass jemand der so ein bisschen hier mit herumspielt ist dass dann schon klar. Also hier auch vielleicht "`Punkt markieren"' w{\"u}rde ich auch in "`Ort markieren"' umbennen. "`Punkt"', ich wei{\ss} nicht ob ich da automatisch was Geographisches mit verbinden w{\"u}rde. Aber ansonsten ist das sehr gut mit der Funktionalit{\"a}t.
    \item[I:] Dann ganz speziell jetzt auf die Anwendung bezogen. Wie werden Dialoge damit vereinfacht?
    \item[P1:] Das kann ich total schwierig nur einsch{\"a}tzen. Also es ist einfach so, da sprech ich jetzt aus eigener Erfahrung: Diskutieren {\"u}ber Tools ist wirklich schwierig. Insbesondere wenn (\dots) Das h{\"a}ngt hier jetzt insbesondere davon ab, ob es Moderationsfunktionalit{\"a}ten noch gibt. Grunds{\"a}tzlich sieht es hier aus dass f{\"u}r Diskussionen eigentlich relativ wenig Platz ist. Also in der Hinsicht, als dass sowohl das Eingabeformular relativ beschr{\"a}nkt ist vom Platz her, als auch der Platz auf dem die Diskussionen angezeigt werden. Kleinere Diskussionen kann ich mir durchaus vorstellen, aber wenn ich mir jetzt wirklich vorstelle zum Beispiel f{\"u}nfzig Akteure versuchen eine Diskussion zu f{\"u}hren, die dann der menschlichen Natur folgend dann auch nicht perfekt strukturiert abl{\"a}uft, also jeder versucht seinen eigenen Standpunkt durchzubringen (\dots) Dann glaube ich, dass durchaus externe Tools sinnvoller sein k{\"o}nnten. Ist aber pure Vermutung.
    \item[I:] Dann die Favorisierung?
    \item[P1:] Also f{\"u}r mich als Orientierungskriterium ist das nat{\"u}rlich super. Das hei{\ss}t also wenn ich die Themen sortieren kann, was hat die meisten Favorisierungspunkte und mir somit irgendwie so auf die Meinung der anderen Beziehen kann, das finde ich gut. Ob ich selber als Nutzer davon so viel Gebrauch machen w{\"u}rde, wei{\ss} ich nicht so genau. Weil zum einen, gibt es ja wie ich das sehe keine eigene Favoritenliste. Also ich verwalte damit nicht meine Favoriten, wie Bookmarksfunktion sozusagen. Das w{\"u}rde Reize f{\"u}r mich erzeugen dann auch Beitr{\"a}ge zu favorisieren. Und zum anderen ist die Funktionalit{\"a}t relativ versteckt finde ich. Ich wei{\ss} gerade schon wieder nicht wie es geht. Das scheint mir auch relativ versteckt. Ich hab ja eben instinktiv versucht in der {\"u}bersicht schon auf das Herzchen zu klicken, aber es funktioniert ja nur in der Detailansicht. Und dann ist es ja sogar komplett ausgeblendet wenn ich da nicht mit der Maus dr{\"u}ber fahre. Auch ist es sehr klein und ausgegraut und so weiter. Also da wei{\ss} ich nicht, ob die Funktion so intuitiv zu entdecken ist. Also wie gesagt ich finde es sehr gut dass die Funktionalit{\"a}t da ist, ich wei{\ss} nur nicht, ob den Nutzern genug Anreize gegeben werden diese Funktionali{\"a}t wirklich aktiv zu verwenden.
    \item[I:] Die Benutzerregistrierung und Anmeldung und auch ganz speziell der Social Login?
    \item[P1:] Alles absolut {\"u}bersichtlich. Genauso wie man das schon von anderen Webseiten her kennt. H{\"a}lt sich an die Standards. Das ist das wichtigste. Das hei{\ss}t also es gibt genau die Felder die ich mir vorstellen w{\"u}rde. Das einzige was mir eben aufgefallen ist, wenn ich mich einlogge, h{\"a}tte ich das registrieren weiter unten im Dialog gesucht. Also nicht oben im Titel, sondern weiter unten. Ich hab das eben {\"u}berlesen und erst nachdem ich aktiv danach gesucht habe, ist mein Blick nochmal auf den Titel gefallen, und dann war da der Link. Also ist aber ja auch nicht total versteckt. Ansonsten, ja, diese Social Login Funktionalit{\"a}ten finde ich gut. Sind ja auch in vielen Seiten verf{\"u}gbar. Ich selber nutze die fast nie, aber finde ich gut. Ist halt so State-of-the-art das zu bieten.
    \item[I:] Haben Sie insgesamt irgendwelche Funktionen vermisst?
    \item[P1:] Hm. (\dots) Ja ich habe ja eben schonmal gesagt, vielleicht dem Benutzer noch so eine "`Pers{\"o}nliche Favoriten"'-Liste anbieten. Oder eben dass mir Dinge zu denen ich selber schon mitgewirkt habe, also Beitr{\"a}ge geschrieben oder favorisiert habe, hervorgehoben werden. Das w{\"a}ren jetzt Dinge die mir direkt einfallen w{\"u}rden. Also nehmen wir jetzt erstmal den Fall an dass hier hundert Projekte eingetragen sind und da kommen jeden Tag neue Kommentare hinzu. Wie finde ich jetzt wieder, f{\"u}r was ich schonmal irgendwie kommentiert habe. Das w{\"u}rde mir fehlen. Ansonsten konkret nichts mehr. Aber es ist nat{\"u}rlich auch so dass ich nichts mit diesem Nachhaltigkeitsprojekt zu tun habe.
    \item[I:] Was f{\"u}r Gr{\"u}nde k{\"o}nnen Sie sich vorstellen die Leute davon abhalten k{\"o}nnten, die Karte zu benutzen?
    \item[P1:] Also es ist nat{\"u}rlich eine Registrierungsschwelle. (\dots) Ach nein, geht es sogar ohne?
    \item[I:] Nein, man kann verfassen, aber nicht absenden.
    \item[P1:] Ah okay. Ja aber das ist sehr gut gemacht. Finde ich richtig gut. Ich hatte jetzt eigentlich erwartet, dass mir direkt verboten werden w{\"u}rde einen neuen Beitrag zu verfassen. Und ich finde das sehr gut diese diesen Login nachdem der Kommentar verfasst wurde. Also ich hab mir die Arbeit schon gemacht und mehr als mein "`asdf"' da rein geschrieben, und dann versuch ich das abzuschicken und dann kommt die Aufforderung dass ich mich einloggen soll, dann tue ich das auch. Also finde ich sehr gut diese Reihenfolge gerade. Ansonten f{\"a}llt mir gerade kein Grund mehr ein. 	 
    \item[I:] Ja gut, gibt dann noch abschlie{\ss}ende Kommentare ihrerseits? Oder Fragen?
    \item[P1:] Eigentlich nicht wirklich. Doch. In Richtung was f{\"u}r Administrative Funktionen gibts da? Zum einen der Administrator der Seite und auch vielleicht f{\"u}r die Leute die da so ein Projekt "`besitzen"'. Hat der auch irgendwelche Moderationsfunktionen? Oder gibt es Exportfunktionen?
    \item[I:] Es gibt ein Admin-Interface mit dem kann man dann alles machen. Moderationsfunktionen f{\"u}r die Projektbesitzer gibt es allerdings nicht. Und Lesezugriff hat man {\"u}ber eine JSON-API. Deswegen k{\"o}nnte man das Interface auch komplett austauschen.
    \item[P1:] Ah okay.
    \item[I:] Gut, dann war es das. Vielen Dank!
\end{itemize}


\subsubsection{Appendix B.X Expert 2}

\begin{itemize}
    \item[I:] Kennen Sie Anwendungen die Diskussionen durch Geoobjekte unterstützen?
    \item[P1:] Ja ArguMap natürlich. (lacht) Sonst, kenn ich natürlich einige. Oder war das jetzt eine Ja-Nein-Frage? (Nein, das sollen Erzählaufforderungen sein.) Was denn noch? (\dots) Hätte ich jetzt vorher noch mal ein meine Diplomarbeit gucken sollen. Also jetzt spontan habe ich jetzt keine mehr so auf dem Schirm. Es gibt auf jeden Fall mehrere.
    \item[I:] Welche Anwendungsfälle gibt es für die Idee der "`Argumentation Map"'?
    \item[P1:] Ja einmal natürlich so Bürgerbeteiligungsgeschichten. Also öffentliche Planung, wo es darum geht irgendwie Bauvorhaben öffentlich zur Diskussion zu stellen, um halt einfach meistens um eine breitere Akzeptanz zu sicherzustellen. Dass man so nicht irgendwas plant und dass dann den Bürgern so vorsetzt, und sagt "`Okay, das ist es, und so wird es gemacht."' Sondern, damit schon möglichst früh die Bürger einzubinden. Das ist einmal so eine offizielle Nutzung und dann gibts natürlich so privat. So "`Freizeit"'-Nutzungen sage ich mal. Wo man irgendwie einen gemeinsamen Urlaub oder eine Radtour plant. Oder solche Sachen.
    \item[I:] Welche Lösungen um Bürger mit Initiativen oder Politik zusammenzubringen kennen Sie?
    \item[P1:] Naja, klassisch läuft das natürlich so (\dots) Also es kommt natürlich darauf an, wo man hinguckt. Hier in Amerika gibt es ja klassisch diese Town Hall Meetings wo man dann einfach hingehen kann, und seinen Senf dazugeben kann. Und in Deutschland läuft das ja glaube ich offiziell eher über Aushänge. Das halt so Entwürfe gemacht werden. Die werden dann halt ausgehängt, und dann kann man da das irgendwie kommentieren auf diesem Weg. Das ist so ein bisschen asynchroner. (\dots) Was gibts noch? (\dots) Ich glaube das jetzt wirklich so mit elektronischen Hilfsmitteln zu gestalten, das waren bislang eher alles so Experimente. Mir ist jetzt kein Beispiel bekannt wo das mal wirklich breit ausgerollt wurde sozusagen, dass man sowas wirklich online machen konnte. Habe ich noch nirgends gesehen.
    \item[I:] Denken Sie die explizite Verknüpfung von Geoobjekten mit Diskussionsgegenständen ist generell hilfreich im Bürgerbeteiligungskontext?
    \item[P1:] Finde ich einerseitz natürlich super (lacht), sonst hätte ich mich auch nicht damit beschäftigt. Aber ist natürlich auch immmer so eine zweischneidige Sache, weil man natürlich so eine gewisse Lernkurve hinzufügt. Also das muss schon extrem einfach zu bedienen sein, damit auch wirklich jeder Zugriff hat. Und man müsste dann wahrscheinlich auch noch so einen Extra-Schritt machen, und irgendwie noch so eine Station oder Kiosk im Rathaus irgendwie zu Verfügung stellen. Für Leute eben die das nicht von zuhause aus machen können. Vielleicht dann auch mit jemandem, der das irgendwie bedienen kann und da helfen kann. Sonst hat man da natürlich die Gefahr, dass man da Nutzergruppen, wahrscheinlich vor allem ältere Leute, einfach ausschließt.
    \item[I:] Dann konkret zur Anwendung im Vergleich zu bestehenden Anwendungen die Sie schon kennen. Was denken Sie speziell zu der Gegebenheit dass in der {\"U}bersicht nur die Geoobjekte des ersten Beitrages zum Thema angezeigt werden und dass in der Themendetailansicht nur die Geoobjekte zu dem Thema angezeigt werden?
    \item[P1:] Ich denke das ist ganz clever gelöst, weil sonst wird es wahrscheinlich sehr schnell unübersichtlich. Wenn du jetzt noch eine räumliche Suche drin hättest, dann würde man das wahrscheinlich haben wollen, dass auch in in den Antworten die Georeferenzen durchsucht werden. Das hast du glaube ich noch nicht, oder? (Nein, da ist keine räumliche Suche drin) Das würde wahrscheinlich Sinn machen. Aber so jetzt zum browsen, sage ich mal, wenn man einsteigt, macht das schon Sinn finde ich, dass man nur das sieht, was auf der obersten Ebene ist.
    \item[I:] Dann die Zwei-Wege Highlights bei Mausinteraktion?
    \item[P1:] Finde ich gut, das hatten wir ja auch in dieser ArguMap-Anwendung schon so. Gibt es denn da auch die Möglichkeit, dass mehrere Beiträge auf das gleiche Objekt referenzieren? (Ja, das geht) Ja das ist auf jeden Fall super dann. Weil dann kann man direkt auf einen Blick sehen, welche Beiträge gibts jetzt zum ifgi, oder zum Schloss.
    \item[I:] Die Filter- und Sortierfunktion?
    \item[P1:] Ja das macht auf jeden Fall beides Sinn. Und wenn man jetzt gerade neu rein kommt, und da ist schon einiges auf der Karte (\dots) Da will man ja wahrscheinlich auch erstmal gucken was schon da ist, bevor man jetzt irgendwas rein schreibt, was schon drei andere geschrieben haben. Macht auf jeden Fall Sinn. Also bis jetzt wie ich es gesehen habe aus so wie ich es erwarten würde.
    \item[I:] Dann diese Verfassen- und Antworten- Funktion?
    \item[P1:] Finde ich gut, aber was auf die Dauer vielleicht ein bisschen verwirrend werden könnte, ist die Möglichkeit da neue Akteure, Aktivitäten und Inhalte hinzuzufügen. Weil, wenn da ordentlich gebrauch von gemacht wird, dann könnte es glaube ich unübersichtlich werden. Also in einer echten Anwendung würde man das vielleicht abschalten wollen könnte ich mir vorstellen.
    \item[I:] Normalerweise ist das auch abgeschalten, aber hier auf meiner lokalen Version ist das noch möglich.
    \item[P1:] Ah okay.
    \item[I:] Dann das Verkn{\"u}pfen von den W{\"o}rtern mit den Geoobjekten, bestehenden Geoobjekten und Hyperlinks?
    \item[P1:] Macht Sinn. Ich weiß jetzt nicht wie intuitiv das ist. Was da vielleicht noch hilfreich wäre, wäre eine Anbindung an einen Geocoder, der dann aus dem aktuellen Kartenausschnitt nach Objekten durchsucht, die dem eingegebenen Wort schon entsprechen. Das würde das ganze wahrscheinlich nochmal ein gutes Stück einfacher machen, als das einzugeben oder manuell den Marker zu setzen. (Das ist eigentlich eine gute Idee)
    \item[I:] Dann die Favorisierung der Beiträge?
    \item[P1:] Finde ich auch gut. Müsste man sich dann konkret mal angucken, wie die Leute das benutzen. Also ist das wahrscheinlich eher so gedacht wie so eine Art "`Finde ich gut. Das unterstütze ich, diese Idee"'-Funktion. Aber ich kann mir auch vorstellen, dass einige Leute das so als eine Art Bookmark benutzen für Diskussionen, die sie irgendwie weiterverfolgen wollen oder so. Müsste man sich dann konkret einfach angucken, wie die Leute das benutzen. Kann ich nicht sagen.
    \item[I:] Die Benutzerregistrierung und Anmeldung?
    \item[P1:] Ja ist ja ziemlich Standard. Was man erwarten würde, würde ich sagen. Schien mir intuitiv zu sein.
    \item[I:] Dann konkret der Social Login?
    \item[P1:] Finde ich gut. Ich glaube das macht es einfacher für viele Leute.
    \item[I:] Dann ganz speziell jetzt auf die Anwendung bezogen. Wie werden Dialoge damit vereinfacht?
    \item[P1:] Kommt auf den Dialog an wahrscheinlich. Wollen die die Anwendung auch für Planung benutzen? (Ja wäre ja prinzipiell möglich) Ist wohl sowohl für den angedachten Einsatz als auch für die Planung nützlich. Wobei ich mir vorstellen könnte, dass es noch für die Planung vielleicht noch hilfreicher wäre. Für mal so eine Gruppe von Leuten, die sich jetzt über Wochen und Monate mit diesem Tag beschäftigen und da irgendwie so festhalten wollen, was es da für Ideen gibt. Und so Pro- und Contra-Argumente zu sammeln. Dann als Rückkanal für die Leute, um jetzt hinterher zu sagen "`Fand ich gut, da würde ich wieder hingehen"'. Das ist vielleicht sogar so wie "`Kanonen auf Spatzen geschossen."' Aber das ist auch nur eine Vermutung. Muss man ausprobieren. 
    \item[I:] Haben Sie Funktionen vermisst?
    \item[P1:] In diesem Kontext jetzt nicht. Also diese ganze ArguMap-Sache, die wir da damals gemacht haben, die war ja eher so in einem Bürgerbeteiligungskontext. Und dann war so das Grundszenario dass es da so verschiedene Planungsvarianten auf die man dann antworten konnte. Da hatten wir dann auch diese Funktion, dass man angeben konnte, ob das jetzt ein Pro oder Kontra Argument ist. Dass man halt schnell einen Überblick kriegt, das eine hat irgendwie eine breite Zustimmung oder breite Ablehnung. Da hatten wir irgendwie glaube ich so Plus und Minus Symbole benutzt. Aber für den Anwendungskontext hier, brauch man das glaube ich nicht. Ja sonst das einzige was ich ja schon gesagt hab, war halt die Einbindung eines Geocoders. Das würde es wahrscheinlich noch ein ganze Ecke benutzbarer machen.
    \item[I:] Was f{\"u}r Gr{\"u}nde k{\"o}nnen Sie sich vorstellen die Leute davon abhalten k{\"o}nnten, die Karte zu benutzen?
    \item[P1:] Also man hat ja immer noch eine gewisse Lernkurve. Ich glaube das ist auch nicht weg zu kriegen bei solchen Tools, weil (\dots) ich wüsste nicht wie man das noch einfacher machen soll. Man muss ja eh immer durch die einzelnen Screens klicken. Und wenn man das ignoriert, dann muss man halt einfach ein bisschen rumgucken, wie das funktioniert. Das könnte ich mir vorstellen, dass Leute, die gerade nicht so viel online machen, (\dots) ein bisschen abgeschreckt sind, und die Finger davon lassen.
    \item[I:] Gut. Das war es von meiner Seite so. Gibt es von Ihrer Seite noch Anmerkungen oder Fragen?
    \item[P1:] Machst du das als Open Source verfügbar? (lacht)
    \item[I:] Ja. Das ist auf jeden Fall geplant.
    \item[P1:] Cool. Könnte ich mir auf jeden Fall auch noch vorstellen, dass man das hier mal irgendwie benutzen könnte. Was man natürlich auch mal überlegen könnte, wäre so langfristig so eine Art Vergleich zu machen zwischen dem ArguMap und deiner Anwendung. Gut, das ArguMap-Tool was ich damals gemacht habe, ist auch schon ein bisschen in die Jahre gekommen. Dass man das vielleicht nochmal reaktiviert. Und dann so eine Art Vergleich macht zwischen den beiden.
    \item[I:] Ja. Ich muss zugeben, ich habe es nicht zum laufen bekommen.
    \item[P1:] Das kann gut sein, das ist ja auch eine Version von der Google Maps API, die schon älter ist auf jeden Fall. Ich weiß auch gar nicht was ich da Serverseitig benutzt habe. Das war glaube ich in PHP.
    \item[I:] Ja, dann war es das schon jetzt. Vielen Dank!        
\end{itemize}

\end{document}
